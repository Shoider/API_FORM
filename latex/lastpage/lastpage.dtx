% \iffalse meta-comment
%
% File: lastpage.dtx
% Version: 2025-01-27 v2.1e
% Info: Refers to last page's name
%
% Copyright © 2010 - 2025 by
%    H.-Martin Münch <Martin dot Muench at Uni-Bonn dot de>
% Portions of code copyrighted by other people as marked.
%
% This package was invented by Jeffrey P. Goldberg.
% I thought that an enhancement was needed and therefore created the pageslts package,
% https://ctan.org/pkg/pageslts. Nevertheless, for compatibility with existing
% documents/packages as well as for the low amount of resources needed by the
% lastpage package (no new counter!), I updated this package.
% Thanks go to Jeffrey P. Goldberg for allowing me to do this.
%
% While this package allows for things like "Page \thepage{} of \pageref{LastPage}"
% to get "Page 7 of 9" or "Page VII of IX",
% the NUMBER of pages is nowadays available from the kernel
% (\@abspage@last, \thetotalpages, \PreviousTotalPages),
% but with pages for example I to X and then 1 to 10, the number of pages would be "20",
% while the name of the last page is "10". Decide what you need/want!
%
% This work may be distributed and/or modified under the
% conditions of the LaTeX Project Public License, either
% version 1.3c of this license or (at your option) any later
% version. This version of this license is in
%    https://www.latex-project.org/lppl/lppl-1-3c.txt
% and the latest version of this license is in
%    https://www.latex-project.org/lppl.txt
% and version 1.3c or later is part of all distributions of
% LaTeX version 2005-12-01 or later.
%
% This work has the LPPL maintenance status "maintained".
%
% The Current Maintainer of this work is H.-Martin Münch.
%
% This work consists of the main source file lastpage.dtx,
% the README, and the derived files
%    lastpage.sty, lastpage209.sty, lastpage2e.sty,
%    lastpageclassic.sty, lastpagemodern.sty,
%    lastpage.pdf, lastpage.ins, lastpage.drv,
%    lastpage-example.tex, lastpage-example.pdf.
%
% "lastpage" is available on CTAN:
% https://ctan.org/pkg/lastpage
%
% Also a TDS.ZIP file is provided that contains all the files
% already sorted in a TDS tree:
% https://mirror.ctan.org/install/macros/latex/contrib/lastpage.tds.zip
%
%<*ignore>
\begingroup
  \catcode123=1 %
  \catcode125=2 %
  \def\x{LaTeX2e}%
\expandafter\endgroup
\ifcase 0\ifx\install y1\fi\expandafter
         \ifx\csname processbatchFile\endcsname\relax\else1\fi
         \ifx\fmtname\x\else 1\fi\relax
\else\csname fi\endcsname
%</ignore>
%<*install>
\input docstrip.tex
\Msg{**********************************************************************}
\Msg{* Installation                                                       *}
\Msg{* Package: lastpage 2025-01-27 v2.1e Refers to last page's name (HMM)*}
\Msg{**********************************************************************}

\keepsilent
\askforoverwritefalse

\let\MetaPrefix\relax
\preamble

This is a generated file.

Project: lastpage
Version: 2025-01-27 v2.1e
Info: Refers to last page's name

Copyright (C) 2010 - 2025 by
    H.-Martin Muench <Martin dot Muench at Uni-Bonn dot de>
Portions of code copyrighted by other people as marked.

This work may be distributed and/or modified under the
conditions of the LaTeX Project Public License, either
version 1.3c of this license or (at your option) any later
version. This version of this license is in
   https://www.latex-project.org/lppl/lppl-1-3c.txt
and the latest version of this license is in
   https://www.latex-project.org/lppl.txt
and version 1.3c or later is part of all distributions of
LaTeX version 2005-12-01 or later.

This work has the LPPL maintenance status "maintained".

The Current Maintainer of this work is H.-Martin Muench.

This package was invented by Jeffrey P. Goldberg.
I thought that an enhancement was needed and therefore created the pageslts package,
https://ctan.org/pkg/pageslts. Nevertheless, for compatibility with existing
documents/packages as well as for the low amount of resources needed by the
lastpage package (no new counter!), I updated this package.
Thanks go to Jeffrey P. Goldberg for allowing me to do this.

While this package allows for things like "Page slash thepage{} of slash pageref{LastPage}"
to get "Page 7 of 9" or "Page VII of IX",
the NUMBER of pages is nowadays available from the kernel
(slash @abspage@last, slash thetotalpages, slash PreviousTotalPages),
but with pages for example I to X and then 1 to 10, the number of pages would be "20",
while the name of the last page is "10". Decide what you need/want!

This work consists of the main source file lastpage.dtx,
the README, and the derived files
   lastpage.sty, lastpage209.sty, lastpage2e.sty,
   lastpageclassic.sty, lastpagemodern.sty,
   lastpage.pdf, lastpage.ins, lastpage.drv,
   lastpage-example.tex, lastpage-example.pdf.

In memoriam
 Claudia Simone Barth + 1996-01-30
 Tommy Muench         + 2014-01-02
 Hans-Klaus Muench    + 2014-08-24

\endpreamble
\let\MetaPrefix\DoubleperCent

\generate{%
  \file{lastpage.ins}{\from{lastpage.dtx}{install}}%
  \file{lastpage.drv}{\from{lastpage.dtx}{driver}}%
  \usedir{tex/latex/lastpage}%
  \file{lastpage.sty}{\from{lastpage.dtx}{package}}%
  \file{lastpage209.sty}{\from{lastpage.dtx}{lastpage209}}%
  \file{lastpage2e.sty}{\from{lastpage.dtx}{lastpage2e}}%
  \file{lastpageclassic.sty}{\from{lastpage.dtx}{lastpageclassic}}%
  \file{lastpagemodern.sty}{\from{lastpage.dtx}{lastpagemodern}}%
  \usedir{doc/latex/lastpage}%
  \file{lastpage-example.tex}{\from{lastpage.dtx}{example}}%
}

\catcode32=13\relax% active space
\let =\space%
\Msg{************************************************************************}
\Msg{*}
\Msg{* To finish the installation you have to move the following}
\Msg{* file into a directory searched by TeX:}
\Msg{*}
\Msg{*  lastpage.sty, lastpage209.sty, lastpage2e.sty,}
\Msg{*  lastpageclassic.sty, and lastpagemodern.sty}
\Msg{*}
\Msg{* To produce the documentation run the file `lastpage.drv'}
\Msg{* through (pdf)LaTeX, e.g.}
\Msg{*  pdflatex lastpage.drv}
\Msg{*  makeindex -s gind.ist lastpage.idx}
\Msg{*  pdflatex lastpage.drv}
\Msg{*  makeindex -s gind.ist lastpage.idx}
\Msg{*  pdflatex lastpage.drv}
\Msg{*}
\Msg{* At least three runs are necessary e.g. to get the}
\Msg{*  references right!}
\Msg{*}
\Msg{* Happy TeXing!}
\Msg{*}
\Msg{************************************************************************}

\endbatchfile
%</install>
%<*ignore>
\fi
%</ignore>
%
% \section{The documentation driver file}
%
% The next bit of code contains the documentation driver file for
% \TeX , i.\,e., the file that will produce the documentation you
% are currently reading. It will be extracted from this file by the
% \texttt{docstrip} programme. That is, run \LaTeX{} on \texttt{docstrip}
% and specify the \texttt{driver} option when \texttt{docstrip}
% asks for options.
%
%    \begin{macrocode}
%<*driver>
\NeedsTeXFormat{LaTeX2e}[2024-11-01]
\ProvidesFile{lastpage.drv}[2025-01-27 v2.1e Refers to last page's name (HMM)]
\documentclass{ltxdoc}[2024/02/08]% v2.1j Standard LaTeX documentation class
\usepackage{holtxdoc}[2019/12/09]%  v0.30 Private additional ltxdoc support (HO)
\hypersetup{%
 pdfsubject={Refers to last page's name (HMM; JPG)},%
 pdfkeywords={LaTeX, lastpage, last page, page number, page name},%
 pdfencoding=auto,%
 pdflang={en},%
 breaklinks=true,%
 linktoc=all,%
 pdfstartview=FitH,%
 pdfpagelayout=OneColumn,%
 bookmarksnumbered=true,%
 bookmarksopen=true,%
 bookmarksopenlevel=2,%
 pdfmenubar=true,%
 pdftoolbar=true,%
 pdfwindowui=true,%
 pdfnewwindow=true%
}
\CodelineIndex
\hyphenation{every-thing}
\begin{document}
  \DocInput{lastpage.dtx}%
\end{document}
%</driver>
%    \end{macrocode}
% \fi
%
% \GetFileInfo{lastpage.drv}
%
% \begingroup
%   \def\x{\#,\$,\^,\_,\~,\ ,\&,\{,\},\%}%
%   \makeatletter
%   \@onelevel@sanitize\x
% \expandafter\endgroup
% \expandafter\DoNotIndex\expandafter{\x}
% \expandafter\DoNotIndex\expandafter{\string\ }
% \begingroup
%   \makeatletter
%     \lccode`9=32\relax
%     \lowercase{%^^A
%       \edef\x{\noexpand\DoNotIndex{\@backslashchar9}}%^^A
%     }%^^A
%   \expandafter\endgroup\x
%
% \DoNotIndex{\@auxout,\@evenfoot,\@firstofone,\@firstoftwo,\@ifl@t@r,\@ifpackagelater,\@ifpackageloaded,%
% \IfPackageAtLeastF,\IfPackageLoadedT,\IfPackageLoadedTF,%
% \@ifundefined,\@kernel@reserved@label@data,\@mainaux,\@number,\@oddfoot,%
% \addtocounter,\AddToHook,\arabic,\begin,%
% \bigskip,\clearpage,\csname,\detokenize,\documentclass,\EdefUnescapeString,\empty,\end,%
% \endcsname,\enddocument,\expanded,\ExplSyntaxOff,\ExplSyntaxOn,\fmtversion,%
% \g,\gdef,\here,\hfil,\Hy@temp,\Hy@unicodefalse,\Hy@EveryPageAnchor,%
% \hypersetup,\if,\if@filesw,\if@nobreak,\ifHy@hypertexnames,\ifHy@pageanchor,\ifHy@plainpages,%
% \ifdim,\ifnum,\ifvmode,\ifx,\immediate,\input,%
% \label,\lastpage,\lastpage-example.tex,\lastpage.dtx,\lastpage.sty,\lastpagee.sty,%
% \lastpageclassic.sty,\lastpagemodern.sty,\LaTeX,\LaTeX-kernel,\listfiles,%
% \makeatletter,\makeatother,\markboth,\meaning,\mbox,\message,\MessageBreak,%
% \NeedsTeXFormat,\newcommand,\newcounter,\newline,\newpage,\nobreak,%
% \noindent,\normalsize,\numexpr,\origenddocument,\PackageError,\PackageNoteNoLine,\PackageWarning,%
% \PackageWarningNoLine,\pagenumbering,\pageref,\pdfstringdef,\protect,\ProvidesPackage,%
% \qquad,\renewcommand,\RequirePackage,\section,\ShowHook,\slshape,\smallskip,\space,\test,\textbf,%
% \textit,\textquotedblleft,\textquotedblright,\textsf,\texttt,\the,\thepage,%
% \today,\upshape,\url,\usepackage,\value,\verb,\wd,\write,\xlastpage@rmpage,\xxlastpage@rmpage}
%
% \title{The \xpackage{lastpage} package}
% \author{H.-Martin Münch\\\xemail{Martin.Muench at Uni-Bonn.de}\\
%   invented by Jeffrey P. Goldberg\\\xemail{jeffrey+news at goldmark.org}}
% \date{2025-01-27 v2.1e}
% \maketitle
%
% \begin{abstract}
% \noindent This \LaTeX{} package puts the label \texttt{LastPage}
% at the end of the document into the \xfile{.aux} file,
% allowing the user to refer to the last page of a document.
% This might be particularly useful in places like headers or footers.~--\newline
% While this package allows for things like\newline
% \hbox{\textquotedblleft Page} \hbox{|\thepage{}|} of |\pageref{LastPage}|\textquotedblright{}
% to get \textquotedblleft Page 7 of 9\textquotedblright{} or
% \textquotedblleft Page VII of IX\textquotedblright,
% the \emph{number} of pages is nowadays available from the kernel
% (|\@abspage@last|, |\thetotalpages|, |\PreviousTotalPages|),
% but when more than one page numbering scheme is used
% (for example pages I~to X and then 1~to 10,
% thus number of pages \hbox{\textquotedblleft 20\textquotedblright ,}
% but name of the last page \hbox{\textquotedblleft 10\textquotedblright ),}
% or another package has output after this package,
% or the page numbers exceed a certain range,
% there might be problems, which can be solved by using the
% \xpackage{pageslts} package instead.
% \end{abstract}
%
% \bigskip
%
% \noindent Disclaimer for web links: The author is not responsible for any contents
% referred to in this work unless having full knowledge of illegal contents.
% If any damage occurs by the use of information presented there,
% only the author of the respective pages might be liable,
% not the one who has referred to those pages.
% \newpage
% \tableofcontents
%
% \section{Introduction}
% \indent This \LaTeX{} package puts the label \texttt{LastPage}
% (at end of the document via hook |enddocument/afterlastpage|,
% for older formats via |\AtEndDocument|, for \LaTeX2.09{} via
% redefining |\enddocument|) into the \xfile{aux} file,
% allowing the user to refer to the last page of a document via
% |\pageref{LastPage}|. This might be particularly useful
% in places like headers or footers.
%
% \bigskip
%
% This package was invented by \textbf{Jeffrey P. Goldberg},
% and is now maintained by \textsc{H.-Martin Münch}. A~big
% \textquotedblleft Thank you!\textquotedblright{} to
% \textsc{Jeffrey P. Goldberg} for granting this.
%
% \bigskip
%
% If you are more ambitious in respect to your aims with this package,
% you might want to have a look at the \xpackage{pageslts} package
% (see section~\ref{sec:Alternatives}: Alternatives).
%
% \section{Usage}
% \indent Just load the package placing
% \begin{quote}
%   |\usepackage{lastpage}|
% \end{quote}
% \noindent in the preamble of your source file
% (or |%%
%% This is file `lastpage.sty',
%% generated with the docstrip utility.
%%
%% The original source files were:
%%
%% lastpage.dtx  (with options: `package')
%% 
%% This is a generated file.
%% 
%% Project: lastpage
%% Version: 2025-01-27 v2.1e
%% Info: Refers to last page's name
%% 
%% Copyright (C) 2010 - 2025 by
%%     H.-Martin Muench <Martin dot Muench at Uni-Bonn dot de>
%% Portions of code copyrighted by other people as marked.
%% 
%% This work may be distributed and/or modified under the
%% conditions of the LaTeX Project Public License, either
%% version 1.3c of this license or (at your option) any later
%% version. This version of this license is in
%%    https://www.latex-project.org/lppl/lppl-1-3c.txt
%% and the latest version of this license is in
%%    https://www.latex-project.org/lppl.txt
%% and version 1.3c or later is part of all distributions of
%% LaTeX version 2005-12-01 or later.
%% 
%% This work has the LPPL maintenance status "maintained".
%% 
%% The Current Maintainer of this work is H.-Martin Muench.
%% 
%% This package was invented by Jeffrey P. Goldberg.
%% I thought that an enhancement was needed and therefore created the pageslts package,
%% https://ctan.org/pkg/pageslts. Nevertheless, for compatibility with existing
%% documents/packages as well as for the low amount of resources needed by the
%% lastpage package (no new counter!), I updated this package.
%% Thanks go to Jeffrey P. Goldberg for allowing me to do this.
%% 
%% While this package allows for things like "Page slash thepage{} of slash pageref{LastPage}"
%% to get "Page 7 of 9" or "Page VII of IX",
%% the NUMBER of pages is nowadays available from the kernel
%% (slash @abspage@last, slash thetotalpages, slash PreviousTotalPages),
%% but with pages for example I to X and then 1 to 10, the number of pages would be "20",
%% while the name of the last page is "10". Decide what you need/want!
%% 
%% This work consists of the main source file lastpage.dtx,
%% the README, and the derived files
%%    lastpage.sty, lastpage209.sty, lastpage2e.sty,
%%    lastpageclassic.sty, lastpagemodern.sty,
%%    lastpage.pdf, lastpage.ins, lastpage.drv,
%%    lastpage-example.tex, lastpage-example.pdf.
%% 
%% In memoriam
%%  Claudia Simone Barth + 1996-01-30
%%  Tommy Muench         + 2014-01-02
%%  Hans-Klaus Muench    + 2014-08-24
%% 
%% Part of the "lastpage" package
%% loads either lastpage2.09.sty for TeX 2.09 or lastpage2e.sty for LaTeX 2e
%% with code from https://groups.google.com/g/comp.text.tex/c/-Qmhj1ZI4xM
\def\loadlastpage{\ProvidesPackage{lastpage}[2025/01/27 v2.1e lastpage: 2.09 or 2e? (HMM)]\relax\RequirePackage{lastpage2e}}
\begingroup \expandafter \ifx \csname documentclass\endcsname\relax
\endgroup \expandafter %%
%% This is file `lastpage209.sty',
%% generated with the docstrip utility.
%%
%% The original source files were:
%%
%% lastpage.dtx  (with options: `lastpage209')
%% 
%% This is a generated file.
%% 
%% Project: lastpage
%% Version: 2025-01-27 v2.1e
%% Info: Refers to last page's name
%% 
%% Copyright (C) 2010 - 2025 by
%%     H.-Martin Muench <Martin dot Muench at Uni-Bonn dot de>
%% Portions of code copyrighted by other people as marked.
%% 
%% This work may be distributed and/or modified under the
%% conditions of the LaTeX Project Public License, either
%% version 1.3c of this license or (at your option) any later
%% version. This version of this license is in
%%    https://www.latex-project.org/lppl/lppl-1-3c.txt
%% and the latest version of this license is in
%%    https://www.latex-project.org/lppl.txt
%% and version 1.3c or later is part of all distributions of
%% LaTeX version 2005-12-01 or later.
%% 
%% This work has the LPPL maintenance status "maintained".
%% 
%% The Current Maintainer of this work is H.-Martin Muench.
%% 
%% This package was invented by Jeffrey P. Goldberg.
%% I thought that an enhancement was needed and therefore created the pageslts package,
%% https://ctan.org/pkg/pageslts. Nevertheless, for compatibility with existing
%% documents/packages as well as for the low amount of resources needed by the
%% lastpage package (no new counter!), I updated this package.
%% Thanks go to Jeffrey P. Goldberg for allowing me to do this.
%% 
%% While this package allows for things like "Page slash thepage{} of slash pageref{LastPage}"
%% to get "Page 7 of 9" or "Page VII of IX",
%% the NUMBER of pages is nowadays available from the kernel
%% (slash @abspage@last, slash thetotalpages, slash PreviousTotalPages),
%% but with pages for example I to X and then 1 to 10, the number of pages would be "20",
%% while the name of the last page is "10". Decide what you need/want!
%% 
%% This work consists of the main source file lastpage.dtx,
%% the README, and the derived files
%%    lastpage.sty, lastpage209.sty, lastpage2e.sty,
%%    lastpageclassic.sty, lastpagemodern.sty,
%%    lastpage.pdf, lastpage.ins, lastpage.drv,
%%    lastpage-example.tex, lastpage-example.pdf.
%% 
%% In memoriam
%%  Claudia Simone Barth + 1996-01-30
%%  Tommy Muench         + 2014-01-02
%%  Hans-Klaus Muench    + 2014-08-24
%% 
%% Part of the "lastpage" package
%% FOR LaTeX 2.09 ONLY - FOR LaTeX 2e USE lastpage2e.sty
%% This is lastpage209.sty invented by Jeffrey P. Goldberg,
%% after Piet van Oostrum: Page layout in LaTeX, March 2, 2004, section 16;
%% fancyhdr.pdf; lastpage209.sty maintained by H.-Martin Muench.
\let\origenddocument=\enddocument%
\def\enddocument{\clearpage%
  {\addtocounter{page}{-1}%
   \immediate\write\@mainaux{\string\newlabel{LastPage}{{}{\thepage}}}%
  }%
  \addtocounter{page}{+1}%
  \origenddocument%
 }
\endinput
%%
%% End of file `lastpage209.sty'.

\else \endgroup \expandafter \loadlastpage
\fi
\endinput
%%
%% End of file `lastpage.sty'.
| if |\usepackage| is unknown).
%
% \indent For example for various draft forms it is desirable to have a
% page reference to the last page, so that e.\,g.\ page footers can
% contain something like \textquotedblleft page $N$ of $K$\textquotedblright,
% where $N$ is the current page and $K$ is the last page. Once the package
% is loaded, anywhere in the text references can be made to the label
% \texttt{LastPage}. In particular one can use the \xpackage{fancyhdr}
% or \xpackage{nccfancyhdr} package, or redefinitions of the page headings
% and footings to get a reference to the last page.
%
% \noindent In your document the code
% \begin{verbatim}
% \makeatletter
% \renewcommand{\@evenfoot}{%
%  \normalsize\slshape DRAFT \today\hfil \upshape %
%  page \thepage{} of \pageref{LastPage}}
% \renewcommand{\@oddfoot}{\@evenfoot}
% \makeatother
%\end{verbatim}
% \noindent creates footers like\smallbreak
% \textquotedblleft\mbox{\textsl{DRAFT \today}\hspace{1cm}page 7 of 9}\textquotedblright\smallbreak
% \noindent in the compiled document (cf.~the \texttt{lastpage-example} file).\newline
% If the \xpackage{hyperref} package is used, the references are hyperlinked
% to their aims. If these hyperlinks shall be suppressed, |\pageref*{...}|
% instead of |\pageref{...}| can be used.\bigbreak
%
% The \xpackage{lastpage} package does not provide the words
% \textquotedblleft page\textquotedblright{} or \textquotedblleft of\textquotedblright{},
% but e.\,g.\ the \xclass{handout} class uses \textquotedblleft of\textquotedblright{} in
% the definition of the footer. (In the \texttt{lastpage-example} also
% |\@evenfoot| is redefined, but it is not the \xpackage{lastpage} \emph{package}
% redefining this.) If you want to change \textquotedblleft page\textquotedblright{} or
% \textquotedblleft of\textquotedblright{} (e.\,g.\ to another language), you therefore
% have got to look in the used class/package(s)/preamble instead of in the
% \xpackage{lastpage} package.\bigbreak
%
% If the total \emph{number} of pages of a document is needed,
% the kernel already gives this by |\makeatletter\@abspage@last\makeatother|,
% |\thetotalpages|, and |\PreviousTotalPages| (needing at least two compiler runs).
%
% \section{Some \textsc{Warnings}\label{sec:warn}}
% \subsection{\texttt{\textbackslash AtEndDocument}\label{ssec:aed}}
% \indent {\bfseries |\AtEndDocument| is not used by the \pkg{lastpagemodern.sty}
% version of the lastpage package, requiring \LaTeX -format 2024-11-01
% or newer. Instead |\AddToHook{enddocument/afterlastpage}| is used
% and the problem does not arise.}
% \bigskip
%
% \pkg{lastpageclassic.sty} uses |\AtEndDocument| and
% \pkg{lastpage209.sty} redefines \linebreak |\enddocument|.
% The last two cases are problematic:\newline
% The output of a \LaTeXe{} run is not independent of the order
% in which the packages are loaded. It is often the case that the same
% formats for which one must put tables and figure at the end,
% are the ones in which endnotes are also required.
% If one wants to use |\AtEndDocument| here as well (as done for
% |\pageref{LastPage}|), then it is easy to get to three separate
% uses of |\AtEndDocument| (assuming one uses this for the endnotes
% as well). Clearly it is not safe for any package writer or user
% to assume that no material will follow what they put into
% |\AtEndDocument|. Therefore a message, which begins with
% \texttt{AED}, is included in every usage of |\AtEndDocument|.
% \xpackage{lastpage} uses |\AtEndDocument{|\ldots|\clearpage|\ldots|}|, thus\newline
%|\usepackage{lastpage}|\ldots|\AtEndDocument{|\textit{something}|}|
% will place \textit{something} after the |\clearpage|. To place it earlier, use\newline
% |\AtEndDocument{|\textit{something}|}|\ldots|\usepackage{lastpage}|.
% If the \textit{something} is not known before |\usepackage{lastpage}|,
% you can use for example
% \begin{verbatim}
% ...
% \def\beforeLastpageClearpage{\relax}
% \AtEndDocument{\beforeLastpageClearpage}
% \usepackage{lastpage}
% \begin{document}
% ...
% \def\beforeLastpageClearpage{\textit{something}}%
% ...
% \end{document}
% \end{verbatim}
% (might need a protected and/or expanded |\def|). When |\clearpage| leads
% to some output, |\clearpage\textit{something}| instead of |\textit{something}|
% might be wanted.
%
% \subsection{Interaction with ancient versions of the \xpackage{endfloat} package\label{sec:endfloat}}
%
% \indent {\bfseries |\AtEndDocument| is not used by the \pkg{lastpagemodern.sty}
% version of the lastpage package, requiring \LaTeX -format 2024-11-01
% or newer. Instead |\AddToHook{enddocument/afterlastpage}| is used
% and the problem does not arise.}
%
% \bigskip
% The \emph{ancient} version~2.0 (and earlier; 2.0~from 1992;
% current version at the time of updating this documentation: 2.7~from 2019)
% of the \xpackage{endfloat} package actually redefined the |\enddocument| command,
% and so interfered drastically with the \LaTeXe{} commands which make use of
% |\AtEndDocument|.
% If you want your \texttt{LastPage} to label the last page of these end floats,
% you need to load \xpackage{lastpage} after loading \xpackage{endfloat}
% (or use \texttt{VeryLastPage} from the \xpackage{pageslts} package instead).
% If, on the other hand, you \emph{want} \texttt{LastPage} to refer
% to the (not so) last page, exclusive of the floats at the end,
% then load in the reverse order. Independent from the order of
% \xpackage{lastpage} and \xpackage{endfloat}, you will still need
% a version of \xpackage{endfloat} later than 2.0 from 1992.
%
% Other \LaTeX2.09{} (!) packages also seem to like to redefine
% |\enddocument|. In addition to the old \xpackage{endfloat},
% \xpackage{harvard} comes to mind. All of these will need to be
% modified swiftly.
%
% \subsection{Page name instead of page number}
% When any page numbering scheme other than \texttt{arabic} is used
% at the page, which |\pageref{LastPage}| refers to, the \textit{name}
% and not the \textit{number} of the page is given. For example,
% \texttt{Alph} page numbering scheme and $10$ pages will give \texttt{J} instead of 10,
% \texttt{Roman} page numbering scheme and $10$ pages will give \texttt{X} instead of 10,
% and so on.\newline
% \indent (The \xpackage{pageslts} package puts |\lastpageref{LastPages}|
% (with \textbf{s} at the end) at your disposal for remediation.)
%
% \subsection{No write access to the \xfile{aux} file}
% Some packages (e.\,g.\ \xpackage{tikz} and \xpackage{selectp}) sometimes prevent
% the output to the \xfile{aux} file. In that case a warning is issued. This is
% no problem as long as there is another compilation run where the label to the
% last page can be placed via the \xfile{aux} file.
%
% \subsection[Wrong last page number with \textgreater 1 page numbering scheme]{%
%  Wrong last page number with more than one page numbering scheme}
% When more than one page numbering scheme is used,
% \texttt{LastPage} does not give the total \textbf{number} of pages
% (even if \texttt{arabic} is the page numbering scheme of that page).
% For example, for a document with VI+36 pages, it gives
% \textquotedblleft 36\textquotedblright{} as reference to the last page.
% While this is correct, the total number of pages is $42$.\newline
% If the total \emph{number} of pages of a document is needed,
% the kernel already gives this by |\makeatletter\@abspage@last\makeatother|,
% |\thetotalpages|, and |\PreviousTotalPages|.
% The \xpackage{pageslts} package puts |\lastpageref{LastPages}|
% (with \textbf{s} at the end) at your disposal for remediation,
% giving the number of pages and linking to the last page, if linking is provided
% for examaple by the \xpackage{hyperref} package.
%
% \subsection{\texttt{\textbackslash addtocounter\{page\}\{\ldots\} and %
%  \texttt{\textbackslash setcounter\{page\}\{\ldots\}}}}
% When the page number was manipulated by |\addtocounter{page}{...}| or
% |\setcounter{page}{...}|, \texttt{LastPage} does not give the total
% \textbf{number} of pages (even if \texttt{arabic} is the page numbering
% scheme of that page).\newline
% The \xpackage{pageslts} package puts |\lastpageref{LastPages}|
% (with \textbf{s} at the end) at your disposal for remediation:
% \texttt{LastPages} ignores page number manipulation. Also |\@abspage@last|,
% |\thetotalpages|, and |\PreviousTotalPages| from the kernel
% are not influenced by page number manipulation.
%
% \subsection{Page number reset by \texttt{\textbackslash pagenumbering\{\ldots\}}}
% At a page numbering change the page number is reset to one.
% Therefore \texttt{LastPage} does not give the total \textbf{number} of pages
% (even if \texttt{arabic} is the page numbering scheme of that page).
% Furthermore, now two pages have the same name.\newline
% The \xpackage{pageslts} package does not only put
% |\lastpageref{LastPages}| (with \textbf{s} at the end) at your disposal
% for remediation: \texttt{LastPages} also ignores page number manipulation.
% It furthermore offers the option |pagecontinue| to continue the
% page numbering, when |\pagenumbering{...}| is used.
%
% \subsection{Last pages of different page numbering schemes}
% |\pageref{LastPage}| refers to the (maybe not so) last page of the last
% page numbering scheme. References to the respective last page of the other
% page numbering schemes are not provided.\newline
% The \xpackage{pageslts} package does this with labels
% \texttt{pagesLTS.<numbering scheme>}, where \texttt{<numbering scheme>} is
% e.\,g.\ arabic, roman, Roman, alph, or Alph.\linebreak
% For fnsymbol please use |\lastpageref{pagesLTS.fnsymbol}| instead of\newline
% |\pageref{pagesLTS.fnsymbol}|.
%
% \subsection{Current page}
% The command |\thepage| gives the \textbf{name} of the current page
% in the current page numbering scheme, which is different from the
% current total/absolute page number e.\,g.\ with a second
% page numbering scheme, |\addtocounter{page}{...}|, or |\setcounter{page}{...}|,
% and it will not be an arabic number at all,
% if the current page numbering scheme is not arabic.\newline
% The \xpackage{pageslts} package provides the command
% |\theCurrentPage| and for the current number of pages in the current
% page numbering scheme\newline
% |\theCurrentPageLocal|. The kernel already provides the
% number of pages, which have been shipped out, as |\the\ReadonlyShipoutCounter|.
% The current page is always |ReadonlyShipoutCounter +1|.
%
% \subsection{First page}
% There is no special label at the first page.
% (This is the \textbf{last}\textsf{page} package, after all.)\newline
% The \xpackage{pageslts} package creates a label \texttt{pagesLTS.0}
% at the first page of the document.
%
% \subsection{Page counter overflow\label{ss:overflow}}
% \indent \textquotedblleft The ranges of supported counter values are more or less
% restricted. Only \cs{arabic} can be used with any counter value \TeX{} supports.
% \begin{quote}
% \catcode`\|=12 %
% \begin{tabular}{@{}l|l|l|l@{}}
% Presentation & Supported & Ignored & Error message\\
% command      & domain    & values  & `Counter too large'\\
% \hline
% \cs{arabic}
%   & \ttfamily -MAX..MAX & &\\
% \cs{roman}, \cs{Roman}
%   & \ttfamily 1..MAX & \ttfamily -MAX..0 &\\
% \cs{alph}, \cs{Alph}
%   & \ttfamily 1..26 & 0 & \ttfamily -MAX..-1, 27..MAX\\
% \cs{fnsymbol}
%   & \ttfamily 1..9 & 0 & \ttfamily -MAX..-1, 10..MAX\\
% \hline
% \end{tabular}\\[1ex]
% \texttt{MAX} = \texttt{2147483647}
% \end{quote}
% \textquotedblright\ (\xpackage{alphalph} package manual, 2019/12/09, v2.6,
% first table, p.~2).
%
% \noindent When \textit{any} page is out of that range, there will be a counter overflow.\newline
% \xpackage{lastpage} probably is not the right package to be asked
% to correct this anyway, but the \xpackage{pageslts} package
% (with appropriate options) can do this.
%
% When MAX is exceeded via
% |\setcounter{<name>}{| something greater than MAX (or smaller then $-$MAX) |}|,
% then the error
% \begin{quote}
% \begin{verbatim}
% ! Number too big.
% I can only go up to 2147483647='17777777777="7FFFFFFF,
% so I'm using that number instead of yours.
% \end{verbatim}
% \end{quote}
% \vspace{-\baselineskip}
% will arise. But if the counter has a value of $2\,147\,483\,647 = {}$MAX,
% and |\addtocounter{<name>}{+1}| is tried, no error is issued, but
% |\arabic{<name>}| prints $-2147483648$, and further |\addtocounter{<name>}{+1}|s
% give $-2147483647$, $-2147483646$ and so on.\newline
% For a counter value of $-2\,147\,483\,647 = -$MAX and |\addtocounter{<name>}{-1}|s
% after $-2147483647$ it is printed $-2147483648$, $2147483647$, $2147483646$ and so on
% (without any message in the \xfile{log} file about any possible issue).
%
% \subsection{Other packages manipulating \texttt{\textbackslash lastpage@putlabel}}
% The \xpackage{revtex4} class redefines the |\lastpage@putlabel| command
% to place a label |LastPage|.\newline
% |\lastpage@putlabel| in the \xpackage{lastpage} package was replaced by |\lastpage@putl@bel|,
% but the |LastPage| label could become defined more than once.
%
% \subsection{\texttt{\textbackslash pagenumbering\{fnsymbol\}}\label{subsec:fnsymbol}}
% When using the foot-note-symbols as page numbers,
% it can be necessary to declare in the document's preamble:
% \begin{verbatim}
% \ProvideTextCommand{\textasteriskcentered}{PD1}{*}
% \ProvideTextCommand{\textdagger}{PD1}{†}
% %\ProvideTextCommand{\textdaggerdbl}{PD1}{‡)% seems to neither work nor be necessary
% \ProvideTextCommand{\textsection}{PD1}{§}
% \ProvideTextCommand{\textparagraph}{PD1}{¶}
% \ProvideTextCommand{\textbardbl}{PD1}{‖}
% \end{verbatim}
%
% \section{Alternatives\label{sec:Alternatives}}
% There are similar packages, which do (or do not) similar things (or even more).
% As I neither know what exactly you want to accomplish when using this package
% (e.\,g.~page number vs. page name, hyperlinks or not), nor what resources
% your system has (e.\,g.~\TeX, \LaTeX2e, $\varepsilon$-\TeX{}, \LaTeX-format as recent
% as 2024-11-01 or newer), here is a list of some possible alternatives:
% \begin{description}
% \item[-]\DescribeMacro{pageslts}
%  The \xpackage{pageslts} package first started as a revision of this
%  \xpackage{lastpage} package, but I~thought that an enhancement was needed
%  to accomplish what the \xpackage{pageslts} package does. For backward
%  compatibility, a label named |LastPage| is provided.
%  Thus |\usepackage{lastpage}| can be replaced by\newline
%  |\usepackage[pagecontinue=false,alphMult=0,AlphMulti=0,|\newline
%  | fnsymbolmult=false,romanMult=false,RomanMulti=false]{pageslts}|,\newline
%  if the behaviour of the \xpackage{lastpage} package should be simulated.
%  The default options are\newline
%  |\usepackage[pagecontinue=true,alphMult=ab,AlphMulti=AB,|\newline
%  |fnsymbolmult=true,romanMult=true,RomanMulti=true]{pageslts}|.\newline
%  Benefits of \xpackage{pageslts} package (with appropriate options) are:
%  \begin{description}
%  \item[+] Labels \texttt{LastPage} (|\AddToHook{enddocument/afterlastpage}|,
%   formerly |\AtEndDocument|; same as the \xpackage{LastPage} package) and
%   \texttt{VeryLastPage} (also |\AddToHook{enddocument/afterlastpage}|,
%   but formerly\newline%
%   |\AfterLastShipout|), allowing the user to refer to the
%   (very) last page of a document.
%  \item[+] For example, when more than one page numbering scheme is used,
%    the label \mbox{\texttt{LastPage}\textbf{s}} gives the total \textit{number} of pages.
%  \item[+] At the last page of each page numbering scheme a label\newline
%   \texttt{pagesLTS.<numbering scheme>} is placed, where
%   \texttt{<numbering scheme>} is e.\,g.\ arabic, roman, Roman, alph, or Alph.
%   For fnsymbol please use |\lastpageref{pagesLTS.fnsymbol}| instead of\newline
%   |\pageref{pagesLTS.fnsymbol}|.
%  \item[+] When the same numbering scheme is used twice, the page numbers
%   are either reset to one or continued automatically, depending on the option
%   given when the package is called.
%  \item[+] The command |\theCurrentPage| prints the current total/absolute
%   page number -- in contrast to |\thepage|, which gives only the page
%   \textit{name} in the current page numbering scheme.
%   |\theCurrentPageLocal| gives the current number of pages in the current
%   page numbering scheme. |\thepage| and |\theCurrentPageLocal| are different
%   e.\,g.\ when |\addtocounter{page}{...}| or |\setcounter{page}{...}| were used.
%  \item[+] At the first page of the document a label \texttt{pagesLTS.0} is created.
%  \item[+] The \xpackage{alphalph} package is supported, i.\,e.\ %
%   page numbers alph or Alph $>26$ and fnsymbol $>9$ can be used
%   (with according options set). Even zero and negative page numbers can be used
%   with \texttt{arabic}, \texttt{alph}, \texttt{Alph}, \texttt{roman}, \texttt{Roman},
%   and \texttt{fnsymbol} page numbering (with \xpackage{alphalph} package and
%   according options).
%  \end{description}
%  Further labels are provided for special cases.\newline
%  \url{https://ctan.org/pkg/pageslts}
%
% \item[-]\DescribeMacro{LaTeX-kernel}
%  The number of pages is nowadays available via |\@abspage@last|,\linebreak
%  |\thetotalpages|, and |\PreviousTotalPages| from the kernel,
%  but when more than one page numbering scheme is used
%  (for example pages I~to X and then 1~to 10, thus number of pages
%  \hbox{\textquotedblleft 20\textquotedblright ,} but name of the last page
%  \hbox{\textquotedblleft 10\textquotedblright ),}
%  or when or the fnsymbol page numbering scheme is used, or another package
%  has output after this package, or the page numbers exceed a certain range,
%  there might be issues. (Is the total number of pages wanted? Or is the name
%  of the last page sought?)\newline
%  |\the\ReadonlyShipoutCounter| contains the number of currently shipped out pages,
%  i.\,e.\ current page minus one.
%
% \item[-]\DescribeMacro{totpages}
%  The \xpackage{totpages} package provides a \texttt{totpages} label similar to
%  \texttt{LastPages}, but |\AtEndDocument| instead of hook |enddocument/afterlastpage|
%  of the \pkg{pageslts} package.
%  The \xpackage{totpages} package additionally computes the number of paper sheets
%  needed to (double) print the document (with one, two, three,\ldots{} pages on
%  one sheet of paper) (which can be achieved also with the \xpackage{papermas} package,
%  an extension of the \xpackage{pageslts} package, which further allows to compute
%  the mass of that printed version of the document, useful e.\,g.\ when sending it
%  by mail to determine the postage).\newline
%  \url{https://ctan.org/pkg/totpages}
%
% \item[-]\DescribeMacro{totalcount}
%  The \xpackage{totalcount} package provides |\totalpages|.
%  If there are only arabic page numbers consecutively running from 1 to the last page,
%  this works. But for example
%
% \begin{quote}
%\begin{verbatim}
%\documentclass{article}
%\usepackage[page]{totalcount}
%\pagenumbering{Roman}
%\begin{document}
%\addtocounter{page}{49}
%Page \thepage{} of \totalpages
%\end{document}
%\end{verbatim}
% \end{quote}
%
%  prints \textquotedblleft Page L of 50\textquotedblright,
%  where the number of pages is one (and no hyperlink is provided to the
%  last page even if \xpackage{hyperref} is used).\newline
%  \url{https://ctan.org/pkg/totalcount}
%
% \item[-]\DescribeMacro{totcount}
%  The \xpackage{totcount} package provides the last value of a counter, thus also the value
%  of the \texttt{page} counter. You do not get a hyperlink to the last page, only the numerical
%  value of the last page name is given \mbox{(i.\,e.~X+72} pages gives 72 instead of 82
%  as total number of pages), and the number of pages can be changed for example
%  by |\addtocounter|.\newline
%  \url{https://ctan.org/pkg/totcount}
% \pagebreak
%
% \item[-]\DescribeMacro{nofm}
%  \textquotedblleft There is a package \xpackage{nofm.sty} available, but some versions
%  of it are defective, and most don't work with \xpackage{fancyhdr} because they take over
%  the complete page layout.\textquotedblright{} (\textsc{Piet van Oostrum}:
%  Page layout in \LaTeX , March~2, 2004, section~16; fancyhdr.pdf)\newline
%  \xpackage{nofm} as of 1991/02/25 (without version number), available at\newline
%  \url{https://mirror.ctan.org/obsolete/macros/latex209/contrib/misc/nofm.sty},\newline
%  does not work with e.\,g.\ \xpackage{hyperref}, redefines |\enddocument|
%  as well as |\@oddhead|, |\@evenhead|, |\@oddfoot|, and |\@evenfoot|.\newline
%  If you know the (\url{https://CTAN.org}) location of a
%  \textbf{working}~(!) version, please send me an e-mail, thanks!
%
% \item[-]\DescribeMacro{count1to}
%  The \xpackage{count1to} package \textquotedblleft sets |\count1| to |\count8|
%  with the values of page to subparagraph. |\count9| is used to flag odd
%  pages. \ldots\ [T]he code for the TotalPages
%  label\textquotedblright\ (package manual, 2024-06-13) has been removed
%  from the current package version.\newline
%  \url{https://ctan.org/pkg/count1to}
%
% \item[-]\DescribeMacro{zref}
%  The \xpackage{zref} package \textquotedblleft implements an extensible referencing
%  system\textquotedblright\ (package manual, 2023-09-14).\newline
%  \url{https://ctan.org/pkg/zref}
%
% \item[-]\DescribeMacro{memoir}
%  The \xpackage{memoir} \emph{class} provides |\thelastpage| (page number printed
%  on last page) and |\thelastsheet| (number of pages).\newline
%  \url{https://ctan.org/pkg/memoir}
% \end{description}
%
% \bigskip
%
% \noindent (You programmed or found another alternative,
%  which is available at\newline
%  \url{https://CTAN.org}?
%  OK, send an e-mail to me with the name, location at CTAN,
%  and a short notice, and I will probably include it in the list above.)
% \newpage
%
% \section{Example}
%    \begin{macrocode}
%<*example>
\documentclass[british]{article}[2024/06/29]% v1.4n Standard LaTeX document class
\makeatletter
\usepackage[draft]{showkeys}[2024/05/23]% v3.21 Show cite and label keys (DPC, MH)
%%      Use final instead of draft to hide the keys. %%
\usepackage[pdfpagelabels=true,hyperindex=false]{hyperref}[2024-10-30]% v7.01k
\@ifpackageloaded{hyperref}{% Hypertext links for LaTeX
\hypersetup{extension=pdf,%
 plainpages=false,%
 pdflang={en},%
 pdftitle={lastpage package example},%
 pdfauthor={H.-Martin Muench},%
 pdfsubject={Example for the lastpage package},%
 pdfkeywords={LaTeX, lastpage},%
 pdfview=Fit,%
 pdfstartview=Fit,%
 pdfpagelayout=SinglePage%
}}{\usepackage{url}[2013/09/16]}%  v3.4  Verb mode for urls, etc.
\usepackage{lastpage}[2025/01/27]% v2.1e Refers to last page's name (HMM; JPG)]
\renewcommand{\@evenfoot}{{%
 \normalsize\slshape \today\hfil \upshape %
 page \thepage{} of \pageref{LastPage}}}
\renewcommand{\@oddfoot}{\@evenfoot}
\makeatother
\listfiles
\begin{document}
\pagenumbering{Roman}
\section*{Example for lastpage}
\markboth{Example for lastpage}{Example for lastpage}
This example demonstrates the use of package\newline
\textsf{lastpage}, v2.1e as of 2025-01-27 (HMM; JPG).\newline
The package takes no options.\newline
For more details please see the documentation!\newline

\noindent \label{keys} To hide the \pageref{keys}{\qquad } use option
\texttt{final} instead of \texttt{draft} with the \textsf{showkeys}
package (or remove the package call from the preamble of
this document).\newline

\textbf{Hyperlinks or not:} If the \textsf{hyperref} package is loaded,
the references are also hyperlinked:\newline
\smallskip
Last page's name (LastPage): \pageref{LastPage}\newline
\noindent If the \textsf{hyperref} package is loaded, but the hyperlinks
of the references shall be suppressed, \verb|\pageref*{...}|
can be used:\newline
\smallskip
Last page's name (LastPage): \pageref*{LastPage}\newline

\textbf{Trademarks} appear throughout this example without any
trademark symbol; they are the property of their respective
trademark owner. There is no intention of infringement; the
usage is to the benefit of the trademark owner.\newline

\textbf{Tip}: Use \textit{logical page numbers}
for the display of the pdf (in Adobe Acrobat Reader 2024.005.20392:
Edit $>$ Preferences $>$ Page Display $>$
Page Content and Information: Use logical page numbers)!\newline

If you are more ambitious in respect to your aims with this package,
you might want to have a look at the \textsf{pageslts} package:\newline
\url{https://ctan.org/pkg/pageslts}.
\bigskip

\noindent The page (\verb|\thepage|): \thepage \newline
Last page's name (LastPage): \pageref{LastPage}
\newpage

\noindent The page (\verb|\thepage|): \thepage \newline
Last page's name (LastPage): \pageref{LastPage}

\bigskip

\noindent There was the question:

\begin{quote}
\begin{verbatim}
\documentclass{article}
\usepackage{hyperref}
\usepackage{lastpage}
\begin{document}
\ifnum\thepage=\pageref{LastPage} foo \else bar \fi
\end{document}
\end{verbatim}

producing the error
\textquotedblleft missing number, treated as zero\textquotedblright.
\end{quote}

\noindent \verb|\pageref| inserts a hyperlink, \verb|\pageref{LastPage}|
is not expandable and the code breaks.\newline
The code does not generally work even without hyperref.

\begin{quote}
\begin{verbatim}
\documentclass{article}
\usepackage{hyperref}
\usepackage{lastpage}
\pagenumbering{Roman}
\begin{document}
\addtocounter{page}{8}
\edef\here{\thepage}
\makeatletter
\ifx\here\lastpage@lastpage\relax foo\else bar\fi
\makeatother
\end{document}
\end{verbatim}
\end{quote}

\noindent does work (two compilations needed), because \verb|\lastpage@lastpage|
contains the name of the page, \mbox{example:}
\begin{verbatim}
Page \thepage{} is (not) page
\makeatletter\lastpage@lastpage\makeatother.
\end{verbatim}
prints:\newline
Page \thepage{} is (not) page
\makeatletter\lastpage@lastpage\makeatother.
\newline
This can be broken for example by \verb|\pagenumbering{fnsymbol}|
(because then \verb|\edef\here{\thepage}| does not work).
\newpage

\noindent The page (\verb|\thepage|): \thepage\newline
Last page's name (LastPage): \pageref{LastPage}
\bigskip

With modern \LaTeX{} it is possible to say:
\begin{quote}
\begin{verbatim}
\NeedsTeXFormat{LaTeX2e}[2024-11-01]
\documentclass{article}
\pagenumbering{fnsymbol}
\begin{document}
\addtocounter{page}{8}%
\ExplSyntaxOn%
\xdef\test{\numexpr\the\g_shipout_readonly_int +1\relax}%
\ExplSyntaxOff%
\ifnum\PreviousTotalPages=\test\relax%
This is the last page.%
\else%
This is not the last page
(or it is but \LaTeX{} needs another compilation run
to detect this).
\fi
\end{document}
\end{verbatim}
\end{quote}

\newpage
\section*{The End}
\noindent The page (\verb|\thepage|): \thepage \newline
Last page's name (LastPage): \pageref{LastPage}
\bigskip

To see the content of the \texttt{enddocument/afterlastpage}-hook
(for a recent \LaTeX-format!) use % without the \verb||, of course!
\verb|\ShowHook{enddocument/afterlastpage}|.
\end{document}
%</example>
%    \end{macrocode}
% \StopEventually{}
% \newpage
%
% \section{The implementation}
% \DescribeMacro{lastpage.sty}
% We first need to determine whether we are on \TeX~2.09 or \LaTeX2e.\newline
% (That line, which is too long for the documentation, reads:\newline
% |\def\loadlastpage{\ProvidesPackage{lastpage}[2025/01/27 v2.1e lastpage:|\linebreak
% | 2.09 or 2e? (HMM)]\relax\RequirePackage{lastpage2e}}|.)
%
%    \begin{macrocode}
%<*package>
%% Part of the "lastpage" package
%% loads either lastpage2.09.sty for TeX 2.09 or lastpage2e.sty for LaTeX 2e
%% with code from https://groups.google.com/g/comp.text.tex/c/-Qmhj1ZI4xM
\def\loadlastpage{\ProvidesPackage{lastpage}[2025/01/27 v2.1e lastpage: 2.09 or 2e? (HMM)]\relax\RequirePackage{lastpage2e}}
\begingroup \expandafter \ifx \csname documentclass\endcsname\relax
\endgroup \expandafter %%
%% This is file `lastpage209.sty',
%% generated with the docstrip utility.
%%
%% The original source files were:
%%
%% lastpage.dtx  (with options: `lastpage209')
%% 
%% This is a generated file.
%% 
%% Project: lastpage
%% Version: 2025-01-27 v2.1e
%% Info: Refers to last page's name
%% 
%% Copyright (C) 2010 - 2025 by
%%     H.-Martin Muench <Martin dot Muench at Uni-Bonn dot de>
%% Portions of code copyrighted by other people as marked.
%% 
%% This work may be distributed and/or modified under the
%% conditions of the LaTeX Project Public License, either
%% version 1.3c of this license or (at your option) any later
%% version. This version of this license is in
%%    https://www.latex-project.org/lppl/lppl-1-3c.txt
%% and the latest version of this license is in
%%    https://www.latex-project.org/lppl.txt
%% and version 1.3c or later is part of all distributions of
%% LaTeX version 2005-12-01 or later.
%% 
%% This work has the LPPL maintenance status "maintained".
%% 
%% The Current Maintainer of this work is H.-Martin Muench.
%% 
%% This package was invented by Jeffrey P. Goldberg.
%% I thought that an enhancement was needed and therefore created the pageslts package,
%% https://ctan.org/pkg/pageslts. Nevertheless, for compatibility with existing
%% documents/packages as well as for the low amount of resources needed by the
%% lastpage package (no new counter!), I updated this package.
%% Thanks go to Jeffrey P. Goldberg for allowing me to do this.
%% 
%% While this package allows for things like "Page slash thepage{} of slash pageref{LastPage}"
%% to get "Page 7 of 9" or "Page VII of IX",
%% the NUMBER of pages is nowadays available from the kernel
%% (slash @abspage@last, slash thetotalpages, slash PreviousTotalPages),
%% but with pages for example I to X and then 1 to 10, the number of pages would be "20",
%% while the name of the last page is "10". Decide what you need/want!
%% 
%% This work consists of the main source file lastpage.dtx,
%% the README, and the derived files
%%    lastpage.sty, lastpage209.sty, lastpage2e.sty,
%%    lastpageclassic.sty, lastpagemodern.sty,
%%    lastpage.pdf, lastpage.ins, lastpage.drv,
%%    lastpage-example.tex, lastpage-example.pdf.
%% 
%% In memoriam
%%  Claudia Simone Barth + 1996-01-30
%%  Tommy Muench         + 2014-01-02
%%  Hans-Klaus Muench    + 2014-08-24
%% 
%% Part of the "lastpage" package
%% FOR LaTeX 2.09 ONLY - FOR LaTeX 2e USE lastpage2e.sty
%% This is lastpage209.sty invented by Jeffrey P. Goldberg,
%% after Piet van Oostrum: Page layout in LaTeX, March 2, 2004, section 16;
%% fancyhdr.pdf; lastpage209.sty maintained by H.-Martin Muench.
\let\origenddocument=\enddocument%
\def\enddocument{\clearpage%
  {\addtocounter{page}{-1}%
   \immediate\write\@mainaux{\string\newlabel{LastPage}{{}{\thepage}}}%
  }%
  \addtocounter{page}{+1}%
  \origenddocument%
 }
\endinput
%%
%% End of file `lastpage209.sty'.

\else \endgroup \expandafter \loadlastpage
\fi
%</package>
%    \end{macrocode}
%
% \DescribeMacro{lastpage209.sty}
% If we are on \TeX~2.09 (really?!), we load the 2.09 version \pkg{lastpage209.sty}:
%
%    \begin{macrocode}
%<*lastpage209>
%% Part of the "lastpage" package
%% FOR LaTeX 2.09 ONLY - FOR LaTeX 2e USE lastpage2e.sty
%% This is lastpage209.sty invented by Jeffrey P. Goldberg,
%% after Piet van Oostrum: Page layout in LaTeX, March 2, 2004, section 16;
%% fancyhdr.pdf; lastpage209.sty maintained by H.-Martin Muench.
\let\origenddocument=\enddocument%
\def\enddocument{\clearpage%
  {\addtocounter{page}{-1}%
   \immediate\write\@mainaux{\string\newlabel{LastPage}{{}{\thepage}}}%
  }%
  \addtocounter{page}{+1}%
  \origenddocument%
 }
%</lastpage209>
%    \end{macrocode}
%
% \DescribeMacro{lastpage2e.sty}
% If |\documentclass| is known, we are in \LaTeXe{} -- but which one?
% For modern versions with \eTeX{} and hook management etc.
% we load version \pkg{lastpagemodern.sty}, otherwise
% \pkg{lastpageclassic.sty}. We start off by checking that we are
% loading into \LaTeXe{} and announcing the name and version of this package.
%
%    \begin{macrocode}
%<*lastpage2e>
%% Part of the "lastpage" package
\NeedsTeXFormat{LaTeX2e}[1994/12/01]
\ProvidesPackage{lastpage2e}[2025/01/27 v2.1e %
 Decide which 2e lastpage version to use (HMM)]
\@ifl@t@r\fmtversion{2024/06/01}{\RequirePackage{lastpagemodern}}{%
                                 \RequirePackage{lastpageclassic}}
\message{^^J}
%</lastpage2e>
%    \end{macrocode}
% \newpage
%
% \DescribeMacro{lastpageclassic.sty}
% In case of older \LaTeX-formats \pkg{lastpageclassic.sty} is loaded:
%
%    \begin{macrocode}
%<*lastpageclassic>
%% Part of the "lastpage" package
\NeedsTeXFormat{LaTeX2e}[1994/12/01]
\ProvidesPackage{lastpageclassic}[2025/01/27 v2.1e %
  Refers to last page's name (HMM; JPG)]
%% allows for things like "Page \thepage{} of \pageref{LastPage}"
%% to get "Page 7 of 9"

%    \end{macrocode}
%
% \indent For comparisons, \textquotedblleft one\textquotedblright{} is defined
% (|\@ne| does not work for this).
%
%    \begin{macrocode}
\gdef\lastpage@one{1}
%    \end{macrocode}
%
% We define |\lastpage@hyper|, |\lastpage@nameref|, |\lastpage@french|, and
% |\lastpage@LTS| to be \textquotedblleft \texttt{0}\textquotedblright{}.
%
%    \begin{macrocode}
\gdef\lastpage@hyper{0}
\gdef\lastpage@nameref{0}
\gdef\lastpage@french{0}
\gdef\lastpage@LTS{0}
%    \end{macrocode}
%
% We define |\lastpage@firstpage| to be \textquotedblleft \texttt{1}\textquotedblright{},
% and before re-definition via the \xfile{.aux} file,
% |\lastpage@lastpage| and |\lastpage@lastpageHy| are unknown.
%
%    \begin{macrocode}
\gdef\lastpage@firstpage{1}
\gdef\lastpage@lastpage{??}
\gdef\lastpage@lastpageHy{??}

%    \end{macrocode}
%
% \begin{macro}{\AtBeginDocument}
% \indent |\AtBeginDocument| we give a warning about ancient versions of the \pkg{endfloat} package.
% Then it is checked whether various packages are loaded.
% (|\@ifpackageloaded| cannot be used later than |\AtBeginDocument|.)
% If this is the case,\newline
% |\lastpage@<package abbreviation>| is defined as \texttt{1}
% (otherwise it stays~\texttt{0}).
%
%    \begin{macrocode}
\AtBeginDocument{%
  \@ifpackageloaded{endfloat}{%
    \@ifpackagelater{endfloat}{1993/04/30}{\relax}{%
      \PackageError{lastpage}{%
        Incompatibility with outdated version of endfloat package}{%
        lastpage is not fully compatible with a version\MessageBreak%
        before 2.1 of the endfloat package,\MessageBreak%
        because those versions redefined\MessageBreak%
        the \string\enddocument\space command.}%
    }}{}%
  \@ifpackageloaded{tikz}{\gdef\lastpage@tikz{1}}{}%
  \@ifpackageloaded{hyperref}{\gdef\lastpage@hyper{1}}{}%
  \@ifpackageloaded{nameref}{\gdef\lastpage@nameref{1}}{}%
  \@ifpackageloaded{french}{\gdef\lastpage@french{1}}{}%
  \@ifpackageloaded{frenchle}{\gdef\lastpage@french{1}}{}%
  \@ifpackageloaded{pagesLTS}{\gdef\lastpage@LTS{1}}{}%
  \@ifpackageloaded{pageslts}{\gdef\lastpage@LTS{1}}{}%
%    \end{macrocode}
%
% |\lastpage@putlabel|, used by older versions of this package,
% is redefined e.\,g.\ by \xpackage{revtex}, \xpackage{frenchle},
% \xpackage{PPRcorners}, and old versions of \xpackage{hyperref}.
% While now |\lastpage@putl@bel| is used instead, \xpackage{revtex}
% could also define a label \texttt{LastPage},
% which then would be multiply defined. (Which is no big issue,
% if it is associated with the same page.) Therefore we define
%
%    \begin{macrocode}
  \gdef\lastpage@putlabel{\relax}%
  }

%    \end{macrocode}
%
% \noindent Because |\lastpage@putlabel| might be (re)defined later,
% depending on the order in which the packages are loaded,
% we will do this again |\AtEndDocument|.
% \end{macro}
%
% \begin{macro}{\lastpage@putl@bel}
% \indent This command does the writing of the label:
%
%    \begin{macrocode}
\newcommand{\lastpage@putl@bel}{%
%    \end{macrocode}
%
% |\AtBeginDocument| it is checked whether the \xpackage{hyperref} package is loaded,\newline
% |\@ifpackageloaded{hyperref}{\gdef\lastpage@hyper{1}}{}|.\newline
% |\@ifpackageloaded| cannot be used later than |\AtBeginDocument|.\newline
% User \textsc{Sebastian Bank} found and reported (Thanks!) a~case, when this check is not
% sufficient. Using a class with\newline
% |\usepackage{lastpage}|\newline
% |\AtBeginDocument{\usepackage{hyperref}}|\newline
% leads to failed detection of the \xpackage{hyperref} package, because
% |\AtBeginDocument| \textit{first} the check for \xpackage{hyperref} is performed,
% and \textit{then} \xpackage{hyperref} is loaded. As mentioned above,
% |\@ifpackageloaded| cannot be used later, so here we do not check for the
% \xpackage{hyperref} package again, but for its |\Hy@Warning| command.
% In version~1.2c of the \xpackage{lastpage} package, it was checked for
% the |\hyperref| command, but as it turned out, \xpackage{tcilatex} \textit{is}
% defining that. If some other package or user is defining |\Hy@Warning|,
% \xpackage{lastpage} will falsely assume, that \xpackage{hyperref} has been loaded,
% but in my humble opinion, defining |\Hy@Warning| does not make sense and
% is bad style (except definition by the \xpackage{hyperref} package itself,
% of course).
%
%    \begin{macrocode}
  \@ifundefined{Hy@Warning}{%  hyperref not loaded
    }{\gdef\lastpage@hyper{1}% hyperref loaded
     }%
%    \end{macrocode}
%
% If the \xpackage{pageslts} package is used, this \xpackage{lastpage} package is
% not needed at all. The \xpackage{LastPage} label would even be defined twice.
% Thus, if \xpackage{pageslts} is used, here nothing is done:
%
%    \begin{macrocode}
  \ifx\lastpage@LTS\lastpage@one%
  \else%
%    \end{macrocode}
%
% Otherwise the label is set:\newline
% We have got to distinguish whether \xpackage{hyperref} has been loaded or not:
%
%    \begin{macrocode}
    \ifx\lastpage@hyper\lastpage@one%
      \lastpage@putlabelhyper%
    \else%
%    \end{macrocode}
%
% and also need to treat documents with \xpackage{nameref} differently:
%
%    \begin{macrocode}
      \ifx\lastpage@nameref\lastpage@one%
        \lastpage@putlabelNR%
      \else%
%    \end{macrocode}
%
% When those packages have not been loaded, we just write the
% simple label into the \xfile{aux} file (and store the value of the page):
%
%    \begin{macrocode}
        \begingroup%
          \addtocounter{page}{-1}%
          \immediate\write\@auxout{\string\newlabel{LastPage}{{}{\thepage}}}%
          \immediate\write\@auxout{\string\xdef\string\lastpage@lastpage{\thepage}}%
          \immediate\write\@auxout{\string\gdef\string\lastpage@lastpageHy{}}%
          \addtocounter{page}{+1}%
        \endgroup%
      \fi%
    \fi%
  \fi%
  }

%    \end{macrocode}
% \end{macro}
% \newpage
%
% \begin{macro}{\lastpage@putlabelhyper}%
% \indent When \xpackage{hyperref} has been loaded, the label is set with the
% |\lastpage@putlabelhyper| command. If the \xpackage{hyperref} package is used,
% but page-anchors are disabled, the hyperlinking will not work. (The warning will
% also be shown, when only |\pageref*{LastPage}| is used (or neither one),
% but without messing with |\pageref| we cannot detect this.)
%
%    \begin{macrocode}
\newcommand{\lastpage@putlabelhyper}{%
  \ifHy@pageanchor%
  \else%
    \PackageWarningNoLine{lastpage}{%
      The \string\pageref{LastPage} link does not work\MessageBreak%
      using hyperref with disabled option `pageanchor'.\MessageBreak%
      Better enable `pageanchor' or use\MessageBreak%
      \string\pageref*{LastPage} (not generating a link)%
      }%
  \fi%
%    \end{macrocode}
%
% Since the page has been put out, we are on the page \textit{after} that page.
% We therefore subtract one from the page counter. (For the compiler,
% this is equal to |\advance\c@page\m@ne|, but for human readers of the code
% it is probably easier to understand.)
%
%    \begin{macrocode}
  \begingroup%
    \addtocounter{page}{-1}%
%    \end{macrocode}
%
% Simply using |\label| for \texttt{LastPage} would not work,
% because labels wait for the output routines to work, and there
% may be no more invocations of the output routines. To force
% the write out, we need to do an |\immediate| write into the \xfile{aux} file.
%
%    \begin{macrocode}
%% with code from \Hy@EveryPageAnchor of the hyperref package,
%% 2010/04/17 v6.80x; newer versions are available
    \let\@number\@firstofone%
    \ifHy@pageanchor%
      \ifHy@hypertexnames%
        \ifHy@plainpages%
          \def\Hy@temp{\arabic{page}}%
        \else%
          \Hy@unicodefalse%
          \ifnum \value{page}=1\relax%
%    \end{macrocode}
%
% We do not count the pages ourselves, and so they could have been changed by
% e.\,g.\ |\pagenumbering{...}|, |\addtocounter{page}{...}|,\newline
% |\setcounter{page}{...}|. Thus the page might have the number one
% while not being the first page at all. Using the \xpackage{everyshi}
% package would help, but this package should not require other packages.
% The \xpackage{pageslts} package does a better handling.\newline
% We will make a mistake here at most once:
%
%    \begin{macrocode}
            \ifx \lastpage@firstpage\lastpage@one\relax%
              \def\Hy@temp{\thepage}%
              \gdef\lastpage@firstpage{0}%
            \else%
                \pdfstringdef\Hy@temp{\thepage}%
          \fi%
          \else%
            \pdfstringdef\Hy@temp{\thepage}%
          \fi%
        \fi%
      \else%
        \def\Hy@temp{\the\Hy@pagecounter}%
      \fi%
    \fi%
    \immediate\write\@auxout{%
      \string\newlabel{LastPage}{{}{\thepage}{}{%
          \ifHy@pageanchor page.\Hy@temp\fi}{}}%
    }%
%    \end{macrocode}
%
% We also save the values, so that we can later (next rerun) check,
% whether they have been saved in the \xfile{aux} file.
%
%    \begin{macrocode}
    \immediate\write\@auxout{\string\xdef\string\lastpage@lastpage{\thepage}}%
    \ifHy@pageanchor%
      \immediate\write\@auxout{\string\xdef\string\lastpage@lastpageHy{\Hy@temp}}%
    \else%
      \immediate\write\@auxout{\string\gdef\string\lastpage@lastpageHy{}}%
    \fi%
%    \end{macrocode}
%
% After the writeout we restore the page number again,
% since there might be other things still to be done.
%
%    \begin{macrocode}
    \addtocounter{page}{+1}%
  \endgroup%
  }

%    \end{macrocode}
% \end{macro}
%
% \begin{macro}{\lastpage@putlabelNR}
% \indent The \xpackage{nameref} package redefines |\label| to have five arguments
% instead of two, therefore\newline
% |\newlabel{LastPage}{{}{\thepage}{}{}{}}|
% instead of\newline
% |\newlabel{LastPage}{{}{\thepage}}| must be used:
%
%    \begin{macrocode}
\newcommand{\lastpage@putlabelNR}{%
  \begingroup%
    \addtocounter{page}{-1}%
    \immediate\write\@auxout{\string\newlabel{LastPage}{{}{\thepage}{}{}{}}}%
    \immediate\write\@auxout{\string\xdef\string\lastpage@lastpage{\thepage}}%
    \immediate\write\@auxout{\string\gdef\string\lastpage@lastpageHy{}}%
    \addtocounter{page}{+1}%
  \endgroup%
  }

%    \end{macrocode}
% \end{macro}
%
% \begin{macro}{\lastpage@fileswtest}
% \indent Later it will be determined whether it is allowed to write
% to the \xfile{aux} file. If it was \emph{not} allowed, it is checked
% whether the label was already set via the \xfile{aux} file in some
% earlier compilation run. (There are packages where the document
% is compiled with access to the \xfile{aux} file, and then there is
% an additional compiler run, where the \xfile{aux} file cannot be changed,
% but in that run there is also no need to change it.) The \xpackage{tikz}
% package is somewhat different, therefore we only give a warning instead
% of an error (and hope that there is another compiler run where the
% \xfile{aux} file can be written).
%
%    \begin{macrocode}
\newcommand{\lastpage@fileswtest}[2]{%
  \edef\lastpage@testa{#1}%
  \edef\lastpage@testb{#2}%
  \ifx\lastpage@testa\lastpage@testb%
  \else%
    \ifx\lastpage@tikz\lastpage@one\relax%
      \PackageWarning{lastpage}%
       {The lastpage package was not allowed to write to an\MessageBreak%
        .aux file. This package does not work without access\MessageBreak%
        to an .aux file.\MessageBreak%
        It is OK if the .aux file was already updated\MessageBreak%
        by a previous compiler run\MessageBreak%
        and would not have changed anyway.\MessageBreak%
       }%
    \else%
      \PackageError{lastpage}{No auxiliary file allowed}%
       {The lastpage package was not allowed to write to an .aux file.\MessageBreak%
        This package does not work without access to an .aux file.\MessageBreak%
        Press Ctrl+Z to exit.\MessageBreak%
        But it is OK if the .aux file was already updated\MessageBreak%
        by a previous compiler run\MessageBreak%
        and would not have changed anyway.}%
    \fi%
  \fi%
  }

%    \end{macrocode}
% \end{macro}
%
% \begin{macro}{\lastpage@fileswtestHy}
% \indent When the \xpackage{hyperref} package has been loaded,
% |\lastpage@lastpageHy| must be tested additionally. (And a
% |\newcommand| is needed, because |\ifHy@pageanchor| is not even
% defined when \xpackage{hyperref} has not been loaded.)
%
%    \begin{macrocode}
\newcommand{\lastpage@fileswtestHy}{%
  \ifHy@pageanchor%
    \lastpage@fileswtest{\Hy@temp}{\lastpage@lastpageHy}%
  \else%
    \lastpage@fileswtest{\empty}{\lastpage@lastpageHy}%
  \fi%
  }
%    \end{macrocode}
% \end{macro}
%
% \begin{macro}{\AtEndDocument}
% \indent |\AtEndDocument| we again (re)define |\lastpage@putlabel|
% to do nothing and check |\lastpage@lastpage|,
% whether it is still unchanged, which is OK for the first run only.
%
%    \begin{macrocode}

\AtEndDocument{%
  \ifx\lastpage@LTS\lastpage@one%
  \else%
    \gdef\lastpage@putlabel{??}%
    \ifx\lastpage@lastpage\lastpage@putlabel\relax%
      \PackageWarning{lastpage}{Rerun to get the references right}%
    \fi%
  \fi%
  \gdef\lastpage@putlabel{\relax}%
%    \end{macrocode}
%
% It is checked whether writing to files is allowed
% (otherwise, only an error message is issued and nothing is done).
%
%    \begin{macrocode}
  \if@filesw%
%    \end{macrocode}
%
% We put in a |\message| to show, in what order things (which were called)
% are done (see subsection~\ref{ssec:aed}).
%
%    \begin{macrocode}
    \message{^^JAED: lastpage setting LastPage^^J}%
%    \end{macrocode}
%
% After this we issue a |\clearpage| to put out all floats,
% which are still floating, and place the \texttt{LastPage} label.
% Sometimes |\clearpage| might be undefined.
%
%    \begin{macrocode}
    \@ifundefined{clearpage}{\relax}{\clearpage}%
    \ifx\lastpage@french\lastpage@one% french or frenchle loaded
      \addtocounter{page}{+1}\lastpage@putl@bel\addtocounter{page}{-1}%
    \else% neither one loaded
      \lastpage@putl@bel%
    \fi%
%    \end{macrocode}
%
% When writing to files is not allowed, nothing can be done. But when
% the label was already set via the \xfile{aux} file, nothing needs
% to be done. We check for this with |\lastpage@fileswtest| and
% (if \xpackage{hyperref} has been loaded) |\lastpage@fileswtestHy|.
%
%    \begin{macrocode}
  \else%
    \ifx\lastpage@LTS\lastpage@one%
    \else%
      \lastpage@fileswtest{\thepage}{\lastpage@lastpage}%
      \ifx\lastpage@hyper\lastpage@one%
        \lastpage@fileswtestHy%
      \fi%
    \fi%
  \fi%
  }
%    \end{macrocode}
% \end{macro}
%    \begin{macrocode}
%</lastpageclassic>
%    \end{macrocode}
% \newpage
%
% \DescribeMacro{lastpagemodern.sty}
% In case of a recent \LaTeX-format, \pkg{lastpagemodern.sty} is loaded:
%
%    \begin{macrocode}
%<*lastpagemodern>
%% Part of the "lastpage" package
\NeedsTeXFormat{LaTeX2e}[2024-11-01]
\ProvidesPackage{lastpagemodern}[2025-01-27 v2.1e %
  Refers to last page's name (HMM; JPG)]
%% allows for things like "Page \thepage{} of \pageref{LastPage}"
%% to get "Page 7 of 9" or "Page VII of IX";
%% the NUMBER of pages is available via \@abspage@last, but with pages
%% for example I to X and then 1 to 10, the number of pages would be "20",
%% while the name of the last page is "10". Decide what you need/want!

%    \end{macrocode}
%
% \indent For comparisons, \textquotedblleft one\textquotedblright{} is defined
% (|\@ne| does not work for this).
%
%    \begin{macrocode}
\gdef\lastpage@one{1}
%    \end{macrocode}
%
% We define |\lastpage@firstpage| to be \textquotedblleft \texttt{1}\textquotedblright{},
% and before re-definition via the \xfile{.aux} file,
% |\lastpage@lastpage| and |\lastpage@lastpageHy| are unknown.
%
%    \begin{macrocode}
\gdef\lastpage@firstpage{1}
\gdef\lastpage@lastpage{??}
\gdef\lastpage@lastpageHy{??}

%    \end{macrocode}
%
% \begin{macro}{\lastpage@IfNumericTF}
% |\lastpage@IfNumericTF| was needed to determine whether a page number was numeric or not.
% Although it is no longer used, I am reluctant to remove it because it may have been used
% in other places in documents.
%
%    \begin{macrocode}
\newcommand\lastpage@gobbleminus[1]{\ifx-#1\else#1\fi}
%    \end{macrocode}
% which is from \url{https://texfaq.org/FAQ-isitanum},
%    \begin{macrocode}
\newcounter{lastpagecount}
%    \end{macrocode}
% and similar to \url{https://tex.stackexchange.com/a/12811/17119}
%    \begin{macrocode}
\newcommand{\lastpage@IfNumericTF}[3]{%
\sbox\z@{\c@lastpagecount=0\lastpage@gobbleminus{#1}\relax}%
\ifdim\wd0>\z@\relax#3% is not numeric
\else#2% is numeric
\fi}

%    \end{macrocode}
% \end{macro}
%
% \DescribeMacro{\AddToHook\{begindocument/end\}}
% \indent |\lastpage@putlabel|, used by older versions of this package,
% is/was redefined by other packages.
% While now |\lastpage@putl@bel| is used instead, other packages
% could also define a label \texttt{LastPage},
% which then would be multiply defined. (Which is no big issue,
% if it is associated with the same page.) Therefore we define
%
%    \begin{macrocode}
\AddToHook{begindocument/end}{%
  \IfPackageLoadedT{pageslts}{%
    \PackageNoteNoLine{lastpage}{Packages pageslts and lastpage used.\MessageBreak%
      lastpage is not necessary when loading pageslts}%
  }%
  \gdef\lastpage@putlabel{\relax}%
  }

%    \end{macrocode}
%
% \begin{macro}{\protected@iwrite}
% \indent We need an |\immediate\protected@write|. Just |\immediate\write| had led to errors,
% for example when packages like \pkg{babel-greek} re-defined |\roman|\ %
% (thanks to \textsc{Ulrike Fischer} for the report).
%
%    \begin{macrocode}
%% Code provided by Prof. Enrico Gregorio at https://tex.stackexchange.com/a/542425
\long\def\protected@iwrite#1#2#3{%
  \begingroup%
  #2%
  \let\protect\@unexpandable@protect%
  \edef\reserved@a{\immediate\write#1{#3}}%
  \reserved@a%
  \endgroup%
  \if@nobreak\ifvmode\nobreak\fi\fi%
  }

%    \end{macrocode}
% \end{macro}
%
% \begin{macro}{\lastpage@makeHy}
% Just once we need the page from |\@currentHpage| without any \mbox{\textquotedblleft |page.|\textquotedblright :}
%    \begin{macrocode}
\newcommand{\lastpage@makeHy}{%
%    \end{macrocode}
% |\gdef\lastpage@Hy{}|, but that was already done before this command.
%    \begin{macrocode}
  \def\lastpage@Hptest{Doc-Start}%
  \ifx\lastpage@Hptest\@currentHpage\relax%
    \gdef\lastpage@Hy{\@currentHpage}%
  \else%
    \edef\lastpage@Hptest{\@currentHpage}%
    \ifx\lastpage@Hptest\empty\relax%
%    \end{macrocode}
% |\gdef\lastpage@Hy{}|, but that was already done before this command.
%    \begin{macrocode}
    \else%
      \def\lastpage@Hptest{page.}%
      \ifx\lastpage@Hptest\@currentHpage\relax
        \def\lastpage@Hptest{\csname @fnsymbol\endcsname \c@page }%
        \ifx\lastpage@Hptest\thepage\relax%
          \ifnum\value{page}=3\else%
            \PackageWarningNoLine{lastpage}{You should add a\MessageBreak
              \string\ProvideTextCommand{...}{PD1}{...}\MessageBreak%
              (see the lastpage package manual, 3.13 %
              \string\pagenumbering{fnsymbol})\MessageBreak%
              to your document's preamble}%
%    \end{macrocode}
% See \autoref{subsec:fnsymbol}: \nameref{subsec:fnsymbol}, page~\pageref{subsec:fnsymbol}.
%    \begin{macrocode}
          \fi%
        \fi%
        \PackageWarningNoLine{lastpage}{%
          \string\@currentHpage\space is\MessageBreak%
          just "page." without number,\MessageBreak%
          \string\lastpage@lastpageHy\space is now let empty}%
%    \end{macrocode}
% |\gdef\lastpage@Hy{}|, but that was already done before this command.
%    \begin{macrocode}
      \else%
%    \end{macrocode}
% |\@currentHpage| should be |page.<some number>|, |\lastpage@rmpage| removes the
% \mbox{\textquotedblleft |page.|\textquotedblright .} Next compilation run,
% |\lastpage@lastpageHy| gets defined via the \xfile{aux} file.
% If we arrived at this place, but the definition is still empty,
% then |\@currentHpage| has some unexpected content.
%    \begin{macrocode}
        \gdef\lastpage@Hy{\lastpage@rmpage{\@currentHpage}}%
        \ifx\lastpage@lastpageHy\empty\relax%
          \PackageWarningNoLine{lastpage}{%
            \string\@currentHpage\space is\MessageBreak%
            \meaning\@currentHpage\MessageBreak%
            not beginning with "page.",\MessageBreak%
            \string\lastpage@lastpageHy\space is now let empty}%
  \fi\fi\fi\fi%
  }

%    \end{macrocode}
% \end{macro}
% \begin{macro}{\lastpage@rmpage}
% |\lastpage@rmpage| removes the \mbox{\textquotedblleft |page.|\textquotedblright.}
%
%    \begin{macrocode}
%% Code provided by David Carlisle at https://tex.stackexchange.com/a/721877
\def\lastpage@rmpage#1{%
  \expandafter\xlastpage@rmpage\expanded{#1}\xlastpage@rmpage page.%
  \xlastpage@rmpage\xxlastpage@rmpage{#1}}
\def\xlastpage@rmpage #1page.#2\xlastpage@rmpage#3\xxlastpage@rmpage#4{%
  \if$\detokenize{#1}$#2%\else#4
  \fi}

%    \end{macrocode}
% |\else#4| means, that it did not start with |page.|, and whatever it is,
% we cannot use this |#4| for |\lastpage@Hy|.
% \end{macro}
%
% \begin{macro}{\lastpage@putl@bel}
% \indent This command does the writing of the label.
% If the \xpackage{hyperref} package is used, but page-anchors are disabled,
% the hyperlinking will not work. (The warning will also be shown, when only
% |\pageref*{LastPage}| is used (or neither |\pageref{LastPage}| nor |\pageref*{LastPage}|),
% but without messing with |\pageref| we cannot detect this.)
%
%    \begin{macrocode}
\newcommand{\lastpage@putl@bel}{%
  \IfPackageLoadedT{hyperref}{%
    \IfPackageAtLeastF{hyperref}{2024-10-30}{%
      \PackageError{lastpage}{hyperref package version too old}{%
        required version: 2024-10-30 or newer, found version:\MessageBreak%
        \csname ver@hyperref.sty\endcsname\MessageBreak%
        Update hyperref or use lastpageclassic.sty instead of\MessageBreak%
        lastpagemodern.sty!}}%
    \ifHy@pageanchor\else%
      \PackageWarningNoLine{lastpage}{%
        The \string\pageref{LastPage} link does not work\MessageBreak%
        using hyperref with disabled option `pageanchor'.\MessageBreak%
        Better enable `pageanchor' or use\MessageBreak%
        \string\pageref*{LastPage} (not generating a link)}%
    \fi%
   }%
  \begingroup%
%    \end{macrocode}
%
% \noindent Since the page has been put out, we are on the page \textit{after} that page.
% We therefore subtract one from the page counter. (For the compiler,
% this is equal to |\advance\c@page\m@ne|, but for human readers of the code
% it is probably easier to understand.)
%
%    \begin{macrocode}
    \addtocounter{page}{-1}%
%    \end{macrocode}
%
% If the \xpackage{pageslts} package is used, this \xpackage{lastpage} package
% is not needed at all. The \xpackage{LastPage} label would even be defined twice.
% Thus, if \xpackage{pageslts} is used, here nothing is done.
%
%    \begin{macrocode}
    \IfPackageLoadedTF{pageslts}{% then pageslts writes the label for  "LastPage".
%    \end{macrocode}
%
% \noindent Simply using |\label| for \texttt{LastPage} would not work,
% because labels wait for the output routines to work, and there
% may be no more invocations of the output routines.
% To force the write out, we need to do an |\immediate| protected write
% into the \xfile{aux} file.
%
%    \begin{macrocode}
     }{\protected@iwrite\@auxout{}{\string\newlabel{LastPage}{%
         {\@currentlabel}{\thepage}{\@currentlabelname}%
         {\IfPackageLoadedTF{hyperref}{\ifHy@pageanchor\@currentHpage\fi%
                                      }{\@currentHref}}%
         {\@kernel@reserved@label@data}}%
       }%
      }%
%    \end{macrocode}
% \noindent We also save the values, so that we can later (next rerun) check,
% whether they have been saved in the \xfile{aux} file.
%
%    \begin{macrocode}
    \protected@iwrite\@auxout{}{%
      \string\gdef\string\lastpage@lastpage{\thepage}}%
    \gdef\lastpage@Hy{}%
    \IfPackageLoadedT{hyperref}{\ifHy@pageanchor\lastpage@makeHy\fi}%
    \protected@iwrite\@auxout{}{%
      \string\gdef\string\lastpage@lastpageHy{\lastpage@Hy}}%
%    \end{macrocode}
%
% \noindent After the write-out we restore the page number again,
% since there might be other things still to be done.
%
%    \begin{macrocode}
    \addtocounter{page}{+1}%
  \endgroup%
  }

%    \end{macrocode}
% \end{macro}
%
% \begin{macro}{\lastpage@fileswtest}
% \indent \hspace*{2em}Later it will be determined whether it is allowed to write
% to the \xfile{aux} file. If it was \emph{not} allowed, it is checked
% whether the label was already set via the \xfile{aux} file in some
% earlier compilation run. (There are packages where the document
% is compiled with access to the \xfile{aux} file, and then there is
% an additional compiler run, where the \xfile{aux} file cannot be changed,
% but in that run there is also no need to change it.) The \xpackage{tikz}
% package is somewhat different, therefore we only give a warning instead
% of an error (and hope that there is another compiler run where the
% \xfile{aux} file can be written).
%
%    \begin{macrocode}
\newcommand{\lastpage@fileswtest}[2]{%
  \edef\lastpage@testa{#1}%
  \edef\lastpage@testb{#2}%
  \ifx\lastpage@testa\lastpage@testb%
  \else%
    \IfPackageLoadedTF{tikz}{%
      \PackageWarning{lastpage}%
       {The lastpage package was not allowed to write to an\MessageBreak%
        .aux file. This package does not work without access\MessageBreak%
        to an .aux file.\MessageBreak%
        It is OK if the .aux file was already updated\MessageBreak%
        by a previous compiler run\MessageBreak%
        and would not have changed anyway.\MessageBreak%
       }%
    }{\PackageError{lastpage}{No auxiliary file allowed}%
       {The lastpage package was not allowed to write to an .aux file.\MessageBreak%
        This package does not work without access to an .aux file.\MessageBreak%
        Press Ctrl+Z to exit.\MessageBreak%
        But it is OK if the .aux file was already updated\MessageBreak%
        by a previous compiler run\MessageBreak%
        and would not have changed anyway.\MessageBreak%
       }%
     }%
  \fi%
  }

%    \end{macrocode}
% \end{macro}
%
% \begin{macro}{\lastpage@fileswtestHy}
% \indent \hspace*{2em}When the \xpackage{hyperref} package has been loaded,
% |\lastpage@lastpageHy| must be tested additionally. (And a
% |\newcommand| is needed, because |\ifHy@pageanchor| is not even
% defined when \xpackage{hyperref} has not been loaded.)
%
%    \begin{macrocode}
\newcommand{\lastpage@fileswtestHy}{%
  \ifHy@pageanchor%
    \lastpage@fileswtest{\@currentHpage}{\@the@H@page}%
  \else%
    \lastpage@fileswtest{\empty}{\lastpage@lastpageHy}%
  \fi%
  }

%    \end{macrocode}
% \end{macro}
%
% \DescribeMacro{enddocument/afterlastpage}
% \indent |enddocument/afterlastpage| we again (re)define |\lastpage@putlabel|
% to do nothing, but first use it to check whether |\lastpage@lastpage|
% is still unchanged, which is OK for the first run only.
%
%    \begin{macrocode}
\AddToHook{enddocument/afterlastpage}{%
  \gdef\lastpage@putlabel{??}%
  \ifx\lastpage@lastpage\lastpage@putlabel\relax%
    \AddToHook{enddocument/info}{%
      \PackageWarning{lastpage}{Rerun to get the references right}%
      }%
  \fi%
  \gdef\lastpage@putlabel{\relax}%
%    \end{macrocode}
%
% It is checked whether writing to files is allowed
% (otherwise, only an error message is issued and nothing is done).
%
%    \begin{macrocode}
  \if@filesw%
%    \end{macrocode}
%
% We put in a |\message| to show, in what order things (which were called) are done.
%
%    \begin{macrocode}
    \message{^^Jenddocument/afterlastpage (AED): lastpage setting LastPage.^^J}%
    \IfPackageLoadedTF{french}{%
         \addtocounter{page}{+1}\lastpage@putl@bel\addtocounter{page}{-1}%
     }{\IfPackageLoadedTF{frenchle}{%
         \addtocounter{page}{+1}\lastpage@putl@bel\addtocounter{page}{-1}%
       }{\lastpage@putl@bel%
        }%
      }%
  \else%
%    \end{macrocode}
%
% When writing to files is not allowed, nothing can be done. But when
% the label was already set via the \xfile{aux} file, nothing needs
% to be done. We check for this with |\lastpage@fileswtest| and
% (if \xpackage{hyperref} has been loaded) |\lastpage@fileswtestHy|.
%
%    \begin{macrocode}
    \lastpage@fileswtest{\thepage}{\lastpage@lastpage}%
    \IfPackageLoadedT{hyperref}{\lastpage@fileswtestHy}%
  \fi%
  }
%</lastpagemodern>
%    \end{macrocode}
% \newpage
%
% \section{Installation}
% \subsection{Downloads\label{ss:Downloads}}
% Everything is available at \url{https://www.ctan.org},
% but may need additional packages themselves.
%
% \DescribeMacro{lastpage.dtx}
% For unpacking the |lastpage.dtx| file and constructing the documentation it is required:
% \begin{description}
% \item[-] \TeX Format \LaTeXe{}: \url{https://www.CTAN.org}
%
% \item[-] document class \xclass{ltxdoc}, 2024/02/08, v2.1j,
%   \url{https://ctan.org/pkg/ltxdoc}
%
% \item[-] package \xpackage{holtxdoc}, 2019/12/09, v0.30,
%   \url{https://ctan.org/pkg/holtxdoc}
% \end{description}
%
% \DescribeMacro{lastpage.sty}
% The |lastpage.sty| (i.\,e.\ each document using the \xpackage{lastpage} package)
% requires:
% \begin{description}
% \item[-] \TeX, \url{https://www.CTAN.org}
%
% \item[-] package \xpackage{lastpage}, 2025-01-27, v2.1e,
%   \url{https://ctan.org/pkg/lastpage}
% \end{description}
%
% \DescribeMacro{lastpage209.sty}
% The |lastpage209.sty| for \LaTeX2.09{} (i.\,e.\ each document using
% the \xpackage{lastpage209} package) requires:
% \begin{description}
% \item[-] \TeX-format \LaTeX{}, v2.09
%
% \item[-] package \xpackage{lastpage209}, 2025-01-27, v2.1e,
%   \url{https://ctan.org/pkg/lastpage}
% \end{description}
% and does not work with \xpackage{hyperref}, which needs \LaTeX2e{}.\smallskip
%
% \DescribeMacro{lastpage2e.sty}
% The |lastpage2e.sty| for \LaTeXe{} (i.\,e.\ each document using
% the \xpackage{lastpage2e} package) requires:
% \begin{description}
% \item[-] \TeX-format \LaTeXe{} 1994/12/01 or newer,
%   \url{https://www.CTAN.org}
%
% \item[-] package \xpackage{lastpage}, 2025-01-27, v2.1e,
%   \url{https://ctan.org/pkg/lastpage}
% \end{description}
%
% \DescribeMacro{lastpageclassic.sty}
% The |lastpageclassic.sty| for \LaTeXe{} (i.\,e.\ each document using
% the \xpackage{lastpageclassic} package) requires:
% \begin{description}
% \item[-] \TeX-format \LaTeXe{} between 1994/12/01 and 2024-05-31,\newline
%   \url{https://www.CTAN.org}
%
% \item[-] package \xpackage{lastpage}, 2025-01-27, v2.1e,
%   \url{https://ctan.org/pkg/lastpage}
% \end{description}
% and can use
% \begin{description}
% \item[-] package \xpackage{hyperref}, 2023-07-08, v7.01b,
%   \url{https://ctan.org/pkg/hyperref}\newline
%   (probably also some older and newer versions)
% \end{description}
%
% \DescribeMacro{lastpagemodern.sty}
% The |lastpagemodern.sty| for \LaTeXe{} (i.\,e.\ each document using
% the \xpackage{lastpagemodern} package) requires:
% \begin{description}
% \item[-] \TeX-format \LaTeXe{} 2024-11-01 or newer,\newline
%   \url{https://www.CTAN.org}
%
% \item[-] package \xpackage{lastpage}, 2025-01-27, v2.1e,
%   \url{https://ctan.org/pkg/lastpage}
% \end{description}
% and can use
% \begin{description}
% \item[-] package \xpackage{hyperref}, probably 2023-11-07 and newer
%   (tested with: 2024-11-05 v7.01l),
%   \url{https://ctan.org/pkg/hyperref}
% \end{description}
% \pagebreak
%
% \DescribeMacro{lastpage-example.tex}
% The |lastpage-example.tex| requires the same file as all
% documents using the \xpackage{lastpage} package, i.\,e.
% \begin{description}
% \item[-] package \xpackage{lastpage}, 2025-01-27, v2.1e,
%   \url{https://ctan.org/pkg/lastpage}\newline
%   (Well, it is the example file for this package, and because you are reading the
%    documentation for the \xpackage{lastpage} package, it can be assumed that you already
%    have some version of it -- is it the current one?)
% \end{description}
% and additionally:
% \begin{description}
% \item[-] class \xpackage{article}, 2024/06/29, v1.4n, from \xpackage{classes}:
%           \url{https://ctan.org/pkg/classes}
%
% \item[-] package \xpackage{showkeys}, 2024-05-23, v3.21,
%   \url{https://ctan.org/pkg/showkeys}
%
% \item[-] package \xpackage{hyperref}, 2024-11-05, v7.01l,
%   \url{https://ctan.org/pkg/hyperref}
% \end{description}
%
% \DescribeMacro{Münch}
% A hyperlinked list of my (other) packages can be found at\newline
% \url{https://ctan.org/author/muench-hm}.
%
% \subsection{Package, unpacking TDS}
% \paragraph{Package.} This package is available on \url{https://www.CTAN.org}.
% \begin{description}
% \item[\url{https://mirror.ctan.org/macros/latex/contrib/lastpage/lastpage.dtx}]\hspace*{0.1cm} \newline
%       The source file.
% \item[\url{https://mirror.ctan.org/macros/latex/contrib/lastpage/lastpage.pdf}]\hspace*{0.1cm} \newline
%       The documentation.
% \item[\url{https://mirror.ctan.org/macros/latex/contrib/lastpage/lastpage-example.pdf}]\hspace*{0.1cm}
%       The compiled example file, as it should look like.
% \item[\url{https://mirror.ctan.org/macros/latex/contrib/lastpage/README}]\hspace*{0.1cm} \newline
%       The README file.
% \end{description}
% There is also a |lastpage.tds.zip| available:
% \begin{description}
% \item[\url{https://mirror.ctan.org/install/macros/latex/contrib/lastpage.tds.zip}]\hspace*{0.1cm}
%       Everything in TDS compliant, compiled format.
% \end{description}
% which additionally contains\newline
% \begin{tabular}{ll}
% lastpage.ins         & The \texttt{ins}tallation file.\\
% lastpage.drv         & The \mbox{\texttt{dr}i\texttt{v}er} to generate the documentation.\\
% lastpage.sty         & The \texttt{sty}le file.\\
% lastpage209.sty      & The \texttt{sty}le file for \LaTeX2.09{} \textbf{only}.\\
% lastpage2e.sty       & The \texttt{sty}le file to determine which 2e-style to use.\\
% lastpageclassic.sty  & The \texttt{sty}le file for older \LaTeX-formats.\\
% lastpagemodern.sty   & The \texttt{sty}le file for the recent \LaTeX-format.\\
% lastpage-example.tex & The example file.
% \end{tabular}
%
% \bigskip
%
% \noindent For required other packages please see the preceding subsection.
%
% \paragraph{Unpacking.} The \xfile{.dtx} file is a self-extracting
% \docstrip{} archive. The files are extracted by running the
% \xext{.dtx} through \plainTeX :
% \begin{quote}
%   \verb|tex lastpage.dtx|
% \end{quote}
%
% \noindent About generating the documentation see paragraph~\ref{GenDoc} below.\par
%
% \paragraph{TDS.} Now the different files must be moved into
% the different directories in your installation TDS tree
% (also known as \xfile{texmf} tree):
% \begin{quote}
% \def\t{^^A
% \begin{tabular}{@{}>{\ttfamily}l@{ $\rightarrow$ }>{\ttfamily}l@{}}
%   lastpage.sty & tex/latex/lastpage.sty\\
%   lastpage209.sty & tex/latex/lastpage.sty\\
%   lastpage2e.sty & tex/latex/lastpage.sty\\
%   lastpageclassic.sty & tex/latex/lastpage.sty\\
%   lastpagemodern.sty & tex/latex/lastpage.sty\\
%   lastpage.pdf & doc/latex/lastpage.pdf\\
%   lastpage-example.tex & doc/latex/lastpage-example.tex\\
%   lastpage-example.pdf & doc/latex/lastpage-example.pdf\\
%   lastpage.dtx & source/latex/lastpage.dtx
% \end{tabular}^^A
% }^^A
% \sbox0{\t}^^A
% \ifdim\wd0>\linewidth
%   \begingroup
%     \advance\linewidth by\leftmargin
%     \advance\linewidth by\rightmargin
%   \edef\x{\endgroup
%     \def\noexpand\lw{\the\linewidth}^^A
%   }\x
%   \def\lwbox{^^A
%     \leavevmode
%     \hbox to \linewidth{^^A
%       \kern-\leftmargin\relax
%       \hss
%       \usebox0
%       \hss
%       \kern-\rightmargin\relax
%     }^^A
%   }^^A
%   \ifdim\wd0>\lw
%     \sbox0{\small\t}^^A
%     \ifdim\wd0>\linewidth
%       \ifdim\wd0>\lw
%         \sbox0{\footnotesize\t}^^A
%         \ifdim\wd0>\linewidth
%           \ifdim\wd0>\lw
%             \sbox0{\scriptsize\t}^^A
%             \ifdim\wd0>\linewidth
%               \ifdim\wd0>\lw
%                 \sbox0{\tiny\t}^^A
%                 \ifdim\wd0>\linewidth
%                   \lwbox
%                 \else
%                   \usebox0
%                 \fi
%               \else
%                 \lwbox
%               \fi
%             \else
%               \usebox0
%             \fi
%           \else
%             \lwbox
%           \fi
%         \else
%           \usebox0
%         \fi
%       \else
%         \lwbox
%       \fi
%     \else
%       \usebox0
%     \fi
%   \else
%     \lwbox
%   \fi
% \else
%   \usebox0
% \fi
% \end{quote}
% If you have a \xfile{docstrip.cfg} that configures and enables \docstrip's
% TDS installing feature, then some files can already be in the right
% place, see the documentation of \docstrip.
%
% \subsection{Refresh file name databases}
% If your \TeX~distribution (\TeX\,Live, \mikTeX, \dots) relies on
% file name databases, you must refresh these. For example, \TeX\,Live\ users
% run \verb|texhash| or \verb|mktexlsr|.
%
% \subsection{Some details for the interested}
% \paragraph{Unpacking with \LaTeX.}
% The \xfile{.dtx} chooses its action depending on the format:
% \begin{description}
% \item[\plainTeX:] Run \docstrip\ and extract the files.
% \item[\LaTeX:] Generate the documentation.
% \end{description}
% If you insist on using \LaTeX\ for \docstrip\ (really,
% \docstrip\ does not need \LaTeX ), then inform the autodetect routine
% about your intention:
% \begin{quote}
%   \verb|latex \let\install=y% \iffalse meta-comment
%
% File: lastpage.dtx
% Version: 2025-01-27 v2.1e
% Info: Refers to last page's name
%
% Copyright © 2010 - 2025 by
%    H.-Martin Münch <Martin dot Muench at Uni-Bonn dot de>
% Portions of code copyrighted by other people as marked.
%
% This package was invented by Jeffrey P. Goldberg.
% I thought that an enhancement was needed and therefore created the pageslts package,
% https://ctan.org/pkg/pageslts. Nevertheless, for compatibility with existing
% documents/packages as well as for the low amount of resources needed by the
% lastpage package (no new counter!), I updated this package.
% Thanks go to Jeffrey P. Goldberg for allowing me to do this.
%
% While this package allows for things like "Page \thepage{} of \pageref{LastPage}"
% to get "Page 7 of 9" or "Page VII of IX",
% the NUMBER of pages is nowadays available from the kernel
% (\@abspage@last, \thetotalpages, \PreviousTotalPages),
% but with pages for example I to X and then 1 to 10, the number of pages would be "20",
% while the name of the last page is "10". Decide what you need/want!
%
% This work may be distributed and/or modified under the
% conditions of the LaTeX Project Public License, either
% version 1.3c of this license or (at your option) any later
% version. This version of this license is in
%    https://www.latex-project.org/lppl/lppl-1-3c.txt
% and the latest version of this license is in
%    https://www.latex-project.org/lppl.txt
% and version 1.3c or later is part of all distributions of
% LaTeX version 2005-12-01 or later.
%
% This work has the LPPL maintenance status "maintained".
%
% The Current Maintainer of this work is H.-Martin Münch.
%
% This work consists of the main source file lastpage.dtx,
% the README, and the derived files
%    lastpage.sty, lastpage209.sty, lastpage2e.sty,
%    lastpageclassic.sty, lastpagemodern.sty,
%    lastpage.pdf, lastpage.ins, lastpage.drv,
%    lastpage-example.tex, lastpage-example.pdf.
%
% "lastpage" is available on CTAN:
% https://ctan.org/pkg/lastpage
%
% Also a TDS.ZIP file is provided that contains all the files
% already sorted in a TDS tree:
% https://mirror.ctan.org/install/macros/latex/contrib/lastpage.tds.zip
%
%<*ignore>
\begingroup
  \catcode123=1 %
  \catcode125=2 %
  \def\x{LaTeX2e}%
\expandafter\endgroup
\ifcase 0\ifx\install y1\fi\expandafter
         \ifx\csname processbatchFile\endcsname\relax\else1\fi
         \ifx\fmtname\x\else 1\fi\relax
\else\csname fi\endcsname
%</ignore>
%<*install>
\input docstrip.tex
\Msg{**********************************************************************}
\Msg{* Installation                                                       *}
\Msg{* Package: lastpage 2025-01-27 v2.1e Refers to last page's name (HMM)*}
\Msg{**********************************************************************}

\keepsilent
\askforoverwritefalse

\let\MetaPrefix\relax
\preamble

This is a generated file.

Project: lastpage
Version: 2025-01-27 v2.1e
Info: Refers to last page's name

Copyright (C) 2010 - 2025 by
    H.-Martin Muench <Martin dot Muench at Uni-Bonn dot de>
Portions of code copyrighted by other people as marked.

This work may be distributed and/or modified under the
conditions of the LaTeX Project Public License, either
version 1.3c of this license or (at your option) any later
version. This version of this license is in
   https://www.latex-project.org/lppl/lppl-1-3c.txt
and the latest version of this license is in
   https://www.latex-project.org/lppl.txt
and version 1.3c or later is part of all distributions of
LaTeX version 2005-12-01 or later.

This work has the LPPL maintenance status "maintained".

The Current Maintainer of this work is H.-Martin Muench.

This package was invented by Jeffrey P. Goldberg.
I thought that an enhancement was needed and therefore created the pageslts package,
https://ctan.org/pkg/pageslts. Nevertheless, for compatibility with existing
documents/packages as well as for the low amount of resources needed by the
lastpage package (no new counter!), I updated this package.
Thanks go to Jeffrey P. Goldberg for allowing me to do this.

While this package allows for things like "Page slash thepage{} of slash pageref{LastPage}"
to get "Page 7 of 9" or "Page VII of IX",
the NUMBER of pages is nowadays available from the kernel
(slash @abspage@last, slash thetotalpages, slash PreviousTotalPages),
but with pages for example I to X and then 1 to 10, the number of pages would be "20",
while the name of the last page is "10". Decide what you need/want!

This work consists of the main source file lastpage.dtx,
the README, and the derived files
   lastpage.sty, lastpage209.sty, lastpage2e.sty,
   lastpageclassic.sty, lastpagemodern.sty,
   lastpage.pdf, lastpage.ins, lastpage.drv,
   lastpage-example.tex, lastpage-example.pdf.

In memoriam
 Claudia Simone Barth + 1996-01-30
 Tommy Muench         + 2014-01-02
 Hans-Klaus Muench    + 2014-08-24

\endpreamble
\let\MetaPrefix\DoubleperCent

\generate{%
  \file{lastpage.ins}{\from{lastpage.dtx}{install}}%
  \file{lastpage.drv}{\from{lastpage.dtx}{driver}}%
  \usedir{tex/latex/lastpage}%
  \file{lastpage.sty}{\from{lastpage.dtx}{package}}%
  \file{lastpage209.sty}{\from{lastpage.dtx}{lastpage209}}%
  \file{lastpage2e.sty}{\from{lastpage.dtx}{lastpage2e}}%
  \file{lastpageclassic.sty}{\from{lastpage.dtx}{lastpageclassic}}%
  \file{lastpagemodern.sty}{\from{lastpage.dtx}{lastpagemodern}}%
  \usedir{doc/latex/lastpage}%
  \file{lastpage-example.tex}{\from{lastpage.dtx}{example}}%
}

\catcode32=13\relax% active space
\let =\space%
\Msg{************************************************************************}
\Msg{*}
\Msg{* To finish the installation you have to move the following}
\Msg{* file into a directory searched by TeX:}
\Msg{*}
\Msg{*  lastpage.sty, lastpage209.sty, lastpage2e.sty,}
\Msg{*  lastpageclassic.sty, and lastpagemodern.sty}
\Msg{*}
\Msg{* To produce the documentation run the file `lastpage.drv'}
\Msg{* through (pdf)LaTeX, e.g.}
\Msg{*  pdflatex lastpage.drv}
\Msg{*  makeindex -s gind.ist lastpage.idx}
\Msg{*  pdflatex lastpage.drv}
\Msg{*  makeindex -s gind.ist lastpage.idx}
\Msg{*  pdflatex lastpage.drv}
\Msg{*}
\Msg{* At least three runs are necessary e.g. to get the}
\Msg{*  references right!}
\Msg{*}
\Msg{* Happy TeXing!}
\Msg{*}
\Msg{************************************************************************}

\endbatchfile
%</install>
%<*ignore>
\fi
%</ignore>
%
% \section{The documentation driver file}
%
% The next bit of code contains the documentation driver file for
% \TeX , i.\,e., the file that will produce the documentation you
% are currently reading. It will be extracted from this file by the
% \texttt{docstrip} programme. That is, run \LaTeX{} on \texttt{docstrip}
% and specify the \texttt{driver} option when \texttt{docstrip}
% asks for options.
%
%    \begin{macrocode}
%<*driver>
\NeedsTeXFormat{LaTeX2e}[2024-11-01]
\ProvidesFile{lastpage.drv}[2025-01-27 v2.1e Refers to last page's name (HMM)]
\documentclass{ltxdoc}[2024/02/08]% v2.1j Standard LaTeX documentation class
\usepackage{holtxdoc}[2019/12/09]%  v0.30 Private additional ltxdoc support (HO)
\hypersetup{%
 pdfsubject={Refers to last page's name (HMM; JPG)},%
 pdfkeywords={LaTeX, lastpage, last page, page number, page name},%
 pdfencoding=auto,%
 pdflang={en},%
 breaklinks=true,%
 linktoc=all,%
 pdfstartview=FitH,%
 pdfpagelayout=OneColumn,%
 bookmarksnumbered=true,%
 bookmarksopen=true,%
 bookmarksopenlevel=2,%
 pdfmenubar=true,%
 pdftoolbar=true,%
 pdfwindowui=true,%
 pdfnewwindow=true%
}
\CodelineIndex
\hyphenation{every-thing}
\begin{document}
  \DocInput{lastpage.dtx}%
\end{document}
%</driver>
%    \end{macrocode}
% \fi
%
% \GetFileInfo{lastpage.drv}
%
% \begingroup
%   \def\x{\#,\$,\^,\_,\~,\ ,\&,\{,\},\%}%
%   \makeatletter
%   \@onelevel@sanitize\x
% \expandafter\endgroup
% \expandafter\DoNotIndex\expandafter{\x}
% \expandafter\DoNotIndex\expandafter{\string\ }
% \begingroup
%   \makeatletter
%     \lccode`9=32\relax
%     \lowercase{%^^A
%       \edef\x{\noexpand\DoNotIndex{\@backslashchar9}}%^^A
%     }%^^A
%   \expandafter\endgroup\x
%
% \DoNotIndex{\@auxout,\@evenfoot,\@firstofone,\@firstoftwo,\@ifl@t@r,\@ifpackagelater,\@ifpackageloaded,%
% \IfPackageAtLeastF,\IfPackageLoadedT,\IfPackageLoadedTF,%
% \@ifundefined,\@kernel@reserved@label@data,\@mainaux,\@number,\@oddfoot,%
% \addtocounter,\AddToHook,\arabic,\begin,%
% \bigskip,\clearpage,\csname,\detokenize,\documentclass,\EdefUnescapeString,\empty,\end,%
% \endcsname,\enddocument,\expanded,\ExplSyntaxOff,\ExplSyntaxOn,\fmtversion,%
% \g,\gdef,\here,\hfil,\Hy@temp,\Hy@unicodefalse,\Hy@EveryPageAnchor,%
% \hypersetup,\if,\if@filesw,\if@nobreak,\ifHy@hypertexnames,\ifHy@pageanchor,\ifHy@plainpages,%
% \ifdim,\ifnum,\ifvmode,\ifx,\immediate,\input,%
% \label,\lastpage,\lastpage-example.tex,\lastpage.dtx,\lastpage.sty,\lastpagee.sty,%
% \lastpageclassic.sty,\lastpagemodern.sty,\LaTeX,\LaTeX-kernel,\listfiles,%
% \makeatletter,\makeatother,\markboth,\meaning,\mbox,\message,\MessageBreak,%
% \NeedsTeXFormat,\newcommand,\newcounter,\newline,\newpage,\nobreak,%
% \noindent,\normalsize,\numexpr,\origenddocument,\PackageError,\PackageNoteNoLine,\PackageWarning,%
% \PackageWarningNoLine,\pagenumbering,\pageref,\pdfstringdef,\protect,\ProvidesPackage,%
% \qquad,\renewcommand,\RequirePackage,\section,\ShowHook,\slshape,\smallskip,\space,\test,\textbf,%
% \textit,\textquotedblleft,\textquotedblright,\textsf,\texttt,\the,\thepage,%
% \today,\upshape,\url,\usepackage,\value,\verb,\wd,\write,\xlastpage@rmpage,\xxlastpage@rmpage}
%
% \title{The \xpackage{lastpage} package}
% \author{H.-Martin Münch\\\xemail{Martin.Muench at Uni-Bonn.de}\\
%   invented by Jeffrey P. Goldberg\\\xemail{jeffrey+news at goldmark.org}}
% \date{2025-01-27 v2.1e}
% \maketitle
%
% \begin{abstract}
% \noindent This \LaTeX{} package puts the label \texttt{LastPage}
% at the end of the document into the \xfile{.aux} file,
% allowing the user to refer to the last page of a document.
% This might be particularly useful in places like headers or footers.~--\newline
% While this package allows for things like\newline
% \hbox{\textquotedblleft Page} \hbox{|\thepage{}|} of |\pageref{LastPage}|\textquotedblright{}
% to get \textquotedblleft Page 7 of 9\textquotedblright{} or
% \textquotedblleft Page VII of IX\textquotedblright,
% the \emph{number} of pages is nowadays available from the kernel
% (|\@abspage@last|, |\thetotalpages|, |\PreviousTotalPages|),
% but when more than one page numbering scheme is used
% (for example pages I~to X and then 1~to 10,
% thus number of pages \hbox{\textquotedblleft 20\textquotedblright ,}
% but name of the last page \hbox{\textquotedblleft 10\textquotedblright ),}
% or another package has output after this package,
% or the page numbers exceed a certain range,
% there might be problems, which can be solved by using the
% \xpackage{pageslts} package instead.
% \end{abstract}
%
% \bigskip
%
% \noindent Disclaimer for web links: The author is not responsible for any contents
% referred to in this work unless having full knowledge of illegal contents.
% If any damage occurs by the use of information presented there,
% only the author of the respective pages might be liable,
% not the one who has referred to those pages.
% \newpage
% \tableofcontents
%
% \section{Introduction}
% \indent This \LaTeX{} package puts the label \texttt{LastPage}
% (at end of the document via hook |enddocument/afterlastpage|,
% for older formats via |\AtEndDocument|, for \LaTeX2.09{} via
% redefining |\enddocument|) into the \xfile{aux} file,
% allowing the user to refer to the last page of a document via
% |\pageref{LastPage}|. This might be particularly useful
% in places like headers or footers.
%
% \bigskip
%
% This package was invented by \textbf{Jeffrey P. Goldberg},
% and is now maintained by \textsc{H.-Martin Münch}. A~big
% \textquotedblleft Thank you!\textquotedblright{} to
% \textsc{Jeffrey P. Goldberg} for granting this.
%
% \bigskip
%
% If you are more ambitious in respect to your aims with this package,
% you might want to have a look at the \xpackage{pageslts} package
% (see section~\ref{sec:Alternatives}: Alternatives).
%
% \section{Usage}
% \indent Just load the package placing
% \begin{quote}
%   |\usepackage{lastpage}|
% \end{quote}
% \noindent in the preamble of your source file
% (or |%%
%% This is file `lastpage.sty',
%% generated with the docstrip utility.
%%
%% The original source files were:
%%
%% lastpage.dtx  (with options: `package')
%% 
%% This is a generated file.
%% 
%% Project: lastpage
%% Version: 2025-01-27 v2.1e
%% Info: Refers to last page's name
%% 
%% Copyright (C) 2010 - 2025 by
%%     H.-Martin Muench <Martin dot Muench at Uni-Bonn dot de>
%% Portions of code copyrighted by other people as marked.
%% 
%% This work may be distributed and/or modified under the
%% conditions of the LaTeX Project Public License, either
%% version 1.3c of this license or (at your option) any later
%% version. This version of this license is in
%%    https://www.latex-project.org/lppl/lppl-1-3c.txt
%% and the latest version of this license is in
%%    https://www.latex-project.org/lppl.txt
%% and version 1.3c or later is part of all distributions of
%% LaTeX version 2005-12-01 or later.
%% 
%% This work has the LPPL maintenance status "maintained".
%% 
%% The Current Maintainer of this work is H.-Martin Muench.
%% 
%% This package was invented by Jeffrey P. Goldberg.
%% I thought that an enhancement was needed and therefore created the pageslts package,
%% https://ctan.org/pkg/pageslts. Nevertheless, for compatibility with existing
%% documents/packages as well as for the low amount of resources needed by the
%% lastpage package (no new counter!), I updated this package.
%% Thanks go to Jeffrey P. Goldberg for allowing me to do this.
%% 
%% While this package allows for things like "Page slash thepage{} of slash pageref{LastPage}"
%% to get "Page 7 of 9" or "Page VII of IX",
%% the NUMBER of pages is nowadays available from the kernel
%% (slash @abspage@last, slash thetotalpages, slash PreviousTotalPages),
%% but with pages for example I to X and then 1 to 10, the number of pages would be "20",
%% while the name of the last page is "10". Decide what you need/want!
%% 
%% This work consists of the main source file lastpage.dtx,
%% the README, and the derived files
%%    lastpage.sty, lastpage209.sty, lastpage2e.sty,
%%    lastpageclassic.sty, lastpagemodern.sty,
%%    lastpage.pdf, lastpage.ins, lastpage.drv,
%%    lastpage-example.tex, lastpage-example.pdf.
%% 
%% In memoriam
%%  Claudia Simone Barth + 1996-01-30
%%  Tommy Muench         + 2014-01-02
%%  Hans-Klaus Muench    + 2014-08-24
%% 
%% Part of the "lastpage" package
%% loads either lastpage2.09.sty for TeX 2.09 or lastpage2e.sty for LaTeX 2e
%% with code from https://groups.google.com/g/comp.text.tex/c/-Qmhj1ZI4xM
\def\loadlastpage{\ProvidesPackage{lastpage}[2025/01/27 v2.1e lastpage: 2.09 or 2e? (HMM)]\relax\RequirePackage{lastpage2e}}
\begingroup \expandafter \ifx \csname documentclass\endcsname\relax
\endgroup \expandafter %%
%% This is file `lastpage209.sty',
%% generated with the docstrip utility.
%%
%% The original source files were:
%%
%% lastpage.dtx  (with options: `lastpage209')
%% 
%% This is a generated file.
%% 
%% Project: lastpage
%% Version: 2025-01-27 v2.1e
%% Info: Refers to last page's name
%% 
%% Copyright (C) 2010 - 2025 by
%%     H.-Martin Muench <Martin dot Muench at Uni-Bonn dot de>
%% Portions of code copyrighted by other people as marked.
%% 
%% This work may be distributed and/or modified under the
%% conditions of the LaTeX Project Public License, either
%% version 1.3c of this license or (at your option) any later
%% version. This version of this license is in
%%    https://www.latex-project.org/lppl/lppl-1-3c.txt
%% and the latest version of this license is in
%%    https://www.latex-project.org/lppl.txt
%% and version 1.3c or later is part of all distributions of
%% LaTeX version 2005-12-01 or later.
%% 
%% This work has the LPPL maintenance status "maintained".
%% 
%% The Current Maintainer of this work is H.-Martin Muench.
%% 
%% This package was invented by Jeffrey P. Goldberg.
%% I thought that an enhancement was needed and therefore created the pageslts package,
%% https://ctan.org/pkg/pageslts. Nevertheless, for compatibility with existing
%% documents/packages as well as for the low amount of resources needed by the
%% lastpage package (no new counter!), I updated this package.
%% Thanks go to Jeffrey P. Goldberg for allowing me to do this.
%% 
%% While this package allows for things like "Page slash thepage{} of slash pageref{LastPage}"
%% to get "Page 7 of 9" or "Page VII of IX",
%% the NUMBER of pages is nowadays available from the kernel
%% (slash @abspage@last, slash thetotalpages, slash PreviousTotalPages),
%% but with pages for example I to X and then 1 to 10, the number of pages would be "20",
%% while the name of the last page is "10". Decide what you need/want!
%% 
%% This work consists of the main source file lastpage.dtx,
%% the README, and the derived files
%%    lastpage.sty, lastpage209.sty, lastpage2e.sty,
%%    lastpageclassic.sty, lastpagemodern.sty,
%%    lastpage.pdf, lastpage.ins, lastpage.drv,
%%    lastpage-example.tex, lastpage-example.pdf.
%% 
%% In memoriam
%%  Claudia Simone Barth + 1996-01-30
%%  Tommy Muench         + 2014-01-02
%%  Hans-Klaus Muench    + 2014-08-24
%% 
%% Part of the "lastpage" package
%% FOR LaTeX 2.09 ONLY - FOR LaTeX 2e USE lastpage2e.sty
%% This is lastpage209.sty invented by Jeffrey P. Goldberg,
%% after Piet van Oostrum: Page layout in LaTeX, March 2, 2004, section 16;
%% fancyhdr.pdf; lastpage209.sty maintained by H.-Martin Muench.
\let\origenddocument=\enddocument%
\def\enddocument{\clearpage%
  {\addtocounter{page}{-1}%
   \immediate\write\@mainaux{\string\newlabel{LastPage}{{}{\thepage}}}%
  }%
  \addtocounter{page}{+1}%
  \origenddocument%
 }
\endinput
%%
%% End of file `lastpage209.sty'.

\else \endgroup \expandafter \loadlastpage
\fi
\endinput
%%
%% End of file `lastpage.sty'.
| if |\usepackage| is unknown).
%
% \indent For example for various draft forms it is desirable to have a
% page reference to the last page, so that e.\,g.\ page footers can
% contain something like \textquotedblleft page $N$ of $K$\textquotedblright,
% where $N$ is the current page and $K$ is the last page. Once the package
% is loaded, anywhere in the text references can be made to the label
% \texttt{LastPage}. In particular one can use the \xpackage{fancyhdr}
% or \xpackage{nccfancyhdr} package, or redefinitions of the page headings
% and footings to get a reference to the last page.
%
% \noindent In your document the code
% \begin{verbatim}
% \makeatletter
% \renewcommand{\@evenfoot}{%
%  \normalsize\slshape DRAFT \today\hfil \upshape %
%  page \thepage{} of \pageref{LastPage}}
% \renewcommand{\@oddfoot}{\@evenfoot}
% \makeatother
%\end{verbatim}
% \noindent creates footers like\smallbreak
% \textquotedblleft\mbox{\textsl{DRAFT \today}\hspace{1cm}page 7 of 9}\textquotedblright\smallbreak
% \noindent in the compiled document (cf.~the \texttt{lastpage-example} file).\newline
% If the \xpackage{hyperref} package is used, the references are hyperlinked
% to their aims. If these hyperlinks shall be suppressed, |\pageref*{...}|
% instead of |\pageref{...}| can be used.\bigbreak
%
% The \xpackage{lastpage} package does not provide the words
% \textquotedblleft page\textquotedblright{} or \textquotedblleft of\textquotedblright{},
% but e.\,g.\ the \xclass{handout} class uses \textquotedblleft of\textquotedblright{} in
% the definition of the footer. (In the \texttt{lastpage-example} also
% |\@evenfoot| is redefined, but it is not the \xpackage{lastpage} \emph{package}
% redefining this.) If you want to change \textquotedblleft page\textquotedblright{} or
% \textquotedblleft of\textquotedblright{} (e.\,g.\ to another language), you therefore
% have got to look in the used class/package(s)/preamble instead of in the
% \xpackage{lastpage} package.\bigbreak
%
% If the total \emph{number} of pages of a document is needed,
% the kernel already gives this by |\makeatletter\@abspage@last\makeatother|,
% |\thetotalpages|, and |\PreviousTotalPages| (needing at least two compiler runs).
%
% \section{Some \textsc{Warnings}\label{sec:warn}}
% \subsection{\texttt{\textbackslash AtEndDocument}\label{ssec:aed}}
% \indent {\bfseries |\AtEndDocument| is not used by the \pkg{lastpagemodern.sty}
% version of the lastpage package, requiring \LaTeX -format 2024-11-01
% or newer. Instead |\AddToHook{enddocument/afterlastpage}| is used
% and the problem does not arise.}
% \bigskip
%
% \pkg{lastpageclassic.sty} uses |\AtEndDocument| and
% \pkg{lastpage209.sty} redefines \linebreak |\enddocument|.
% The last two cases are problematic:\newline
% The output of a \LaTeXe{} run is not independent of the order
% in which the packages are loaded. It is often the case that the same
% formats for which one must put tables and figure at the end,
% are the ones in which endnotes are also required.
% If one wants to use |\AtEndDocument| here as well (as done for
% |\pageref{LastPage}|), then it is easy to get to three separate
% uses of |\AtEndDocument| (assuming one uses this for the endnotes
% as well). Clearly it is not safe for any package writer or user
% to assume that no material will follow what they put into
% |\AtEndDocument|. Therefore a message, which begins with
% \texttt{AED}, is included in every usage of |\AtEndDocument|.
% \xpackage{lastpage} uses |\AtEndDocument{|\ldots|\clearpage|\ldots|}|, thus\newline
%|\usepackage{lastpage}|\ldots|\AtEndDocument{|\textit{something}|}|
% will place \textit{something} after the |\clearpage|. To place it earlier, use\newline
% |\AtEndDocument{|\textit{something}|}|\ldots|\usepackage{lastpage}|.
% If the \textit{something} is not known before |\usepackage{lastpage}|,
% you can use for example
% \begin{verbatim}
% ...
% \def\beforeLastpageClearpage{\relax}
% \AtEndDocument{\beforeLastpageClearpage}
% \usepackage{lastpage}
% \begin{document}
% ...
% \def\beforeLastpageClearpage{\textit{something}}%
% ...
% \end{document}
% \end{verbatim}
% (might need a protected and/or expanded |\def|). When |\clearpage| leads
% to some output, |\clearpage\textit{something}| instead of |\textit{something}|
% might be wanted.
%
% \subsection{Interaction with ancient versions of the \xpackage{endfloat} package\label{sec:endfloat}}
%
% \indent {\bfseries |\AtEndDocument| is not used by the \pkg{lastpagemodern.sty}
% version of the lastpage package, requiring \LaTeX -format 2024-11-01
% or newer. Instead |\AddToHook{enddocument/afterlastpage}| is used
% and the problem does not arise.}
%
% \bigskip
% The \emph{ancient} version~2.0 (and earlier; 2.0~from 1992;
% current version at the time of updating this documentation: 2.7~from 2019)
% of the \xpackage{endfloat} package actually redefined the |\enddocument| command,
% and so interfered drastically with the \LaTeXe{} commands which make use of
% |\AtEndDocument|.
% If you want your \texttt{LastPage} to label the last page of these end floats,
% you need to load \xpackage{lastpage} after loading \xpackage{endfloat}
% (or use \texttt{VeryLastPage} from the \xpackage{pageslts} package instead).
% If, on the other hand, you \emph{want} \texttt{LastPage} to refer
% to the (not so) last page, exclusive of the floats at the end,
% then load in the reverse order. Independent from the order of
% \xpackage{lastpage} and \xpackage{endfloat}, you will still need
% a version of \xpackage{endfloat} later than 2.0 from 1992.
%
% Other \LaTeX2.09{} (!) packages also seem to like to redefine
% |\enddocument|. In addition to the old \xpackage{endfloat},
% \xpackage{harvard} comes to mind. All of these will need to be
% modified swiftly.
%
% \subsection{Page name instead of page number}
% When any page numbering scheme other than \texttt{arabic} is used
% at the page, which |\pageref{LastPage}| refers to, the \textit{name}
% and not the \textit{number} of the page is given. For example,
% \texttt{Alph} page numbering scheme and $10$ pages will give \texttt{J} instead of 10,
% \texttt{Roman} page numbering scheme and $10$ pages will give \texttt{X} instead of 10,
% and so on.\newline
% \indent (The \xpackage{pageslts} package puts |\lastpageref{LastPages}|
% (with \textbf{s} at the end) at your disposal for remediation.)
%
% \subsection{No write access to the \xfile{aux} file}
% Some packages (e.\,g.\ \xpackage{tikz} and \xpackage{selectp}) sometimes prevent
% the output to the \xfile{aux} file. In that case a warning is issued. This is
% no problem as long as there is another compilation run where the label to the
% last page can be placed via the \xfile{aux} file.
%
% \subsection[Wrong last page number with \textgreater 1 page numbering scheme]{%
%  Wrong last page number with more than one page numbering scheme}
% When more than one page numbering scheme is used,
% \texttt{LastPage} does not give the total \textbf{number} of pages
% (even if \texttt{arabic} is the page numbering scheme of that page).
% For example, for a document with VI+36 pages, it gives
% \textquotedblleft 36\textquotedblright{} as reference to the last page.
% While this is correct, the total number of pages is $42$.\newline
% If the total \emph{number} of pages of a document is needed,
% the kernel already gives this by |\makeatletter\@abspage@last\makeatother|,
% |\thetotalpages|, and |\PreviousTotalPages|.
% The \xpackage{pageslts} package puts |\lastpageref{LastPages}|
% (with \textbf{s} at the end) at your disposal for remediation,
% giving the number of pages and linking to the last page, if linking is provided
% for examaple by the \xpackage{hyperref} package.
%
% \subsection{\texttt{\textbackslash addtocounter\{page\}\{\ldots\} and %
%  \texttt{\textbackslash setcounter\{page\}\{\ldots\}}}}
% When the page number was manipulated by |\addtocounter{page}{...}| or
% |\setcounter{page}{...}|, \texttt{LastPage} does not give the total
% \textbf{number} of pages (even if \texttt{arabic} is the page numbering
% scheme of that page).\newline
% The \xpackage{pageslts} package puts |\lastpageref{LastPages}|
% (with \textbf{s} at the end) at your disposal for remediation:
% \texttt{LastPages} ignores page number manipulation. Also |\@abspage@last|,
% |\thetotalpages|, and |\PreviousTotalPages| from the kernel
% are not influenced by page number manipulation.
%
% \subsection{Page number reset by \texttt{\textbackslash pagenumbering\{\ldots\}}}
% At a page numbering change the page number is reset to one.
% Therefore \texttt{LastPage} does not give the total \textbf{number} of pages
% (even if \texttt{arabic} is the page numbering scheme of that page).
% Furthermore, now two pages have the same name.\newline
% The \xpackage{pageslts} package does not only put
% |\lastpageref{LastPages}| (with \textbf{s} at the end) at your disposal
% for remediation: \texttt{LastPages} also ignores page number manipulation.
% It furthermore offers the option |pagecontinue| to continue the
% page numbering, when |\pagenumbering{...}| is used.
%
% \subsection{Last pages of different page numbering schemes}
% |\pageref{LastPage}| refers to the (maybe not so) last page of the last
% page numbering scheme. References to the respective last page of the other
% page numbering schemes are not provided.\newline
% The \xpackage{pageslts} package does this with labels
% \texttt{pagesLTS.<numbering scheme>}, where \texttt{<numbering scheme>} is
% e.\,g.\ arabic, roman, Roman, alph, or Alph.\linebreak
% For fnsymbol please use |\lastpageref{pagesLTS.fnsymbol}| instead of\newline
% |\pageref{pagesLTS.fnsymbol}|.
%
% \subsection{Current page}
% The command |\thepage| gives the \textbf{name} of the current page
% in the current page numbering scheme, which is different from the
% current total/absolute page number e.\,g.\ with a second
% page numbering scheme, |\addtocounter{page}{...}|, or |\setcounter{page}{...}|,
% and it will not be an arabic number at all,
% if the current page numbering scheme is not arabic.\newline
% The \xpackage{pageslts} package provides the command
% |\theCurrentPage| and for the current number of pages in the current
% page numbering scheme\newline
% |\theCurrentPageLocal|. The kernel already provides the
% number of pages, which have been shipped out, as |\the\ReadonlyShipoutCounter|.
% The current page is always |ReadonlyShipoutCounter +1|.
%
% \subsection{First page}
% There is no special label at the first page.
% (This is the \textbf{last}\textsf{page} package, after all.)\newline
% The \xpackage{pageslts} package creates a label \texttt{pagesLTS.0}
% at the first page of the document.
%
% \subsection{Page counter overflow\label{ss:overflow}}
% \indent \textquotedblleft The ranges of supported counter values are more or less
% restricted. Only \cs{arabic} can be used with any counter value \TeX{} supports.
% \begin{quote}
% \catcode`\|=12 %
% \begin{tabular}{@{}l|l|l|l@{}}
% Presentation & Supported & Ignored & Error message\\
% command      & domain    & values  & `Counter too large'\\
% \hline
% \cs{arabic}
%   & \ttfamily -MAX..MAX & &\\
% \cs{roman}, \cs{Roman}
%   & \ttfamily 1..MAX & \ttfamily -MAX..0 &\\
% \cs{alph}, \cs{Alph}
%   & \ttfamily 1..26 & 0 & \ttfamily -MAX..-1, 27..MAX\\
% \cs{fnsymbol}
%   & \ttfamily 1..9 & 0 & \ttfamily -MAX..-1, 10..MAX\\
% \hline
% \end{tabular}\\[1ex]
% \texttt{MAX} = \texttt{2147483647}
% \end{quote}
% \textquotedblright\ (\xpackage{alphalph} package manual, 2019/12/09, v2.6,
% first table, p.~2).
%
% \noindent When \textit{any} page is out of that range, there will be a counter overflow.\newline
% \xpackage{lastpage} probably is not the right package to be asked
% to correct this anyway, but the \xpackage{pageslts} package
% (with appropriate options) can do this.
%
% When MAX is exceeded via
% |\setcounter{<name>}{| something greater than MAX (or smaller then $-$MAX) |}|,
% then the error
% \begin{quote}
% \begin{verbatim}
% ! Number too big.
% I can only go up to 2147483647='17777777777="7FFFFFFF,
% so I'm using that number instead of yours.
% \end{verbatim}
% \end{quote}
% \vspace{-\baselineskip}
% will arise. But if the counter has a value of $2\,147\,483\,647 = {}$MAX,
% and |\addtocounter{<name>}{+1}| is tried, no error is issued, but
% |\arabic{<name>}| prints $-2147483648$, and further |\addtocounter{<name>}{+1}|s
% give $-2147483647$, $-2147483646$ and so on.\newline
% For a counter value of $-2\,147\,483\,647 = -$MAX and |\addtocounter{<name>}{-1}|s
% after $-2147483647$ it is printed $-2147483648$, $2147483647$, $2147483646$ and so on
% (without any message in the \xfile{log} file about any possible issue).
%
% \subsection{Other packages manipulating \texttt{\textbackslash lastpage@putlabel}}
% The \xpackage{revtex4} class redefines the |\lastpage@putlabel| command
% to place a label |LastPage|.\newline
% |\lastpage@putlabel| in the \xpackage{lastpage} package was replaced by |\lastpage@putl@bel|,
% but the |LastPage| label could become defined more than once.
%
% \subsection{\texttt{\textbackslash pagenumbering\{fnsymbol\}}\label{subsec:fnsymbol}}
% When using the foot-note-symbols as page numbers,
% it can be necessary to declare in the document's preamble:
% \begin{verbatim}
% \ProvideTextCommand{\textasteriskcentered}{PD1}{*}
% \ProvideTextCommand{\textdagger}{PD1}{†}
% %\ProvideTextCommand{\textdaggerdbl}{PD1}{‡)% seems to neither work nor be necessary
% \ProvideTextCommand{\textsection}{PD1}{§}
% \ProvideTextCommand{\textparagraph}{PD1}{¶}
% \ProvideTextCommand{\textbardbl}{PD1}{‖}
% \end{verbatim}
%
% \section{Alternatives\label{sec:Alternatives}}
% There are similar packages, which do (or do not) similar things (or even more).
% As I neither know what exactly you want to accomplish when using this package
% (e.\,g.~page number vs. page name, hyperlinks or not), nor what resources
% your system has (e.\,g.~\TeX, \LaTeX2e, $\varepsilon$-\TeX{}, \LaTeX-format as recent
% as 2024-11-01 or newer), here is a list of some possible alternatives:
% \begin{description}
% \item[-]\DescribeMacro{pageslts}
%  The \xpackage{pageslts} package first started as a revision of this
%  \xpackage{lastpage} package, but I~thought that an enhancement was needed
%  to accomplish what the \xpackage{pageslts} package does. For backward
%  compatibility, a label named |LastPage| is provided.
%  Thus |\usepackage{lastpage}| can be replaced by\newline
%  |\usepackage[pagecontinue=false,alphMult=0,AlphMulti=0,|\newline
%  | fnsymbolmult=false,romanMult=false,RomanMulti=false]{pageslts}|,\newline
%  if the behaviour of the \xpackage{lastpage} package should be simulated.
%  The default options are\newline
%  |\usepackage[pagecontinue=true,alphMult=ab,AlphMulti=AB,|\newline
%  |fnsymbolmult=true,romanMult=true,RomanMulti=true]{pageslts}|.\newline
%  Benefits of \xpackage{pageslts} package (with appropriate options) are:
%  \begin{description}
%  \item[+] Labels \texttt{LastPage} (|\AddToHook{enddocument/afterlastpage}|,
%   formerly |\AtEndDocument|; same as the \xpackage{LastPage} package) and
%   \texttt{VeryLastPage} (also |\AddToHook{enddocument/afterlastpage}|,
%   but formerly\newline%
%   |\AfterLastShipout|), allowing the user to refer to the
%   (very) last page of a document.
%  \item[+] For example, when more than one page numbering scheme is used,
%    the label \mbox{\texttt{LastPage}\textbf{s}} gives the total \textit{number} of pages.
%  \item[+] At the last page of each page numbering scheme a label\newline
%   \texttt{pagesLTS.<numbering scheme>} is placed, where
%   \texttt{<numbering scheme>} is e.\,g.\ arabic, roman, Roman, alph, or Alph.
%   For fnsymbol please use |\lastpageref{pagesLTS.fnsymbol}| instead of\newline
%   |\pageref{pagesLTS.fnsymbol}|.
%  \item[+] When the same numbering scheme is used twice, the page numbers
%   are either reset to one or continued automatically, depending on the option
%   given when the package is called.
%  \item[+] The command |\theCurrentPage| prints the current total/absolute
%   page number -- in contrast to |\thepage|, which gives only the page
%   \textit{name} in the current page numbering scheme.
%   |\theCurrentPageLocal| gives the current number of pages in the current
%   page numbering scheme. |\thepage| and |\theCurrentPageLocal| are different
%   e.\,g.\ when |\addtocounter{page}{...}| or |\setcounter{page}{...}| were used.
%  \item[+] At the first page of the document a label \texttt{pagesLTS.0} is created.
%  \item[+] The \xpackage{alphalph} package is supported, i.\,e.\ %
%   page numbers alph or Alph $>26$ and fnsymbol $>9$ can be used
%   (with according options set). Even zero and negative page numbers can be used
%   with \texttt{arabic}, \texttt{alph}, \texttt{Alph}, \texttt{roman}, \texttt{Roman},
%   and \texttt{fnsymbol} page numbering (with \xpackage{alphalph} package and
%   according options).
%  \end{description}
%  Further labels are provided for special cases.\newline
%  \url{https://ctan.org/pkg/pageslts}
%
% \item[-]\DescribeMacro{LaTeX-kernel}
%  The number of pages is nowadays available via |\@abspage@last|,\linebreak
%  |\thetotalpages|, and |\PreviousTotalPages| from the kernel,
%  but when more than one page numbering scheme is used
%  (for example pages I~to X and then 1~to 10, thus number of pages
%  \hbox{\textquotedblleft 20\textquotedblright ,} but name of the last page
%  \hbox{\textquotedblleft 10\textquotedblright ),}
%  or when or the fnsymbol page numbering scheme is used, or another package
%  has output after this package, or the page numbers exceed a certain range,
%  there might be issues. (Is the total number of pages wanted? Or is the name
%  of the last page sought?)\newline
%  |\the\ReadonlyShipoutCounter| contains the number of currently shipped out pages,
%  i.\,e.\ current page minus one.
%
% \item[-]\DescribeMacro{totpages}
%  The \xpackage{totpages} package provides a \texttt{totpages} label similar to
%  \texttt{LastPages}, but |\AtEndDocument| instead of hook |enddocument/afterlastpage|
%  of the \pkg{pageslts} package.
%  The \xpackage{totpages} package additionally computes the number of paper sheets
%  needed to (double) print the document (with one, two, three,\ldots{} pages on
%  one sheet of paper) (which can be achieved also with the \xpackage{papermas} package,
%  an extension of the \xpackage{pageslts} package, which further allows to compute
%  the mass of that printed version of the document, useful e.\,g.\ when sending it
%  by mail to determine the postage).\newline
%  \url{https://ctan.org/pkg/totpages}
%
% \item[-]\DescribeMacro{totalcount}
%  The \xpackage{totalcount} package provides |\totalpages|.
%  If there are only arabic page numbers consecutively running from 1 to the last page,
%  this works. But for example
%
% \begin{quote}
%\begin{verbatim}
%\documentclass{article}
%\usepackage[page]{totalcount}
%\pagenumbering{Roman}
%\begin{document}
%\addtocounter{page}{49}
%Page \thepage{} of \totalpages
%\end{document}
%\end{verbatim}
% \end{quote}
%
%  prints \textquotedblleft Page L of 50\textquotedblright,
%  where the number of pages is one (and no hyperlink is provided to the
%  last page even if \xpackage{hyperref} is used).\newline
%  \url{https://ctan.org/pkg/totalcount}
%
% \item[-]\DescribeMacro{totcount}
%  The \xpackage{totcount} package provides the last value of a counter, thus also the value
%  of the \texttt{page} counter. You do not get a hyperlink to the last page, only the numerical
%  value of the last page name is given \mbox{(i.\,e.~X+72} pages gives 72 instead of 82
%  as total number of pages), and the number of pages can be changed for example
%  by |\addtocounter|.\newline
%  \url{https://ctan.org/pkg/totcount}
% \pagebreak
%
% \item[-]\DescribeMacro{nofm}
%  \textquotedblleft There is a package \xpackage{nofm.sty} available, but some versions
%  of it are defective, and most don't work with \xpackage{fancyhdr} because they take over
%  the complete page layout.\textquotedblright{} (\textsc{Piet van Oostrum}:
%  Page layout in \LaTeX , March~2, 2004, section~16; fancyhdr.pdf)\newline
%  \xpackage{nofm} as of 1991/02/25 (without version number), available at\newline
%  \url{https://mirror.ctan.org/obsolete/macros/latex209/contrib/misc/nofm.sty},\newline
%  does not work with e.\,g.\ \xpackage{hyperref}, redefines |\enddocument|
%  as well as |\@oddhead|, |\@evenhead|, |\@oddfoot|, and |\@evenfoot|.\newline
%  If you know the (\url{https://CTAN.org}) location of a
%  \textbf{working}~(!) version, please send me an e-mail, thanks!
%
% \item[-]\DescribeMacro{count1to}
%  The \xpackage{count1to} package \textquotedblleft sets |\count1| to |\count8|
%  with the values of page to subparagraph. |\count9| is used to flag odd
%  pages. \ldots\ [T]he code for the TotalPages
%  label\textquotedblright\ (package manual, 2024-06-13) has been removed
%  from the current package version.\newline
%  \url{https://ctan.org/pkg/count1to}
%
% \item[-]\DescribeMacro{zref}
%  The \xpackage{zref} package \textquotedblleft implements an extensible referencing
%  system\textquotedblright\ (package manual, 2023-09-14).\newline
%  \url{https://ctan.org/pkg/zref}
%
% \item[-]\DescribeMacro{memoir}
%  The \xpackage{memoir} \emph{class} provides |\thelastpage| (page number printed
%  on last page) and |\thelastsheet| (number of pages).\newline
%  \url{https://ctan.org/pkg/memoir}
% \end{description}
%
% \bigskip
%
% \noindent (You programmed or found another alternative,
%  which is available at\newline
%  \url{https://CTAN.org}?
%  OK, send an e-mail to me with the name, location at CTAN,
%  and a short notice, and I will probably include it in the list above.)
% \newpage
%
% \section{Example}
%    \begin{macrocode}
%<*example>
\documentclass[british]{article}[2024/06/29]% v1.4n Standard LaTeX document class
\makeatletter
\usepackage[draft]{showkeys}[2024/05/23]% v3.21 Show cite and label keys (DPC, MH)
%%      Use final instead of draft to hide the keys. %%
\usepackage[pdfpagelabels=true,hyperindex=false]{hyperref}[2024-10-30]% v7.01k
\@ifpackageloaded{hyperref}{% Hypertext links for LaTeX
\hypersetup{extension=pdf,%
 plainpages=false,%
 pdflang={en},%
 pdftitle={lastpage package example},%
 pdfauthor={H.-Martin Muench},%
 pdfsubject={Example for the lastpage package},%
 pdfkeywords={LaTeX, lastpage},%
 pdfview=Fit,%
 pdfstartview=Fit,%
 pdfpagelayout=SinglePage%
}}{\usepackage{url}[2013/09/16]}%  v3.4  Verb mode for urls, etc.
\usepackage{lastpage}[2025/01/27]% v2.1e Refers to last page's name (HMM; JPG)]
\renewcommand{\@evenfoot}{{%
 \normalsize\slshape \today\hfil \upshape %
 page \thepage{} of \pageref{LastPage}}}
\renewcommand{\@oddfoot}{\@evenfoot}
\makeatother
\listfiles
\begin{document}
\pagenumbering{Roman}
\section*{Example for lastpage}
\markboth{Example for lastpage}{Example for lastpage}
This example demonstrates the use of package\newline
\textsf{lastpage}, v2.1e as of 2025-01-27 (HMM; JPG).\newline
The package takes no options.\newline
For more details please see the documentation!\newline

\noindent \label{keys} To hide the \pageref{keys}{\qquad } use option
\texttt{final} instead of \texttt{draft} with the \textsf{showkeys}
package (or remove the package call from the preamble of
this document).\newline

\textbf{Hyperlinks or not:} If the \textsf{hyperref} package is loaded,
the references are also hyperlinked:\newline
\smallskip
Last page's name (LastPage): \pageref{LastPage}\newline
\noindent If the \textsf{hyperref} package is loaded, but the hyperlinks
of the references shall be suppressed, \verb|\pageref*{...}|
can be used:\newline
\smallskip
Last page's name (LastPage): \pageref*{LastPage}\newline

\textbf{Trademarks} appear throughout this example without any
trademark symbol; they are the property of their respective
trademark owner. There is no intention of infringement; the
usage is to the benefit of the trademark owner.\newline

\textbf{Tip}: Use \textit{logical page numbers}
for the display of the pdf (in Adobe Acrobat Reader 2024.005.20392:
Edit $>$ Preferences $>$ Page Display $>$
Page Content and Information: Use logical page numbers)!\newline

If you are more ambitious in respect to your aims with this package,
you might want to have a look at the \textsf{pageslts} package:\newline
\url{https://ctan.org/pkg/pageslts}.
\bigskip

\noindent The page (\verb|\thepage|): \thepage \newline
Last page's name (LastPage): \pageref{LastPage}
\newpage

\noindent The page (\verb|\thepage|): \thepage \newline
Last page's name (LastPage): \pageref{LastPage}

\bigskip

\noindent There was the question:

\begin{quote}
\begin{verbatim}
\documentclass{article}
\usepackage{hyperref}
\usepackage{lastpage}
\begin{document}
\ifnum\thepage=\pageref{LastPage} foo \else bar \fi
\end{document}
\end{verbatim}

producing the error
\textquotedblleft missing number, treated as zero\textquotedblright.
\end{quote}

\noindent \verb|\pageref| inserts a hyperlink, \verb|\pageref{LastPage}|
is not expandable and the code breaks.\newline
The code does not generally work even without hyperref.

\begin{quote}
\begin{verbatim}
\documentclass{article}
\usepackage{hyperref}
\usepackage{lastpage}
\pagenumbering{Roman}
\begin{document}
\addtocounter{page}{8}
\edef\here{\thepage}
\makeatletter
\ifx\here\lastpage@lastpage\relax foo\else bar\fi
\makeatother
\end{document}
\end{verbatim}
\end{quote}

\noindent does work (two compilations needed), because \verb|\lastpage@lastpage|
contains the name of the page, \mbox{example:}
\begin{verbatim}
Page \thepage{} is (not) page
\makeatletter\lastpage@lastpage\makeatother.
\end{verbatim}
prints:\newline
Page \thepage{} is (not) page
\makeatletter\lastpage@lastpage\makeatother.
\newline
This can be broken for example by \verb|\pagenumbering{fnsymbol}|
(because then \verb|\edef\here{\thepage}| does not work).
\newpage

\noindent The page (\verb|\thepage|): \thepage\newline
Last page's name (LastPage): \pageref{LastPage}
\bigskip

With modern \LaTeX{} it is possible to say:
\begin{quote}
\begin{verbatim}
\NeedsTeXFormat{LaTeX2e}[2024-11-01]
\documentclass{article}
\pagenumbering{fnsymbol}
\begin{document}
\addtocounter{page}{8}%
\ExplSyntaxOn%
\xdef\test{\numexpr\the\g_shipout_readonly_int +1\relax}%
\ExplSyntaxOff%
\ifnum\PreviousTotalPages=\test\relax%
This is the last page.%
\else%
This is not the last page
(or it is but \LaTeX{} needs another compilation run
to detect this).
\fi
\end{document}
\end{verbatim}
\end{quote}

\newpage
\section*{The End}
\noindent The page (\verb|\thepage|): \thepage \newline
Last page's name (LastPage): \pageref{LastPage}
\bigskip

To see the content of the \texttt{enddocument/afterlastpage}-hook
(for a recent \LaTeX-format!) use % without the \verb||, of course!
\verb|\ShowHook{enddocument/afterlastpage}|.
\end{document}
%</example>
%    \end{macrocode}
% \StopEventually{}
% \newpage
%
% \section{The implementation}
% \DescribeMacro{lastpage.sty}
% We first need to determine whether we are on \TeX~2.09 or \LaTeX2e.\newline
% (That line, which is too long for the documentation, reads:\newline
% |\def\loadlastpage{\ProvidesPackage{lastpage}[2025/01/27 v2.1e lastpage:|\linebreak
% | 2.09 or 2e? (HMM)]\relax\RequirePackage{lastpage2e}}|.)
%
%    \begin{macrocode}
%<*package>
%% Part of the "lastpage" package
%% loads either lastpage2.09.sty for TeX 2.09 or lastpage2e.sty for LaTeX 2e
%% with code from https://groups.google.com/g/comp.text.tex/c/-Qmhj1ZI4xM
\def\loadlastpage{\ProvidesPackage{lastpage}[2025/01/27 v2.1e lastpage: 2.09 or 2e? (HMM)]\relax\RequirePackage{lastpage2e}}
\begingroup \expandafter \ifx \csname documentclass\endcsname\relax
\endgroup \expandafter %%
%% This is file `lastpage209.sty',
%% generated with the docstrip utility.
%%
%% The original source files were:
%%
%% lastpage.dtx  (with options: `lastpage209')
%% 
%% This is a generated file.
%% 
%% Project: lastpage
%% Version: 2025-01-27 v2.1e
%% Info: Refers to last page's name
%% 
%% Copyright (C) 2010 - 2025 by
%%     H.-Martin Muench <Martin dot Muench at Uni-Bonn dot de>
%% Portions of code copyrighted by other people as marked.
%% 
%% This work may be distributed and/or modified under the
%% conditions of the LaTeX Project Public License, either
%% version 1.3c of this license or (at your option) any later
%% version. This version of this license is in
%%    https://www.latex-project.org/lppl/lppl-1-3c.txt
%% and the latest version of this license is in
%%    https://www.latex-project.org/lppl.txt
%% and version 1.3c or later is part of all distributions of
%% LaTeX version 2005-12-01 or later.
%% 
%% This work has the LPPL maintenance status "maintained".
%% 
%% The Current Maintainer of this work is H.-Martin Muench.
%% 
%% This package was invented by Jeffrey P. Goldberg.
%% I thought that an enhancement was needed and therefore created the pageslts package,
%% https://ctan.org/pkg/pageslts. Nevertheless, for compatibility with existing
%% documents/packages as well as for the low amount of resources needed by the
%% lastpage package (no new counter!), I updated this package.
%% Thanks go to Jeffrey P. Goldberg for allowing me to do this.
%% 
%% While this package allows for things like "Page slash thepage{} of slash pageref{LastPage}"
%% to get "Page 7 of 9" or "Page VII of IX",
%% the NUMBER of pages is nowadays available from the kernel
%% (slash @abspage@last, slash thetotalpages, slash PreviousTotalPages),
%% but with pages for example I to X and then 1 to 10, the number of pages would be "20",
%% while the name of the last page is "10". Decide what you need/want!
%% 
%% This work consists of the main source file lastpage.dtx,
%% the README, and the derived files
%%    lastpage.sty, lastpage209.sty, lastpage2e.sty,
%%    lastpageclassic.sty, lastpagemodern.sty,
%%    lastpage.pdf, lastpage.ins, lastpage.drv,
%%    lastpage-example.tex, lastpage-example.pdf.
%% 
%% In memoriam
%%  Claudia Simone Barth + 1996-01-30
%%  Tommy Muench         + 2014-01-02
%%  Hans-Klaus Muench    + 2014-08-24
%% 
%% Part of the "lastpage" package
%% FOR LaTeX 2.09 ONLY - FOR LaTeX 2e USE lastpage2e.sty
%% This is lastpage209.sty invented by Jeffrey P. Goldberg,
%% after Piet van Oostrum: Page layout in LaTeX, March 2, 2004, section 16;
%% fancyhdr.pdf; lastpage209.sty maintained by H.-Martin Muench.
\let\origenddocument=\enddocument%
\def\enddocument{\clearpage%
  {\addtocounter{page}{-1}%
   \immediate\write\@mainaux{\string\newlabel{LastPage}{{}{\thepage}}}%
  }%
  \addtocounter{page}{+1}%
  \origenddocument%
 }
\endinput
%%
%% End of file `lastpage209.sty'.

\else \endgroup \expandafter \loadlastpage
\fi
%</package>
%    \end{macrocode}
%
% \DescribeMacro{lastpage209.sty}
% If we are on \TeX~2.09 (really?!), we load the 2.09 version \pkg{lastpage209.sty}:
%
%    \begin{macrocode}
%<*lastpage209>
%% Part of the "lastpage" package
%% FOR LaTeX 2.09 ONLY - FOR LaTeX 2e USE lastpage2e.sty
%% This is lastpage209.sty invented by Jeffrey P. Goldberg,
%% after Piet van Oostrum: Page layout in LaTeX, March 2, 2004, section 16;
%% fancyhdr.pdf; lastpage209.sty maintained by H.-Martin Muench.
\let\origenddocument=\enddocument%
\def\enddocument{\clearpage%
  {\addtocounter{page}{-1}%
   \immediate\write\@mainaux{\string\newlabel{LastPage}{{}{\thepage}}}%
  }%
  \addtocounter{page}{+1}%
  \origenddocument%
 }
%</lastpage209>
%    \end{macrocode}
%
% \DescribeMacro{lastpage2e.sty}
% If |\documentclass| is known, we are in \LaTeXe{} -- but which one?
% For modern versions with \eTeX{} and hook management etc.
% we load version \pkg{lastpagemodern.sty}, otherwise
% \pkg{lastpageclassic.sty}. We start off by checking that we are
% loading into \LaTeXe{} and announcing the name and version of this package.
%
%    \begin{macrocode}
%<*lastpage2e>
%% Part of the "lastpage" package
\NeedsTeXFormat{LaTeX2e}[1994/12/01]
\ProvidesPackage{lastpage2e}[2025/01/27 v2.1e %
 Decide which 2e lastpage version to use (HMM)]
\@ifl@t@r\fmtversion{2024/06/01}{\RequirePackage{lastpagemodern}}{%
                                 \RequirePackage{lastpageclassic}}
\message{^^J}
%</lastpage2e>
%    \end{macrocode}
% \newpage
%
% \DescribeMacro{lastpageclassic.sty}
% In case of older \LaTeX-formats \pkg{lastpageclassic.sty} is loaded:
%
%    \begin{macrocode}
%<*lastpageclassic>
%% Part of the "lastpage" package
\NeedsTeXFormat{LaTeX2e}[1994/12/01]
\ProvidesPackage{lastpageclassic}[2025/01/27 v2.1e %
  Refers to last page's name (HMM; JPG)]
%% allows for things like "Page \thepage{} of \pageref{LastPage}"
%% to get "Page 7 of 9"

%    \end{macrocode}
%
% \indent For comparisons, \textquotedblleft one\textquotedblright{} is defined
% (|\@ne| does not work for this).
%
%    \begin{macrocode}
\gdef\lastpage@one{1}
%    \end{macrocode}
%
% We define |\lastpage@hyper|, |\lastpage@nameref|, |\lastpage@french|, and
% |\lastpage@LTS| to be \textquotedblleft \texttt{0}\textquotedblright{}.
%
%    \begin{macrocode}
\gdef\lastpage@hyper{0}
\gdef\lastpage@nameref{0}
\gdef\lastpage@french{0}
\gdef\lastpage@LTS{0}
%    \end{macrocode}
%
% We define |\lastpage@firstpage| to be \textquotedblleft \texttt{1}\textquotedblright{},
% and before re-definition via the \xfile{.aux} file,
% |\lastpage@lastpage| and |\lastpage@lastpageHy| are unknown.
%
%    \begin{macrocode}
\gdef\lastpage@firstpage{1}
\gdef\lastpage@lastpage{??}
\gdef\lastpage@lastpageHy{??}

%    \end{macrocode}
%
% \begin{macro}{\AtBeginDocument}
% \indent |\AtBeginDocument| we give a warning about ancient versions of the \pkg{endfloat} package.
% Then it is checked whether various packages are loaded.
% (|\@ifpackageloaded| cannot be used later than |\AtBeginDocument|.)
% If this is the case,\newline
% |\lastpage@<package abbreviation>| is defined as \texttt{1}
% (otherwise it stays~\texttt{0}).
%
%    \begin{macrocode}
\AtBeginDocument{%
  \@ifpackageloaded{endfloat}{%
    \@ifpackagelater{endfloat}{1993/04/30}{\relax}{%
      \PackageError{lastpage}{%
        Incompatibility with outdated version of endfloat package}{%
        lastpage is not fully compatible with a version\MessageBreak%
        before 2.1 of the endfloat package,\MessageBreak%
        because those versions redefined\MessageBreak%
        the \string\enddocument\space command.}%
    }}{}%
  \@ifpackageloaded{tikz}{\gdef\lastpage@tikz{1}}{}%
  \@ifpackageloaded{hyperref}{\gdef\lastpage@hyper{1}}{}%
  \@ifpackageloaded{nameref}{\gdef\lastpage@nameref{1}}{}%
  \@ifpackageloaded{french}{\gdef\lastpage@french{1}}{}%
  \@ifpackageloaded{frenchle}{\gdef\lastpage@french{1}}{}%
  \@ifpackageloaded{pagesLTS}{\gdef\lastpage@LTS{1}}{}%
  \@ifpackageloaded{pageslts}{\gdef\lastpage@LTS{1}}{}%
%    \end{macrocode}
%
% |\lastpage@putlabel|, used by older versions of this package,
% is redefined e.\,g.\ by \xpackage{revtex}, \xpackage{frenchle},
% \xpackage{PPRcorners}, and old versions of \xpackage{hyperref}.
% While now |\lastpage@putl@bel| is used instead, \xpackage{revtex}
% could also define a label \texttt{LastPage},
% which then would be multiply defined. (Which is no big issue,
% if it is associated with the same page.) Therefore we define
%
%    \begin{macrocode}
  \gdef\lastpage@putlabel{\relax}%
  }

%    \end{macrocode}
%
% \noindent Because |\lastpage@putlabel| might be (re)defined later,
% depending on the order in which the packages are loaded,
% we will do this again |\AtEndDocument|.
% \end{macro}
%
% \begin{macro}{\lastpage@putl@bel}
% \indent This command does the writing of the label:
%
%    \begin{macrocode}
\newcommand{\lastpage@putl@bel}{%
%    \end{macrocode}
%
% |\AtBeginDocument| it is checked whether the \xpackage{hyperref} package is loaded,\newline
% |\@ifpackageloaded{hyperref}{\gdef\lastpage@hyper{1}}{}|.\newline
% |\@ifpackageloaded| cannot be used later than |\AtBeginDocument|.\newline
% User \textsc{Sebastian Bank} found and reported (Thanks!) a~case, when this check is not
% sufficient. Using a class with\newline
% |\usepackage{lastpage}|\newline
% |\AtBeginDocument{\usepackage{hyperref}}|\newline
% leads to failed detection of the \xpackage{hyperref} package, because
% |\AtBeginDocument| \textit{first} the check for \xpackage{hyperref} is performed,
% and \textit{then} \xpackage{hyperref} is loaded. As mentioned above,
% |\@ifpackageloaded| cannot be used later, so here we do not check for the
% \xpackage{hyperref} package again, but for its |\Hy@Warning| command.
% In version~1.2c of the \xpackage{lastpage} package, it was checked for
% the |\hyperref| command, but as it turned out, \xpackage{tcilatex} \textit{is}
% defining that. If some other package or user is defining |\Hy@Warning|,
% \xpackage{lastpage} will falsely assume, that \xpackage{hyperref} has been loaded,
% but in my humble opinion, defining |\Hy@Warning| does not make sense and
% is bad style (except definition by the \xpackage{hyperref} package itself,
% of course).
%
%    \begin{macrocode}
  \@ifundefined{Hy@Warning}{%  hyperref not loaded
    }{\gdef\lastpage@hyper{1}% hyperref loaded
     }%
%    \end{macrocode}
%
% If the \xpackage{pageslts} package is used, this \xpackage{lastpage} package is
% not needed at all. The \xpackage{LastPage} label would even be defined twice.
% Thus, if \xpackage{pageslts} is used, here nothing is done:
%
%    \begin{macrocode}
  \ifx\lastpage@LTS\lastpage@one%
  \else%
%    \end{macrocode}
%
% Otherwise the label is set:\newline
% We have got to distinguish whether \xpackage{hyperref} has been loaded or not:
%
%    \begin{macrocode}
    \ifx\lastpage@hyper\lastpage@one%
      \lastpage@putlabelhyper%
    \else%
%    \end{macrocode}
%
% and also need to treat documents with \xpackage{nameref} differently:
%
%    \begin{macrocode}
      \ifx\lastpage@nameref\lastpage@one%
        \lastpage@putlabelNR%
      \else%
%    \end{macrocode}
%
% When those packages have not been loaded, we just write the
% simple label into the \xfile{aux} file (and store the value of the page):
%
%    \begin{macrocode}
        \begingroup%
          \addtocounter{page}{-1}%
          \immediate\write\@auxout{\string\newlabel{LastPage}{{}{\thepage}}}%
          \immediate\write\@auxout{\string\xdef\string\lastpage@lastpage{\thepage}}%
          \immediate\write\@auxout{\string\gdef\string\lastpage@lastpageHy{}}%
          \addtocounter{page}{+1}%
        \endgroup%
      \fi%
    \fi%
  \fi%
  }

%    \end{macrocode}
% \end{macro}
% \newpage
%
% \begin{macro}{\lastpage@putlabelhyper}%
% \indent When \xpackage{hyperref} has been loaded, the label is set with the
% |\lastpage@putlabelhyper| command. If the \xpackage{hyperref} package is used,
% but page-anchors are disabled, the hyperlinking will not work. (The warning will
% also be shown, when only |\pageref*{LastPage}| is used (or neither one),
% but without messing with |\pageref| we cannot detect this.)
%
%    \begin{macrocode}
\newcommand{\lastpage@putlabelhyper}{%
  \ifHy@pageanchor%
  \else%
    \PackageWarningNoLine{lastpage}{%
      The \string\pageref{LastPage} link does not work\MessageBreak%
      using hyperref with disabled option `pageanchor'.\MessageBreak%
      Better enable `pageanchor' or use\MessageBreak%
      \string\pageref*{LastPage} (not generating a link)%
      }%
  \fi%
%    \end{macrocode}
%
% Since the page has been put out, we are on the page \textit{after} that page.
% We therefore subtract one from the page counter. (For the compiler,
% this is equal to |\advance\c@page\m@ne|, but for human readers of the code
% it is probably easier to understand.)
%
%    \begin{macrocode}
  \begingroup%
    \addtocounter{page}{-1}%
%    \end{macrocode}
%
% Simply using |\label| for \texttt{LastPage} would not work,
% because labels wait for the output routines to work, and there
% may be no more invocations of the output routines. To force
% the write out, we need to do an |\immediate| write into the \xfile{aux} file.
%
%    \begin{macrocode}
%% with code from \Hy@EveryPageAnchor of the hyperref package,
%% 2010/04/17 v6.80x; newer versions are available
    \let\@number\@firstofone%
    \ifHy@pageanchor%
      \ifHy@hypertexnames%
        \ifHy@plainpages%
          \def\Hy@temp{\arabic{page}}%
        \else%
          \Hy@unicodefalse%
          \ifnum \value{page}=1\relax%
%    \end{macrocode}
%
% We do not count the pages ourselves, and so they could have been changed by
% e.\,g.\ |\pagenumbering{...}|, |\addtocounter{page}{...}|,\newline
% |\setcounter{page}{...}|. Thus the page might have the number one
% while not being the first page at all. Using the \xpackage{everyshi}
% package would help, but this package should not require other packages.
% The \xpackage{pageslts} package does a better handling.\newline
% We will make a mistake here at most once:
%
%    \begin{macrocode}
            \ifx \lastpage@firstpage\lastpage@one\relax%
              \def\Hy@temp{\thepage}%
              \gdef\lastpage@firstpage{0}%
            \else%
                \pdfstringdef\Hy@temp{\thepage}%
          \fi%
          \else%
            \pdfstringdef\Hy@temp{\thepage}%
          \fi%
        \fi%
      \else%
        \def\Hy@temp{\the\Hy@pagecounter}%
      \fi%
    \fi%
    \immediate\write\@auxout{%
      \string\newlabel{LastPage}{{}{\thepage}{}{%
          \ifHy@pageanchor page.\Hy@temp\fi}{}}%
    }%
%    \end{macrocode}
%
% We also save the values, so that we can later (next rerun) check,
% whether they have been saved in the \xfile{aux} file.
%
%    \begin{macrocode}
    \immediate\write\@auxout{\string\xdef\string\lastpage@lastpage{\thepage}}%
    \ifHy@pageanchor%
      \immediate\write\@auxout{\string\xdef\string\lastpage@lastpageHy{\Hy@temp}}%
    \else%
      \immediate\write\@auxout{\string\gdef\string\lastpage@lastpageHy{}}%
    \fi%
%    \end{macrocode}
%
% After the writeout we restore the page number again,
% since there might be other things still to be done.
%
%    \begin{macrocode}
    \addtocounter{page}{+1}%
  \endgroup%
  }

%    \end{macrocode}
% \end{macro}
%
% \begin{macro}{\lastpage@putlabelNR}
% \indent The \xpackage{nameref} package redefines |\label| to have five arguments
% instead of two, therefore\newline
% |\newlabel{LastPage}{{}{\thepage}{}{}{}}|
% instead of\newline
% |\newlabel{LastPage}{{}{\thepage}}| must be used:
%
%    \begin{macrocode}
\newcommand{\lastpage@putlabelNR}{%
  \begingroup%
    \addtocounter{page}{-1}%
    \immediate\write\@auxout{\string\newlabel{LastPage}{{}{\thepage}{}{}{}}}%
    \immediate\write\@auxout{\string\xdef\string\lastpage@lastpage{\thepage}}%
    \immediate\write\@auxout{\string\gdef\string\lastpage@lastpageHy{}}%
    \addtocounter{page}{+1}%
  \endgroup%
  }

%    \end{macrocode}
% \end{macro}
%
% \begin{macro}{\lastpage@fileswtest}
% \indent Later it will be determined whether it is allowed to write
% to the \xfile{aux} file. If it was \emph{not} allowed, it is checked
% whether the label was already set via the \xfile{aux} file in some
% earlier compilation run. (There are packages where the document
% is compiled with access to the \xfile{aux} file, and then there is
% an additional compiler run, where the \xfile{aux} file cannot be changed,
% but in that run there is also no need to change it.) The \xpackage{tikz}
% package is somewhat different, therefore we only give a warning instead
% of an error (and hope that there is another compiler run where the
% \xfile{aux} file can be written).
%
%    \begin{macrocode}
\newcommand{\lastpage@fileswtest}[2]{%
  \edef\lastpage@testa{#1}%
  \edef\lastpage@testb{#2}%
  \ifx\lastpage@testa\lastpage@testb%
  \else%
    \ifx\lastpage@tikz\lastpage@one\relax%
      \PackageWarning{lastpage}%
       {The lastpage package was not allowed to write to an\MessageBreak%
        .aux file. This package does not work without access\MessageBreak%
        to an .aux file.\MessageBreak%
        It is OK if the .aux file was already updated\MessageBreak%
        by a previous compiler run\MessageBreak%
        and would not have changed anyway.\MessageBreak%
       }%
    \else%
      \PackageError{lastpage}{No auxiliary file allowed}%
       {The lastpage package was not allowed to write to an .aux file.\MessageBreak%
        This package does not work without access to an .aux file.\MessageBreak%
        Press Ctrl+Z to exit.\MessageBreak%
        But it is OK if the .aux file was already updated\MessageBreak%
        by a previous compiler run\MessageBreak%
        and would not have changed anyway.}%
    \fi%
  \fi%
  }

%    \end{macrocode}
% \end{macro}
%
% \begin{macro}{\lastpage@fileswtestHy}
% \indent When the \xpackage{hyperref} package has been loaded,
% |\lastpage@lastpageHy| must be tested additionally. (And a
% |\newcommand| is needed, because |\ifHy@pageanchor| is not even
% defined when \xpackage{hyperref} has not been loaded.)
%
%    \begin{macrocode}
\newcommand{\lastpage@fileswtestHy}{%
  \ifHy@pageanchor%
    \lastpage@fileswtest{\Hy@temp}{\lastpage@lastpageHy}%
  \else%
    \lastpage@fileswtest{\empty}{\lastpage@lastpageHy}%
  \fi%
  }
%    \end{macrocode}
% \end{macro}
%
% \begin{macro}{\AtEndDocument}
% \indent |\AtEndDocument| we again (re)define |\lastpage@putlabel|
% to do nothing and check |\lastpage@lastpage|,
% whether it is still unchanged, which is OK for the first run only.
%
%    \begin{macrocode}

\AtEndDocument{%
  \ifx\lastpage@LTS\lastpage@one%
  \else%
    \gdef\lastpage@putlabel{??}%
    \ifx\lastpage@lastpage\lastpage@putlabel\relax%
      \PackageWarning{lastpage}{Rerun to get the references right}%
    \fi%
  \fi%
  \gdef\lastpage@putlabel{\relax}%
%    \end{macrocode}
%
% It is checked whether writing to files is allowed
% (otherwise, only an error message is issued and nothing is done).
%
%    \begin{macrocode}
  \if@filesw%
%    \end{macrocode}
%
% We put in a |\message| to show, in what order things (which were called)
% are done (see subsection~\ref{ssec:aed}).
%
%    \begin{macrocode}
    \message{^^JAED: lastpage setting LastPage^^J}%
%    \end{macrocode}
%
% After this we issue a |\clearpage| to put out all floats,
% which are still floating, and place the \texttt{LastPage} label.
% Sometimes |\clearpage| might be undefined.
%
%    \begin{macrocode}
    \@ifundefined{clearpage}{\relax}{\clearpage}%
    \ifx\lastpage@french\lastpage@one% french or frenchle loaded
      \addtocounter{page}{+1}\lastpage@putl@bel\addtocounter{page}{-1}%
    \else% neither one loaded
      \lastpage@putl@bel%
    \fi%
%    \end{macrocode}
%
% When writing to files is not allowed, nothing can be done. But when
% the label was already set via the \xfile{aux} file, nothing needs
% to be done. We check for this with |\lastpage@fileswtest| and
% (if \xpackage{hyperref} has been loaded) |\lastpage@fileswtestHy|.
%
%    \begin{macrocode}
  \else%
    \ifx\lastpage@LTS\lastpage@one%
    \else%
      \lastpage@fileswtest{\thepage}{\lastpage@lastpage}%
      \ifx\lastpage@hyper\lastpage@one%
        \lastpage@fileswtestHy%
      \fi%
    \fi%
  \fi%
  }
%    \end{macrocode}
% \end{macro}
%    \begin{macrocode}
%</lastpageclassic>
%    \end{macrocode}
% \newpage
%
% \DescribeMacro{lastpagemodern.sty}
% In case of a recent \LaTeX-format, \pkg{lastpagemodern.sty} is loaded:
%
%    \begin{macrocode}
%<*lastpagemodern>
%% Part of the "lastpage" package
\NeedsTeXFormat{LaTeX2e}[2024-11-01]
\ProvidesPackage{lastpagemodern}[2025-01-27 v2.1e %
  Refers to last page's name (HMM; JPG)]
%% allows for things like "Page \thepage{} of \pageref{LastPage}"
%% to get "Page 7 of 9" or "Page VII of IX";
%% the NUMBER of pages is available via \@abspage@last, but with pages
%% for example I to X and then 1 to 10, the number of pages would be "20",
%% while the name of the last page is "10". Decide what you need/want!

%    \end{macrocode}
%
% \indent For comparisons, \textquotedblleft one\textquotedblright{} is defined
% (|\@ne| does not work for this).
%
%    \begin{macrocode}
\gdef\lastpage@one{1}
%    \end{macrocode}
%
% We define |\lastpage@firstpage| to be \textquotedblleft \texttt{1}\textquotedblright{},
% and before re-definition via the \xfile{.aux} file,
% |\lastpage@lastpage| and |\lastpage@lastpageHy| are unknown.
%
%    \begin{macrocode}
\gdef\lastpage@firstpage{1}
\gdef\lastpage@lastpage{??}
\gdef\lastpage@lastpageHy{??}

%    \end{macrocode}
%
% \begin{macro}{\lastpage@IfNumericTF}
% |\lastpage@IfNumericTF| was needed to determine whether a page number was numeric or not.
% Although it is no longer used, I am reluctant to remove it because it may have been used
% in other places in documents.
%
%    \begin{macrocode}
\newcommand\lastpage@gobbleminus[1]{\ifx-#1\else#1\fi}
%    \end{macrocode}
% which is from \url{https://texfaq.org/FAQ-isitanum},
%    \begin{macrocode}
\newcounter{lastpagecount}
%    \end{macrocode}
% and similar to \url{https://tex.stackexchange.com/a/12811/17119}
%    \begin{macrocode}
\newcommand{\lastpage@IfNumericTF}[3]{%
\sbox\z@{\c@lastpagecount=0\lastpage@gobbleminus{#1}\relax}%
\ifdim\wd0>\z@\relax#3% is not numeric
\else#2% is numeric
\fi}

%    \end{macrocode}
% \end{macro}
%
% \DescribeMacro{\AddToHook\{begindocument/end\}}
% \indent |\lastpage@putlabel|, used by older versions of this package,
% is/was redefined by other packages.
% While now |\lastpage@putl@bel| is used instead, other packages
% could also define a label \texttt{LastPage},
% which then would be multiply defined. (Which is no big issue,
% if it is associated with the same page.) Therefore we define
%
%    \begin{macrocode}
\AddToHook{begindocument/end}{%
  \IfPackageLoadedT{pageslts}{%
    \PackageNoteNoLine{lastpage}{Packages pageslts and lastpage used.\MessageBreak%
      lastpage is not necessary when loading pageslts}%
  }%
  \gdef\lastpage@putlabel{\relax}%
  }

%    \end{macrocode}
%
% \begin{macro}{\protected@iwrite}
% \indent We need an |\immediate\protected@write|. Just |\immediate\write| had led to errors,
% for example when packages like \pkg{babel-greek} re-defined |\roman|\ %
% (thanks to \textsc{Ulrike Fischer} for the report).
%
%    \begin{macrocode}
%% Code provided by Prof. Enrico Gregorio at https://tex.stackexchange.com/a/542425
\long\def\protected@iwrite#1#2#3{%
  \begingroup%
  #2%
  \let\protect\@unexpandable@protect%
  \edef\reserved@a{\immediate\write#1{#3}}%
  \reserved@a%
  \endgroup%
  \if@nobreak\ifvmode\nobreak\fi\fi%
  }

%    \end{macrocode}
% \end{macro}
%
% \begin{macro}{\lastpage@makeHy}
% Just once we need the page from |\@currentHpage| without any \mbox{\textquotedblleft |page.|\textquotedblright :}
%    \begin{macrocode}
\newcommand{\lastpage@makeHy}{%
%    \end{macrocode}
% |\gdef\lastpage@Hy{}|, but that was already done before this command.
%    \begin{macrocode}
  \def\lastpage@Hptest{Doc-Start}%
  \ifx\lastpage@Hptest\@currentHpage\relax%
    \gdef\lastpage@Hy{\@currentHpage}%
  \else%
    \edef\lastpage@Hptest{\@currentHpage}%
    \ifx\lastpage@Hptest\empty\relax%
%    \end{macrocode}
% |\gdef\lastpage@Hy{}|, but that was already done before this command.
%    \begin{macrocode}
    \else%
      \def\lastpage@Hptest{page.}%
      \ifx\lastpage@Hptest\@currentHpage\relax
        \def\lastpage@Hptest{\csname @fnsymbol\endcsname \c@page }%
        \ifx\lastpage@Hptest\thepage\relax%
          \ifnum\value{page}=3\else%
            \PackageWarningNoLine{lastpage}{You should add a\MessageBreak
              \string\ProvideTextCommand{...}{PD1}{...}\MessageBreak%
              (see the lastpage package manual, 3.13 %
              \string\pagenumbering{fnsymbol})\MessageBreak%
              to your document's preamble}%
%    \end{macrocode}
% See \autoref{subsec:fnsymbol}: \nameref{subsec:fnsymbol}, page~\pageref{subsec:fnsymbol}.
%    \begin{macrocode}
          \fi%
        \fi%
        \PackageWarningNoLine{lastpage}{%
          \string\@currentHpage\space is\MessageBreak%
          just "page." without number,\MessageBreak%
          \string\lastpage@lastpageHy\space is now let empty}%
%    \end{macrocode}
% |\gdef\lastpage@Hy{}|, but that was already done before this command.
%    \begin{macrocode}
      \else%
%    \end{macrocode}
% |\@currentHpage| should be |page.<some number>|, |\lastpage@rmpage| removes the
% \mbox{\textquotedblleft |page.|\textquotedblright .} Next compilation run,
% |\lastpage@lastpageHy| gets defined via the \xfile{aux} file.
% If we arrived at this place, but the definition is still empty,
% then |\@currentHpage| has some unexpected content.
%    \begin{macrocode}
        \gdef\lastpage@Hy{\lastpage@rmpage{\@currentHpage}}%
        \ifx\lastpage@lastpageHy\empty\relax%
          \PackageWarningNoLine{lastpage}{%
            \string\@currentHpage\space is\MessageBreak%
            \meaning\@currentHpage\MessageBreak%
            not beginning with "page.",\MessageBreak%
            \string\lastpage@lastpageHy\space is now let empty}%
  \fi\fi\fi\fi%
  }

%    \end{macrocode}
% \end{macro}
% \begin{macro}{\lastpage@rmpage}
% |\lastpage@rmpage| removes the \mbox{\textquotedblleft |page.|\textquotedblright.}
%
%    \begin{macrocode}
%% Code provided by David Carlisle at https://tex.stackexchange.com/a/721877
\def\lastpage@rmpage#1{%
  \expandafter\xlastpage@rmpage\expanded{#1}\xlastpage@rmpage page.%
  \xlastpage@rmpage\xxlastpage@rmpage{#1}}
\def\xlastpage@rmpage #1page.#2\xlastpage@rmpage#3\xxlastpage@rmpage#4{%
  \if$\detokenize{#1}$#2%\else#4
  \fi}

%    \end{macrocode}
% |\else#4| means, that it did not start with |page.|, and whatever it is,
% we cannot use this |#4| for |\lastpage@Hy|.
% \end{macro}
%
% \begin{macro}{\lastpage@putl@bel}
% \indent This command does the writing of the label.
% If the \xpackage{hyperref} package is used, but page-anchors are disabled,
% the hyperlinking will not work. (The warning will also be shown, when only
% |\pageref*{LastPage}| is used (or neither |\pageref{LastPage}| nor |\pageref*{LastPage}|),
% but without messing with |\pageref| we cannot detect this.)
%
%    \begin{macrocode}
\newcommand{\lastpage@putl@bel}{%
  \IfPackageLoadedT{hyperref}{%
    \IfPackageAtLeastF{hyperref}{2024-10-30}{%
      \PackageError{lastpage}{hyperref package version too old}{%
        required version: 2024-10-30 or newer, found version:\MessageBreak%
        \csname ver@hyperref.sty\endcsname\MessageBreak%
        Update hyperref or use lastpageclassic.sty instead of\MessageBreak%
        lastpagemodern.sty!}}%
    \ifHy@pageanchor\else%
      \PackageWarningNoLine{lastpage}{%
        The \string\pageref{LastPage} link does not work\MessageBreak%
        using hyperref with disabled option `pageanchor'.\MessageBreak%
        Better enable `pageanchor' or use\MessageBreak%
        \string\pageref*{LastPage} (not generating a link)}%
    \fi%
   }%
  \begingroup%
%    \end{macrocode}
%
% \noindent Since the page has been put out, we are on the page \textit{after} that page.
% We therefore subtract one from the page counter. (For the compiler,
% this is equal to |\advance\c@page\m@ne|, but for human readers of the code
% it is probably easier to understand.)
%
%    \begin{macrocode}
    \addtocounter{page}{-1}%
%    \end{macrocode}
%
% If the \xpackage{pageslts} package is used, this \xpackage{lastpage} package
% is not needed at all. The \xpackage{LastPage} label would even be defined twice.
% Thus, if \xpackage{pageslts} is used, here nothing is done.
%
%    \begin{macrocode}
    \IfPackageLoadedTF{pageslts}{% then pageslts writes the label for  "LastPage".
%    \end{macrocode}
%
% \noindent Simply using |\label| for \texttt{LastPage} would not work,
% because labels wait for the output routines to work, and there
% may be no more invocations of the output routines.
% To force the write out, we need to do an |\immediate| protected write
% into the \xfile{aux} file.
%
%    \begin{macrocode}
     }{\protected@iwrite\@auxout{}{\string\newlabel{LastPage}{%
         {\@currentlabel}{\thepage}{\@currentlabelname}%
         {\IfPackageLoadedTF{hyperref}{\ifHy@pageanchor\@currentHpage\fi%
                                      }{\@currentHref}}%
         {\@kernel@reserved@label@data}}%
       }%
      }%
%    \end{macrocode}
% \noindent We also save the values, so that we can later (next rerun) check,
% whether they have been saved in the \xfile{aux} file.
%
%    \begin{macrocode}
    \protected@iwrite\@auxout{}{%
      \string\gdef\string\lastpage@lastpage{\thepage}}%
    \gdef\lastpage@Hy{}%
    \IfPackageLoadedT{hyperref}{\ifHy@pageanchor\lastpage@makeHy\fi}%
    \protected@iwrite\@auxout{}{%
      \string\gdef\string\lastpage@lastpageHy{\lastpage@Hy}}%
%    \end{macrocode}
%
% \noindent After the write-out we restore the page number again,
% since there might be other things still to be done.
%
%    \begin{macrocode}
    \addtocounter{page}{+1}%
  \endgroup%
  }

%    \end{macrocode}
% \end{macro}
%
% \begin{macro}{\lastpage@fileswtest}
% \indent \hspace*{2em}Later it will be determined whether it is allowed to write
% to the \xfile{aux} file. If it was \emph{not} allowed, it is checked
% whether the label was already set via the \xfile{aux} file in some
% earlier compilation run. (There are packages where the document
% is compiled with access to the \xfile{aux} file, and then there is
% an additional compiler run, where the \xfile{aux} file cannot be changed,
% but in that run there is also no need to change it.) The \xpackage{tikz}
% package is somewhat different, therefore we only give a warning instead
% of an error (and hope that there is another compiler run where the
% \xfile{aux} file can be written).
%
%    \begin{macrocode}
\newcommand{\lastpage@fileswtest}[2]{%
  \edef\lastpage@testa{#1}%
  \edef\lastpage@testb{#2}%
  \ifx\lastpage@testa\lastpage@testb%
  \else%
    \IfPackageLoadedTF{tikz}{%
      \PackageWarning{lastpage}%
       {The lastpage package was not allowed to write to an\MessageBreak%
        .aux file. This package does not work without access\MessageBreak%
        to an .aux file.\MessageBreak%
        It is OK if the .aux file was already updated\MessageBreak%
        by a previous compiler run\MessageBreak%
        and would not have changed anyway.\MessageBreak%
       }%
    }{\PackageError{lastpage}{No auxiliary file allowed}%
       {The lastpage package was not allowed to write to an .aux file.\MessageBreak%
        This package does not work without access to an .aux file.\MessageBreak%
        Press Ctrl+Z to exit.\MessageBreak%
        But it is OK if the .aux file was already updated\MessageBreak%
        by a previous compiler run\MessageBreak%
        and would not have changed anyway.\MessageBreak%
       }%
     }%
  \fi%
  }

%    \end{macrocode}
% \end{macro}
%
% \begin{macro}{\lastpage@fileswtestHy}
% \indent \hspace*{2em}When the \xpackage{hyperref} package has been loaded,
% |\lastpage@lastpageHy| must be tested additionally. (And a
% |\newcommand| is needed, because |\ifHy@pageanchor| is not even
% defined when \xpackage{hyperref} has not been loaded.)
%
%    \begin{macrocode}
\newcommand{\lastpage@fileswtestHy}{%
  \ifHy@pageanchor%
    \lastpage@fileswtest{\@currentHpage}{\@the@H@page}%
  \else%
    \lastpage@fileswtest{\empty}{\lastpage@lastpageHy}%
  \fi%
  }

%    \end{macrocode}
% \end{macro}
%
% \DescribeMacro{enddocument/afterlastpage}
% \indent |enddocument/afterlastpage| we again (re)define |\lastpage@putlabel|
% to do nothing, but first use it to check whether |\lastpage@lastpage|
% is still unchanged, which is OK for the first run only.
%
%    \begin{macrocode}
\AddToHook{enddocument/afterlastpage}{%
  \gdef\lastpage@putlabel{??}%
  \ifx\lastpage@lastpage\lastpage@putlabel\relax%
    \AddToHook{enddocument/info}{%
      \PackageWarning{lastpage}{Rerun to get the references right}%
      }%
  \fi%
  \gdef\lastpage@putlabel{\relax}%
%    \end{macrocode}
%
% It is checked whether writing to files is allowed
% (otherwise, only an error message is issued and nothing is done).
%
%    \begin{macrocode}
  \if@filesw%
%    \end{macrocode}
%
% We put in a |\message| to show, in what order things (which were called) are done.
%
%    \begin{macrocode}
    \message{^^Jenddocument/afterlastpage (AED): lastpage setting LastPage.^^J}%
    \IfPackageLoadedTF{french}{%
         \addtocounter{page}{+1}\lastpage@putl@bel\addtocounter{page}{-1}%
     }{\IfPackageLoadedTF{frenchle}{%
         \addtocounter{page}{+1}\lastpage@putl@bel\addtocounter{page}{-1}%
       }{\lastpage@putl@bel%
        }%
      }%
  \else%
%    \end{macrocode}
%
% When writing to files is not allowed, nothing can be done. But when
% the label was already set via the \xfile{aux} file, nothing needs
% to be done. We check for this with |\lastpage@fileswtest| and
% (if \xpackage{hyperref} has been loaded) |\lastpage@fileswtestHy|.
%
%    \begin{macrocode}
    \lastpage@fileswtest{\thepage}{\lastpage@lastpage}%
    \IfPackageLoadedT{hyperref}{\lastpage@fileswtestHy}%
  \fi%
  }
%</lastpagemodern>
%    \end{macrocode}
% \newpage
%
% \section{Installation}
% \subsection{Downloads\label{ss:Downloads}}
% Everything is available at \url{https://www.ctan.org},
% but may need additional packages themselves.
%
% \DescribeMacro{lastpage.dtx}
% For unpacking the |lastpage.dtx| file and constructing the documentation it is required:
% \begin{description}
% \item[-] \TeX Format \LaTeXe{}: \url{https://www.CTAN.org}
%
% \item[-] document class \xclass{ltxdoc}, 2024/02/08, v2.1j,
%   \url{https://ctan.org/pkg/ltxdoc}
%
% \item[-] package \xpackage{holtxdoc}, 2019/12/09, v0.30,
%   \url{https://ctan.org/pkg/holtxdoc}
% \end{description}
%
% \DescribeMacro{lastpage.sty}
% The |lastpage.sty| (i.\,e.\ each document using the \xpackage{lastpage} package)
% requires:
% \begin{description}
% \item[-] \TeX, \url{https://www.CTAN.org}
%
% \item[-] package \xpackage{lastpage}, 2025-01-27, v2.1e,
%   \url{https://ctan.org/pkg/lastpage}
% \end{description}
%
% \DescribeMacro{lastpage209.sty}
% The |lastpage209.sty| for \LaTeX2.09{} (i.\,e.\ each document using
% the \xpackage{lastpage209} package) requires:
% \begin{description}
% \item[-] \TeX-format \LaTeX{}, v2.09
%
% \item[-] package \xpackage{lastpage209}, 2025-01-27, v2.1e,
%   \url{https://ctan.org/pkg/lastpage}
% \end{description}
% and does not work with \xpackage{hyperref}, which needs \LaTeX2e{}.\smallskip
%
% \DescribeMacro{lastpage2e.sty}
% The |lastpage2e.sty| for \LaTeXe{} (i.\,e.\ each document using
% the \xpackage{lastpage2e} package) requires:
% \begin{description}
% \item[-] \TeX-format \LaTeXe{} 1994/12/01 or newer,
%   \url{https://www.CTAN.org}
%
% \item[-] package \xpackage{lastpage}, 2025-01-27, v2.1e,
%   \url{https://ctan.org/pkg/lastpage}
% \end{description}
%
% \DescribeMacro{lastpageclassic.sty}
% The |lastpageclassic.sty| for \LaTeXe{} (i.\,e.\ each document using
% the \xpackage{lastpageclassic} package) requires:
% \begin{description}
% \item[-] \TeX-format \LaTeXe{} between 1994/12/01 and 2024-05-31,\newline
%   \url{https://www.CTAN.org}
%
% \item[-] package \xpackage{lastpage}, 2025-01-27, v2.1e,
%   \url{https://ctan.org/pkg/lastpage}
% \end{description}
% and can use
% \begin{description}
% \item[-] package \xpackage{hyperref}, 2023-07-08, v7.01b,
%   \url{https://ctan.org/pkg/hyperref}\newline
%   (probably also some older and newer versions)
% \end{description}
%
% \DescribeMacro{lastpagemodern.sty}
% The |lastpagemodern.sty| for \LaTeXe{} (i.\,e.\ each document using
% the \xpackage{lastpagemodern} package) requires:
% \begin{description}
% \item[-] \TeX-format \LaTeXe{} 2024-11-01 or newer,\newline
%   \url{https://www.CTAN.org}
%
% \item[-] package \xpackage{lastpage}, 2025-01-27, v2.1e,
%   \url{https://ctan.org/pkg/lastpage}
% \end{description}
% and can use
% \begin{description}
% \item[-] package \xpackage{hyperref}, probably 2023-11-07 and newer
%   (tested with: 2024-11-05 v7.01l),
%   \url{https://ctan.org/pkg/hyperref}
% \end{description}
% \pagebreak
%
% \DescribeMacro{lastpage-example.tex}
% The |lastpage-example.tex| requires the same file as all
% documents using the \xpackage{lastpage} package, i.\,e.
% \begin{description}
% \item[-] package \xpackage{lastpage}, 2025-01-27, v2.1e,
%   \url{https://ctan.org/pkg/lastpage}\newline
%   (Well, it is the example file for this package, and because you are reading the
%    documentation for the \xpackage{lastpage} package, it can be assumed that you already
%    have some version of it -- is it the current one?)
% \end{description}
% and additionally:
% \begin{description}
% \item[-] class \xpackage{article}, 2024/06/29, v1.4n, from \xpackage{classes}:
%           \url{https://ctan.org/pkg/classes}
%
% \item[-] package \xpackage{showkeys}, 2024-05-23, v3.21,
%   \url{https://ctan.org/pkg/showkeys}
%
% \item[-] package \xpackage{hyperref}, 2024-11-05, v7.01l,
%   \url{https://ctan.org/pkg/hyperref}
% \end{description}
%
% \DescribeMacro{Münch}
% A hyperlinked list of my (other) packages can be found at\newline
% \url{https://ctan.org/author/muench-hm}.
%
% \subsection{Package, unpacking TDS}
% \paragraph{Package.} This package is available on \url{https://www.CTAN.org}.
% \begin{description}
% \item[\url{https://mirror.ctan.org/macros/latex/contrib/lastpage/lastpage.dtx}]\hspace*{0.1cm} \newline
%       The source file.
% \item[\url{https://mirror.ctan.org/macros/latex/contrib/lastpage/lastpage.pdf}]\hspace*{0.1cm} \newline
%       The documentation.
% \item[\url{https://mirror.ctan.org/macros/latex/contrib/lastpage/lastpage-example.pdf}]\hspace*{0.1cm}
%       The compiled example file, as it should look like.
% \item[\url{https://mirror.ctan.org/macros/latex/contrib/lastpage/README}]\hspace*{0.1cm} \newline
%       The README file.
% \end{description}
% There is also a |lastpage.tds.zip| available:
% \begin{description}
% \item[\url{https://mirror.ctan.org/install/macros/latex/contrib/lastpage.tds.zip}]\hspace*{0.1cm}
%       Everything in TDS compliant, compiled format.
% \end{description}
% which additionally contains\newline
% \begin{tabular}{ll}
% lastpage.ins         & The \texttt{ins}tallation file.\\
% lastpage.drv         & The \mbox{\texttt{dr}i\texttt{v}er} to generate the documentation.\\
% lastpage.sty         & The \texttt{sty}le file.\\
% lastpage209.sty      & The \texttt{sty}le file for \LaTeX2.09{} \textbf{only}.\\
% lastpage2e.sty       & The \texttt{sty}le file to determine which 2e-style to use.\\
% lastpageclassic.sty  & The \texttt{sty}le file for older \LaTeX-formats.\\
% lastpagemodern.sty   & The \texttt{sty}le file for the recent \LaTeX-format.\\
% lastpage-example.tex & The example file.
% \end{tabular}
%
% \bigskip
%
% \noindent For required other packages please see the preceding subsection.
%
% \paragraph{Unpacking.} The \xfile{.dtx} file is a self-extracting
% \docstrip{} archive. The files are extracted by running the
% \xext{.dtx} through \plainTeX :
% \begin{quote}
%   \verb|tex lastpage.dtx|
% \end{quote}
%
% \noindent About generating the documentation see paragraph~\ref{GenDoc} below.\par
%
% \paragraph{TDS.} Now the different files must be moved into
% the different directories in your installation TDS tree
% (also known as \xfile{texmf} tree):
% \begin{quote}
% \def\t{^^A
% \begin{tabular}{@{}>{\ttfamily}l@{ $\rightarrow$ }>{\ttfamily}l@{}}
%   lastpage.sty & tex/latex/lastpage.sty\\
%   lastpage209.sty & tex/latex/lastpage.sty\\
%   lastpage2e.sty & tex/latex/lastpage.sty\\
%   lastpageclassic.sty & tex/latex/lastpage.sty\\
%   lastpagemodern.sty & tex/latex/lastpage.sty\\
%   lastpage.pdf & doc/latex/lastpage.pdf\\
%   lastpage-example.tex & doc/latex/lastpage-example.tex\\
%   lastpage-example.pdf & doc/latex/lastpage-example.pdf\\
%   lastpage.dtx & source/latex/lastpage.dtx
% \end{tabular}^^A
% }^^A
% \sbox0{\t}^^A
% \ifdim\wd0>\linewidth
%   \begingroup
%     \advance\linewidth by\leftmargin
%     \advance\linewidth by\rightmargin
%   \edef\x{\endgroup
%     \def\noexpand\lw{\the\linewidth}^^A
%   }\x
%   \def\lwbox{^^A
%     \leavevmode
%     \hbox to \linewidth{^^A
%       \kern-\leftmargin\relax
%       \hss
%       \usebox0
%       \hss
%       \kern-\rightmargin\relax
%     }^^A
%   }^^A
%   \ifdim\wd0>\lw
%     \sbox0{\small\t}^^A
%     \ifdim\wd0>\linewidth
%       \ifdim\wd0>\lw
%         \sbox0{\footnotesize\t}^^A
%         \ifdim\wd0>\linewidth
%           \ifdim\wd0>\lw
%             \sbox0{\scriptsize\t}^^A
%             \ifdim\wd0>\linewidth
%               \ifdim\wd0>\lw
%                 \sbox0{\tiny\t}^^A
%                 \ifdim\wd0>\linewidth
%                   \lwbox
%                 \else
%                   \usebox0
%                 \fi
%               \else
%                 \lwbox
%               \fi
%             \else
%               \usebox0
%             \fi
%           \else
%             \lwbox
%           \fi
%         \else
%           \usebox0
%         \fi
%       \else
%         \lwbox
%       \fi
%     \else
%       \usebox0
%     \fi
%   \else
%     \lwbox
%   \fi
% \else
%   \usebox0
% \fi
% \end{quote}
% If you have a \xfile{docstrip.cfg} that configures and enables \docstrip's
% TDS installing feature, then some files can already be in the right
% place, see the documentation of \docstrip.
%
% \subsection{Refresh file name databases}
% If your \TeX~distribution (\TeX\,Live, \mikTeX, \dots) relies on
% file name databases, you must refresh these. For example, \TeX\,Live\ users
% run \verb|texhash| or \verb|mktexlsr|.
%
% \subsection{Some details for the interested}
% \paragraph{Unpacking with \LaTeX.}
% The \xfile{.dtx} chooses its action depending on the format:
% \begin{description}
% \item[\plainTeX:] Run \docstrip\ and extract the files.
% \item[\LaTeX:] Generate the documentation.
% \end{description}
% If you insist on using \LaTeX\ for \docstrip\ (really,
% \docstrip\ does not need \LaTeX ), then inform the autodetect routine
% about your intention:
% \begin{quote}
%   \verb|latex \let\install=y% \iffalse meta-comment
%
% File: lastpage.dtx
% Version: 2025-01-27 v2.1e
% Info: Refers to last page's name
%
% Copyright © 2010 - 2025 by
%    H.-Martin Münch <Martin dot Muench at Uni-Bonn dot de>
% Portions of code copyrighted by other people as marked.
%
% This package was invented by Jeffrey P. Goldberg.
% I thought that an enhancement was needed and therefore created the pageslts package,
% https://ctan.org/pkg/pageslts. Nevertheless, for compatibility with existing
% documents/packages as well as for the low amount of resources needed by the
% lastpage package (no new counter!), I updated this package.
% Thanks go to Jeffrey P. Goldberg for allowing me to do this.
%
% While this package allows for things like "Page \thepage{} of \pageref{LastPage}"
% to get "Page 7 of 9" or "Page VII of IX",
% the NUMBER of pages is nowadays available from the kernel
% (\@abspage@last, \thetotalpages, \PreviousTotalPages),
% but with pages for example I to X and then 1 to 10, the number of pages would be "20",
% while the name of the last page is "10". Decide what you need/want!
%
% This work may be distributed and/or modified under the
% conditions of the LaTeX Project Public License, either
% version 1.3c of this license or (at your option) any later
% version. This version of this license is in
%    https://www.latex-project.org/lppl/lppl-1-3c.txt
% and the latest version of this license is in
%    https://www.latex-project.org/lppl.txt
% and version 1.3c or later is part of all distributions of
% LaTeX version 2005-12-01 or later.
%
% This work has the LPPL maintenance status "maintained".
%
% The Current Maintainer of this work is H.-Martin Münch.
%
% This work consists of the main source file lastpage.dtx,
% the README, and the derived files
%    lastpage.sty, lastpage209.sty, lastpage2e.sty,
%    lastpageclassic.sty, lastpagemodern.sty,
%    lastpage.pdf, lastpage.ins, lastpage.drv,
%    lastpage-example.tex, lastpage-example.pdf.
%
% "lastpage" is available on CTAN:
% https://ctan.org/pkg/lastpage
%
% Also a TDS.ZIP file is provided that contains all the files
% already sorted in a TDS tree:
% https://mirror.ctan.org/install/macros/latex/contrib/lastpage.tds.zip
%
%<*ignore>
\begingroup
  \catcode123=1 %
  \catcode125=2 %
  \def\x{LaTeX2e}%
\expandafter\endgroup
\ifcase 0\ifx\install y1\fi\expandafter
         \ifx\csname processbatchFile\endcsname\relax\else1\fi
         \ifx\fmtname\x\else 1\fi\relax
\else\csname fi\endcsname
%</ignore>
%<*install>
\input docstrip.tex
\Msg{**********************************************************************}
\Msg{* Installation                                                       *}
\Msg{* Package: lastpage 2025-01-27 v2.1e Refers to last page's name (HMM)*}
\Msg{**********************************************************************}

\keepsilent
\askforoverwritefalse

\let\MetaPrefix\relax
\preamble

This is a generated file.

Project: lastpage
Version: 2025-01-27 v2.1e
Info: Refers to last page's name

Copyright (C) 2010 - 2025 by
    H.-Martin Muench <Martin dot Muench at Uni-Bonn dot de>
Portions of code copyrighted by other people as marked.

This work may be distributed and/or modified under the
conditions of the LaTeX Project Public License, either
version 1.3c of this license or (at your option) any later
version. This version of this license is in
   https://www.latex-project.org/lppl/lppl-1-3c.txt
and the latest version of this license is in
   https://www.latex-project.org/lppl.txt
and version 1.3c or later is part of all distributions of
LaTeX version 2005-12-01 or later.

This work has the LPPL maintenance status "maintained".

The Current Maintainer of this work is H.-Martin Muench.

This package was invented by Jeffrey P. Goldberg.
I thought that an enhancement was needed and therefore created the pageslts package,
https://ctan.org/pkg/pageslts. Nevertheless, for compatibility with existing
documents/packages as well as for the low amount of resources needed by the
lastpage package (no new counter!), I updated this package.
Thanks go to Jeffrey P. Goldberg for allowing me to do this.

While this package allows for things like "Page slash thepage{} of slash pageref{LastPage}"
to get "Page 7 of 9" or "Page VII of IX",
the NUMBER of pages is nowadays available from the kernel
(slash @abspage@last, slash thetotalpages, slash PreviousTotalPages),
but with pages for example I to X and then 1 to 10, the number of pages would be "20",
while the name of the last page is "10". Decide what you need/want!

This work consists of the main source file lastpage.dtx,
the README, and the derived files
   lastpage.sty, lastpage209.sty, lastpage2e.sty,
   lastpageclassic.sty, lastpagemodern.sty,
   lastpage.pdf, lastpage.ins, lastpage.drv,
   lastpage-example.tex, lastpage-example.pdf.

In memoriam
 Claudia Simone Barth + 1996-01-30
 Tommy Muench         + 2014-01-02
 Hans-Klaus Muench    + 2014-08-24

\endpreamble
\let\MetaPrefix\DoubleperCent

\generate{%
  \file{lastpage.ins}{\from{lastpage.dtx}{install}}%
  \file{lastpage.drv}{\from{lastpage.dtx}{driver}}%
  \usedir{tex/latex/lastpage}%
  \file{lastpage.sty}{\from{lastpage.dtx}{package}}%
  \file{lastpage209.sty}{\from{lastpage.dtx}{lastpage209}}%
  \file{lastpage2e.sty}{\from{lastpage.dtx}{lastpage2e}}%
  \file{lastpageclassic.sty}{\from{lastpage.dtx}{lastpageclassic}}%
  \file{lastpagemodern.sty}{\from{lastpage.dtx}{lastpagemodern}}%
  \usedir{doc/latex/lastpage}%
  \file{lastpage-example.tex}{\from{lastpage.dtx}{example}}%
}

\catcode32=13\relax% active space
\let =\space%
\Msg{************************************************************************}
\Msg{*}
\Msg{* To finish the installation you have to move the following}
\Msg{* file into a directory searched by TeX:}
\Msg{*}
\Msg{*  lastpage.sty, lastpage209.sty, lastpage2e.sty,}
\Msg{*  lastpageclassic.sty, and lastpagemodern.sty}
\Msg{*}
\Msg{* To produce the documentation run the file `lastpage.drv'}
\Msg{* through (pdf)LaTeX, e.g.}
\Msg{*  pdflatex lastpage.drv}
\Msg{*  makeindex -s gind.ist lastpage.idx}
\Msg{*  pdflatex lastpage.drv}
\Msg{*  makeindex -s gind.ist lastpage.idx}
\Msg{*  pdflatex lastpage.drv}
\Msg{*}
\Msg{* At least three runs are necessary e.g. to get the}
\Msg{*  references right!}
\Msg{*}
\Msg{* Happy TeXing!}
\Msg{*}
\Msg{************************************************************************}

\endbatchfile
%</install>
%<*ignore>
\fi
%</ignore>
%
% \section{The documentation driver file}
%
% The next bit of code contains the documentation driver file for
% \TeX , i.\,e., the file that will produce the documentation you
% are currently reading. It will be extracted from this file by the
% \texttt{docstrip} programme. That is, run \LaTeX{} on \texttt{docstrip}
% and specify the \texttt{driver} option when \texttt{docstrip}
% asks for options.
%
%    \begin{macrocode}
%<*driver>
\NeedsTeXFormat{LaTeX2e}[2024-11-01]
\ProvidesFile{lastpage.drv}[2025-01-27 v2.1e Refers to last page's name (HMM)]
\documentclass{ltxdoc}[2024/02/08]% v2.1j Standard LaTeX documentation class
\usepackage{holtxdoc}[2019/12/09]%  v0.30 Private additional ltxdoc support (HO)
\hypersetup{%
 pdfsubject={Refers to last page's name (HMM; JPG)},%
 pdfkeywords={LaTeX, lastpage, last page, page number, page name},%
 pdfencoding=auto,%
 pdflang={en},%
 breaklinks=true,%
 linktoc=all,%
 pdfstartview=FitH,%
 pdfpagelayout=OneColumn,%
 bookmarksnumbered=true,%
 bookmarksopen=true,%
 bookmarksopenlevel=2,%
 pdfmenubar=true,%
 pdftoolbar=true,%
 pdfwindowui=true,%
 pdfnewwindow=true%
}
\CodelineIndex
\hyphenation{every-thing}
\begin{document}
  \DocInput{lastpage.dtx}%
\end{document}
%</driver>
%    \end{macrocode}
% \fi
%
% \GetFileInfo{lastpage.drv}
%
% \begingroup
%   \def\x{\#,\$,\^,\_,\~,\ ,\&,\{,\},\%}%
%   \makeatletter
%   \@onelevel@sanitize\x
% \expandafter\endgroup
% \expandafter\DoNotIndex\expandafter{\x}
% \expandafter\DoNotIndex\expandafter{\string\ }
% \begingroup
%   \makeatletter
%     \lccode`9=32\relax
%     \lowercase{%^^A
%       \edef\x{\noexpand\DoNotIndex{\@backslashchar9}}%^^A
%     }%^^A
%   \expandafter\endgroup\x
%
% \DoNotIndex{\@auxout,\@evenfoot,\@firstofone,\@firstoftwo,\@ifl@t@r,\@ifpackagelater,\@ifpackageloaded,%
% \IfPackageAtLeastF,\IfPackageLoadedT,\IfPackageLoadedTF,%
% \@ifundefined,\@kernel@reserved@label@data,\@mainaux,\@number,\@oddfoot,%
% \addtocounter,\AddToHook,\arabic,\begin,%
% \bigskip,\clearpage,\csname,\detokenize,\documentclass,\EdefUnescapeString,\empty,\end,%
% \endcsname,\enddocument,\expanded,\ExplSyntaxOff,\ExplSyntaxOn,\fmtversion,%
% \g,\gdef,\here,\hfil,\Hy@temp,\Hy@unicodefalse,\Hy@EveryPageAnchor,%
% \hypersetup,\if,\if@filesw,\if@nobreak,\ifHy@hypertexnames,\ifHy@pageanchor,\ifHy@plainpages,%
% \ifdim,\ifnum,\ifvmode,\ifx,\immediate,\input,%
% \label,\lastpage,\lastpage-example.tex,\lastpage.dtx,\lastpage.sty,\lastpagee.sty,%
% \lastpageclassic.sty,\lastpagemodern.sty,\LaTeX,\LaTeX-kernel,\listfiles,%
% \makeatletter,\makeatother,\markboth,\meaning,\mbox,\message,\MessageBreak,%
% \NeedsTeXFormat,\newcommand,\newcounter,\newline,\newpage,\nobreak,%
% \noindent,\normalsize,\numexpr,\origenddocument,\PackageError,\PackageNoteNoLine,\PackageWarning,%
% \PackageWarningNoLine,\pagenumbering,\pageref,\pdfstringdef,\protect,\ProvidesPackage,%
% \qquad,\renewcommand,\RequirePackage,\section,\ShowHook,\slshape,\smallskip,\space,\test,\textbf,%
% \textit,\textquotedblleft,\textquotedblright,\textsf,\texttt,\the,\thepage,%
% \today,\upshape,\url,\usepackage,\value,\verb,\wd,\write,\xlastpage@rmpage,\xxlastpage@rmpage}
%
% \title{The \xpackage{lastpage} package}
% \author{H.-Martin Münch\\\xemail{Martin.Muench at Uni-Bonn.de}\\
%   invented by Jeffrey P. Goldberg\\\xemail{jeffrey+news at goldmark.org}}
% \date{2025-01-27 v2.1e}
% \maketitle
%
% \begin{abstract}
% \noindent This \LaTeX{} package puts the label \texttt{LastPage}
% at the end of the document into the \xfile{.aux} file,
% allowing the user to refer to the last page of a document.
% This might be particularly useful in places like headers or footers.~--\newline
% While this package allows for things like\newline
% \hbox{\textquotedblleft Page} \hbox{|\thepage{}|} of |\pageref{LastPage}|\textquotedblright{}
% to get \textquotedblleft Page 7 of 9\textquotedblright{} or
% \textquotedblleft Page VII of IX\textquotedblright,
% the \emph{number} of pages is nowadays available from the kernel
% (|\@abspage@last|, |\thetotalpages|, |\PreviousTotalPages|),
% but when more than one page numbering scheme is used
% (for example pages I~to X and then 1~to 10,
% thus number of pages \hbox{\textquotedblleft 20\textquotedblright ,}
% but name of the last page \hbox{\textquotedblleft 10\textquotedblright ),}
% or another package has output after this package,
% or the page numbers exceed a certain range,
% there might be problems, which can be solved by using the
% \xpackage{pageslts} package instead.
% \end{abstract}
%
% \bigskip
%
% \noindent Disclaimer for web links: The author is not responsible for any contents
% referred to in this work unless having full knowledge of illegal contents.
% If any damage occurs by the use of information presented there,
% only the author of the respective pages might be liable,
% not the one who has referred to those pages.
% \newpage
% \tableofcontents
%
% \section{Introduction}
% \indent This \LaTeX{} package puts the label \texttt{LastPage}
% (at end of the document via hook |enddocument/afterlastpage|,
% for older formats via |\AtEndDocument|, for \LaTeX2.09{} via
% redefining |\enddocument|) into the \xfile{aux} file,
% allowing the user to refer to the last page of a document via
% |\pageref{LastPage}|. This might be particularly useful
% in places like headers or footers.
%
% \bigskip
%
% This package was invented by \textbf{Jeffrey P. Goldberg},
% and is now maintained by \textsc{H.-Martin Münch}. A~big
% \textquotedblleft Thank you!\textquotedblright{} to
% \textsc{Jeffrey P. Goldberg} for granting this.
%
% \bigskip
%
% If you are more ambitious in respect to your aims with this package,
% you might want to have a look at the \xpackage{pageslts} package
% (see section~\ref{sec:Alternatives}: Alternatives).
%
% \section{Usage}
% \indent Just load the package placing
% \begin{quote}
%   |\usepackage{lastpage}|
% \end{quote}
% \noindent in the preamble of your source file
% (or |%%
%% This is file `lastpage.sty',
%% generated with the docstrip utility.
%%
%% The original source files were:
%%
%% lastpage.dtx  (with options: `package')
%% 
%% This is a generated file.
%% 
%% Project: lastpage
%% Version: 2025-01-27 v2.1e
%% Info: Refers to last page's name
%% 
%% Copyright (C) 2010 - 2025 by
%%     H.-Martin Muench <Martin dot Muench at Uni-Bonn dot de>
%% Portions of code copyrighted by other people as marked.
%% 
%% This work may be distributed and/or modified under the
%% conditions of the LaTeX Project Public License, either
%% version 1.3c of this license or (at your option) any later
%% version. This version of this license is in
%%    https://www.latex-project.org/lppl/lppl-1-3c.txt
%% and the latest version of this license is in
%%    https://www.latex-project.org/lppl.txt
%% and version 1.3c or later is part of all distributions of
%% LaTeX version 2005-12-01 or later.
%% 
%% This work has the LPPL maintenance status "maintained".
%% 
%% The Current Maintainer of this work is H.-Martin Muench.
%% 
%% This package was invented by Jeffrey P. Goldberg.
%% I thought that an enhancement was needed and therefore created the pageslts package,
%% https://ctan.org/pkg/pageslts. Nevertheless, for compatibility with existing
%% documents/packages as well as for the low amount of resources needed by the
%% lastpage package (no new counter!), I updated this package.
%% Thanks go to Jeffrey P. Goldberg for allowing me to do this.
%% 
%% While this package allows for things like "Page slash thepage{} of slash pageref{LastPage}"
%% to get "Page 7 of 9" or "Page VII of IX",
%% the NUMBER of pages is nowadays available from the kernel
%% (slash @abspage@last, slash thetotalpages, slash PreviousTotalPages),
%% but with pages for example I to X and then 1 to 10, the number of pages would be "20",
%% while the name of the last page is "10". Decide what you need/want!
%% 
%% This work consists of the main source file lastpage.dtx,
%% the README, and the derived files
%%    lastpage.sty, lastpage209.sty, lastpage2e.sty,
%%    lastpageclassic.sty, lastpagemodern.sty,
%%    lastpage.pdf, lastpage.ins, lastpage.drv,
%%    lastpage-example.tex, lastpage-example.pdf.
%% 
%% In memoriam
%%  Claudia Simone Barth + 1996-01-30
%%  Tommy Muench         + 2014-01-02
%%  Hans-Klaus Muench    + 2014-08-24
%% 
%% Part of the "lastpage" package
%% loads either lastpage2.09.sty for TeX 2.09 or lastpage2e.sty for LaTeX 2e
%% with code from https://groups.google.com/g/comp.text.tex/c/-Qmhj1ZI4xM
\def\loadlastpage{\ProvidesPackage{lastpage}[2025/01/27 v2.1e lastpage: 2.09 or 2e? (HMM)]\relax\RequirePackage{lastpage2e}}
\begingroup \expandafter \ifx \csname documentclass\endcsname\relax
\endgroup \expandafter \input{lastpage209.sty}
\else \endgroup \expandafter \loadlastpage
\fi
\endinput
%%
%% End of file `lastpage.sty'.
| if |\usepackage| is unknown).
%
% \indent For example for various draft forms it is desirable to have a
% page reference to the last page, so that e.\,g.\ page footers can
% contain something like \textquotedblleft page $N$ of $K$\textquotedblright,
% where $N$ is the current page and $K$ is the last page. Once the package
% is loaded, anywhere in the text references can be made to the label
% \texttt{LastPage}. In particular one can use the \xpackage{fancyhdr}
% or \xpackage{nccfancyhdr} package, or redefinitions of the page headings
% and footings to get a reference to the last page.
%
% \noindent In your document the code
% \begin{verbatim}
% \makeatletter
% \renewcommand{\@evenfoot}{%
%  \normalsize\slshape DRAFT \today\hfil \upshape %
%  page \thepage{} of \pageref{LastPage}}
% \renewcommand{\@oddfoot}{\@evenfoot}
% \makeatother
%\end{verbatim}
% \noindent creates footers like\smallbreak
% \textquotedblleft\mbox{\textsl{DRAFT \today}\hspace{1cm}page 7 of 9}\textquotedblright\smallbreak
% \noindent in the compiled document (cf.~the \texttt{lastpage-example} file).\newline
% If the \xpackage{hyperref} package is used, the references are hyperlinked
% to their aims. If these hyperlinks shall be suppressed, |\pageref*{...}|
% instead of |\pageref{...}| can be used.\bigbreak
%
% The \xpackage{lastpage} package does not provide the words
% \textquotedblleft page\textquotedblright{} or \textquotedblleft of\textquotedblright{},
% but e.\,g.\ the \xclass{handout} class uses \textquotedblleft of\textquotedblright{} in
% the definition of the footer. (In the \texttt{lastpage-example} also
% |\@evenfoot| is redefined, but it is not the \xpackage{lastpage} \emph{package}
% redefining this.) If you want to change \textquotedblleft page\textquotedblright{} or
% \textquotedblleft of\textquotedblright{} (e.\,g.\ to another language), you therefore
% have got to look in the used class/package(s)/preamble instead of in the
% \xpackage{lastpage} package.\bigbreak
%
% If the total \emph{number} of pages of a document is needed,
% the kernel already gives this by |\makeatletter\@abspage@last\makeatother|,
% |\thetotalpages|, and |\PreviousTotalPages| (needing at least two compiler runs).
%
% \section{Some \textsc{Warnings}\label{sec:warn}}
% \subsection{\texttt{\textbackslash AtEndDocument}\label{ssec:aed}}
% \indent {\bfseries |\AtEndDocument| is not used by the \pkg{lastpagemodern.sty}
% version of the lastpage package, requiring \LaTeX -format 2024-11-01
% or newer. Instead |\AddToHook{enddocument/afterlastpage}| is used
% and the problem does not arise.}
% \bigskip
%
% \pkg{lastpageclassic.sty} uses |\AtEndDocument| and
% \pkg{lastpage209.sty} redefines \linebreak |\enddocument|.
% The last two cases are problematic:\newline
% The output of a \LaTeXe{} run is not independent of the order
% in which the packages are loaded. It is often the case that the same
% formats for which one must put tables and figure at the end,
% are the ones in which endnotes are also required.
% If one wants to use |\AtEndDocument| here as well (as done for
% |\pageref{LastPage}|), then it is easy to get to three separate
% uses of |\AtEndDocument| (assuming one uses this for the endnotes
% as well). Clearly it is not safe for any package writer or user
% to assume that no material will follow what they put into
% |\AtEndDocument|. Therefore a message, which begins with
% \texttt{AED}, is included in every usage of |\AtEndDocument|.
% \xpackage{lastpage} uses |\AtEndDocument{|\ldots|\clearpage|\ldots|}|, thus\newline
%|\usepackage{lastpage}|\ldots|\AtEndDocument{|\textit{something}|}|
% will place \textit{something} after the |\clearpage|. To place it earlier, use\newline
% |\AtEndDocument{|\textit{something}|}|\ldots|\usepackage{lastpage}|.
% If the \textit{something} is not known before |\usepackage{lastpage}|,
% you can use for example
% \begin{verbatim}
% ...
% \def\beforeLastpageClearpage{\relax}
% \AtEndDocument{\beforeLastpageClearpage}
% \usepackage{lastpage}
% \begin{document}
% ...
% \def\beforeLastpageClearpage{\textit{something}}%
% ...
% \end{document}
% \end{verbatim}
% (might need a protected and/or expanded |\def|). When |\clearpage| leads
% to some output, |\clearpage\textit{something}| instead of |\textit{something}|
% might be wanted.
%
% \subsection{Interaction with ancient versions of the \xpackage{endfloat} package\label{sec:endfloat}}
%
% \indent {\bfseries |\AtEndDocument| is not used by the \pkg{lastpagemodern.sty}
% version of the lastpage package, requiring \LaTeX -format 2024-11-01
% or newer. Instead |\AddToHook{enddocument/afterlastpage}| is used
% and the problem does not arise.}
%
% \bigskip
% The \emph{ancient} version~2.0 (and earlier; 2.0~from 1992;
% current version at the time of updating this documentation: 2.7~from 2019)
% of the \xpackage{endfloat} package actually redefined the |\enddocument| command,
% and so interfered drastically with the \LaTeXe{} commands which make use of
% |\AtEndDocument|.
% If you want your \texttt{LastPage} to label the last page of these end floats,
% you need to load \xpackage{lastpage} after loading \xpackage{endfloat}
% (or use \texttt{VeryLastPage} from the \xpackage{pageslts} package instead).
% If, on the other hand, you \emph{want} \texttt{LastPage} to refer
% to the (not so) last page, exclusive of the floats at the end,
% then load in the reverse order. Independent from the order of
% \xpackage{lastpage} and \xpackage{endfloat}, you will still need
% a version of \xpackage{endfloat} later than 2.0 from 1992.
%
% Other \LaTeX2.09{} (!) packages also seem to like to redefine
% |\enddocument|. In addition to the old \xpackage{endfloat},
% \xpackage{harvard} comes to mind. All of these will need to be
% modified swiftly.
%
% \subsection{Page name instead of page number}
% When any page numbering scheme other than \texttt{arabic} is used
% at the page, which |\pageref{LastPage}| refers to, the \textit{name}
% and not the \textit{number} of the page is given. For example,
% \texttt{Alph} page numbering scheme and $10$ pages will give \texttt{J} instead of 10,
% \texttt{Roman} page numbering scheme and $10$ pages will give \texttt{X} instead of 10,
% and so on.\newline
% \indent (The \xpackage{pageslts} package puts |\lastpageref{LastPages}|
% (with \textbf{s} at the end) at your disposal for remediation.)
%
% \subsection{No write access to the \xfile{aux} file}
% Some packages (e.\,g.\ \xpackage{tikz} and \xpackage{selectp}) sometimes prevent
% the output to the \xfile{aux} file. In that case a warning is issued. This is
% no problem as long as there is another compilation run where the label to the
% last page can be placed via the \xfile{aux} file.
%
% \subsection[Wrong last page number with \textgreater 1 page numbering scheme]{%
%  Wrong last page number with more than one page numbering scheme}
% When more than one page numbering scheme is used,
% \texttt{LastPage} does not give the total \textbf{number} of pages
% (even if \texttt{arabic} is the page numbering scheme of that page).
% For example, for a document with VI+36 pages, it gives
% \textquotedblleft 36\textquotedblright{} as reference to the last page.
% While this is correct, the total number of pages is $42$.\newline
% If the total \emph{number} of pages of a document is needed,
% the kernel already gives this by |\makeatletter\@abspage@last\makeatother|,
% |\thetotalpages|, and |\PreviousTotalPages|.
% The \xpackage{pageslts} package puts |\lastpageref{LastPages}|
% (with \textbf{s} at the end) at your disposal for remediation,
% giving the number of pages and linking to the last page, if linking is provided
% for examaple by the \xpackage{hyperref} package.
%
% \subsection{\texttt{\textbackslash addtocounter\{page\}\{\ldots\} and %
%  \texttt{\textbackslash setcounter\{page\}\{\ldots\}}}}
% When the page number was manipulated by |\addtocounter{page}{...}| or
% |\setcounter{page}{...}|, \texttt{LastPage} does not give the total
% \textbf{number} of pages (even if \texttt{arabic} is the page numbering
% scheme of that page).\newline
% The \xpackage{pageslts} package puts |\lastpageref{LastPages}|
% (with \textbf{s} at the end) at your disposal for remediation:
% \texttt{LastPages} ignores page number manipulation. Also |\@abspage@last|,
% |\thetotalpages|, and |\PreviousTotalPages| from the kernel
% are not influenced by page number manipulation.
%
% \subsection{Page number reset by \texttt{\textbackslash pagenumbering\{\ldots\}}}
% At a page numbering change the page number is reset to one.
% Therefore \texttt{LastPage} does not give the total \textbf{number} of pages
% (even if \texttt{arabic} is the page numbering scheme of that page).
% Furthermore, now two pages have the same name.\newline
% The \xpackage{pageslts} package does not only put
% |\lastpageref{LastPages}| (with \textbf{s} at the end) at your disposal
% for remediation: \texttt{LastPages} also ignores page number manipulation.
% It furthermore offers the option |pagecontinue| to continue the
% page numbering, when |\pagenumbering{...}| is used.
%
% \subsection{Last pages of different page numbering schemes}
% |\pageref{LastPage}| refers to the (maybe not so) last page of the last
% page numbering scheme. References to the respective last page of the other
% page numbering schemes are not provided.\newline
% The \xpackage{pageslts} package does this with labels
% \texttt{pagesLTS.<numbering scheme>}, where \texttt{<numbering scheme>} is
% e.\,g.\ arabic, roman, Roman, alph, or Alph.\linebreak
% For fnsymbol please use |\lastpageref{pagesLTS.fnsymbol}| instead of\newline
% |\pageref{pagesLTS.fnsymbol}|.
%
% \subsection{Current page}
% The command |\thepage| gives the \textbf{name} of the current page
% in the current page numbering scheme, which is different from the
% current total/absolute page number e.\,g.\ with a second
% page numbering scheme, |\addtocounter{page}{...}|, or |\setcounter{page}{...}|,
% and it will not be an arabic number at all,
% if the current page numbering scheme is not arabic.\newline
% The \xpackage{pageslts} package provides the command
% |\theCurrentPage| and for the current number of pages in the current
% page numbering scheme\newline
% |\theCurrentPageLocal|. The kernel already provides the
% number of pages, which have been shipped out, as |\the\ReadonlyShipoutCounter|.
% The current page is always |ReadonlyShipoutCounter +1|.
%
% \subsection{First page}
% There is no special label at the first page.
% (This is the \textbf{last}\textsf{page} package, after all.)\newline
% The \xpackage{pageslts} package creates a label \texttt{pagesLTS.0}
% at the first page of the document.
%
% \subsection{Page counter overflow\label{ss:overflow}}
% \indent \textquotedblleft The ranges of supported counter values are more or less
% restricted. Only \cs{arabic} can be used with any counter value \TeX{} supports.
% \begin{quote}
% \catcode`\|=12 %
% \begin{tabular}{@{}l|l|l|l@{}}
% Presentation & Supported & Ignored & Error message\\
% command      & domain    & values  & `Counter too large'\\
% \hline
% \cs{arabic}
%   & \ttfamily -MAX..MAX & &\\
% \cs{roman}, \cs{Roman}
%   & \ttfamily 1..MAX & \ttfamily -MAX..0 &\\
% \cs{alph}, \cs{Alph}
%   & \ttfamily 1..26 & 0 & \ttfamily -MAX..-1, 27..MAX\\
% \cs{fnsymbol}
%   & \ttfamily 1..9 & 0 & \ttfamily -MAX..-1, 10..MAX\\
% \hline
% \end{tabular}\\[1ex]
% \texttt{MAX} = \texttt{2147483647}
% \end{quote}
% \textquotedblright\ (\xpackage{alphalph} package manual, 2019/12/09, v2.6,
% first table, p.~2).
%
% \noindent When \textit{any} page is out of that range, there will be a counter overflow.\newline
% \xpackage{lastpage} probably is not the right package to be asked
% to correct this anyway, but the \xpackage{pageslts} package
% (with appropriate options) can do this.
%
% When MAX is exceeded via
% |\setcounter{<name>}{| something greater than MAX (or smaller then $-$MAX) |}|,
% then the error
% \begin{quote}
% \begin{verbatim}
% ! Number too big.
% I can only go up to 2147483647='17777777777="7FFFFFFF,
% so I'm using that number instead of yours.
% \end{verbatim}
% \end{quote}
% \vspace{-\baselineskip}
% will arise. But if the counter has a value of $2\,147\,483\,647 = {}$MAX,
% and |\addtocounter{<name>}{+1}| is tried, no error is issued, but
% |\arabic{<name>}| prints $-2147483648$, and further |\addtocounter{<name>}{+1}|s
% give $-2147483647$, $-2147483646$ and so on.\newline
% For a counter value of $-2\,147\,483\,647 = -$MAX and |\addtocounter{<name>}{-1}|s
% after $-2147483647$ it is printed $-2147483648$, $2147483647$, $2147483646$ and so on
% (without any message in the \xfile{log} file about any possible issue).
%
% \subsection{Other packages manipulating \texttt{\textbackslash lastpage@putlabel}}
% The \xpackage{revtex4} class redefines the |\lastpage@putlabel| command
% to place a label |LastPage|.\newline
% |\lastpage@putlabel| in the \xpackage{lastpage} package was replaced by |\lastpage@putl@bel|,
% but the |LastPage| label could become defined more than once.
%
% \subsection{\texttt{\textbackslash pagenumbering\{fnsymbol\}}\label{subsec:fnsymbol}}
% When using the foot-note-symbols as page numbers,
% it can be necessary to declare in the document's preamble:
% \begin{verbatim}
% \ProvideTextCommand{\textasteriskcentered}{PD1}{*}
% \ProvideTextCommand{\textdagger}{PD1}{†}
% %\ProvideTextCommand{\textdaggerdbl}{PD1}{‡)% seems to neither work nor be necessary
% \ProvideTextCommand{\textsection}{PD1}{§}
% \ProvideTextCommand{\textparagraph}{PD1}{¶}
% \ProvideTextCommand{\textbardbl}{PD1}{‖}
% \end{verbatim}
%
% \section{Alternatives\label{sec:Alternatives}}
% There are similar packages, which do (or do not) similar things (or even more).
% As I neither know what exactly you want to accomplish when using this package
% (e.\,g.~page number vs. page name, hyperlinks or not), nor what resources
% your system has (e.\,g.~\TeX, \LaTeX2e, $\varepsilon$-\TeX{}, \LaTeX-format as recent
% as 2024-11-01 or newer), here is a list of some possible alternatives:
% \begin{description}
% \item[-]\DescribeMacro{pageslts}
%  The \xpackage{pageslts} package first started as a revision of this
%  \xpackage{lastpage} package, but I~thought that an enhancement was needed
%  to accomplish what the \xpackage{pageslts} package does. For backward
%  compatibility, a label named |LastPage| is provided.
%  Thus |\usepackage{lastpage}| can be replaced by\newline
%  |\usepackage[pagecontinue=false,alphMult=0,AlphMulti=0,|\newline
%  | fnsymbolmult=false,romanMult=false,RomanMulti=false]{pageslts}|,\newline
%  if the behaviour of the \xpackage{lastpage} package should be simulated.
%  The default options are\newline
%  |\usepackage[pagecontinue=true,alphMult=ab,AlphMulti=AB,|\newline
%  |fnsymbolmult=true,romanMult=true,RomanMulti=true]{pageslts}|.\newline
%  Benefits of \xpackage{pageslts} package (with appropriate options) are:
%  \begin{description}
%  \item[+] Labels \texttt{LastPage} (|\AddToHook{enddocument/afterlastpage}|,
%   formerly |\AtEndDocument|; same as the \xpackage{LastPage} package) and
%   \texttt{VeryLastPage} (also |\AddToHook{enddocument/afterlastpage}|,
%   but formerly\newline%
%   |\AfterLastShipout|), allowing the user to refer to the
%   (very) last page of a document.
%  \item[+] For example, when more than one page numbering scheme is used,
%    the label \mbox{\texttt{LastPage}\textbf{s}} gives the total \textit{number} of pages.
%  \item[+] At the last page of each page numbering scheme a label\newline
%   \texttt{pagesLTS.<numbering scheme>} is placed, where
%   \texttt{<numbering scheme>} is e.\,g.\ arabic, roman, Roman, alph, or Alph.
%   For fnsymbol please use |\lastpageref{pagesLTS.fnsymbol}| instead of\newline
%   |\pageref{pagesLTS.fnsymbol}|.
%  \item[+] When the same numbering scheme is used twice, the page numbers
%   are either reset to one or continued automatically, depending on the option
%   given when the package is called.
%  \item[+] The command |\theCurrentPage| prints the current total/absolute
%   page number -- in contrast to |\thepage|, which gives only the page
%   \textit{name} in the current page numbering scheme.
%   |\theCurrentPageLocal| gives the current number of pages in the current
%   page numbering scheme. |\thepage| and |\theCurrentPageLocal| are different
%   e.\,g.\ when |\addtocounter{page}{...}| or |\setcounter{page}{...}| were used.
%  \item[+] At the first page of the document a label \texttt{pagesLTS.0} is created.
%  \item[+] The \xpackage{alphalph} package is supported, i.\,e.\ %
%   page numbers alph or Alph $>26$ and fnsymbol $>9$ can be used
%   (with according options set). Even zero and negative page numbers can be used
%   with \texttt{arabic}, \texttt{alph}, \texttt{Alph}, \texttt{roman}, \texttt{Roman},
%   and \texttt{fnsymbol} page numbering (with \xpackage{alphalph} package and
%   according options).
%  \end{description}
%  Further labels are provided for special cases.\newline
%  \url{https://ctan.org/pkg/pageslts}
%
% \item[-]\DescribeMacro{LaTeX-kernel}
%  The number of pages is nowadays available via |\@abspage@last|,\linebreak
%  |\thetotalpages|, and |\PreviousTotalPages| from the kernel,
%  but when more than one page numbering scheme is used
%  (for example pages I~to X and then 1~to 10, thus number of pages
%  \hbox{\textquotedblleft 20\textquotedblright ,} but name of the last page
%  \hbox{\textquotedblleft 10\textquotedblright ),}
%  or when or the fnsymbol page numbering scheme is used, or another package
%  has output after this package, or the page numbers exceed a certain range,
%  there might be issues. (Is the total number of pages wanted? Or is the name
%  of the last page sought?)\newline
%  |\the\ReadonlyShipoutCounter| contains the number of currently shipped out pages,
%  i.\,e.\ current page minus one.
%
% \item[-]\DescribeMacro{totpages}
%  The \xpackage{totpages} package provides a \texttt{totpages} label similar to
%  \texttt{LastPages}, but |\AtEndDocument| instead of hook |enddocument/afterlastpage|
%  of the \pkg{pageslts} package.
%  The \xpackage{totpages} package additionally computes the number of paper sheets
%  needed to (double) print the document (with one, two, three,\ldots{} pages on
%  one sheet of paper) (which can be achieved also with the \xpackage{papermas} package,
%  an extension of the \xpackage{pageslts} package, which further allows to compute
%  the mass of that printed version of the document, useful e.\,g.\ when sending it
%  by mail to determine the postage).\newline
%  \url{https://ctan.org/pkg/totpages}
%
% \item[-]\DescribeMacro{totalcount}
%  The \xpackage{totalcount} package provides |\totalpages|.
%  If there are only arabic page numbers consecutively running from 1 to the last page,
%  this works. But for example
%
% \begin{quote}
%\begin{verbatim}
%\documentclass{article}
%\usepackage[page]{totalcount}
%\pagenumbering{Roman}
%\begin{document}
%\addtocounter{page}{49}
%Page \thepage{} of \totalpages
%\end{document}
%\end{verbatim}
% \end{quote}
%
%  prints \textquotedblleft Page L of 50\textquotedblright,
%  where the number of pages is one (and no hyperlink is provided to the
%  last page even if \xpackage{hyperref} is used).\newline
%  \url{https://ctan.org/pkg/totalcount}
%
% \item[-]\DescribeMacro{totcount}
%  The \xpackage{totcount} package provides the last value of a counter, thus also the value
%  of the \texttt{page} counter. You do not get a hyperlink to the last page, only the numerical
%  value of the last page name is given \mbox{(i.\,e.~X+72} pages gives 72 instead of 82
%  as total number of pages), and the number of pages can be changed for example
%  by |\addtocounter|.\newline
%  \url{https://ctan.org/pkg/totcount}
% \pagebreak
%
% \item[-]\DescribeMacro{nofm}
%  \textquotedblleft There is a package \xpackage{nofm.sty} available, but some versions
%  of it are defective, and most don't work with \xpackage{fancyhdr} because they take over
%  the complete page layout.\textquotedblright{} (\textsc{Piet van Oostrum}:
%  Page layout in \LaTeX , March~2, 2004, section~16; fancyhdr.pdf)\newline
%  \xpackage{nofm} as of 1991/02/25 (without version number), available at\newline
%  \url{https://mirror.ctan.org/obsolete/macros/latex209/contrib/misc/nofm.sty},\newline
%  does not work with e.\,g.\ \xpackage{hyperref}, redefines |\enddocument|
%  as well as |\@oddhead|, |\@evenhead|, |\@oddfoot|, and |\@evenfoot|.\newline
%  If you know the (\url{https://CTAN.org}) location of a
%  \textbf{working}~(!) version, please send me an e-mail, thanks!
%
% \item[-]\DescribeMacro{count1to}
%  The \xpackage{count1to} package \textquotedblleft sets |\count1| to |\count8|
%  with the values of page to subparagraph. |\count9| is used to flag odd
%  pages. \ldots\ [T]he code for the TotalPages
%  label\textquotedblright\ (package manual, 2024-06-13) has been removed
%  from the current package version.\newline
%  \url{https://ctan.org/pkg/count1to}
%
% \item[-]\DescribeMacro{zref}
%  The \xpackage{zref} package \textquotedblleft implements an extensible referencing
%  system\textquotedblright\ (package manual, 2023-09-14).\newline
%  \url{https://ctan.org/pkg/zref}
%
% \item[-]\DescribeMacro{memoir}
%  The \xpackage{memoir} \emph{class} provides |\thelastpage| (page number printed
%  on last page) and |\thelastsheet| (number of pages).\newline
%  \url{https://ctan.org/pkg/memoir}
% \end{description}
%
% \bigskip
%
% \noindent (You programmed or found another alternative,
%  which is available at\newline
%  \url{https://CTAN.org}?
%  OK, send an e-mail to me with the name, location at CTAN,
%  and a short notice, and I will probably include it in the list above.)
% \newpage
%
% \section{Example}
%    \begin{macrocode}
%<*example>
\documentclass[british]{article}[2024/06/29]% v1.4n Standard LaTeX document class
\makeatletter
\usepackage[draft]{showkeys}[2024/05/23]% v3.21 Show cite and label keys (DPC, MH)
%%      Use final instead of draft to hide the keys. %%
\usepackage[pdfpagelabels=true,hyperindex=false]{hyperref}[2024-10-30]% v7.01k
\@ifpackageloaded{hyperref}{% Hypertext links for LaTeX
\hypersetup{extension=pdf,%
 plainpages=false,%
 pdflang={en},%
 pdftitle={lastpage package example},%
 pdfauthor={H.-Martin Muench},%
 pdfsubject={Example for the lastpage package},%
 pdfkeywords={LaTeX, lastpage},%
 pdfview=Fit,%
 pdfstartview=Fit,%
 pdfpagelayout=SinglePage%
}}{\usepackage{url}[2013/09/16]}%  v3.4  Verb mode for urls, etc.
\usepackage{lastpage}[2025/01/27]% v2.1e Refers to last page's name (HMM; JPG)]
\renewcommand{\@evenfoot}{{%
 \normalsize\slshape \today\hfil \upshape %
 page \thepage{} of \pageref{LastPage}}}
\renewcommand{\@oddfoot}{\@evenfoot}
\makeatother
\listfiles
\begin{document}
\pagenumbering{Roman}
\section*{Example for lastpage}
\markboth{Example for lastpage}{Example for lastpage}
This example demonstrates the use of package\newline
\textsf{lastpage}, v2.1e as of 2025-01-27 (HMM; JPG).\newline
The package takes no options.\newline
For more details please see the documentation!\newline

\noindent \label{keys} To hide the \pageref{keys}{\qquad } use option
\texttt{final} instead of \texttt{draft} with the \textsf{showkeys}
package (or remove the package call from the preamble of
this document).\newline

\textbf{Hyperlinks or not:} If the \textsf{hyperref} package is loaded,
the references are also hyperlinked:\newline
\smallskip
Last page's name (LastPage): \pageref{LastPage}\newline
\noindent If the \textsf{hyperref} package is loaded, but the hyperlinks
of the references shall be suppressed, \verb|\pageref*{...}|
can be used:\newline
\smallskip
Last page's name (LastPage): \pageref*{LastPage}\newline

\textbf{Trademarks} appear throughout this example without any
trademark symbol; they are the property of their respective
trademark owner. There is no intention of infringement; the
usage is to the benefit of the trademark owner.\newline

\textbf{Tip}: Use \textit{logical page numbers}
for the display of the pdf (in Adobe Acrobat Reader 2024.005.20392:
Edit $>$ Preferences $>$ Page Display $>$
Page Content and Information: Use logical page numbers)!\newline

If you are more ambitious in respect to your aims with this package,
you might want to have a look at the \textsf{pageslts} package:\newline
\url{https://ctan.org/pkg/pageslts}.
\bigskip

\noindent The page (\verb|\thepage|): \thepage \newline
Last page's name (LastPage): \pageref{LastPage}
\newpage

\noindent The page (\verb|\thepage|): \thepage \newline
Last page's name (LastPage): \pageref{LastPage}

\bigskip

\noindent There was the question:

\begin{quote}
\begin{verbatim}
\documentclass{article}
\usepackage{hyperref}
\usepackage{lastpage}
\begin{document}
\ifnum\thepage=\pageref{LastPage} foo \else bar \fi
\end{document}
\end{verbatim}

producing the error
\textquotedblleft missing number, treated as zero\textquotedblright.
\end{quote}

\noindent \verb|\pageref| inserts a hyperlink, \verb|\pageref{LastPage}|
is not expandable and the code breaks.\newline
The code does not generally work even without hyperref.

\begin{quote}
\begin{verbatim}
\documentclass{article}
\usepackage{hyperref}
\usepackage{lastpage}
\pagenumbering{Roman}
\begin{document}
\addtocounter{page}{8}
\edef\here{\thepage}
\makeatletter
\ifx\here\lastpage@lastpage\relax foo\else bar\fi
\makeatother
\end{document}
\end{verbatim}
\end{quote}

\noindent does work (two compilations needed), because \verb|\lastpage@lastpage|
contains the name of the page, \mbox{example:}
\begin{verbatim}
Page \thepage{} is (not) page
\makeatletter\lastpage@lastpage\makeatother.
\end{verbatim}
prints:\newline
Page \thepage{} is (not) page
\makeatletter\lastpage@lastpage\makeatother.
\newline
This can be broken for example by \verb|\pagenumbering{fnsymbol}|
(because then \verb|\edef\here{\thepage}| does not work).
\newpage

\noindent The page (\verb|\thepage|): \thepage\newline
Last page's name (LastPage): \pageref{LastPage}
\bigskip

With modern \LaTeX{} it is possible to say:
\begin{quote}
\begin{verbatim}
\NeedsTeXFormat{LaTeX2e}[2024-11-01]
\documentclass{article}
\pagenumbering{fnsymbol}
\begin{document}
\addtocounter{page}{8}%
\ExplSyntaxOn%
\xdef\test{\numexpr\the\g_shipout_readonly_int +1\relax}%
\ExplSyntaxOff%
\ifnum\PreviousTotalPages=\test\relax%
This is the last page.%
\else%
This is not the last page
(or it is but \LaTeX{} needs another compilation run
to detect this).
\fi
\end{document}
\end{verbatim}
\end{quote}

\newpage
\section*{The End}
\noindent The page (\verb|\thepage|): \thepage \newline
Last page's name (LastPage): \pageref{LastPage}
\bigskip

To see the content of the \texttt{enddocument/afterlastpage}-hook
(for a recent \LaTeX-format!) use % without the \verb||, of course!
\verb|\ShowHook{enddocument/afterlastpage}|.
\end{document}
%</example>
%    \end{macrocode}
% \StopEventually{}
% \newpage
%
% \section{The implementation}
% \DescribeMacro{lastpage.sty}
% We first need to determine whether we are on \TeX~2.09 or \LaTeX2e.\newline
% (That line, which is too long for the documentation, reads:\newline
% |\def\loadlastpage{\ProvidesPackage{lastpage}[2025/01/27 v2.1e lastpage:|\linebreak
% | 2.09 or 2e? (HMM)]\relax\RequirePackage{lastpage2e}}|.)
%
%    \begin{macrocode}
%<*package>
%% Part of the "lastpage" package
%% loads either lastpage2.09.sty for TeX 2.09 or lastpage2e.sty for LaTeX 2e
%% with code from https://groups.google.com/g/comp.text.tex/c/-Qmhj1ZI4xM
\def\loadlastpage{\ProvidesPackage{lastpage}[2025/01/27 v2.1e lastpage: 2.09 or 2e? (HMM)]\relax\RequirePackage{lastpage2e}}
\begingroup \expandafter \ifx \csname documentclass\endcsname\relax
\endgroup \expandafter %%
%% This is file `lastpage209.sty',
%% generated with the docstrip utility.
%%
%% The original source files were:
%%
%% lastpage.dtx  (with options: `lastpage209')
%% 
%% This is a generated file.
%% 
%% Project: lastpage
%% Version: 2025-01-27 v2.1e
%% Info: Refers to last page's name
%% 
%% Copyright (C) 2010 - 2025 by
%%     H.-Martin Muench <Martin dot Muench at Uni-Bonn dot de>
%% Portions of code copyrighted by other people as marked.
%% 
%% This work may be distributed and/or modified under the
%% conditions of the LaTeX Project Public License, either
%% version 1.3c of this license or (at your option) any later
%% version. This version of this license is in
%%    https://www.latex-project.org/lppl/lppl-1-3c.txt
%% and the latest version of this license is in
%%    https://www.latex-project.org/lppl.txt
%% and version 1.3c or later is part of all distributions of
%% LaTeX version 2005-12-01 or later.
%% 
%% This work has the LPPL maintenance status "maintained".
%% 
%% The Current Maintainer of this work is H.-Martin Muench.
%% 
%% This package was invented by Jeffrey P. Goldberg.
%% I thought that an enhancement was needed and therefore created the pageslts package,
%% https://ctan.org/pkg/pageslts. Nevertheless, for compatibility with existing
%% documents/packages as well as for the low amount of resources needed by the
%% lastpage package (no new counter!), I updated this package.
%% Thanks go to Jeffrey P. Goldberg for allowing me to do this.
%% 
%% While this package allows for things like "Page slash thepage{} of slash pageref{LastPage}"
%% to get "Page 7 of 9" or "Page VII of IX",
%% the NUMBER of pages is nowadays available from the kernel
%% (slash @abspage@last, slash thetotalpages, slash PreviousTotalPages),
%% but with pages for example I to X and then 1 to 10, the number of pages would be "20",
%% while the name of the last page is "10". Decide what you need/want!
%% 
%% This work consists of the main source file lastpage.dtx,
%% the README, and the derived files
%%    lastpage.sty, lastpage209.sty, lastpage2e.sty,
%%    lastpageclassic.sty, lastpagemodern.sty,
%%    lastpage.pdf, lastpage.ins, lastpage.drv,
%%    lastpage-example.tex, lastpage-example.pdf.
%% 
%% In memoriam
%%  Claudia Simone Barth + 1996-01-30
%%  Tommy Muench         + 2014-01-02
%%  Hans-Klaus Muench    + 2014-08-24
%% 
%% Part of the "lastpage" package
%% FOR LaTeX 2.09 ONLY - FOR LaTeX 2e USE lastpage2e.sty
%% This is lastpage209.sty invented by Jeffrey P. Goldberg,
%% after Piet van Oostrum: Page layout in LaTeX, March 2, 2004, section 16;
%% fancyhdr.pdf; lastpage209.sty maintained by H.-Martin Muench.
\let\origenddocument=\enddocument%
\def\enddocument{\clearpage%
  {\addtocounter{page}{-1}%
   \immediate\write\@mainaux{\string\newlabel{LastPage}{{}{\thepage}}}%
  }%
  \addtocounter{page}{+1}%
  \origenddocument%
 }
\endinput
%%
%% End of file `lastpage209.sty'.

\else \endgroup \expandafter \loadlastpage
\fi
%</package>
%    \end{macrocode}
%
% \DescribeMacro{lastpage209.sty}
% If we are on \TeX~2.09 (really?!), we load the 2.09 version \pkg{lastpage209.sty}:
%
%    \begin{macrocode}
%<*lastpage209>
%% Part of the "lastpage" package
%% FOR LaTeX 2.09 ONLY - FOR LaTeX 2e USE lastpage2e.sty
%% This is lastpage209.sty invented by Jeffrey P. Goldberg,
%% after Piet van Oostrum: Page layout in LaTeX, March 2, 2004, section 16;
%% fancyhdr.pdf; lastpage209.sty maintained by H.-Martin Muench.
\let\origenddocument=\enddocument%
\def\enddocument{\clearpage%
  {\addtocounter{page}{-1}%
   \immediate\write\@mainaux{\string\newlabel{LastPage}{{}{\thepage}}}%
  }%
  \addtocounter{page}{+1}%
  \origenddocument%
 }
%</lastpage209>
%    \end{macrocode}
%
% \DescribeMacro{lastpage2e.sty}
% If |\documentclass| is known, we are in \LaTeXe{} -- but which one?
% For modern versions with \eTeX{} and hook management etc.
% we load version \pkg{lastpagemodern.sty}, otherwise
% \pkg{lastpageclassic.sty}. We start off by checking that we are
% loading into \LaTeXe{} and announcing the name and version of this package.
%
%    \begin{macrocode}
%<*lastpage2e>
%% Part of the "lastpage" package
\NeedsTeXFormat{LaTeX2e}[1994/12/01]
\ProvidesPackage{lastpage2e}[2025/01/27 v2.1e %
 Decide which 2e lastpage version to use (HMM)]
\@ifl@t@r\fmtversion{2024/06/01}{\RequirePackage{lastpagemodern}}{%
                                 \RequirePackage{lastpageclassic}}
\message{^^J}
%</lastpage2e>
%    \end{macrocode}
% \newpage
%
% \DescribeMacro{lastpageclassic.sty}
% In case of older \LaTeX-formats \pkg{lastpageclassic.sty} is loaded:
%
%    \begin{macrocode}
%<*lastpageclassic>
%% Part of the "lastpage" package
\NeedsTeXFormat{LaTeX2e}[1994/12/01]
\ProvidesPackage{lastpageclassic}[2025/01/27 v2.1e %
  Refers to last page's name (HMM; JPG)]
%% allows for things like "Page \thepage{} of \pageref{LastPage}"
%% to get "Page 7 of 9"

%    \end{macrocode}
%
% \indent For comparisons, \textquotedblleft one\textquotedblright{} is defined
% (|\@ne| does not work for this).
%
%    \begin{macrocode}
\gdef\lastpage@one{1}
%    \end{macrocode}
%
% We define |\lastpage@hyper|, |\lastpage@nameref|, |\lastpage@french|, and
% |\lastpage@LTS| to be \textquotedblleft \texttt{0}\textquotedblright{}.
%
%    \begin{macrocode}
\gdef\lastpage@hyper{0}
\gdef\lastpage@nameref{0}
\gdef\lastpage@french{0}
\gdef\lastpage@LTS{0}
%    \end{macrocode}
%
% We define |\lastpage@firstpage| to be \textquotedblleft \texttt{1}\textquotedblright{},
% and before re-definition via the \xfile{.aux} file,
% |\lastpage@lastpage| and |\lastpage@lastpageHy| are unknown.
%
%    \begin{macrocode}
\gdef\lastpage@firstpage{1}
\gdef\lastpage@lastpage{??}
\gdef\lastpage@lastpageHy{??}

%    \end{macrocode}
%
% \begin{macro}{\AtBeginDocument}
% \indent |\AtBeginDocument| we give a warning about ancient versions of the \pkg{endfloat} package.
% Then it is checked whether various packages are loaded.
% (|\@ifpackageloaded| cannot be used later than |\AtBeginDocument|.)
% If this is the case,\newline
% |\lastpage@<package abbreviation>| is defined as \texttt{1}
% (otherwise it stays~\texttt{0}).
%
%    \begin{macrocode}
\AtBeginDocument{%
  \@ifpackageloaded{endfloat}{%
    \@ifpackagelater{endfloat}{1993/04/30}{\relax}{%
      \PackageError{lastpage}{%
        Incompatibility with outdated version of endfloat package}{%
        lastpage is not fully compatible with a version\MessageBreak%
        before 2.1 of the endfloat package,\MessageBreak%
        because those versions redefined\MessageBreak%
        the \string\enddocument\space command.}%
    }}{}%
  \@ifpackageloaded{tikz}{\gdef\lastpage@tikz{1}}{}%
  \@ifpackageloaded{hyperref}{\gdef\lastpage@hyper{1}}{}%
  \@ifpackageloaded{nameref}{\gdef\lastpage@nameref{1}}{}%
  \@ifpackageloaded{french}{\gdef\lastpage@french{1}}{}%
  \@ifpackageloaded{frenchle}{\gdef\lastpage@french{1}}{}%
  \@ifpackageloaded{pagesLTS}{\gdef\lastpage@LTS{1}}{}%
  \@ifpackageloaded{pageslts}{\gdef\lastpage@LTS{1}}{}%
%    \end{macrocode}
%
% |\lastpage@putlabel|, used by older versions of this package,
% is redefined e.\,g.\ by \xpackage{revtex}, \xpackage{frenchle},
% \xpackage{PPRcorners}, and old versions of \xpackage{hyperref}.
% While now |\lastpage@putl@bel| is used instead, \xpackage{revtex}
% could also define a label \texttt{LastPage},
% which then would be multiply defined. (Which is no big issue,
% if it is associated with the same page.) Therefore we define
%
%    \begin{macrocode}
  \gdef\lastpage@putlabel{\relax}%
  }

%    \end{macrocode}
%
% \noindent Because |\lastpage@putlabel| might be (re)defined later,
% depending on the order in which the packages are loaded,
% we will do this again |\AtEndDocument|.
% \end{macro}
%
% \begin{macro}{\lastpage@putl@bel}
% \indent This command does the writing of the label:
%
%    \begin{macrocode}
\newcommand{\lastpage@putl@bel}{%
%    \end{macrocode}
%
% |\AtBeginDocument| it is checked whether the \xpackage{hyperref} package is loaded,\newline
% |\@ifpackageloaded{hyperref}{\gdef\lastpage@hyper{1}}{}|.\newline
% |\@ifpackageloaded| cannot be used later than |\AtBeginDocument|.\newline
% User \textsc{Sebastian Bank} found and reported (Thanks!) a~case, when this check is not
% sufficient. Using a class with\newline
% |\usepackage{lastpage}|\newline
% |\AtBeginDocument{\usepackage{hyperref}}|\newline
% leads to failed detection of the \xpackage{hyperref} package, because
% |\AtBeginDocument| \textit{first} the check for \xpackage{hyperref} is performed,
% and \textit{then} \xpackage{hyperref} is loaded. As mentioned above,
% |\@ifpackageloaded| cannot be used later, so here we do not check for the
% \xpackage{hyperref} package again, but for its |\Hy@Warning| command.
% In version~1.2c of the \xpackage{lastpage} package, it was checked for
% the |\hyperref| command, but as it turned out, \xpackage{tcilatex} \textit{is}
% defining that. If some other package or user is defining |\Hy@Warning|,
% \xpackage{lastpage} will falsely assume, that \xpackage{hyperref} has been loaded,
% but in my humble opinion, defining |\Hy@Warning| does not make sense and
% is bad style (except definition by the \xpackage{hyperref} package itself,
% of course).
%
%    \begin{macrocode}
  \@ifundefined{Hy@Warning}{%  hyperref not loaded
    }{\gdef\lastpage@hyper{1}% hyperref loaded
     }%
%    \end{macrocode}
%
% If the \xpackage{pageslts} package is used, this \xpackage{lastpage} package is
% not needed at all. The \xpackage{LastPage} label would even be defined twice.
% Thus, if \xpackage{pageslts} is used, here nothing is done:
%
%    \begin{macrocode}
  \ifx\lastpage@LTS\lastpage@one%
  \else%
%    \end{macrocode}
%
% Otherwise the label is set:\newline
% We have got to distinguish whether \xpackage{hyperref} has been loaded or not:
%
%    \begin{macrocode}
    \ifx\lastpage@hyper\lastpage@one%
      \lastpage@putlabelhyper%
    \else%
%    \end{macrocode}
%
% and also need to treat documents with \xpackage{nameref} differently:
%
%    \begin{macrocode}
      \ifx\lastpage@nameref\lastpage@one%
        \lastpage@putlabelNR%
      \else%
%    \end{macrocode}
%
% When those packages have not been loaded, we just write the
% simple label into the \xfile{aux} file (and store the value of the page):
%
%    \begin{macrocode}
        \begingroup%
          \addtocounter{page}{-1}%
          \immediate\write\@auxout{\string\newlabel{LastPage}{{}{\thepage}}}%
          \immediate\write\@auxout{\string\xdef\string\lastpage@lastpage{\thepage}}%
          \immediate\write\@auxout{\string\gdef\string\lastpage@lastpageHy{}}%
          \addtocounter{page}{+1}%
        \endgroup%
      \fi%
    \fi%
  \fi%
  }

%    \end{macrocode}
% \end{macro}
% \newpage
%
% \begin{macro}{\lastpage@putlabelhyper}%
% \indent When \xpackage{hyperref} has been loaded, the label is set with the
% |\lastpage@putlabelhyper| command. If the \xpackage{hyperref} package is used,
% but page-anchors are disabled, the hyperlinking will not work. (The warning will
% also be shown, when only |\pageref*{LastPage}| is used (or neither one),
% but without messing with |\pageref| we cannot detect this.)
%
%    \begin{macrocode}
\newcommand{\lastpage@putlabelhyper}{%
  \ifHy@pageanchor%
  \else%
    \PackageWarningNoLine{lastpage}{%
      The \string\pageref{LastPage} link does not work\MessageBreak%
      using hyperref with disabled option `pageanchor'.\MessageBreak%
      Better enable `pageanchor' or use\MessageBreak%
      \string\pageref*{LastPage} (not generating a link)%
      }%
  \fi%
%    \end{macrocode}
%
% Since the page has been put out, we are on the page \textit{after} that page.
% We therefore subtract one from the page counter. (For the compiler,
% this is equal to |\advance\c@page\m@ne|, but for human readers of the code
% it is probably easier to understand.)
%
%    \begin{macrocode}
  \begingroup%
    \addtocounter{page}{-1}%
%    \end{macrocode}
%
% Simply using |\label| for \texttt{LastPage} would not work,
% because labels wait for the output routines to work, and there
% may be no more invocations of the output routines. To force
% the write out, we need to do an |\immediate| write into the \xfile{aux} file.
%
%    \begin{macrocode}
%% with code from \Hy@EveryPageAnchor of the hyperref package,
%% 2010/04/17 v6.80x; newer versions are available
    \let\@number\@firstofone%
    \ifHy@pageanchor%
      \ifHy@hypertexnames%
        \ifHy@plainpages%
          \def\Hy@temp{\arabic{page}}%
        \else%
          \Hy@unicodefalse%
          \ifnum \value{page}=1\relax%
%    \end{macrocode}
%
% We do not count the pages ourselves, and so they could have been changed by
% e.\,g.\ |\pagenumbering{...}|, |\addtocounter{page}{...}|,\newline
% |\setcounter{page}{...}|. Thus the page might have the number one
% while not being the first page at all. Using the \xpackage{everyshi}
% package would help, but this package should not require other packages.
% The \xpackage{pageslts} package does a better handling.\newline
% We will make a mistake here at most once:
%
%    \begin{macrocode}
            \ifx \lastpage@firstpage\lastpage@one\relax%
              \def\Hy@temp{\thepage}%
              \gdef\lastpage@firstpage{0}%
            \else%
                \pdfstringdef\Hy@temp{\thepage}%
          \fi%
          \else%
            \pdfstringdef\Hy@temp{\thepage}%
          \fi%
        \fi%
      \else%
        \def\Hy@temp{\the\Hy@pagecounter}%
      \fi%
    \fi%
    \immediate\write\@auxout{%
      \string\newlabel{LastPage}{{}{\thepage}{}{%
          \ifHy@pageanchor page.\Hy@temp\fi}{}}%
    }%
%    \end{macrocode}
%
% We also save the values, so that we can later (next rerun) check,
% whether they have been saved in the \xfile{aux} file.
%
%    \begin{macrocode}
    \immediate\write\@auxout{\string\xdef\string\lastpage@lastpage{\thepage}}%
    \ifHy@pageanchor%
      \immediate\write\@auxout{\string\xdef\string\lastpage@lastpageHy{\Hy@temp}}%
    \else%
      \immediate\write\@auxout{\string\gdef\string\lastpage@lastpageHy{}}%
    \fi%
%    \end{macrocode}
%
% After the writeout we restore the page number again,
% since there might be other things still to be done.
%
%    \begin{macrocode}
    \addtocounter{page}{+1}%
  \endgroup%
  }

%    \end{macrocode}
% \end{macro}
%
% \begin{macro}{\lastpage@putlabelNR}
% \indent The \xpackage{nameref} package redefines |\label| to have five arguments
% instead of two, therefore\newline
% |\newlabel{LastPage}{{}{\thepage}{}{}{}}|
% instead of\newline
% |\newlabel{LastPage}{{}{\thepage}}| must be used:
%
%    \begin{macrocode}
\newcommand{\lastpage@putlabelNR}{%
  \begingroup%
    \addtocounter{page}{-1}%
    \immediate\write\@auxout{\string\newlabel{LastPage}{{}{\thepage}{}{}{}}}%
    \immediate\write\@auxout{\string\xdef\string\lastpage@lastpage{\thepage}}%
    \immediate\write\@auxout{\string\gdef\string\lastpage@lastpageHy{}}%
    \addtocounter{page}{+1}%
  \endgroup%
  }

%    \end{macrocode}
% \end{macro}
%
% \begin{macro}{\lastpage@fileswtest}
% \indent Later it will be determined whether it is allowed to write
% to the \xfile{aux} file. If it was \emph{not} allowed, it is checked
% whether the label was already set via the \xfile{aux} file in some
% earlier compilation run. (There are packages where the document
% is compiled with access to the \xfile{aux} file, and then there is
% an additional compiler run, where the \xfile{aux} file cannot be changed,
% but in that run there is also no need to change it.) The \xpackage{tikz}
% package is somewhat different, therefore we only give a warning instead
% of an error (and hope that there is another compiler run where the
% \xfile{aux} file can be written).
%
%    \begin{macrocode}
\newcommand{\lastpage@fileswtest}[2]{%
  \edef\lastpage@testa{#1}%
  \edef\lastpage@testb{#2}%
  \ifx\lastpage@testa\lastpage@testb%
  \else%
    \ifx\lastpage@tikz\lastpage@one\relax%
      \PackageWarning{lastpage}%
       {The lastpage package was not allowed to write to an\MessageBreak%
        .aux file. This package does not work without access\MessageBreak%
        to an .aux file.\MessageBreak%
        It is OK if the .aux file was already updated\MessageBreak%
        by a previous compiler run\MessageBreak%
        and would not have changed anyway.\MessageBreak%
       }%
    \else%
      \PackageError{lastpage}{No auxiliary file allowed}%
       {The lastpage package was not allowed to write to an .aux file.\MessageBreak%
        This package does not work without access to an .aux file.\MessageBreak%
        Press Ctrl+Z to exit.\MessageBreak%
        But it is OK if the .aux file was already updated\MessageBreak%
        by a previous compiler run\MessageBreak%
        and would not have changed anyway.}%
    \fi%
  \fi%
  }

%    \end{macrocode}
% \end{macro}
%
% \begin{macro}{\lastpage@fileswtestHy}
% \indent When the \xpackage{hyperref} package has been loaded,
% |\lastpage@lastpageHy| must be tested additionally. (And a
% |\newcommand| is needed, because |\ifHy@pageanchor| is not even
% defined when \xpackage{hyperref} has not been loaded.)
%
%    \begin{macrocode}
\newcommand{\lastpage@fileswtestHy}{%
  \ifHy@pageanchor%
    \lastpage@fileswtest{\Hy@temp}{\lastpage@lastpageHy}%
  \else%
    \lastpage@fileswtest{\empty}{\lastpage@lastpageHy}%
  \fi%
  }
%    \end{macrocode}
% \end{macro}
%
% \begin{macro}{\AtEndDocument}
% \indent |\AtEndDocument| we again (re)define |\lastpage@putlabel|
% to do nothing and check |\lastpage@lastpage|,
% whether it is still unchanged, which is OK for the first run only.
%
%    \begin{macrocode}

\AtEndDocument{%
  \ifx\lastpage@LTS\lastpage@one%
  \else%
    \gdef\lastpage@putlabel{??}%
    \ifx\lastpage@lastpage\lastpage@putlabel\relax%
      \PackageWarning{lastpage}{Rerun to get the references right}%
    \fi%
  \fi%
  \gdef\lastpage@putlabel{\relax}%
%    \end{macrocode}
%
% It is checked whether writing to files is allowed
% (otherwise, only an error message is issued and nothing is done).
%
%    \begin{macrocode}
  \if@filesw%
%    \end{macrocode}
%
% We put in a |\message| to show, in what order things (which were called)
% are done (see subsection~\ref{ssec:aed}).
%
%    \begin{macrocode}
    \message{^^JAED: lastpage setting LastPage^^J}%
%    \end{macrocode}
%
% After this we issue a |\clearpage| to put out all floats,
% which are still floating, and place the \texttt{LastPage} label.
% Sometimes |\clearpage| might be undefined.
%
%    \begin{macrocode}
    \@ifundefined{clearpage}{\relax}{\clearpage}%
    \ifx\lastpage@french\lastpage@one% french or frenchle loaded
      \addtocounter{page}{+1}\lastpage@putl@bel\addtocounter{page}{-1}%
    \else% neither one loaded
      \lastpage@putl@bel%
    \fi%
%    \end{macrocode}
%
% When writing to files is not allowed, nothing can be done. But when
% the label was already set via the \xfile{aux} file, nothing needs
% to be done. We check for this with |\lastpage@fileswtest| and
% (if \xpackage{hyperref} has been loaded) |\lastpage@fileswtestHy|.
%
%    \begin{macrocode}
  \else%
    \ifx\lastpage@LTS\lastpage@one%
    \else%
      \lastpage@fileswtest{\thepage}{\lastpage@lastpage}%
      \ifx\lastpage@hyper\lastpage@one%
        \lastpage@fileswtestHy%
      \fi%
    \fi%
  \fi%
  }
%    \end{macrocode}
% \end{macro}
%    \begin{macrocode}
%</lastpageclassic>
%    \end{macrocode}
% \newpage
%
% \DescribeMacro{lastpagemodern.sty}
% In case of a recent \LaTeX-format, \pkg{lastpagemodern.sty} is loaded:
%
%    \begin{macrocode}
%<*lastpagemodern>
%% Part of the "lastpage" package
\NeedsTeXFormat{LaTeX2e}[2024-11-01]
\ProvidesPackage{lastpagemodern}[2025-01-27 v2.1e %
  Refers to last page's name (HMM; JPG)]
%% allows for things like "Page \thepage{} of \pageref{LastPage}"
%% to get "Page 7 of 9" or "Page VII of IX";
%% the NUMBER of pages is available via \@abspage@last, but with pages
%% for example I to X and then 1 to 10, the number of pages would be "20",
%% while the name of the last page is "10". Decide what you need/want!

%    \end{macrocode}
%
% \indent For comparisons, \textquotedblleft one\textquotedblright{} is defined
% (|\@ne| does not work for this).
%
%    \begin{macrocode}
\gdef\lastpage@one{1}
%    \end{macrocode}
%
% We define |\lastpage@firstpage| to be \textquotedblleft \texttt{1}\textquotedblright{},
% and before re-definition via the \xfile{.aux} file,
% |\lastpage@lastpage| and |\lastpage@lastpageHy| are unknown.
%
%    \begin{macrocode}
\gdef\lastpage@firstpage{1}
\gdef\lastpage@lastpage{??}
\gdef\lastpage@lastpageHy{??}

%    \end{macrocode}
%
% \begin{macro}{\lastpage@IfNumericTF}
% |\lastpage@IfNumericTF| was needed to determine whether a page number was numeric or not.
% Although it is no longer used, I am reluctant to remove it because it may have been used
% in other places in documents.
%
%    \begin{macrocode}
\newcommand\lastpage@gobbleminus[1]{\ifx-#1\else#1\fi}
%    \end{macrocode}
% which is from \url{https://texfaq.org/FAQ-isitanum},
%    \begin{macrocode}
\newcounter{lastpagecount}
%    \end{macrocode}
% and similar to \url{https://tex.stackexchange.com/a/12811/17119}
%    \begin{macrocode}
\newcommand{\lastpage@IfNumericTF}[3]{%
\sbox\z@{\c@lastpagecount=0\lastpage@gobbleminus{#1}\relax}%
\ifdim\wd0>\z@\relax#3% is not numeric
\else#2% is numeric
\fi}

%    \end{macrocode}
% \end{macro}
%
% \DescribeMacro{\AddToHook\{begindocument/end\}}
% \indent |\lastpage@putlabel|, used by older versions of this package,
% is/was redefined by other packages.
% While now |\lastpage@putl@bel| is used instead, other packages
% could also define a label \texttt{LastPage},
% which then would be multiply defined. (Which is no big issue,
% if it is associated with the same page.) Therefore we define
%
%    \begin{macrocode}
\AddToHook{begindocument/end}{%
  \IfPackageLoadedT{pageslts}{%
    \PackageNoteNoLine{lastpage}{Packages pageslts and lastpage used.\MessageBreak%
      lastpage is not necessary when loading pageslts}%
  }%
  \gdef\lastpage@putlabel{\relax}%
  }

%    \end{macrocode}
%
% \begin{macro}{\protected@iwrite}
% \indent We need an |\immediate\protected@write|. Just |\immediate\write| had led to errors,
% for example when packages like \pkg{babel-greek} re-defined |\roman|\ %
% (thanks to \textsc{Ulrike Fischer} for the report).
%
%    \begin{macrocode}
%% Code provided by Prof. Enrico Gregorio at https://tex.stackexchange.com/a/542425
\long\def\protected@iwrite#1#2#3{%
  \begingroup%
  #2%
  \let\protect\@unexpandable@protect%
  \edef\reserved@a{\immediate\write#1{#3}}%
  \reserved@a%
  \endgroup%
  \if@nobreak\ifvmode\nobreak\fi\fi%
  }

%    \end{macrocode}
% \end{macro}
%
% \begin{macro}{\lastpage@makeHy}
% Just once we need the page from |\@currentHpage| without any \mbox{\textquotedblleft |page.|\textquotedblright :}
%    \begin{macrocode}
\newcommand{\lastpage@makeHy}{%
%    \end{macrocode}
% |\gdef\lastpage@Hy{}|, but that was already done before this command.
%    \begin{macrocode}
  \def\lastpage@Hptest{Doc-Start}%
  \ifx\lastpage@Hptest\@currentHpage\relax%
    \gdef\lastpage@Hy{\@currentHpage}%
  \else%
    \edef\lastpage@Hptest{\@currentHpage}%
    \ifx\lastpage@Hptest\empty\relax%
%    \end{macrocode}
% |\gdef\lastpage@Hy{}|, but that was already done before this command.
%    \begin{macrocode}
    \else%
      \def\lastpage@Hptest{page.}%
      \ifx\lastpage@Hptest\@currentHpage\relax
        \def\lastpage@Hptest{\csname @fnsymbol\endcsname \c@page }%
        \ifx\lastpage@Hptest\thepage\relax%
          \ifnum\value{page}=3\else%
            \PackageWarningNoLine{lastpage}{You should add a\MessageBreak
              \string\ProvideTextCommand{...}{PD1}{...}\MessageBreak%
              (see the lastpage package manual, 3.13 %
              \string\pagenumbering{fnsymbol})\MessageBreak%
              to your document's preamble}%
%    \end{macrocode}
% See \autoref{subsec:fnsymbol}: \nameref{subsec:fnsymbol}, page~\pageref{subsec:fnsymbol}.
%    \begin{macrocode}
          \fi%
        \fi%
        \PackageWarningNoLine{lastpage}{%
          \string\@currentHpage\space is\MessageBreak%
          just "page." without number,\MessageBreak%
          \string\lastpage@lastpageHy\space is now let empty}%
%    \end{macrocode}
% |\gdef\lastpage@Hy{}|, but that was already done before this command.
%    \begin{macrocode}
      \else%
%    \end{macrocode}
% |\@currentHpage| should be |page.<some number>|, |\lastpage@rmpage| removes the
% \mbox{\textquotedblleft |page.|\textquotedblright .} Next compilation run,
% |\lastpage@lastpageHy| gets defined via the \xfile{aux} file.
% If we arrived at this place, but the definition is still empty,
% then |\@currentHpage| has some unexpected content.
%    \begin{macrocode}
        \gdef\lastpage@Hy{\lastpage@rmpage{\@currentHpage}}%
        \ifx\lastpage@lastpageHy\empty\relax%
          \PackageWarningNoLine{lastpage}{%
            \string\@currentHpage\space is\MessageBreak%
            \meaning\@currentHpage\MessageBreak%
            not beginning with "page.",\MessageBreak%
            \string\lastpage@lastpageHy\space is now let empty}%
  \fi\fi\fi\fi%
  }

%    \end{macrocode}
% \end{macro}
% \begin{macro}{\lastpage@rmpage}
% |\lastpage@rmpage| removes the \mbox{\textquotedblleft |page.|\textquotedblright.}
%
%    \begin{macrocode}
%% Code provided by David Carlisle at https://tex.stackexchange.com/a/721877
\def\lastpage@rmpage#1{%
  \expandafter\xlastpage@rmpage\expanded{#1}\xlastpage@rmpage page.%
  \xlastpage@rmpage\xxlastpage@rmpage{#1}}
\def\xlastpage@rmpage #1page.#2\xlastpage@rmpage#3\xxlastpage@rmpage#4{%
  \if$\detokenize{#1}$#2%\else#4
  \fi}

%    \end{macrocode}
% |\else#4| means, that it did not start with |page.|, and whatever it is,
% we cannot use this |#4| for |\lastpage@Hy|.
% \end{macro}
%
% \begin{macro}{\lastpage@putl@bel}
% \indent This command does the writing of the label.
% If the \xpackage{hyperref} package is used, but page-anchors are disabled,
% the hyperlinking will not work. (The warning will also be shown, when only
% |\pageref*{LastPage}| is used (or neither |\pageref{LastPage}| nor |\pageref*{LastPage}|),
% but without messing with |\pageref| we cannot detect this.)
%
%    \begin{macrocode}
\newcommand{\lastpage@putl@bel}{%
  \IfPackageLoadedT{hyperref}{%
    \IfPackageAtLeastF{hyperref}{2024-10-30}{%
      \PackageError{lastpage}{hyperref package version too old}{%
        required version: 2024-10-30 or newer, found version:\MessageBreak%
        \csname ver@hyperref.sty\endcsname\MessageBreak%
        Update hyperref or use lastpageclassic.sty instead of\MessageBreak%
        lastpagemodern.sty!}}%
    \ifHy@pageanchor\else%
      \PackageWarningNoLine{lastpage}{%
        The \string\pageref{LastPage} link does not work\MessageBreak%
        using hyperref with disabled option `pageanchor'.\MessageBreak%
        Better enable `pageanchor' or use\MessageBreak%
        \string\pageref*{LastPage} (not generating a link)}%
    \fi%
   }%
  \begingroup%
%    \end{macrocode}
%
% \noindent Since the page has been put out, we are on the page \textit{after} that page.
% We therefore subtract one from the page counter. (For the compiler,
% this is equal to |\advance\c@page\m@ne|, but for human readers of the code
% it is probably easier to understand.)
%
%    \begin{macrocode}
    \addtocounter{page}{-1}%
%    \end{macrocode}
%
% If the \xpackage{pageslts} package is used, this \xpackage{lastpage} package
% is not needed at all. The \xpackage{LastPage} label would even be defined twice.
% Thus, if \xpackage{pageslts} is used, here nothing is done.
%
%    \begin{macrocode}
    \IfPackageLoadedTF{pageslts}{% then pageslts writes the label for  "LastPage".
%    \end{macrocode}
%
% \noindent Simply using |\label| for \texttt{LastPage} would not work,
% because labels wait for the output routines to work, and there
% may be no more invocations of the output routines.
% To force the write out, we need to do an |\immediate| protected write
% into the \xfile{aux} file.
%
%    \begin{macrocode}
     }{\protected@iwrite\@auxout{}{\string\newlabel{LastPage}{%
         {\@currentlabel}{\thepage}{\@currentlabelname}%
         {\IfPackageLoadedTF{hyperref}{\ifHy@pageanchor\@currentHpage\fi%
                                      }{\@currentHref}}%
         {\@kernel@reserved@label@data}}%
       }%
      }%
%    \end{macrocode}
% \noindent We also save the values, so that we can later (next rerun) check,
% whether they have been saved in the \xfile{aux} file.
%
%    \begin{macrocode}
    \protected@iwrite\@auxout{}{%
      \string\gdef\string\lastpage@lastpage{\thepage}}%
    \gdef\lastpage@Hy{}%
    \IfPackageLoadedT{hyperref}{\ifHy@pageanchor\lastpage@makeHy\fi}%
    \protected@iwrite\@auxout{}{%
      \string\gdef\string\lastpage@lastpageHy{\lastpage@Hy}}%
%    \end{macrocode}
%
% \noindent After the write-out we restore the page number again,
% since there might be other things still to be done.
%
%    \begin{macrocode}
    \addtocounter{page}{+1}%
  \endgroup%
  }

%    \end{macrocode}
% \end{macro}
%
% \begin{macro}{\lastpage@fileswtest}
% \indent \hspace*{2em}Later it will be determined whether it is allowed to write
% to the \xfile{aux} file. If it was \emph{not} allowed, it is checked
% whether the label was already set via the \xfile{aux} file in some
% earlier compilation run. (There are packages where the document
% is compiled with access to the \xfile{aux} file, and then there is
% an additional compiler run, where the \xfile{aux} file cannot be changed,
% but in that run there is also no need to change it.) The \xpackage{tikz}
% package is somewhat different, therefore we only give a warning instead
% of an error (and hope that there is another compiler run where the
% \xfile{aux} file can be written).
%
%    \begin{macrocode}
\newcommand{\lastpage@fileswtest}[2]{%
  \edef\lastpage@testa{#1}%
  \edef\lastpage@testb{#2}%
  \ifx\lastpage@testa\lastpage@testb%
  \else%
    \IfPackageLoadedTF{tikz}{%
      \PackageWarning{lastpage}%
       {The lastpage package was not allowed to write to an\MessageBreak%
        .aux file. This package does not work without access\MessageBreak%
        to an .aux file.\MessageBreak%
        It is OK if the .aux file was already updated\MessageBreak%
        by a previous compiler run\MessageBreak%
        and would not have changed anyway.\MessageBreak%
       }%
    }{\PackageError{lastpage}{No auxiliary file allowed}%
       {The lastpage package was not allowed to write to an .aux file.\MessageBreak%
        This package does not work without access to an .aux file.\MessageBreak%
        Press Ctrl+Z to exit.\MessageBreak%
        But it is OK if the .aux file was already updated\MessageBreak%
        by a previous compiler run\MessageBreak%
        and would not have changed anyway.\MessageBreak%
       }%
     }%
  \fi%
  }

%    \end{macrocode}
% \end{macro}
%
% \begin{macro}{\lastpage@fileswtestHy}
% \indent \hspace*{2em}When the \xpackage{hyperref} package has been loaded,
% |\lastpage@lastpageHy| must be tested additionally. (And a
% |\newcommand| is needed, because |\ifHy@pageanchor| is not even
% defined when \xpackage{hyperref} has not been loaded.)
%
%    \begin{macrocode}
\newcommand{\lastpage@fileswtestHy}{%
  \ifHy@pageanchor%
    \lastpage@fileswtest{\@currentHpage}{\@the@H@page}%
  \else%
    \lastpage@fileswtest{\empty}{\lastpage@lastpageHy}%
  \fi%
  }

%    \end{macrocode}
% \end{macro}
%
% \DescribeMacro{enddocument/afterlastpage}
% \indent |enddocument/afterlastpage| we again (re)define |\lastpage@putlabel|
% to do nothing, but first use it to check whether |\lastpage@lastpage|
% is still unchanged, which is OK for the first run only.
%
%    \begin{macrocode}
\AddToHook{enddocument/afterlastpage}{%
  \gdef\lastpage@putlabel{??}%
  \ifx\lastpage@lastpage\lastpage@putlabel\relax%
    \AddToHook{enddocument/info}{%
      \PackageWarning{lastpage}{Rerun to get the references right}%
      }%
  \fi%
  \gdef\lastpage@putlabel{\relax}%
%    \end{macrocode}
%
% It is checked whether writing to files is allowed
% (otherwise, only an error message is issued and nothing is done).
%
%    \begin{macrocode}
  \if@filesw%
%    \end{macrocode}
%
% We put in a |\message| to show, in what order things (which were called) are done.
%
%    \begin{macrocode}
    \message{^^Jenddocument/afterlastpage (AED): lastpage setting LastPage.^^J}%
    \IfPackageLoadedTF{french}{%
         \addtocounter{page}{+1}\lastpage@putl@bel\addtocounter{page}{-1}%
     }{\IfPackageLoadedTF{frenchle}{%
         \addtocounter{page}{+1}\lastpage@putl@bel\addtocounter{page}{-1}%
       }{\lastpage@putl@bel%
        }%
      }%
  \else%
%    \end{macrocode}
%
% When writing to files is not allowed, nothing can be done. But when
% the label was already set via the \xfile{aux} file, nothing needs
% to be done. We check for this with |\lastpage@fileswtest| and
% (if \xpackage{hyperref} has been loaded) |\lastpage@fileswtestHy|.
%
%    \begin{macrocode}
    \lastpage@fileswtest{\thepage}{\lastpage@lastpage}%
    \IfPackageLoadedT{hyperref}{\lastpage@fileswtestHy}%
  \fi%
  }
%</lastpagemodern>
%    \end{macrocode}
% \newpage
%
% \section{Installation}
% \subsection{Downloads\label{ss:Downloads}}
% Everything is available at \url{https://www.ctan.org},
% but may need additional packages themselves.
%
% \DescribeMacro{lastpage.dtx}
% For unpacking the |lastpage.dtx| file and constructing the documentation it is required:
% \begin{description}
% \item[-] \TeX Format \LaTeXe{}: \url{https://www.CTAN.org}
%
% \item[-] document class \xclass{ltxdoc}, 2024/02/08, v2.1j,
%   \url{https://ctan.org/pkg/ltxdoc}
%
% \item[-] package \xpackage{holtxdoc}, 2019/12/09, v0.30,
%   \url{https://ctan.org/pkg/holtxdoc}
% \end{description}
%
% \DescribeMacro{lastpage.sty}
% The |lastpage.sty| (i.\,e.\ each document using the \xpackage{lastpage} package)
% requires:
% \begin{description}
% \item[-] \TeX, \url{https://www.CTAN.org}
%
% \item[-] package \xpackage{lastpage}, 2025-01-27, v2.1e,
%   \url{https://ctan.org/pkg/lastpage}
% \end{description}
%
% \DescribeMacro{lastpage209.sty}
% The |lastpage209.sty| for \LaTeX2.09{} (i.\,e.\ each document using
% the \xpackage{lastpage209} package) requires:
% \begin{description}
% \item[-] \TeX-format \LaTeX{}, v2.09
%
% \item[-] package \xpackage{lastpage209}, 2025-01-27, v2.1e,
%   \url{https://ctan.org/pkg/lastpage}
% \end{description}
% and does not work with \xpackage{hyperref}, which needs \LaTeX2e{}.\smallskip
%
% \DescribeMacro{lastpage2e.sty}
% The |lastpage2e.sty| for \LaTeXe{} (i.\,e.\ each document using
% the \xpackage{lastpage2e} package) requires:
% \begin{description}
% \item[-] \TeX-format \LaTeXe{} 1994/12/01 or newer,
%   \url{https://www.CTAN.org}
%
% \item[-] package \xpackage{lastpage}, 2025-01-27, v2.1e,
%   \url{https://ctan.org/pkg/lastpage}
% \end{description}
%
% \DescribeMacro{lastpageclassic.sty}
% The |lastpageclassic.sty| for \LaTeXe{} (i.\,e.\ each document using
% the \xpackage{lastpageclassic} package) requires:
% \begin{description}
% \item[-] \TeX-format \LaTeXe{} between 1994/12/01 and 2024-05-31,\newline
%   \url{https://www.CTAN.org}
%
% \item[-] package \xpackage{lastpage}, 2025-01-27, v2.1e,
%   \url{https://ctan.org/pkg/lastpage}
% \end{description}
% and can use
% \begin{description}
% \item[-] package \xpackage{hyperref}, 2023-07-08, v7.01b,
%   \url{https://ctan.org/pkg/hyperref}\newline
%   (probably also some older and newer versions)
% \end{description}
%
% \DescribeMacro{lastpagemodern.sty}
% The |lastpagemodern.sty| for \LaTeXe{} (i.\,e.\ each document using
% the \xpackage{lastpagemodern} package) requires:
% \begin{description}
% \item[-] \TeX-format \LaTeXe{} 2024-11-01 or newer,\newline
%   \url{https://www.CTAN.org}
%
% \item[-] package \xpackage{lastpage}, 2025-01-27, v2.1e,
%   \url{https://ctan.org/pkg/lastpage}
% \end{description}
% and can use
% \begin{description}
% \item[-] package \xpackage{hyperref}, probably 2023-11-07 and newer
%   (tested with: 2024-11-05 v7.01l),
%   \url{https://ctan.org/pkg/hyperref}
% \end{description}
% \pagebreak
%
% \DescribeMacro{lastpage-example.tex}
% The |lastpage-example.tex| requires the same file as all
% documents using the \xpackage{lastpage} package, i.\,e.
% \begin{description}
% \item[-] package \xpackage{lastpage}, 2025-01-27, v2.1e,
%   \url{https://ctan.org/pkg/lastpage}\newline
%   (Well, it is the example file for this package, and because you are reading the
%    documentation for the \xpackage{lastpage} package, it can be assumed that you already
%    have some version of it -- is it the current one?)
% \end{description}
% and additionally:
% \begin{description}
% \item[-] class \xpackage{article}, 2024/06/29, v1.4n, from \xpackage{classes}:
%           \url{https://ctan.org/pkg/classes}
%
% \item[-] package \xpackage{showkeys}, 2024-05-23, v3.21,
%   \url{https://ctan.org/pkg/showkeys}
%
% \item[-] package \xpackage{hyperref}, 2024-11-05, v7.01l,
%   \url{https://ctan.org/pkg/hyperref}
% \end{description}
%
% \DescribeMacro{Münch}
% A hyperlinked list of my (other) packages can be found at\newline
% \url{https://ctan.org/author/muench-hm}.
%
% \subsection{Package, unpacking TDS}
% \paragraph{Package.} This package is available on \url{https://www.CTAN.org}.
% \begin{description}
% \item[\url{https://mirror.ctan.org/macros/latex/contrib/lastpage/lastpage.dtx}]\hspace*{0.1cm} \newline
%       The source file.
% \item[\url{https://mirror.ctan.org/macros/latex/contrib/lastpage/lastpage.pdf}]\hspace*{0.1cm} \newline
%       The documentation.
% \item[\url{https://mirror.ctan.org/macros/latex/contrib/lastpage/lastpage-example.pdf}]\hspace*{0.1cm}
%       The compiled example file, as it should look like.
% \item[\url{https://mirror.ctan.org/macros/latex/contrib/lastpage/README}]\hspace*{0.1cm} \newline
%       The README file.
% \end{description}
% There is also a |lastpage.tds.zip| available:
% \begin{description}
% \item[\url{https://mirror.ctan.org/install/macros/latex/contrib/lastpage.tds.zip}]\hspace*{0.1cm}
%       Everything in TDS compliant, compiled format.
% \end{description}
% which additionally contains\newline
% \begin{tabular}{ll}
% lastpage.ins         & The \texttt{ins}tallation file.\\
% lastpage.drv         & The \mbox{\texttt{dr}i\texttt{v}er} to generate the documentation.\\
% lastpage.sty         & The \texttt{sty}le file.\\
% lastpage209.sty      & The \texttt{sty}le file for \LaTeX2.09{} \textbf{only}.\\
% lastpage2e.sty       & The \texttt{sty}le file to determine which 2e-style to use.\\
% lastpageclassic.sty  & The \texttt{sty}le file for older \LaTeX-formats.\\
% lastpagemodern.sty   & The \texttt{sty}le file for the recent \LaTeX-format.\\
% lastpage-example.tex & The example file.
% \end{tabular}
%
% \bigskip
%
% \noindent For required other packages please see the preceding subsection.
%
% \paragraph{Unpacking.} The \xfile{.dtx} file is a self-extracting
% \docstrip{} archive. The files are extracted by running the
% \xext{.dtx} through \plainTeX :
% \begin{quote}
%   \verb|tex lastpage.dtx|
% \end{quote}
%
% \noindent About generating the documentation see paragraph~\ref{GenDoc} below.\par
%
% \paragraph{TDS.} Now the different files must be moved into
% the different directories in your installation TDS tree
% (also known as \xfile{texmf} tree):
% \begin{quote}
% \def\t{^^A
% \begin{tabular}{@{}>{\ttfamily}l@{ $\rightarrow$ }>{\ttfamily}l@{}}
%   lastpage.sty & tex/latex/lastpage.sty\\
%   lastpage209.sty & tex/latex/lastpage.sty\\
%   lastpage2e.sty & tex/latex/lastpage.sty\\
%   lastpageclassic.sty & tex/latex/lastpage.sty\\
%   lastpagemodern.sty & tex/latex/lastpage.sty\\
%   lastpage.pdf & doc/latex/lastpage.pdf\\
%   lastpage-example.tex & doc/latex/lastpage-example.tex\\
%   lastpage-example.pdf & doc/latex/lastpage-example.pdf\\
%   lastpage.dtx & source/latex/lastpage.dtx
% \end{tabular}^^A
% }^^A
% \sbox0{\t}^^A
% \ifdim\wd0>\linewidth
%   \begingroup
%     \advance\linewidth by\leftmargin
%     \advance\linewidth by\rightmargin
%   \edef\x{\endgroup
%     \def\noexpand\lw{\the\linewidth}^^A
%   }\x
%   \def\lwbox{^^A
%     \leavevmode
%     \hbox to \linewidth{^^A
%       \kern-\leftmargin\relax
%       \hss
%       \usebox0
%       \hss
%       \kern-\rightmargin\relax
%     }^^A
%   }^^A
%   \ifdim\wd0>\lw
%     \sbox0{\small\t}^^A
%     \ifdim\wd0>\linewidth
%       \ifdim\wd0>\lw
%         \sbox0{\footnotesize\t}^^A
%         \ifdim\wd0>\linewidth
%           \ifdim\wd0>\lw
%             \sbox0{\scriptsize\t}^^A
%             \ifdim\wd0>\linewidth
%               \ifdim\wd0>\lw
%                 \sbox0{\tiny\t}^^A
%                 \ifdim\wd0>\linewidth
%                   \lwbox
%                 \else
%                   \usebox0
%                 \fi
%               \else
%                 \lwbox
%               \fi
%             \else
%               \usebox0
%             \fi
%           \else
%             \lwbox
%           \fi
%         \else
%           \usebox0
%         \fi
%       \else
%         \lwbox
%       \fi
%     \else
%       \usebox0
%     \fi
%   \else
%     \lwbox
%   \fi
% \else
%   \usebox0
% \fi
% \end{quote}
% If you have a \xfile{docstrip.cfg} that configures and enables \docstrip's
% TDS installing feature, then some files can already be in the right
% place, see the documentation of \docstrip.
%
% \subsection{Refresh file name databases}
% If your \TeX~distribution (\TeX\,Live, \mikTeX, \dots) relies on
% file name databases, you must refresh these. For example, \TeX\,Live\ users
% run \verb|texhash| or \verb|mktexlsr|.
%
% \subsection{Some details for the interested}
% \paragraph{Unpacking with \LaTeX.}
% The \xfile{.dtx} chooses its action depending on the format:
% \begin{description}
% \item[\plainTeX:] Run \docstrip\ and extract the files.
% \item[\LaTeX:] Generate the documentation.
% \end{description}
% If you insist on using \LaTeX\ for \docstrip\ (really,
% \docstrip\ does not need \LaTeX ), then inform the autodetect routine
% about your intention:
% \begin{quote}
%   \verb|latex \let\install=y% \iffalse meta-comment
%
% File: lastpage.dtx
% Version: 2025-01-27 v2.1e
% Info: Refers to last page's name
%
% Copyright © 2010 - 2025 by
%    H.-Martin Münch <Martin dot Muench at Uni-Bonn dot de>
% Portions of code copyrighted by other people as marked.
%
% This package was invented by Jeffrey P. Goldberg.
% I thought that an enhancement was needed and therefore created the pageslts package,
% https://ctan.org/pkg/pageslts. Nevertheless, for compatibility with existing
% documents/packages as well as for the low amount of resources needed by the
% lastpage package (no new counter!), I updated this package.
% Thanks go to Jeffrey P. Goldberg for allowing me to do this.
%
% While this package allows for things like "Page \thepage{} of \pageref{LastPage}"
% to get "Page 7 of 9" or "Page VII of IX",
% the NUMBER of pages is nowadays available from the kernel
% (\@abspage@last, \thetotalpages, \PreviousTotalPages),
% but with pages for example I to X and then 1 to 10, the number of pages would be "20",
% while the name of the last page is "10". Decide what you need/want!
%
% This work may be distributed and/or modified under the
% conditions of the LaTeX Project Public License, either
% version 1.3c of this license or (at your option) any later
% version. This version of this license is in
%    https://www.latex-project.org/lppl/lppl-1-3c.txt
% and the latest version of this license is in
%    https://www.latex-project.org/lppl.txt
% and version 1.3c or later is part of all distributions of
% LaTeX version 2005-12-01 or later.
%
% This work has the LPPL maintenance status "maintained".
%
% The Current Maintainer of this work is H.-Martin Münch.
%
% This work consists of the main source file lastpage.dtx,
% the README, and the derived files
%    lastpage.sty, lastpage209.sty, lastpage2e.sty,
%    lastpageclassic.sty, lastpagemodern.sty,
%    lastpage.pdf, lastpage.ins, lastpage.drv,
%    lastpage-example.tex, lastpage-example.pdf.
%
% "lastpage" is available on CTAN:
% https://ctan.org/pkg/lastpage
%
% Also a TDS.ZIP file is provided that contains all the files
% already sorted in a TDS tree:
% https://mirror.ctan.org/install/macros/latex/contrib/lastpage.tds.zip
%
%<*ignore>
\begingroup
  \catcode123=1 %
  \catcode125=2 %
  \def\x{LaTeX2e}%
\expandafter\endgroup
\ifcase 0\ifx\install y1\fi\expandafter
         \ifx\csname processbatchFile\endcsname\relax\else1\fi
         \ifx\fmtname\x\else 1\fi\relax
\else\csname fi\endcsname
%</ignore>
%<*install>
\input docstrip.tex
\Msg{**********************************************************************}
\Msg{* Installation                                                       *}
\Msg{* Package: lastpage 2025-01-27 v2.1e Refers to last page's name (HMM)*}
\Msg{**********************************************************************}

\keepsilent
\askforoverwritefalse

\let\MetaPrefix\relax
\preamble

This is a generated file.

Project: lastpage
Version: 2025-01-27 v2.1e
Info: Refers to last page's name

Copyright (C) 2010 - 2025 by
    H.-Martin Muench <Martin dot Muench at Uni-Bonn dot de>
Portions of code copyrighted by other people as marked.

This work may be distributed and/or modified under the
conditions of the LaTeX Project Public License, either
version 1.3c of this license or (at your option) any later
version. This version of this license is in
   https://www.latex-project.org/lppl/lppl-1-3c.txt
and the latest version of this license is in
   https://www.latex-project.org/lppl.txt
and version 1.3c or later is part of all distributions of
LaTeX version 2005-12-01 or later.

This work has the LPPL maintenance status "maintained".

The Current Maintainer of this work is H.-Martin Muench.

This package was invented by Jeffrey P. Goldberg.
I thought that an enhancement was needed and therefore created the pageslts package,
https://ctan.org/pkg/pageslts. Nevertheless, for compatibility with existing
documents/packages as well as for the low amount of resources needed by the
lastpage package (no new counter!), I updated this package.
Thanks go to Jeffrey P. Goldberg for allowing me to do this.

While this package allows for things like "Page slash thepage{} of slash pageref{LastPage}"
to get "Page 7 of 9" or "Page VII of IX",
the NUMBER of pages is nowadays available from the kernel
(slash @abspage@last, slash thetotalpages, slash PreviousTotalPages),
but with pages for example I to X and then 1 to 10, the number of pages would be "20",
while the name of the last page is "10". Decide what you need/want!

This work consists of the main source file lastpage.dtx,
the README, and the derived files
   lastpage.sty, lastpage209.sty, lastpage2e.sty,
   lastpageclassic.sty, lastpagemodern.sty,
   lastpage.pdf, lastpage.ins, lastpage.drv,
   lastpage-example.tex, lastpage-example.pdf.

In memoriam
 Claudia Simone Barth + 1996-01-30
 Tommy Muench         + 2014-01-02
 Hans-Klaus Muench    + 2014-08-24

\endpreamble
\let\MetaPrefix\DoubleperCent

\generate{%
  \file{lastpage.ins}{\from{lastpage.dtx}{install}}%
  \file{lastpage.drv}{\from{lastpage.dtx}{driver}}%
  \usedir{tex/latex/lastpage}%
  \file{lastpage.sty}{\from{lastpage.dtx}{package}}%
  \file{lastpage209.sty}{\from{lastpage.dtx}{lastpage209}}%
  \file{lastpage2e.sty}{\from{lastpage.dtx}{lastpage2e}}%
  \file{lastpageclassic.sty}{\from{lastpage.dtx}{lastpageclassic}}%
  \file{lastpagemodern.sty}{\from{lastpage.dtx}{lastpagemodern}}%
  \usedir{doc/latex/lastpage}%
  \file{lastpage-example.tex}{\from{lastpage.dtx}{example}}%
}

\catcode32=13\relax% active space
\let =\space%
\Msg{************************************************************************}
\Msg{*}
\Msg{* To finish the installation you have to move the following}
\Msg{* file into a directory searched by TeX:}
\Msg{*}
\Msg{*  lastpage.sty, lastpage209.sty, lastpage2e.sty,}
\Msg{*  lastpageclassic.sty, and lastpagemodern.sty}
\Msg{*}
\Msg{* To produce the documentation run the file `lastpage.drv'}
\Msg{* through (pdf)LaTeX, e.g.}
\Msg{*  pdflatex lastpage.drv}
\Msg{*  makeindex -s gind.ist lastpage.idx}
\Msg{*  pdflatex lastpage.drv}
\Msg{*  makeindex -s gind.ist lastpage.idx}
\Msg{*  pdflatex lastpage.drv}
\Msg{*}
\Msg{* At least three runs are necessary e.g. to get the}
\Msg{*  references right!}
\Msg{*}
\Msg{* Happy TeXing!}
\Msg{*}
\Msg{************************************************************************}

\endbatchfile
%</install>
%<*ignore>
\fi
%</ignore>
%
% \section{The documentation driver file}
%
% The next bit of code contains the documentation driver file for
% \TeX , i.\,e., the file that will produce the documentation you
% are currently reading. It will be extracted from this file by the
% \texttt{docstrip} programme. That is, run \LaTeX{} on \texttt{docstrip}
% and specify the \texttt{driver} option when \texttt{docstrip}
% asks for options.
%
%    \begin{macrocode}
%<*driver>
\NeedsTeXFormat{LaTeX2e}[2024-11-01]
\ProvidesFile{lastpage.drv}[2025-01-27 v2.1e Refers to last page's name (HMM)]
\documentclass{ltxdoc}[2024/02/08]% v2.1j Standard LaTeX documentation class
\usepackage{holtxdoc}[2019/12/09]%  v0.30 Private additional ltxdoc support (HO)
\hypersetup{%
 pdfsubject={Refers to last page's name (HMM; JPG)},%
 pdfkeywords={LaTeX, lastpage, last page, page number, page name},%
 pdfencoding=auto,%
 pdflang={en},%
 breaklinks=true,%
 linktoc=all,%
 pdfstartview=FitH,%
 pdfpagelayout=OneColumn,%
 bookmarksnumbered=true,%
 bookmarksopen=true,%
 bookmarksopenlevel=2,%
 pdfmenubar=true,%
 pdftoolbar=true,%
 pdfwindowui=true,%
 pdfnewwindow=true%
}
\CodelineIndex
\hyphenation{every-thing}
\begin{document}
  \DocInput{lastpage.dtx}%
\end{document}
%</driver>
%    \end{macrocode}
% \fi
%
% \GetFileInfo{lastpage.drv}
%
% \begingroup
%   \def\x{\#,\$,\^,\_,\~,\ ,\&,\{,\},\%}%
%   \makeatletter
%   \@onelevel@sanitize\x
% \expandafter\endgroup
% \expandafter\DoNotIndex\expandafter{\x}
% \expandafter\DoNotIndex\expandafter{\string\ }
% \begingroup
%   \makeatletter
%     \lccode`9=32\relax
%     \lowercase{%^^A
%       \edef\x{\noexpand\DoNotIndex{\@backslashchar9}}%^^A
%     }%^^A
%   \expandafter\endgroup\x
%
% \DoNotIndex{\@auxout,\@evenfoot,\@firstofone,\@firstoftwo,\@ifl@t@r,\@ifpackagelater,\@ifpackageloaded,%
% \IfPackageAtLeastF,\IfPackageLoadedT,\IfPackageLoadedTF,%
% \@ifundefined,\@kernel@reserved@label@data,\@mainaux,\@number,\@oddfoot,%
% \addtocounter,\AddToHook,\arabic,\begin,%
% \bigskip,\clearpage,\csname,\detokenize,\documentclass,\EdefUnescapeString,\empty,\end,%
% \endcsname,\enddocument,\expanded,\ExplSyntaxOff,\ExplSyntaxOn,\fmtversion,%
% \g,\gdef,\here,\hfil,\Hy@temp,\Hy@unicodefalse,\Hy@EveryPageAnchor,%
% \hypersetup,\if,\if@filesw,\if@nobreak,\ifHy@hypertexnames,\ifHy@pageanchor,\ifHy@plainpages,%
% \ifdim,\ifnum,\ifvmode,\ifx,\immediate,\input,%
% \label,\lastpage,\lastpage-example.tex,\lastpage.dtx,\lastpage.sty,\lastpagee.sty,%
% \lastpageclassic.sty,\lastpagemodern.sty,\LaTeX,\LaTeX-kernel,\listfiles,%
% \makeatletter,\makeatother,\markboth,\meaning,\mbox,\message,\MessageBreak,%
% \NeedsTeXFormat,\newcommand,\newcounter,\newline,\newpage,\nobreak,%
% \noindent,\normalsize,\numexpr,\origenddocument,\PackageError,\PackageNoteNoLine,\PackageWarning,%
% \PackageWarningNoLine,\pagenumbering,\pageref,\pdfstringdef,\protect,\ProvidesPackage,%
% \qquad,\renewcommand,\RequirePackage,\section,\ShowHook,\slshape,\smallskip,\space,\test,\textbf,%
% \textit,\textquotedblleft,\textquotedblright,\textsf,\texttt,\the,\thepage,%
% \today,\upshape,\url,\usepackage,\value,\verb,\wd,\write,\xlastpage@rmpage,\xxlastpage@rmpage}
%
% \title{The \xpackage{lastpage} package}
% \author{H.-Martin Münch\\\xemail{Martin.Muench at Uni-Bonn.de}\\
%   invented by Jeffrey P. Goldberg\\\xemail{jeffrey+news at goldmark.org}}
% \date{2025-01-27 v2.1e}
% \maketitle
%
% \begin{abstract}
% \noindent This \LaTeX{} package puts the label \texttt{LastPage}
% at the end of the document into the \xfile{.aux} file,
% allowing the user to refer to the last page of a document.
% This might be particularly useful in places like headers or footers.~--\newline
% While this package allows for things like\newline
% \hbox{\textquotedblleft Page} \hbox{|\thepage{}|} of |\pageref{LastPage}|\textquotedblright{}
% to get \textquotedblleft Page 7 of 9\textquotedblright{} or
% \textquotedblleft Page VII of IX\textquotedblright,
% the \emph{number} of pages is nowadays available from the kernel
% (|\@abspage@last|, |\thetotalpages|, |\PreviousTotalPages|),
% but when more than one page numbering scheme is used
% (for example pages I~to X and then 1~to 10,
% thus number of pages \hbox{\textquotedblleft 20\textquotedblright ,}
% but name of the last page \hbox{\textquotedblleft 10\textquotedblright ),}
% or another package has output after this package,
% or the page numbers exceed a certain range,
% there might be problems, which can be solved by using the
% \xpackage{pageslts} package instead.
% \end{abstract}
%
% \bigskip
%
% \noindent Disclaimer for web links: The author is not responsible for any contents
% referred to in this work unless having full knowledge of illegal contents.
% If any damage occurs by the use of information presented there,
% only the author of the respective pages might be liable,
% not the one who has referred to those pages.
% \newpage
% \tableofcontents
%
% \section{Introduction}
% \indent This \LaTeX{} package puts the label \texttt{LastPage}
% (at end of the document via hook |enddocument/afterlastpage|,
% for older formats via |\AtEndDocument|, for \LaTeX2.09{} via
% redefining |\enddocument|) into the \xfile{aux} file,
% allowing the user to refer to the last page of a document via
% |\pageref{LastPage}|. This might be particularly useful
% in places like headers or footers.
%
% \bigskip
%
% This package was invented by \textbf{Jeffrey P. Goldberg},
% and is now maintained by \textsc{H.-Martin Münch}. A~big
% \textquotedblleft Thank you!\textquotedblright{} to
% \textsc{Jeffrey P. Goldberg} for granting this.
%
% \bigskip
%
% If you are more ambitious in respect to your aims with this package,
% you might want to have a look at the \xpackage{pageslts} package
% (see section~\ref{sec:Alternatives}: Alternatives).
%
% \section{Usage}
% \indent Just load the package placing
% \begin{quote}
%   |\usepackage{lastpage}|
% \end{quote}
% \noindent in the preamble of your source file
% (or |\input{lastpage.sty}| if |\usepackage| is unknown).
%
% \indent For example for various draft forms it is desirable to have a
% page reference to the last page, so that e.\,g.\ page footers can
% contain something like \textquotedblleft page $N$ of $K$\textquotedblright,
% where $N$ is the current page and $K$ is the last page. Once the package
% is loaded, anywhere in the text references can be made to the label
% \texttt{LastPage}. In particular one can use the \xpackage{fancyhdr}
% or \xpackage{nccfancyhdr} package, or redefinitions of the page headings
% and footings to get a reference to the last page.
%
% \noindent In your document the code
% \begin{verbatim}
% \makeatletter
% \renewcommand{\@evenfoot}{%
%  \normalsize\slshape DRAFT \today\hfil \upshape %
%  page \thepage{} of \pageref{LastPage}}
% \renewcommand{\@oddfoot}{\@evenfoot}
% \makeatother
%\end{verbatim}
% \noindent creates footers like\smallbreak
% \textquotedblleft\mbox{\textsl{DRAFT \today}\hspace{1cm}page 7 of 9}\textquotedblright\smallbreak
% \noindent in the compiled document (cf.~the \texttt{lastpage-example} file).\newline
% If the \xpackage{hyperref} package is used, the references are hyperlinked
% to their aims. If these hyperlinks shall be suppressed, |\pageref*{...}|
% instead of |\pageref{...}| can be used.\bigbreak
%
% The \xpackage{lastpage} package does not provide the words
% \textquotedblleft page\textquotedblright{} or \textquotedblleft of\textquotedblright{},
% but e.\,g.\ the \xclass{handout} class uses \textquotedblleft of\textquotedblright{} in
% the definition of the footer. (In the \texttt{lastpage-example} also
% |\@evenfoot| is redefined, but it is not the \xpackage{lastpage} \emph{package}
% redefining this.) If you want to change \textquotedblleft page\textquotedblright{} or
% \textquotedblleft of\textquotedblright{} (e.\,g.\ to another language), you therefore
% have got to look in the used class/package(s)/preamble instead of in the
% \xpackage{lastpage} package.\bigbreak
%
% If the total \emph{number} of pages of a document is needed,
% the kernel already gives this by |\makeatletter\@abspage@last\makeatother|,
% |\thetotalpages|, and |\PreviousTotalPages| (needing at least two compiler runs).
%
% \section{Some \textsc{Warnings}\label{sec:warn}}
% \subsection{\texttt{\textbackslash AtEndDocument}\label{ssec:aed}}
% \indent {\bfseries |\AtEndDocument| is not used by the \pkg{lastpagemodern.sty}
% version of the lastpage package, requiring \LaTeX -format 2024-11-01
% or newer. Instead |\AddToHook{enddocument/afterlastpage}| is used
% and the problem does not arise.}
% \bigskip
%
% \pkg{lastpageclassic.sty} uses |\AtEndDocument| and
% \pkg{lastpage209.sty} redefines \linebreak |\enddocument|.
% The last two cases are problematic:\newline
% The output of a \LaTeXe{} run is not independent of the order
% in which the packages are loaded. It is often the case that the same
% formats for which one must put tables and figure at the end,
% are the ones in which endnotes are also required.
% If one wants to use |\AtEndDocument| here as well (as done for
% |\pageref{LastPage}|), then it is easy to get to three separate
% uses of |\AtEndDocument| (assuming one uses this for the endnotes
% as well). Clearly it is not safe for any package writer or user
% to assume that no material will follow what they put into
% |\AtEndDocument|. Therefore a message, which begins with
% \texttt{AED}, is included in every usage of |\AtEndDocument|.
% \xpackage{lastpage} uses |\AtEndDocument{|\ldots|\clearpage|\ldots|}|, thus\newline
%|\usepackage{lastpage}|\ldots|\AtEndDocument{|\textit{something}|}|
% will place \textit{something} after the |\clearpage|. To place it earlier, use\newline
% |\AtEndDocument{|\textit{something}|}|\ldots|\usepackage{lastpage}|.
% If the \textit{something} is not known before |\usepackage{lastpage}|,
% you can use for example
% \begin{verbatim}
% ...
% \def\beforeLastpageClearpage{\relax}
% \AtEndDocument{\beforeLastpageClearpage}
% \usepackage{lastpage}
% \begin{document}
% ...
% \def\beforeLastpageClearpage{\textit{something}}%
% ...
% \end{document}
% \end{verbatim}
% (might need a protected and/or expanded |\def|). When |\clearpage| leads
% to some output, |\clearpage\textit{something}| instead of |\textit{something}|
% might be wanted.
%
% \subsection{Interaction with ancient versions of the \xpackage{endfloat} package\label{sec:endfloat}}
%
% \indent {\bfseries |\AtEndDocument| is not used by the \pkg{lastpagemodern.sty}
% version of the lastpage package, requiring \LaTeX -format 2024-11-01
% or newer. Instead |\AddToHook{enddocument/afterlastpage}| is used
% and the problem does not arise.}
%
% \bigskip
% The \emph{ancient} version~2.0 (and earlier; 2.0~from 1992;
% current version at the time of updating this documentation: 2.7~from 2019)
% of the \xpackage{endfloat} package actually redefined the |\enddocument| command,
% and so interfered drastically with the \LaTeXe{} commands which make use of
% |\AtEndDocument|.
% If you want your \texttt{LastPage} to label the last page of these end floats,
% you need to load \xpackage{lastpage} after loading \xpackage{endfloat}
% (or use \texttt{VeryLastPage} from the \xpackage{pageslts} package instead).
% If, on the other hand, you \emph{want} \texttt{LastPage} to refer
% to the (not so) last page, exclusive of the floats at the end,
% then load in the reverse order. Independent from the order of
% \xpackage{lastpage} and \xpackage{endfloat}, you will still need
% a version of \xpackage{endfloat} later than 2.0 from 1992.
%
% Other \LaTeX2.09{} (!) packages also seem to like to redefine
% |\enddocument|. In addition to the old \xpackage{endfloat},
% \xpackage{harvard} comes to mind. All of these will need to be
% modified swiftly.
%
% \subsection{Page name instead of page number}
% When any page numbering scheme other than \texttt{arabic} is used
% at the page, which |\pageref{LastPage}| refers to, the \textit{name}
% and not the \textit{number} of the page is given. For example,
% \texttt{Alph} page numbering scheme and $10$ pages will give \texttt{J} instead of 10,
% \texttt{Roman} page numbering scheme and $10$ pages will give \texttt{X} instead of 10,
% and so on.\newline
% \indent (The \xpackage{pageslts} package puts |\lastpageref{LastPages}|
% (with \textbf{s} at the end) at your disposal for remediation.)
%
% \subsection{No write access to the \xfile{aux} file}
% Some packages (e.\,g.\ \xpackage{tikz} and \xpackage{selectp}) sometimes prevent
% the output to the \xfile{aux} file. In that case a warning is issued. This is
% no problem as long as there is another compilation run where the label to the
% last page can be placed via the \xfile{aux} file.
%
% \subsection[Wrong last page number with \textgreater 1 page numbering scheme]{%
%  Wrong last page number with more than one page numbering scheme}
% When more than one page numbering scheme is used,
% \texttt{LastPage} does not give the total \textbf{number} of pages
% (even if \texttt{arabic} is the page numbering scheme of that page).
% For example, for a document with VI+36 pages, it gives
% \textquotedblleft 36\textquotedblright{} as reference to the last page.
% While this is correct, the total number of pages is $42$.\newline
% If the total \emph{number} of pages of a document is needed,
% the kernel already gives this by |\makeatletter\@abspage@last\makeatother|,
% |\thetotalpages|, and |\PreviousTotalPages|.
% The \xpackage{pageslts} package puts |\lastpageref{LastPages}|
% (with \textbf{s} at the end) at your disposal for remediation,
% giving the number of pages and linking to the last page, if linking is provided
% for examaple by the \xpackage{hyperref} package.
%
% \subsection{\texttt{\textbackslash addtocounter\{page\}\{\ldots\} and %
%  \texttt{\textbackslash setcounter\{page\}\{\ldots\}}}}
% When the page number was manipulated by |\addtocounter{page}{...}| or
% |\setcounter{page}{...}|, \texttt{LastPage} does not give the total
% \textbf{number} of pages (even if \texttt{arabic} is the page numbering
% scheme of that page).\newline
% The \xpackage{pageslts} package puts |\lastpageref{LastPages}|
% (with \textbf{s} at the end) at your disposal for remediation:
% \texttt{LastPages} ignores page number manipulation. Also |\@abspage@last|,
% |\thetotalpages|, and |\PreviousTotalPages| from the kernel
% are not influenced by page number manipulation.
%
% \subsection{Page number reset by \texttt{\textbackslash pagenumbering\{\ldots\}}}
% At a page numbering change the page number is reset to one.
% Therefore \texttt{LastPage} does not give the total \textbf{number} of pages
% (even if \texttt{arabic} is the page numbering scheme of that page).
% Furthermore, now two pages have the same name.\newline
% The \xpackage{pageslts} package does not only put
% |\lastpageref{LastPages}| (with \textbf{s} at the end) at your disposal
% for remediation: \texttt{LastPages} also ignores page number manipulation.
% It furthermore offers the option |pagecontinue| to continue the
% page numbering, when |\pagenumbering{...}| is used.
%
% \subsection{Last pages of different page numbering schemes}
% |\pageref{LastPage}| refers to the (maybe not so) last page of the last
% page numbering scheme. References to the respective last page of the other
% page numbering schemes are not provided.\newline
% The \xpackage{pageslts} package does this with labels
% \texttt{pagesLTS.<numbering scheme>}, where \texttt{<numbering scheme>} is
% e.\,g.\ arabic, roman, Roman, alph, or Alph.\linebreak
% For fnsymbol please use |\lastpageref{pagesLTS.fnsymbol}| instead of\newline
% |\pageref{pagesLTS.fnsymbol}|.
%
% \subsection{Current page}
% The command |\thepage| gives the \textbf{name} of the current page
% in the current page numbering scheme, which is different from the
% current total/absolute page number e.\,g.\ with a second
% page numbering scheme, |\addtocounter{page}{...}|, or |\setcounter{page}{...}|,
% and it will not be an arabic number at all,
% if the current page numbering scheme is not arabic.\newline
% The \xpackage{pageslts} package provides the command
% |\theCurrentPage| and for the current number of pages in the current
% page numbering scheme\newline
% |\theCurrentPageLocal|. The kernel already provides the
% number of pages, which have been shipped out, as |\the\ReadonlyShipoutCounter|.
% The current page is always |ReadonlyShipoutCounter +1|.
%
% \subsection{First page}
% There is no special label at the first page.
% (This is the \textbf{last}\textsf{page} package, after all.)\newline
% The \xpackage{pageslts} package creates a label \texttt{pagesLTS.0}
% at the first page of the document.
%
% \subsection{Page counter overflow\label{ss:overflow}}
% \indent \textquotedblleft The ranges of supported counter values are more or less
% restricted. Only \cs{arabic} can be used with any counter value \TeX{} supports.
% \begin{quote}
% \catcode`\|=12 %
% \begin{tabular}{@{}l|l|l|l@{}}
% Presentation & Supported & Ignored & Error message\\
% command      & domain    & values  & `Counter too large'\\
% \hline
% \cs{arabic}
%   & \ttfamily -MAX..MAX & &\\
% \cs{roman}, \cs{Roman}
%   & \ttfamily 1..MAX & \ttfamily -MAX..0 &\\
% \cs{alph}, \cs{Alph}
%   & \ttfamily 1..26 & 0 & \ttfamily -MAX..-1, 27..MAX\\
% \cs{fnsymbol}
%   & \ttfamily 1..9 & 0 & \ttfamily -MAX..-1, 10..MAX\\
% \hline
% \end{tabular}\\[1ex]
% \texttt{MAX} = \texttt{2147483647}
% \end{quote}
% \textquotedblright\ (\xpackage{alphalph} package manual, 2019/12/09, v2.6,
% first table, p.~2).
%
% \noindent When \textit{any} page is out of that range, there will be a counter overflow.\newline
% \xpackage{lastpage} probably is not the right package to be asked
% to correct this anyway, but the \xpackage{pageslts} package
% (with appropriate options) can do this.
%
% When MAX is exceeded via
% |\setcounter{<name>}{| something greater than MAX (or smaller then $-$MAX) |}|,
% then the error
% \begin{quote}
% \begin{verbatim}
% ! Number too big.
% I can only go up to 2147483647='17777777777="7FFFFFFF,
% so I'm using that number instead of yours.
% \end{verbatim}
% \end{quote}
% \vspace{-\baselineskip}
% will arise. But if the counter has a value of $2\,147\,483\,647 = {}$MAX,
% and |\addtocounter{<name>}{+1}| is tried, no error is issued, but
% |\arabic{<name>}| prints $-2147483648$, and further |\addtocounter{<name>}{+1}|s
% give $-2147483647$, $-2147483646$ and so on.\newline
% For a counter value of $-2\,147\,483\,647 = -$MAX and |\addtocounter{<name>}{-1}|s
% after $-2147483647$ it is printed $-2147483648$, $2147483647$, $2147483646$ and so on
% (without any message in the \xfile{log} file about any possible issue).
%
% \subsection{Other packages manipulating \texttt{\textbackslash lastpage@putlabel}}
% The \xpackage{revtex4} class redefines the |\lastpage@putlabel| command
% to place a label |LastPage|.\newline
% |\lastpage@putlabel| in the \xpackage{lastpage} package was replaced by |\lastpage@putl@bel|,
% but the |LastPage| label could become defined more than once.
%
% \subsection{\texttt{\textbackslash pagenumbering\{fnsymbol\}}\label{subsec:fnsymbol}}
% When using the foot-note-symbols as page numbers,
% it can be necessary to declare in the document's preamble:
% \begin{verbatim}
% \ProvideTextCommand{\textasteriskcentered}{PD1}{*}
% \ProvideTextCommand{\textdagger}{PD1}{†}
% %\ProvideTextCommand{\textdaggerdbl}{PD1}{‡)% seems to neither work nor be necessary
% \ProvideTextCommand{\textsection}{PD1}{§}
% \ProvideTextCommand{\textparagraph}{PD1}{¶}
% \ProvideTextCommand{\textbardbl}{PD1}{‖}
% \end{verbatim}
%
% \section{Alternatives\label{sec:Alternatives}}
% There are similar packages, which do (or do not) similar things (or even more).
% As I neither know what exactly you want to accomplish when using this package
% (e.\,g.~page number vs. page name, hyperlinks or not), nor what resources
% your system has (e.\,g.~\TeX, \LaTeX2e, $\varepsilon$-\TeX{}, \LaTeX-format as recent
% as 2024-11-01 or newer), here is a list of some possible alternatives:
% \begin{description}
% \item[-]\DescribeMacro{pageslts}
%  The \xpackage{pageslts} package first started as a revision of this
%  \xpackage{lastpage} package, but I~thought that an enhancement was needed
%  to accomplish what the \xpackage{pageslts} package does. For backward
%  compatibility, a label named |LastPage| is provided.
%  Thus |\usepackage{lastpage}| can be replaced by\newline
%  |\usepackage[pagecontinue=false,alphMult=0,AlphMulti=0,|\newline
%  | fnsymbolmult=false,romanMult=false,RomanMulti=false]{pageslts}|,\newline
%  if the behaviour of the \xpackage{lastpage} package should be simulated.
%  The default options are\newline
%  |\usepackage[pagecontinue=true,alphMult=ab,AlphMulti=AB,|\newline
%  |fnsymbolmult=true,romanMult=true,RomanMulti=true]{pageslts}|.\newline
%  Benefits of \xpackage{pageslts} package (with appropriate options) are:
%  \begin{description}
%  \item[+] Labels \texttt{LastPage} (|\AddToHook{enddocument/afterlastpage}|,
%   formerly |\AtEndDocument|; same as the \xpackage{LastPage} package) and
%   \texttt{VeryLastPage} (also |\AddToHook{enddocument/afterlastpage}|,
%   but formerly\newline%
%   |\AfterLastShipout|), allowing the user to refer to the
%   (very) last page of a document.
%  \item[+] For example, when more than one page numbering scheme is used,
%    the label \mbox{\texttt{LastPage}\textbf{s}} gives the total \textit{number} of pages.
%  \item[+] At the last page of each page numbering scheme a label\newline
%   \texttt{pagesLTS.<numbering scheme>} is placed, where
%   \texttt{<numbering scheme>} is e.\,g.\ arabic, roman, Roman, alph, or Alph.
%   For fnsymbol please use |\lastpageref{pagesLTS.fnsymbol}| instead of\newline
%   |\pageref{pagesLTS.fnsymbol}|.
%  \item[+] When the same numbering scheme is used twice, the page numbers
%   are either reset to one or continued automatically, depending on the option
%   given when the package is called.
%  \item[+] The command |\theCurrentPage| prints the current total/absolute
%   page number -- in contrast to |\thepage|, which gives only the page
%   \textit{name} in the current page numbering scheme.
%   |\theCurrentPageLocal| gives the current number of pages in the current
%   page numbering scheme. |\thepage| and |\theCurrentPageLocal| are different
%   e.\,g.\ when |\addtocounter{page}{...}| or |\setcounter{page}{...}| were used.
%  \item[+] At the first page of the document a label \texttt{pagesLTS.0} is created.
%  \item[+] The \xpackage{alphalph} package is supported, i.\,e.\ %
%   page numbers alph or Alph $>26$ and fnsymbol $>9$ can be used
%   (with according options set). Even zero and negative page numbers can be used
%   with \texttt{arabic}, \texttt{alph}, \texttt{Alph}, \texttt{roman}, \texttt{Roman},
%   and \texttt{fnsymbol} page numbering (with \xpackage{alphalph} package and
%   according options).
%  \end{description}
%  Further labels are provided for special cases.\newline
%  \url{https://ctan.org/pkg/pageslts}
%
% \item[-]\DescribeMacro{LaTeX-kernel}
%  The number of pages is nowadays available via |\@abspage@last|,\linebreak
%  |\thetotalpages|, and |\PreviousTotalPages| from the kernel,
%  but when more than one page numbering scheme is used
%  (for example pages I~to X and then 1~to 10, thus number of pages
%  \hbox{\textquotedblleft 20\textquotedblright ,} but name of the last page
%  \hbox{\textquotedblleft 10\textquotedblright ),}
%  or when or the fnsymbol page numbering scheme is used, or another package
%  has output after this package, or the page numbers exceed a certain range,
%  there might be issues. (Is the total number of pages wanted? Or is the name
%  of the last page sought?)\newline
%  |\the\ReadonlyShipoutCounter| contains the number of currently shipped out pages,
%  i.\,e.\ current page minus one.
%
% \item[-]\DescribeMacro{totpages}
%  The \xpackage{totpages} package provides a \texttt{totpages} label similar to
%  \texttt{LastPages}, but |\AtEndDocument| instead of hook |enddocument/afterlastpage|
%  of the \pkg{pageslts} package.
%  The \xpackage{totpages} package additionally computes the number of paper sheets
%  needed to (double) print the document (with one, two, three,\ldots{} pages on
%  one sheet of paper) (which can be achieved also with the \xpackage{papermas} package,
%  an extension of the \xpackage{pageslts} package, which further allows to compute
%  the mass of that printed version of the document, useful e.\,g.\ when sending it
%  by mail to determine the postage).\newline
%  \url{https://ctan.org/pkg/totpages}
%
% \item[-]\DescribeMacro{totalcount}
%  The \xpackage{totalcount} package provides |\totalpages|.
%  If there are only arabic page numbers consecutively running from 1 to the last page,
%  this works. But for example
%
% \begin{quote}
%\begin{verbatim}
%\documentclass{article}
%\usepackage[page]{totalcount}
%\pagenumbering{Roman}
%\begin{document}
%\addtocounter{page}{49}
%Page \thepage{} of \totalpages
%\end{document}
%\end{verbatim}
% \end{quote}
%
%  prints \textquotedblleft Page L of 50\textquotedblright,
%  where the number of pages is one (and no hyperlink is provided to the
%  last page even if \xpackage{hyperref} is used).\newline
%  \url{https://ctan.org/pkg/totalcount}
%
% \item[-]\DescribeMacro{totcount}
%  The \xpackage{totcount} package provides the last value of a counter, thus also the value
%  of the \texttt{page} counter. You do not get a hyperlink to the last page, only the numerical
%  value of the last page name is given \mbox{(i.\,e.~X+72} pages gives 72 instead of 82
%  as total number of pages), and the number of pages can be changed for example
%  by |\addtocounter|.\newline
%  \url{https://ctan.org/pkg/totcount}
% \pagebreak
%
% \item[-]\DescribeMacro{nofm}
%  \textquotedblleft There is a package \xpackage{nofm.sty} available, but some versions
%  of it are defective, and most don't work with \xpackage{fancyhdr} because they take over
%  the complete page layout.\textquotedblright{} (\textsc{Piet van Oostrum}:
%  Page layout in \LaTeX , March~2, 2004, section~16; fancyhdr.pdf)\newline
%  \xpackage{nofm} as of 1991/02/25 (without version number), available at\newline
%  \url{https://mirror.ctan.org/obsolete/macros/latex209/contrib/misc/nofm.sty},\newline
%  does not work with e.\,g.\ \xpackage{hyperref}, redefines |\enddocument|
%  as well as |\@oddhead|, |\@evenhead|, |\@oddfoot|, and |\@evenfoot|.\newline
%  If you know the (\url{https://CTAN.org}) location of a
%  \textbf{working}~(!) version, please send me an e-mail, thanks!
%
% \item[-]\DescribeMacro{count1to}
%  The \xpackage{count1to} package \textquotedblleft sets |\count1| to |\count8|
%  with the values of page to subparagraph. |\count9| is used to flag odd
%  pages. \ldots\ [T]he code for the TotalPages
%  label\textquotedblright\ (package manual, 2024-06-13) has been removed
%  from the current package version.\newline
%  \url{https://ctan.org/pkg/count1to}
%
% \item[-]\DescribeMacro{zref}
%  The \xpackage{zref} package \textquotedblleft implements an extensible referencing
%  system\textquotedblright\ (package manual, 2023-09-14).\newline
%  \url{https://ctan.org/pkg/zref}
%
% \item[-]\DescribeMacro{memoir}
%  The \xpackage{memoir} \emph{class} provides |\thelastpage| (page number printed
%  on last page) and |\thelastsheet| (number of pages).\newline
%  \url{https://ctan.org/pkg/memoir}
% \end{description}
%
% \bigskip
%
% \noindent (You programmed or found another alternative,
%  which is available at\newline
%  \url{https://CTAN.org}?
%  OK, send an e-mail to me with the name, location at CTAN,
%  and a short notice, and I will probably include it in the list above.)
% \newpage
%
% \section{Example}
%    \begin{macrocode}
%<*example>
\documentclass[british]{article}[2024/06/29]% v1.4n Standard LaTeX document class
\makeatletter
\usepackage[draft]{showkeys}[2024/05/23]% v3.21 Show cite and label keys (DPC, MH)
%%      Use final instead of draft to hide the keys. %%
\usepackage[pdfpagelabels=true,hyperindex=false]{hyperref}[2024-10-30]% v7.01k
\@ifpackageloaded{hyperref}{% Hypertext links for LaTeX
\hypersetup{extension=pdf,%
 plainpages=false,%
 pdflang={en},%
 pdftitle={lastpage package example},%
 pdfauthor={H.-Martin Muench},%
 pdfsubject={Example for the lastpage package},%
 pdfkeywords={LaTeX, lastpage},%
 pdfview=Fit,%
 pdfstartview=Fit,%
 pdfpagelayout=SinglePage%
}}{\usepackage{url}[2013/09/16]}%  v3.4  Verb mode for urls, etc.
\usepackage{lastpage}[2025/01/27]% v2.1e Refers to last page's name (HMM; JPG)]
\renewcommand{\@evenfoot}{{%
 \normalsize\slshape \today\hfil \upshape %
 page \thepage{} of \pageref{LastPage}}}
\renewcommand{\@oddfoot}{\@evenfoot}
\makeatother
\listfiles
\begin{document}
\pagenumbering{Roman}
\section*{Example for lastpage}
\markboth{Example for lastpage}{Example for lastpage}
This example demonstrates the use of package\newline
\textsf{lastpage}, v2.1e as of 2025-01-27 (HMM; JPG).\newline
The package takes no options.\newline
For more details please see the documentation!\newline

\noindent \label{keys} To hide the \pageref{keys}{\qquad } use option
\texttt{final} instead of \texttt{draft} with the \textsf{showkeys}
package (or remove the package call from the preamble of
this document).\newline

\textbf{Hyperlinks or not:} If the \textsf{hyperref} package is loaded,
the references are also hyperlinked:\newline
\smallskip
Last page's name (LastPage): \pageref{LastPage}\newline
\noindent If the \textsf{hyperref} package is loaded, but the hyperlinks
of the references shall be suppressed, \verb|\pageref*{...}|
can be used:\newline
\smallskip
Last page's name (LastPage): \pageref*{LastPage}\newline

\textbf{Trademarks} appear throughout this example without any
trademark symbol; they are the property of their respective
trademark owner. There is no intention of infringement; the
usage is to the benefit of the trademark owner.\newline

\textbf{Tip}: Use \textit{logical page numbers}
for the display of the pdf (in Adobe Acrobat Reader 2024.005.20392:
Edit $>$ Preferences $>$ Page Display $>$
Page Content and Information: Use logical page numbers)!\newline

If you are more ambitious in respect to your aims with this package,
you might want to have a look at the \textsf{pageslts} package:\newline
\url{https://ctan.org/pkg/pageslts}.
\bigskip

\noindent The page (\verb|\thepage|): \thepage \newline
Last page's name (LastPage): \pageref{LastPage}
\newpage

\noindent The page (\verb|\thepage|): \thepage \newline
Last page's name (LastPage): \pageref{LastPage}

\bigskip

\noindent There was the question:

\begin{quote}
\begin{verbatim}
\documentclass{article}
\usepackage{hyperref}
\usepackage{lastpage}
\begin{document}
\ifnum\thepage=\pageref{LastPage} foo \else bar \fi
\end{document}
\end{verbatim}

producing the error
\textquotedblleft missing number, treated as zero\textquotedblright.
\end{quote}

\noindent \verb|\pageref| inserts a hyperlink, \verb|\pageref{LastPage}|
is not expandable and the code breaks.\newline
The code does not generally work even without hyperref.

\begin{quote}
\begin{verbatim}
\documentclass{article}
\usepackage{hyperref}
\usepackage{lastpage}
\pagenumbering{Roman}
\begin{document}
\addtocounter{page}{8}
\edef\here{\thepage}
\makeatletter
\ifx\here\lastpage@lastpage\relax foo\else bar\fi
\makeatother
\end{document}
\end{verbatim}
\end{quote}

\noindent does work (two compilations needed), because \verb|\lastpage@lastpage|
contains the name of the page, \mbox{example:}
\begin{verbatim}
Page \thepage{} is (not) page
\makeatletter\lastpage@lastpage\makeatother.
\end{verbatim}
prints:\newline
Page \thepage{} is (not) page
\makeatletter\lastpage@lastpage\makeatother.
\newline
This can be broken for example by \verb|\pagenumbering{fnsymbol}|
(because then \verb|\edef\here{\thepage}| does not work).
\newpage

\noindent The page (\verb|\thepage|): \thepage\newline
Last page's name (LastPage): \pageref{LastPage}
\bigskip

With modern \LaTeX{} it is possible to say:
\begin{quote}
\begin{verbatim}
\NeedsTeXFormat{LaTeX2e}[2024-11-01]
\documentclass{article}
\pagenumbering{fnsymbol}
\begin{document}
\addtocounter{page}{8}%
\ExplSyntaxOn%
\xdef\test{\numexpr\the\g_shipout_readonly_int +1\relax}%
\ExplSyntaxOff%
\ifnum\PreviousTotalPages=\test\relax%
This is the last page.%
\else%
This is not the last page
(or it is but \LaTeX{} needs another compilation run
to detect this).
\fi
\end{document}
\end{verbatim}
\end{quote}

\newpage
\section*{The End}
\noindent The page (\verb|\thepage|): \thepage \newline
Last page's name (LastPage): \pageref{LastPage}
\bigskip

To see the content of the \texttt{enddocument/afterlastpage}-hook
(for a recent \LaTeX-format!) use % without the \verb||, of course!
\verb|\ShowHook{enddocument/afterlastpage}|.
\end{document}
%</example>
%    \end{macrocode}
% \StopEventually{}
% \newpage
%
% \section{The implementation}
% \DescribeMacro{lastpage.sty}
% We first need to determine whether we are on \TeX~2.09 or \LaTeX2e.\newline
% (That line, which is too long for the documentation, reads:\newline
% |\def\loadlastpage{\ProvidesPackage{lastpage}[2025/01/27 v2.1e lastpage:|\linebreak
% | 2.09 or 2e? (HMM)]\relax\RequirePackage{lastpage2e}}|.)
%
%    \begin{macrocode}
%<*package>
%% Part of the "lastpage" package
%% loads either lastpage2.09.sty for TeX 2.09 or lastpage2e.sty for LaTeX 2e
%% with code from https://groups.google.com/g/comp.text.tex/c/-Qmhj1ZI4xM
\def\loadlastpage{\ProvidesPackage{lastpage}[2025/01/27 v2.1e lastpage: 2.09 or 2e? (HMM)]\relax\RequirePackage{lastpage2e}}
\begingroup \expandafter \ifx \csname documentclass\endcsname\relax
\endgroup \expandafter \input{lastpage209.sty}
\else \endgroup \expandafter \loadlastpage
\fi
%</package>
%    \end{macrocode}
%
% \DescribeMacro{lastpage209.sty}
% If we are on \TeX~2.09 (really?!), we load the 2.09 version \pkg{lastpage209.sty}:
%
%    \begin{macrocode}
%<*lastpage209>
%% Part of the "lastpage" package
%% FOR LaTeX 2.09 ONLY - FOR LaTeX 2e USE lastpage2e.sty
%% This is lastpage209.sty invented by Jeffrey P. Goldberg,
%% after Piet van Oostrum: Page layout in LaTeX, March 2, 2004, section 16;
%% fancyhdr.pdf; lastpage209.sty maintained by H.-Martin Muench.
\let\origenddocument=\enddocument%
\def\enddocument{\clearpage%
  {\addtocounter{page}{-1}%
   \immediate\write\@mainaux{\string\newlabel{LastPage}{{}{\thepage}}}%
  }%
  \addtocounter{page}{+1}%
  \origenddocument%
 }
%</lastpage209>
%    \end{macrocode}
%
% \DescribeMacro{lastpage2e.sty}
% If |\documentclass| is known, we are in \LaTeXe{} -- but which one?
% For modern versions with \eTeX{} and hook management etc.
% we load version \pkg{lastpagemodern.sty}, otherwise
% \pkg{lastpageclassic.sty}. We start off by checking that we are
% loading into \LaTeXe{} and announcing the name and version of this package.
%
%    \begin{macrocode}
%<*lastpage2e>
%% Part of the "lastpage" package
\NeedsTeXFormat{LaTeX2e}[1994/12/01]
\ProvidesPackage{lastpage2e}[2025/01/27 v2.1e %
 Decide which 2e lastpage version to use (HMM)]
\@ifl@t@r\fmtversion{2024/06/01}{\RequirePackage{lastpagemodern}}{%
                                 \RequirePackage{lastpageclassic}}
\message{^^J}
%</lastpage2e>
%    \end{macrocode}
% \newpage
%
% \DescribeMacro{lastpageclassic.sty}
% In case of older \LaTeX-formats \pkg{lastpageclassic.sty} is loaded:
%
%    \begin{macrocode}
%<*lastpageclassic>
%% Part of the "lastpage" package
\NeedsTeXFormat{LaTeX2e}[1994/12/01]
\ProvidesPackage{lastpageclassic}[2025/01/27 v2.1e %
  Refers to last page's name (HMM; JPG)]
%% allows for things like "Page \thepage{} of \pageref{LastPage}"
%% to get "Page 7 of 9"

%    \end{macrocode}
%
% \indent For comparisons, \textquotedblleft one\textquotedblright{} is defined
% (|\@ne| does not work for this).
%
%    \begin{macrocode}
\gdef\lastpage@one{1}
%    \end{macrocode}
%
% We define |\lastpage@hyper|, |\lastpage@nameref|, |\lastpage@french|, and
% |\lastpage@LTS| to be \textquotedblleft \texttt{0}\textquotedblright{}.
%
%    \begin{macrocode}
\gdef\lastpage@hyper{0}
\gdef\lastpage@nameref{0}
\gdef\lastpage@french{0}
\gdef\lastpage@LTS{0}
%    \end{macrocode}
%
% We define |\lastpage@firstpage| to be \textquotedblleft \texttt{1}\textquotedblright{},
% and before re-definition via the \xfile{.aux} file,
% |\lastpage@lastpage| and |\lastpage@lastpageHy| are unknown.
%
%    \begin{macrocode}
\gdef\lastpage@firstpage{1}
\gdef\lastpage@lastpage{??}
\gdef\lastpage@lastpageHy{??}

%    \end{macrocode}
%
% \begin{macro}{\AtBeginDocument}
% \indent |\AtBeginDocument| we give a warning about ancient versions of the \pkg{endfloat} package.
% Then it is checked whether various packages are loaded.
% (|\@ifpackageloaded| cannot be used later than |\AtBeginDocument|.)
% If this is the case,\newline
% |\lastpage@<package abbreviation>| is defined as \texttt{1}
% (otherwise it stays~\texttt{0}).
%
%    \begin{macrocode}
\AtBeginDocument{%
  \@ifpackageloaded{endfloat}{%
    \@ifpackagelater{endfloat}{1993/04/30}{\relax}{%
      \PackageError{lastpage}{%
        Incompatibility with outdated version of endfloat package}{%
        lastpage is not fully compatible with a version\MessageBreak%
        before 2.1 of the endfloat package,\MessageBreak%
        because those versions redefined\MessageBreak%
        the \string\enddocument\space command.}%
    }}{}%
  \@ifpackageloaded{tikz}{\gdef\lastpage@tikz{1}}{}%
  \@ifpackageloaded{hyperref}{\gdef\lastpage@hyper{1}}{}%
  \@ifpackageloaded{nameref}{\gdef\lastpage@nameref{1}}{}%
  \@ifpackageloaded{french}{\gdef\lastpage@french{1}}{}%
  \@ifpackageloaded{frenchle}{\gdef\lastpage@french{1}}{}%
  \@ifpackageloaded{pagesLTS}{\gdef\lastpage@LTS{1}}{}%
  \@ifpackageloaded{pageslts}{\gdef\lastpage@LTS{1}}{}%
%    \end{macrocode}
%
% |\lastpage@putlabel|, used by older versions of this package,
% is redefined e.\,g.\ by \xpackage{revtex}, \xpackage{frenchle},
% \xpackage{PPRcorners}, and old versions of \xpackage{hyperref}.
% While now |\lastpage@putl@bel| is used instead, \xpackage{revtex}
% could also define a label \texttt{LastPage},
% which then would be multiply defined. (Which is no big issue,
% if it is associated with the same page.) Therefore we define
%
%    \begin{macrocode}
  \gdef\lastpage@putlabel{\relax}%
  }

%    \end{macrocode}
%
% \noindent Because |\lastpage@putlabel| might be (re)defined later,
% depending on the order in which the packages are loaded,
% we will do this again |\AtEndDocument|.
% \end{macro}
%
% \begin{macro}{\lastpage@putl@bel}
% \indent This command does the writing of the label:
%
%    \begin{macrocode}
\newcommand{\lastpage@putl@bel}{%
%    \end{macrocode}
%
% |\AtBeginDocument| it is checked whether the \xpackage{hyperref} package is loaded,\newline
% |\@ifpackageloaded{hyperref}{\gdef\lastpage@hyper{1}}{}|.\newline
% |\@ifpackageloaded| cannot be used later than |\AtBeginDocument|.\newline
% User \textsc{Sebastian Bank} found and reported (Thanks!) a~case, when this check is not
% sufficient. Using a class with\newline
% |\usepackage{lastpage}|\newline
% |\AtBeginDocument{\usepackage{hyperref}}|\newline
% leads to failed detection of the \xpackage{hyperref} package, because
% |\AtBeginDocument| \textit{first} the check for \xpackage{hyperref} is performed,
% and \textit{then} \xpackage{hyperref} is loaded. As mentioned above,
% |\@ifpackageloaded| cannot be used later, so here we do not check for the
% \xpackage{hyperref} package again, but for its |\Hy@Warning| command.
% In version~1.2c of the \xpackage{lastpage} package, it was checked for
% the |\hyperref| command, but as it turned out, \xpackage{tcilatex} \textit{is}
% defining that. If some other package or user is defining |\Hy@Warning|,
% \xpackage{lastpage} will falsely assume, that \xpackage{hyperref} has been loaded,
% but in my humble opinion, defining |\Hy@Warning| does not make sense and
% is bad style (except definition by the \xpackage{hyperref} package itself,
% of course).
%
%    \begin{macrocode}
  \@ifundefined{Hy@Warning}{%  hyperref not loaded
    }{\gdef\lastpage@hyper{1}% hyperref loaded
     }%
%    \end{macrocode}
%
% If the \xpackage{pageslts} package is used, this \xpackage{lastpage} package is
% not needed at all. The \xpackage{LastPage} label would even be defined twice.
% Thus, if \xpackage{pageslts} is used, here nothing is done:
%
%    \begin{macrocode}
  \ifx\lastpage@LTS\lastpage@one%
  \else%
%    \end{macrocode}
%
% Otherwise the label is set:\newline
% We have got to distinguish whether \xpackage{hyperref} has been loaded or not:
%
%    \begin{macrocode}
    \ifx\lastpage@hyper\lastpage@one%
      \lastpage@putlabelhyper%
    \else%
%    \end{macrocode}
%
% and also need to treat documents with \xpackage{nameref} differently:
%
%    \begin{macrocode}
      \ifx\lastpage@nameref\lastpage@one%
        \lastpage@putlabelNR%
      \else%
%    \end{macrocode}
%
% When those packages have not been loaded, we just write the
% simple label into the \xfile{aux} file (and store the value of the page):
%
%    \begin{macrocode}
        \begingroup%
          \addtocounter{page}{-1}%
          \immediate\write\@auxout{\string\newlabel{LastPage}{{}{\thepage}}}%
          \immediate\write\@auxout{\string\xdef\string\lastpage@lastpage{\thepage}}%
          \immediate\write\@auxout{\string\gdef\string\lastpage@lastpageHy{}}%
          \addtocounter{page}{+1}%
        \endgroup%
      \fi%
    \fi%
  \fi%
  }

%    \end{macrocode}
% \end{macro}
% \newpage
%
% \begin{macro}{\lastpage@putlabelhyper}%
% \indent When \xpackage{hyperref} has been loaded, the label is set with the
% |\lastpage@putlabelhyper| command. If the \xpackage{hyperref} package is used,
% but page-anchors are disabled, the hyperlinking will not work. (The warning will
% also be shown, when only |\pageref*{LastPage}| is used (or neither one),
% but without messing with |\pageref| we cannot detect this.)
%
%    \begin{macrocode}
\newcommand{\lastpage@putlabelhyper}{%
  \ifHy@pageanchor%
  \else%
    \PackageWarningNoLine{lastpage}{%
      The \string\pageref{LastPage} link does not work\MessageBreak%
      using hyperref with disabled option `pageanchor'.\MessageBreak%
      Better enable `pageanchor' or use\MessageBreak%
      \string\pageref*{LastPage} (not generating a link)%
      }%
  \fi%
%    \end{macrocode}
%
% Since the page has been put out, we are on the page \textit{after} that page.
% We therefore subtract one from the page counter. (For the compiler,
% this is equal to |\advance\c@page\m@ne|, but for human readers of the code
% it is probably easier to understand.)
%
%    \begin{macrocode}
  \begingroup%
    \addtocounter{page}{-1}%
%    \end{macrocode}
%
% Simply using |\label| for \texttt{LastPage} would not work,
% because labels wait for the output routines to work, and there
% may be no more invocations of the output routines. To force
% the write out, we need to do an |\immediate| write into the \xfile{aux} file.
%
%    \begin{macrocode}
%% with code from \Hy@EveryPageAnchor of the hyperref package,
%% 2010/04/17 v6.80x; newer versions are available
    \let\@number\@firstofone%
    \ifHy@pageanchor%
      \ifHy@hypertexnames%
        \ifHy@plainpages%
          \def\Hy@temp{\arabic{page}}%
        \else%
          \Hy@unicodefalse%
          \ifnum \value{page}=1\relax%
%    \end{macrocode}
%
% We do not count the pages ourselves, and so they could have been changed by
% e.\,g.\ |\pagenumbering{...}|, |\addtocounter{page}{...}|,\newline
% |\setcounter{page}{...}|. Thus the page might have the number one
% while not being the first page at all. Using the \xpackage{everyshi}
% package would help, but this package should not require other packages.
% The \xpackage{pageslts} package does a better handling.\newline
% We will make a mistake here at most once:
%
%    \begin{macrocode}
            \ifx \lastpage@firstpage\lastpage@one\relax%
              \def\Hy@temp{\thepage}%
              \gdef\lastpage@firstpage{0}%
            \else%
                \pdfstringdef\Hy@temp{\thepage}%
          \fi%
          \else%
            \pdfstringdef\Hy@temp{\thepage}%
          \fi%
        \fi%
      \else%
        \def\Hy@temp{\the\Hy@pagecounter}%
      \fi%
    \fi%
    \immediate\write\@auxout{%
      \string\newlabel{LastPage}{{}{\thepage}{}{%
          \ifHy@pageanchor page.\Hy@temp\fi}{}}%
    }%
%    \end{macrocode}
%
% We also save the values, so that we can later (next rerun) check,
% whether they have been saved in the \xfile{aux} file.
%
%    \begin{macrocode}
    \immediate\write\@auxout{\string\xdef\string\lastpage@lastpage{\thepage}}%
    \ifHy@pageanchor%
      \immediate\write\@auxout{\string\xdef\string\lastpage@lastpageHy{\Hy@temp}}%
    \else%
      \immediate\write\@auxout{\string\gdef\string\lastpage@lastpageHy{}}%
    \fi%
%    \end{macrocode}
%
% After the writeout we restore the page number again,
% since there might be other things still to be done.
%
%    \begin{macrocode}
    \addtocounter{page}{+1}%
  \endgroup%
  }

%    \end{macrocode}
% \end{macro}
%
% \begin{macro}{\lastpage@putlabelNR}
% \indent The \xpackage{nameref} package redefines |\label| to have five arguments
% instead of two, therefore\newline
% |\newlabel{LastPage}{{}{\thepage}{}{}{}}|
% instead of\newline
% |\newlabel{LastPage}{{}{\thepage}}| must be used:
%
%    \begin{macrocode}
\newcommand{\lastpage@putlabelNR}{%
  \begingroup%
    \addtocounter{page}{-1}%
    \immediate\write\@auxout{\string\newlabel{LastPage}{{}{\thepage}{}{}{}}}%
    \immediate\write\@auxout{\string\xdef\string\lastpage@lastpage{\thepage}}%
    \immediate\write\@auxout{\string\gdef\string\lastpage@lastpageHy{}}%
    \addtocounter{page}{+1}%
  \endgroup%
  }

%    \end{macrocode}
% \end{macro}
%
% \begin{macro}{\lastpage@fileswtest}
% \indent Later it will be determined whether it is allowed to write
% to the \xfile{aux} file. If it was \emph{not} allowed, it is checked
% whether the label was already set via the \xfile{aux} file in some
% earlier compilation run. (There are packages where the document
% is compiled with access to the \xfile{aux} file, and then there is
% an additional compiler run, where the \xfile{aux} file cannot be changed,
% but in that run there is also no need to change it.) The \xpackage{tikz}
% package is somewhat different, therefore we only give a warning instead
% of an error (and hope that there is another compiler run where the
% \xfile{aux} file can be written).
%
%    \begin{macrocode}
\newcommand{\lastpage@fileswtest}[2]{%
  \edef\lastpage@testa{#1}%
  \edef\lastpage@testb{#2}%
  \ifx\lastpage@testa\lastpage@testb%
  \else%
    \ifx\lastpage@tikz\lastpage@one\relax%
      \PackageWarning{lastpage}%
       {The lastpage package was not allowed to write to an\MessageBreak%
        .aux file. This package does not work without access\MessageBreak%
        to an .aux file.\MessageBreak%
        It is OK if the .aux file was already updated\MessageBreak%
        by a previous compiler run\MessageBreak%
        and would not have changed anyway.\MessageBreak%
       }%
    \else%
      \PackageError{lastpage}{No auxiliary file allowed}%
       {The lastpage package was not allowed to write to an .aux file.\MessageBreak%
        This package does not work without access to an .aux file.\MessageBreak%
        Press Ctrl+Z to exit.\MessageBreak%
        But it is OK if the .aux file was already updated\MessageBreak%
        by a previous compiler run\MessageBreak%
        and would not have changed anyway.}%
    \fi%
  \fi%
  }

%    \end{macrocode}
% \end{macro}
%
% \begin{macro}{\lastpage@fileswtestHy}
% \indent When the \xpackage{hyperref} package has been loaded,
% |\lastpage@lastpageHy| must be tested additionally. (And a
% |\newcommand| is needed, because |\ifHy@pageanchor| is not even
% defined when \xpackage{hyperref} has not been loaded.)
%
%    \begin{macrocode}
\newcommand{\lastpage@fileswtestHy}{%
  \ifHy@pageanchor%
    \lastpage@fileswtest{\Hy@temp}{\lastpage@lastpageHy}%
  \else%
    \lastpage@fileswtest{\empty}{\lastpage@lastpageHy}%
  \fi%
  }
%    \end{macrocode}
% \end{macro}
%
% \begin{macro}{\AtEndDocument}
% \indent |\AtEndDocument| we again (re)define |\lastpage@putlabel|
% to do nothing and check |\lastpage@lastpage|,
% whether it is still unchanged, which is OK for the first run only.
%
%    \begin{macrocode}

\AtEndDocument{%
  \ifx\lastpage@LTS\lastpage@one%
  \else%
    \gdef\lastpage@putlabel{??}%
    \ifx\lastpage@lastpage\lastpage@putlabel\relax%
      \PackageWarning{lastpage}{Rerun to get the references right}%
    \fi%
  \fi%
  \gdef\lastpage@putlabel{\relax}%
%    \end{macrocode}
%
% It is checked whether writing to files is allowed
% (otherwise, only an error message is issued and nothing is done).
%
%    \begin{macrocode}
  \if@filesw%
%    \end{macrocode}
%
% We put in a |\message| to show, in what order things (which were called)
% are done (see subsection~\ref{ssec:aed}).
%
%    \begin{macrocode}
    \message{^^JAED: lastpage setting LastPage^^J}%
%    \end{macrocode}
%
% After this we issue a |\clearpage| to put out all floats,
% which are still floating, and place the \texttt{LastPage} label.
% Sometimes |\clearpage| might be undefined.
%
%    \begin{macrocode}
    \@ifundefined{clearpage}{\relax}{\clearpage}%
    \ifx\lastpage@french\lastpage@one% french or frenchle loaded
      \addtocounter{page}{+1}\lastpage@putl@bel\addtocounter{page}{-1}%
    \else% neither one loaded
      \lastpage@putl@bel%
    \fi%
%    \end{macrocode}
%
% When writing to files is not allowed, nothing can be done. But when
% the label was already set via the \xfile{aux} file, nothing needs
% to be done. We check for this with |\lastpage@fileswtest| and
% (if \xpackage{hyperref} has been loaded) |\lastpage@fileswtestHy|.
%
%    \begin{macrocode}
  \else%
    \ifx\lastpage@LTS\lastpage@one%
    \else%
      \lastpage@fileswtest{\thepage}{\lastpage@lastpage}%
      \ifx\lastpage@hyper\lastpage@one%
        \lastpage@fileswtestHy%
      \fi%
    \fi%
  \fi%
  }
%    \end{macrocode}
% \end{macro}
%    \begin{macrocode}
%</lastpageclassic>
%    \end{macrocode}
% \newpage
%
% \DescribeMacro{lastpagemodern.sty}
% In case of a recent \LaTeX-format, \pkg{lastpagemodern.sty} is loaded:
%
%    \begin{macrocode}
%<*lastpagemodern>
%% Part of the "lastpage" package
\NeedsTeXFormat{LaTeX2e}[2024-11-01]
\ProvidesPackage{lastpagemodern}[2025-01-27 v2.1e %
  Refers to last page's name (HMM; JPG)]
%% allows for things like "Page \thepage{} of \pageref{LastPage}"
%% to get "Page 7 of 9" or "Page VII of IX";
%% the NUMBER of pages is available via \@abspage@last, but with pages
%% for example I to X and then 1 to 10, the number of pages would be "20",
%% while the name of the last page is "10". Decide what you need/want!

%    \end{macrocode}
%
% \indent For comparisons, \textquotedblleft one\textquotedblright{} is defined
% (|\@ne| does not work for this).
%
%    \begin{macrocode}
\gdef\lastpage@one{1}
%    \end{macrocode}
%
% We define |\lastpage@firstpage| to be \textquotedblleft \texttt{1}\textquotedblright{},
% and before re-definition via the \xfile{.aux} file,
% |\lastpage@lastpage| and |\lastpage@lastpageHy| are unknown.
%
%    \begin{macrocode}
\gdef\lastpage@firstpage{1}
\gdef\lastpage@lastpage{??}
\gdef\lastpage@lastpageHy{??}

%    \end{macrocode}
%
% \begin{macro}{\lastpage@IfNumericTF}
% |\lastpage@IfNumericTF| was needed to determine whether a page number was numeric or not.
% Although it is no longer used, I am reluctant to remove it because it may have been used
% in other places in documents.
%
%    \begin{macrocode}
\newcommand\lastpage@gobbleminus[1]{\ifx-#1\else#1\fi}
%    \end{macrocode}
% which is from \url{https://texfaq.org/FAQ-isitanum},
%    \begin{macrocode}
\newcounter{lastpagecount}
%    \end{macrocode}
% and similar to \url{https://tex.stackexchange.com/a/12811/17119}
%    \begin{macrocode}
\newcommand{\lastpage@IfNumericTF}[3]{%
\sbox\z@{\c@lastpagecount=0\lastpage@gobbleminus{#1}\relax}%
\ifdim\wd0>\z@\relax#3% is not numeric
\else#2% is numeric
\fi}

%    \end{macrocode}
% \end{macro}
%
% \DescribeMacro{\AddToHook\{begindocument/end\}}
% \indent |\lastpage@putlabel|, used by older versions of this package,
% is/was redefined by other packages.
% While now |\lastpage@putl@bel| is used instead, other packages
% could also define a label \texttt{LastPage},
% which then would be multiply defined. (Which is no big issue,
% if it is associated with the same page.) Therefore we define
%
%    \begin{macrocode}
\AddToHook{begindocument/end}{%
  \IfPackageLoadedT{pageslts}{%
    \PackageNoteNoLine{lastpage}{Packages pageslts and lastpage used.\MessageBreak%
      lastpage is not necessary when loading pageslts}%
  }%
  \gdef\lastpage@putlabel{\relax}%
  }

%    \end{macrocode}
%
% \begin{macro}{\protected@iwrite}
% \indent We need an |\immediate\protected@write|. Just |\immediate\write| had led to errors,
% for example when packages like \pkg{babel-greek} re-defined |\roman|\ %
% (thanks to \textsc{Ulrike Fischer} for the report).
%
%    \begin{macrocode}
%% Code provided by Prof. Enrico Gregorio at https://tex.stackexchange.com/a/542425
\long\def\protected@iwrite#1#2#3{%
  \begingroup%
  #2%
  \let\protect\@unexpandable@protect%
  \edef\reserved@a{\immediate\write#1{#3}}%
  \reserved@a%
  \endgroup%
  \if@nobreak\ifvmode\nobreak\fi\fi%
  }

%    \end{macrocode}
% \end{macro}
%
% \begin{macro}{\lastpage@makeHy}
% Just once we need the page from |\@currentHpage| without any \mbox{\textquotedblleft |page.|\textquotedblright :}
%    \begin{macrocode}
\newcommand{\lastpage@makeHy}{%
%    \end{macrocode}
% |\gdef\lastpage@Hy{}|, but that was already done before this command.
%    \begin{macrocode}
  \def\lastpage@Hptest{Doc-Start}%
  \ifx\lastpage@Hptest\@currentHpage\relax%
    \gdef\lastpage@Hy{\@currentHpage}%
  \else%
    \edef\lastpage@Hptest{\@currentHpage}%
    \ifx\lastpage@Hptest\empty\relax%
%    \end{macrocode}
% |\gdef\lastpage@Hy{}|, but that was already done before this command.
%    \begin{macrocode}
    \else%
      \def\lastpage@Hptest{page.}%
      \ifx\lastpage@Hptest\@currentHpage\relax
        \def\lastpage@Hptest{\csname @fnsymbol\endcsname \c@page }%
        \ifx\lastpage@Hptest\thepage\relax%
          \ifnum\value{page}=3\else%
            \PackageWarningNoLine{lastpage}{You should add a\MessageBreak
              \string\ProvideTextCommand{...}{PD1}{...}\MessageBreak%
              (see the lastpage package manual, 3.13 %
              \string\pagenumbering{fnsymbol})\MessageBreak%
              to your document's preamble}%
%    \end{macrocode}
% See \autoref{subsec:fnsymbol}: \nameref{subsec:fnsymbol}, page~\pageref{subsec:fnsymbol}.
%    \begin{macrocode}
          \fi%
        \fi%
        \PackageWarningNoLine{lastpage}{%
          \string\@currentHpage\space is\MessageBreak%
          just "page." without number,\MessageBreak%
          \string\lastpage@lastpageHy\space is now let empty}%
%    \end{macrocode}
% |\gdef\lastpage@Hy{}|, but that was already done before this command.
%    \begin{macrocode}
      \else%
%    \end{macrocode}
% |\@currentHpage| should be |page.<some number>|, |\lastpage@rmpage| removes the
% \mbox{\textquotedblleft |page.|\textquotedblright .} Next compilation run,
% |\lastpage@lastpageHy| gets defined via the \xfile{aux} file.
% If we arrived at this place, but the definition is still empty,
% then |\@currentHpage| has some unexpected content.
%    \begin{macrocode}
        \gdef\lastpage@Hy{\lastpage@rmpage{\@currentHpage}}%
        \ifx\lastpage@lastpageHy\empty\relax%
          \PackageWarningNoLine{lastpage}{%
            \string\@currentHpage\space is\MessageBreak%
            \meaning\@currentHpage\MessageBreak%
            not beginning with "page.",\MessageBreak%
            \string\lastpage@lastpageHy\space is now let empty}%
  \fi\fi\fi\fi%
  }

%    \end{macrocode}
% \end{macro}
% \begin{macro}{\lastpage@rmpage}
% |\lastpage@rmpage| removes the \mbox{\textquotedblleft |page.|\textquotedblright.}
%
%    \begin{macrocode}
%% Code provided by David Carlisle at https://tex.stackexchange.com/a/721877
\def\lastpage@rmpage#1{%
  \expandafter\xlastpage@rmpage\expanded{#1}\xlastpage@rmpage page.%
  \xlastpage@rmpage\xxlastpage@rmpage{#1}}
\def\xlastpage@rmpage #1page.#2\xlastpage@rmpage#3\xxlastpage@rmpage#4{%
  \if$\detokenize{#1}$#2%\else#4
  \fi}

%    \end{macrocode}
% |\else#4| means, that it did not start with |page.|, and whatever it is,
% we cannot use this |#4| for |\lastpage@Hy|.
% \end{macro}
%
% \begin{macro}{\lastpage@putl@bel}
% \indent This command does the writing of the label.
% If the \xpackage{hyperref} package is used, but page-anchors are disabled,
% the hyperlinking will not work. (The warning will also be shown, when only
% |\pageref*{LastPage}| is used (or neither |\pageref{LastPage}| nor |\pageref*{LastPage}|),
% but without messing with |\pageref| we cannot detect this.)
%
%    \begin{macrocode}
\newcommand{\lastpage@putl@bel}{%
  \IfPackageLoadedT{hyperref}{%
    \IfPackageAtLeastF{hyperref}{2024-10-30}{%
      \PackageError{lastpage}{hyperref package version too old}{%
        required version: 2024-10-30 or newer, found version:\MessageBreak%
        \csname ver@hyperref.sty\endcsname\MessageBreak%
        Update hyperref or use lastpageclassic.sty instead of\MessageBreak%
        lastpagemodern.sty!}}%
    \ifHy@pageanchor\else%
      \PackageWarningNoLine{lastpage}{%
        The \string\pageref{LastPage} link does not work\MessageBreak%
        using hyperref with disabled option `pageanchor'.\MessageBreak%
        Better enable `pageanchor' or use\MessageBreak%
        \string\pageref*{LastPage} (not generating a link)}%
    \fi%
   }%
  \begingroup%
%    \end{macrocode}
%
% \noindent Since the page has been put out, we are on the page \textit{after} that page.
% We therefore subtract one from the page counter. (For the compiler,
% this is equal to |\advance\c@page\m@ne|, but for human readers of the code
% it is probably easier to understand.)
%
%    \begin{macrocode}
    \addtocounter{page}{-1}%
%    \end{macrocode}
%
% If the \xpackage{pageslts} package is used, this \xpackage{lastpage} package
% is not needed at all. The \xpackage{LastPage} label would even be defined twice.
% Thus, if \xpackage{pageslts} is used, here nothing is done.
%
%    \begin{macrocode}
    \IfPackageLoadedTF{pageslts}{% then pageslts writes the label for  "LastPage".
%    \end{macrocode}
%
% \noindent Simply using |\label| for \texttt{LastPage} would not work,
% because labels wait for the output routines to work, and there
% may be no more invocations of the output routines.
% To force the write out, we need to do an |\immediate| protected write
% into the \xfile{aux} file.
%
%    \begin{macrocode}
     }{\protected@iwrite\@auxout{}{\string\newlabel{LastPage}{%
         {\@currentlabel}{\thepage}{\@currentlabelname}%
         {\IfPackageLoadedTF{hyperref}{\ifHy@pageanchor\@currentHpage\fi%
                                      }{\@currentHref}}%
         {\@kernel@reserved@label@data}}%
       }%
      }%
%    \end{macrocode}
% \noindent We also save the values, so that we can later (next rerun) check,
% whether they have been saved in the \xfile{aux} file.
%
%    \begin{macrocode}
    \protected@iwrite\@auxout{}{%
      \string\gdef\string\lastpage@lastpage{\thepage}}%
    \gdef\lastpage@Hy{}%
    \IfPackageLoadedT{hyperref}{\ifHy@pageanchor\lastpage@makeHy\fi}%
    \protected@iwrite\@auxout{}{%
      \string\gdef\string\lastpage@lastpageHy{\lastpage@Hy}}%
%    \end{macrocode}
%
% \noindent After the write-out we restore the page number again,
% since there might be other things still to be done.
%
%    \begin{macrocode}
    \addtocounter{page}{+1}%
  \endgroup%
  }

%    \end{macrocode}
% \end{macro}
%
% \begin{macro}{\lastpage@fileswtest}
% \indent \hspace*{2em}Later it will be determined whether it is allowed to write
% to the \xfile{aux} file. If it was \emph{not} allowed, it is checked
% whether the label was already set via the \xfile{aux} file in some
% earlier compilation run. (There are packages where the document
% is compiled with access to the \xfile{aux} file, and then there is
% an additional compiler run, where the \xfile{aux} file cannot be changed,
% but in that run there is also no need to change it.) The \xpackage{tikz}
% package is somewhat different, therefore we only give a warning instead
% of an error (and hope that there is another compiler run where the
% \xfile{aux} file can be written).
%
%    \begin{macrocode}
\newcommand{\lastpage@fileswtest}[2]{%
  \edef\lastpage@testa{#1}%
  \edef\lastpage@testb{#2}%
  \ifx\lastpage@testa\lastpage@testb%
  \else%
    \IfPackageLoadedTF{tikz}{%
      \PackageWarning{lastpage}%
       {The lastpage package was not allowed to write to an\MessageBreak%
        .aux file. This package does not work without access\MessageBreak%
        to an .aux file.\MessageBreak%
        It is OK if the .aux file was already updated\MessageBreak%
        by a previous compiler run\MessageBreak%
        and would not have changed anyway.\MessageBreak%
       }%
    }{\PackageError{lastpage}{No auxiliary file allowed}%
       {The lastpage package was not allowed to write to an .aux file.\MessageBreak%
        This package does not work without access to an .aux file.\MessageBreak%
        Press Ctrl+Z to exit.\MessageBreak%
        But it is OK if the .aux file was already updated\MessageBreak%
        by a previous compiler run\MessageBreak%
        and would not have changed anyway.\MessageBreak%
       }%
     }%
  \fi%
  }

%    \end{macrocode}
% \end{macro}
%
% \begin{macro}{\lastpage@fileswtestHy}
% \indent \hspace*{2em}When the \xpackage{hyperref} package has been loaded,
% |\lastpage@lastpageHy| must be tested additionally. (And a
% |\newcommand| is needed, because |\ifHy@pageanchor| is not even
% defined when \xpackage{hyperref} has not been loaded.)
%
%    \begin{macrocode}
\newcommand{\lastpage@fileswtestHy}{%
  \ifHy@pageanchor%
    \lastpage@fileswtest{\@currentHpage}{\@the@H@page}%
  \else%
    \lastpage@fileswtest{\empty}{\lastpage@lastpageHy}%
  \fi%
  }

%    \end{macrocode}
% \end{macro}
%
% \DescribeMacro{enddocument/afterlastpage}
% \indent |enddocument/afterlastpage| we again (re)define |\lastpage@putlabel|
% to do nothing, but first use it to check whether |\lastpage@lastpage|
% is still unchanged, which is OK for the first run only.
%
%    \begin{macrocode}
\AddToHook{enddocument/afterlastpage}{%
  \gdef\lastpage@putlabel{??}%
  \ifx\lastpage@lastpage\lastpage@putlabel\relax%
    \AddToHook{enddocument/info}{%
      \PackageWarning{lastpage}{Rerun to get the references right}%
      }%
  \fi%
  \gdef\lastpage@putlabel{\relax}%
%    \end{macrocode}
%
% It is checked whether writing to files is allowed
% (otherwise, only an error message is issued and nothing is done).
%
%    \begin{macrocode}
  \if@filesw%
%    \end{macrocode}
%
% We put in a |\message| to show, in what order things (which were called) are done.
%
%    \begin{macrocode}
    \message{^^Jenddocument/afterlastpage (AED): lastpage setting LastPage.^^J}%
    \IfPackageLoadedTF{french}{%
         \addtocounter{page}{+1}\lastpage@putl@bel\addtocounter{page}{-1}%
     }{\IfPackageLoadedTF{frenchle}{%
         \addtocounter{page}{+1}\lastpage@putl@bel\addtocounter{page}{-1}%
       }{\lastpage@putl@bel%
        }%
      }%
  \else%
%    \end{macrocode}
%
% When writing to files is not allowed, nothing can be done. But when
% the label was already set via the \xfile{aux} file, nothing needs
% to be done. We check for this with |\lastpage@fileswtest| and
% (if \xpackage{hyperref} has been loaded) |\lastpage@fileswtestHy|.
%
%    \begin{macrocode}
    \lastpage@fileswtest{\thepage}{\lastpage@lastpage}%
    \IfPackageLoadedT{hyperref}{\lastpage@fileswtestHy}%
  \fi%
  }
%</lastpagemodern>
%    \end{macrocode}
% \newpage
%
% \section{Installation}
% \subsection{Downloads\label{ss:Downloads}}
% Everything is available at \url{https://www.ctan.org},
% but may need additional packages themselves.
%
% \DescribeMacro{lastpage.dtx}
% For unpacking the |lastpage.dtx| file and constructing the documentation it is required:
% \begin{description}
% \item[-] \TeX Format \LaTeXe{}: \url{https://www.CTAN.org}
%
% \item[-] document class \xclass{ltxdoc}, 2024/02/08, v2.1j,
%   \url{https://ctan.org/pkg/ltxdoc}
%
% \item[-] package \xpackage{holtxdoc}, 2019/12/09, v0.30,
%   \url{https://ctan.org/pkg/holtxdoc}
% \end{description}
%
% \DescribeMacro{lastpage.sty}
% The |lastpage.sty| (i.\,e.\ each document using the \xpackage{lastpage} package)
% requires:
% \begin{description}
% \item[-] \TeX, \url{https://www.CTAN.org}
%
% \item[-] package \xpackage{lastpage}, 2025-01-27, v2.1e,
%   \url{https://ctan.org/pkg/lastpage}
% \end{description}
%
% \DescribeMacro{lastpage209.sty}
% The |lastpage209.sty| for \LaTeX2.09{} (i.\,e.\ each document using
% the \xpackage{lastpage209} package) requires:
% \begin{description}
% \item[-] \TeX-format \LaTeX{}, v2.09
%
% \item[-] package \xpackage{lastpage209}, 2025-01-27, v2.1e,
%   \url{https://ctan.org/pkg/lastpage}
% \end{description}
% and does not work with \xpackage{hyperref}, which needs \LaTeX2e{}.\smallskip
%
% \DescribeMacro{lastpage2e.sty}
% The |lastpage2e.sty| for \LaTeXe{} (i.\,e.\ each document using
% the \xpackage{lastpage2e} package) requires:
% \begin{description}
% \item[-] \TeX-format \LaTeXe{} 1994/12/01 or newer,
%   \url{https://www.CTAN.org}
%
% \item[-] package \xpackage{lastpage}, 2025-01-27, v2.1e,
%   \url{https://ctan.org/pkg/lastpage}
% \end{description}
%
% \DescribeMacro{lastpageclassic.sty}
% The |lastpageclassic.sty| for \LaTeXe{} (i.\,e.\ each document using
% the \xpackage{lastpageclassic} package) requires:
% \begin{description}
% \item[-] \TeX-format \LaTeXe{} between 1994/12/01 and 2024-05-31,\newline
%   \url{https://www.CTAN.org}
%
% \item[-] package \xpackage{lastpage}, 2025-01-27, v2.1e,
%   \url{https://ctan.org/pkg/lastpage}
% \end{description}
% and can use
% \begin{description}
% \item[-] package \xpackage{hyperref}, 2023-07-08, v7.01b,
%   \url{https://ctan.org/pkg/hyperref}\newline
%   (probably also some older and newer versions)
% \end{description}
%
% \DescribeMacro{lastpagemodern.sty}
% The |lastpagemodern.sty| for \LaTeXe{} (i.\,e.\ each document using
% the \xpackage{lastpagemodern} package) requires:
% \begin{description}
% \item[-] \TeX-format \LaTeXe{} 2024-11-01 or newer,\newline
%   \url{https://www.CTAN.org}
%
% \item[-] package \xpackage{lastpage}, 2025-01-27, v2.1e,
%   \url{https://ctan.org/pkg/lastpage}
% \end{description}
% and can use
% \begin{description}
% \item[-] package \xpackage{hyperref}, probably 2023-11-07 and newer
%   (tested with: 2024-11-05 v7.01l),
%   \url{https://ctan.org/pkg/hyperref}
% \end{description}
% \pagebreak
%
% \DescribeMacro{lastpage-example.tex}
% The |lastpage-example.tex| requires the same file as all
% documents using the \xpackage{lastpage} package, i.\,e.
% \begin{description}
% \item[-] package \xpackage{lastpage}, 2025-01-27, v2.1e,
%   \url{https://ctan.org/pkg/lastpage}\newline
%   (Well, it is the example file for this package, and because you are reading the
%    documentation for the \xpackage{lastpage} package, it can be assumed that you already
%    have some version of it -- is it the current one?)
% \end{description}
% and additionally:
% \begin{description}
% \item[-] class \xpackage{article}, 2024/06/29, v1.4n, from \xpackage{classes}:
%           \url{https://ctan.org/pkg/classes}
%
% \item[-] package \xpackage{showkeys}, 2024-05-23, v3.21,
%   \url{https://ctan.org/pkg/showkeys}
%
% \item[-] package \xpackage{hyperref}, 2024-11-05, v7.01l,
%   \url{https://ctan.org/pkg/hyperref}
% \end{description}
%
% \DescribeMacro{Münch}
% A hyperlinked list of my (other) packages can be found at\newline
% \url{https://ctan.org/author/muench-hm}.
%
% \subsection{Package, unpacking TDS}
% \paragraph{Package.} This package is available on \url{https://www.CTAN.org}.
% \begin{description}
% \item[\url{https://mirror.ctan.org/macros/latex/contrib/lastpage/lastpage.dtx}]\hspace*{0.1cm} \newline
%       The source file.
% \item[\url{https://mirror.ctan.org/macros/latex/contrib/lastpage/lastpage.pdf}]\hspace*{0.1cm} \newline
%       The documentation.
% \item[\url{https://mirror.ctan.org/macros/latex/contrib/lastpage/lastpage-example.pdf}]\hspace*{0.1cm}
%       The compiled example file, as it should look like.
% \item[\url{https://mirror.ctan.org/macros/latex/contrib/lastpage/README}]\hspace*{0.1cm} \newline
%       The README file.
% \end{description}
% There is also a |lastpage.tds.zip| available:
% \begin{description}
% \item[\url{https://mirror.ctan.org/install/macros/latex/contrib/lastpage.tds.zip}]\hspace*{0.1cm}
%       Everything in TDS compliant, compiled format.
% \end{description}
% which additionally contains\newline
% \begin{tabular}{ll}
% lastpage.ins         & The \texttt{ins}tallation file.\\
% lastpage.drv         & The \mbox{\texttt{dr}i\texttt{v}er} to generate the documentation.\\
% lastpage.sty         & The \texttt{sty}le file.\\
% lastpage209.sty      & The \texttt{sty}le file for \LaTeX2.09{} \textbf{only}.\\
% lastpage2e.sty       & The \texttt{sty}le file to determine which 2e-style to use.\\
% lastpageclassic.sty  & The \texttt{sty}le file for older \LaTeX-formats.\\
% lastpagemodern.sty   & The \texttt{sty}le file for the recent \LaTeX-format.\\
% lastpage-example.tex & The example file.
% \end{tabular}
%
% \bigskip
%
% \noindent For required other packages please see the preceding subsection.
%
% \paragraph{Unpacking.} The \xfile{.dtx} file is a self-extracting
% \docstrip{} archive. The files are extracted by running the
% \xext{.dtx} through \plainTeX :
% \begin{quote}
%   \verb|tex lastpage.dtx|
% \end{quote}
%
% \noindent About generating the documentation see paragraph~\ref{GenDoc} below.\par
%
% \paragraph{TDS.} Now the different files must be moved into
% the different directories in your installation TDS tree
% (also known as \xfile{texmf} tree):
% \begin{quote}
% \def\t{^^A
% \begin{tabular}{@{}>{\ttfamily}l@{ $\rightarrow$ }>{\ttfamily}l@{}}
%   lastpage.sty & tex/latex/lastpage.sty\\
%   lastpage209.sty & tex/latex/lastpage.sty\\
%   lastpage2e.sty & tex/latex/lastpage.sty\\
%   lastpageclassic.sty & tex/latex/lastpage.sty\\
%   lastpagemodern.sty & tex/latex/lastpage.sty\\
%   lastpage.pdf & doc/latex/lastpage.pdf\\
%   lastpage-example.tex & doc/latex/lastpage-example.tex\\
%   lastpage-example.pdf & doc/latex/lastpage-example.pdf\\
%   lastpage.dtx & source/latex/lastpage.dtx
% \end{tabular}^^A
% }^^A
% \sbox0{\t}^^A
% \ifdim\wd0>\linewidth
%   \begingroup
%     \advance\linewidth by\leftmargin
%     \advance\linewidth by\rightmargin
%   \edef\x{\endgroup
%     \def\noexpand\lw{\the\linewidth}^^A
%   }\x
%   \def\lwbox{^^A
%     \leavevmode
%     \hbox to \linewidth{^^A
%       \kern-\leftmargin\relax
%       \hss
%       \usebox0
%       \hss
%       \kern-\rightmargin\relax
%     }^^A
%   }^^A
%   \ifdim\wd0>\lw
%     \sbox0{\small\t}^^A
%     \ifdim\wd0>\linewidth
%       \ifdim\wd0>\lw
%         \sbox0{\footnotesize\t}^^A
%         \ifdim\wd0>\linewidth
%           \ifdim\wd0>\lw
%             \sbox0{\scriptsize\t}^^A
%             \ifdim\wd0>\linewidth
%               \ifdim\wd0>\lw
%                 \sbox0{\tiny\t}^^A
%                 \ifdim\wd0>\linewidth
%                   \lwbox
%                 \else
%                   \usebox0
%                 \fi
%               \else
%                 \lwbox
%               \fi
%             \else
%               \usebox0
%             \fi
%           \else
%             \lwbox
%           \fi
%         \else
%           \usebox0
%         \fi
%       \else
%         \lwbox
%       \fi
%     \else
%       \usebox0
%     \fi
%   \else
%     \lwbox
%   \fi
% \else
%   \usebox0
% \fi
% \end{quote}
% If you have a \xfile{docstrip.cfg} that configures and enables \docstrip's
% TDS installing feature, then some files can already be in the right
% place, see the documentation of \docstrip.
%
% \subsection{Refresh file name databases}
% If your \TeX~distribution (\TeX\,Live, \mikTeX, \dots) relies on
% file name databases, you must refresh these. For example, \TeX\,Live\ users
% run \verb|texhash| or \verb|mktexlsr|.
%
% \subsection{Some details for the interested}
% \paragraph{Unpacking with \LaTeX.}
% The \xfile{.dtx} chooses its action depending on the format:
% \begin{description}
% \item[\plainTeX:] Run \docstrip\ and extract the files.
% \item[\LaTeX:] Generate the documentation.
% \end{description}
% If you insist on using \LaTeX\ for \docstrip\ (really,
% \docstrip\ does not need \LaTeX ), then inform the autodetect routine
% about your intention:
% \begin{quote}
%   \verb|latex \let\install=y\input{lastpage.dtx}|
% \end{quote}
% Do not forget to quote the argument according to the demands
% of your shell.
%
% \paragraph{Generating the documentation.\label{GenDoc}}
% You can use both the \xfile{.dtx} or the \xfile{.drv} to generate
% the documentation. The process can be configured by a
% configuration file \xfile{ltxdoc.cfg}. For instance, put the following
% line into this file, if you want to have A4 as paper format:
% \begin{quote}
%   \verb|\PassOptionsToClass{a4paper}{article}|
% \end{quote}
%
% \noindent An example follows how to generate the
% documentation with \pdfLaTeX :
%
% \begin{quote}
%\begin{verbatim}
%pdflatex lastpage.dtx
%makeindex -s gind.ist lastpage.idx
%pdflatex lastpage.dtx
%makeindex -s gind.ist lastpage.idx
%pdflatex lastpage.dtx
%\end{verbatim}
% \end{quote}
%
% \subsection{Compiling the example}
% The example file, \textsf{lastpage-example.tex}, can be compiled via\newline
% \indent |latex lastpage-example.tex|\newline
% or (recommended)\newline
% \indent |pdflatex lastpage-example.tex|\newline
% and will need at least two compiler runs to get all references right.
%
% \section{Acknowledgements}
%
% I (\textsc{H.-Martin Münch}) would like to thank
% \textsc{Jeffrey P. Goldberg} \mbox{(jeffrey+}\allowbreak news at goldmark dot org) for
% inventing the \xpackage{lastpage} package as well as for allowing me
% to update it. Further I would like to thank \textsc{Heiko Oberdiek}
% for providing a~lot~(!) of useful packages (from which I also learned everything
% I know about creating a file in \xfile{dtx} format, OK, say it: copying).
% Thanks to \textsc{David Carlisle} for the new code for |\lastpage@rmpage|.
% Thanks for bug reports go to \textsc{Ulrike Fischer} (several times),
% \textsc{Sebastian Bank}, \textsc{James Hedges}, \textsc{Mikhail Titov},
% \textsc{Micha\l{} Herman}, \textsc{Matteo Gamboz}, and \textsc{James Small}.
% Thanks to \textsc{Sven Siegmund} for pointing out a necessary further
% explanation in the documentation.
%
% \phantomsection
% \begin{History}\label{History}
%   \begin{Version}{1994/06/17 v0.99a}
%     \item First shot by \textsc{Jeffrey P. Goldberg}.
%   \end{Version}
%   \begin{Version}{1994/06/25 v0.1b}
%     \item Last version number created by \textsc{Jeffrey P. Goldberg}.
%   \end{Version}
%   \begin{Version}{1994/07/20 v0.1b (again)}
%     \item Documentation updated by \textsc{Jeffrey P. Goldberg}.\newline
%             The main source code of the \xpackage{lastpage} package 1994/07/20,
%             v0.1b, was:
%             \begin{verbatim}
%              \NeedsTeXFormat{LaTeX2e}[1994/06/01]
%              \ProvidesPackage{lastpage}[1994/07/20 v0.1b
%                LaTeX2e package for refs to last page number (JPG)]
%              \def\lastpage@putlabel{\addtocounter{page}{-1}%
%                \immediate\write\@auxout{\string
%                \newlabel{LastPage}{{}{\thepage}}}%
%                \addtocounter{page}{1}}
%              \AtEndDocument{%
%                \message{AED: lastpage setting LastPage}%
%                \clearpage\lastpage@putlabel}%
%              \endinput
%             \end{verbatim}
%             and then the \xpackage{hyperref} package (2012/11/06, v6.83m)
%             even redefined |\lastpage@putlabel|; the \xpackage{REV\TeX{}4}
%             class 2022-06-05, v4.2f, still does this.
%   \end{Version}
%   \begin{Version}{2010/02/18 v1.1}
%     \item Proposed |LastPages| label by \textsc{H.-Martin Münch}
%             on \Newsgroup{comp.text.tex}, see e.\,g.\ %
%             \url{https://groups.google.com/g/comp.text.tex/c/Ad8pO2Rw_HY/m/8EfHqT1JB0QJ};
%             now available in the \xpackage{pageslts} package.
%   \end{Version}
%   \begin{Version}{2010/07/29 v1.2a}
%     \item Complete rewriting of the package; upgrade from \xpackage{fancyheadings}
%             to \xpackage{fancyhdr} package, then removed the need for the
%             \xpackage{fancyhdr} package at all.
%     \item Included \textsf{lastpage209.sty} for \LaTeX2.09{}.
%     \item Replacement of |\filedate|, |-version|, |-name|,\ldots{} because
%             of \LaTeX{}~bug 2705:\newline
%             Synopsis: Possible problem with |\fileversion| and |\filedate|\newline
%             \url{https://www.latex-project.org/cgi-bin/ltxbugs2html?category=LaTeX&responsible=anyone&state=anything&keyword=lastpage&pr=latex/2705}
%     \item Example |lastpage-example.tex|.
%     \item Alternatives listing (section \ref{sec:Alternatives}).
%     \item Listing of \TeX{} sources (subsection \ref{ss:Downloads}).
%     \item Really a lot of details.
%     \item Complete rewriting of the documentation.
%     \item Everything in \texttt{DTX} framework.
%     \item Included a |\CheckSum|. [Removed in v2.0a.]
%     \item Complete rewriting of the README file.
%   \end{Version}
%   \begin{Version}{2010/08/12 v1.2b}
%     \item Bug fix: |\@PackageInfoNoLine| is only available,
%             if the \xpackage{hyperref} package is loaded.
%             (Bug reported by \textsc{Ulrike Fischer}, thanks!)
%     \item Bug fix: |\ifHy@pageanchor| etc. do not work without
%             \xpackage{hyperref}, and |\else| related to |\ifHy@pageanchor| was
%             wrongly associated with a preceding |\if|, and everything went wrong.
%             Now everything should work again also without \xpackage{hyperref}.
%     \item Renamed |\lastpage@putlabel| to |\lastpage@putl@bel| to get rid of
%             the conflicts with other classes and packages and resulting
%             multiple definitions of the \texttt{lastpage} label.
%   \end{Version}
%   \begin{Version}{2010/08/23 v1.2c}
%     \item Bug fix: Additionally to checking for the \xpackage{hyperref} package
%             |\AtBeginDocument|, when placing the \texttt{lastpage} label it is also
%             checked for the |\hyperref| command, in case \xpackage{hyperref} was not
%             loaded at |\begin{document}| yet. (Bug reported by \textsc{Sebastian Bank},
%             thanks!)\newline
%             [\pkg{lastpagemodern.sty} just uses |\IfPackageLoadedT{hyperref}| and
%              |\IfPackageLoadedTF{hyperref}|.]
%     \item Changed the |\unit| definition (got rid of an old |\rm|). [Removed in v2.0a.]
%     \item Changed |\lastpage@puthyperlabel| to |\lastpage@putlabelhyper| analogous to
%             |\pagesLTS@putlabelhyper| of the \xpackage{pageslts} package.
%     \item Updated version number and date of \xpackage{pagesLTS} package (especially
%             for the check for outdated versions). [Removed in v2.0a.]
%     \item Removed wrong \%\ from the driver file.
%   \end{Version}
%   \begin{Version}{2010/08/25 v1.2d}
%     \item Bug fix: also \xpackage{tcilatex} defines the |\hyperref| command,
%             therefore for \xpackage{hyperref} package detection this had to be
%             changed to |\Hy@Warning|.\newline
%             [\pkg{lastpagemodern.sty} just uses |\IfPackageLoadedTF{hyperref}|.]
%   \end{Version}
%   \begin{Version}{2010/09/12 v1.2e}
%     \item \textsc{James Hedges} pointed out, that there was no instruction in the
%             documentation about suppressing hyperlinks: added (also to the example).
%     \item Diverse small changes.
%   \end{Version}
%   \begin{Version}{2010/09/24 v1.2f}
%     \item Updated to version 2010/09/13 v6.81n of the \xpackage{hyperref} package.
%     \item New version of REV\TeX{}4\ 2010/07/25, v4.1r, old problem.
%     \item New version of \xpackage{pagesLTS} package, 2010/09/22, v1.1k.
%     \item Moved the package from \texttt{.../latex/muench/lastpage/...} to\newline
%             \texttt{.../latex/lastpage/...}.
%   \end{Version}
%   \begin{Version}{2011/02/01 v1.2g}
%     \item Updated to version 2010/04/24 v0.19 of the \xpackage{holtxdoc} package.
%     \item New version of \xpackage{pagesLTS} package, 2011/02/01, v1.1m.
%     \item Updated to version 2010/12/16 v6.81z of the \xpackage{hyperref} package.
%     \item Minor details.
%   \end{Version}
%   \begin{Version}{2011/07/03 v1.2h}
%     \item The \xpackage{holtxdoc} package was fixed, therefore the warning
%             in \xfile{drv} could be removed.~-- Adapted the style of this documentation
%             to new \textsc{Oberdiek} \xfile{dtx} style.
%     \item New versions of \xpackage{pagesLTS}, \xpackage{ulem}, \xpackage{hyperref},
%             \xpackage{papermas} packages.
%     \item Corrected references in the README and manual.
%   \end{Version}
%   \begin{Version}{2011/08/08 v1.2i}
%     \item The \xpackage{pagesLTS} package has been renamed to \xpackage{pageslts}:
%             2011/08/08, v1.2a.
%     \item Some details.
%   \end{Version}
%   \begin{Version}{2011/08/31 v1.2j}
%     \item Updated to \TeX{} Live 2011 (for compiling the documentation and example).
%     \item New version of \xpackage{papermas} package, 2011/08/22, v1.0h.
%     \item Adapted for the use together with packages, which sometimes prevent writing
%             to the \xfile{aux} file. (Bug reported by \textsc{Mikhail Titov}.)
%   \end{Version}
%   \begin{Version}{2011/09/01 v1.2k}
%     \item Fixed |\thepage{}| to \mbox{|\thepage{}|\textvisiblespace,}
%             where there should be a space.
%     \item New version of the \xpackage{hyperref} package, 2011/08/19, v6.82h,
%             but still problem with links to pages with
%             page-\textquotedblleft number\textquotedblright{} in \texttt{fnsymbol}
%             pagenumbering scheme. [Fixed since v6.83m as of 2012/11/06.]
%     \item Documentation update about \textquotedblleft No write access to the
%             \xfile{aux} file\textquotedblright .
%   \end{Version}
%   \begin{Version}{2013/01/28 v1.2l}
%     \item Updated to \TeX{} Live 2012 (for compiling the documentation and example).
%     \item New versions of the packages
%             \xpackage{endfloat}, \xpackage{holtxdoc}, \xpackage{hypdoc},
%             \xpackage{hyperref}, \xpackage{pageslts}, \xpackage{regstats},
%             \xpackage{ulem}, and \xpackage{zref} have become available.
%     \item The \xpackage{nameref} package redefines |\label| to have five arguments
%           instead of two, therefore |\newlabel{LastPage}{{}{\thepage}{}{}{}}| instead of
%           |\newlabel{LastPage}{{}{\thepage}}| must be used.
%           (Bug reported at \url{https://tex.stackexchange.com/q/95541},
%           thanks to \textsc{Micha\l{} Herman}!) Fixed.\newline
%           [Since \LaTeX{} release 2023-06-01 five arguments are standard.]
%   \end{Version}
%   \begin{Version}{2015/03/29 v1.2m}
%     \item Updated to \TeX{} Live 2014 (for compiling the documentation and example).
%     \item Updates to really a lot of details in the documentation (manual \& README).
%   \end{Version}
%   \begin{Version}{2021/09/03 v1.2n}
%     \item Updates to the documentation (manual \& README), to the example, and
%             several small changes in code.
%   \end{Version}
%   \begin{Version}{2023-03-07 v2.0a}
%     \item Removed use of \pkg{ulem}.
%     \item Removed |\unit|.
%     \item \pkg{lastpage} should now determine automatically, whether to load
%             its \TeX~2.09 version, classic \LaTeX2e-version, or modern version
%             with \eTeX, hook{-}man\-age\-ment etc.
%     \item Converted to UTF-8.
%     \item Updated to [then] current \LaTeX{} format 2022-11-01.
%     \item Extensive updates to the documentation (manual \& README) and to the example.
%   \end{Version}
%   \begin{Version}{2023-04-12 v2.0b}
%     \item Bug fix: What should have been |}}{}| was |}{}}|.
%             Thanks to \textsc{Matteo Gamboz} for the bug report!
%   \end{Version}
%   \begin{Version}{2023-07-24 v2.0c}
%     \item Since \LaTeX{} release 2023-06-01 labels have always five arguments.
%     \item Removed the incompatibility warning regarding \pkg{endfloat}
%             before 1994-06-01~(!) from \pkg{lastpagemodern.sty}.
%     \item Replaced |\immediate\write| by a form of (pseudo-code!) |\immediate\protected@write|.
%             (Thanks to \textsc{Ulrike Fischer} for the error report and solution!)
%     \item Update of the |\lastpage@putl@bel| code to \pkg{hyperref} 2023-07-08, v7.01b,
%             which now allows fnsymbol as page numbering scheme.
%   \end{Version}
%   \begin{Version}{2023-10-06 v2.0d}
%     \item For plain arabic page numbers only, \pkg{lastpage} again writes the page number
%             in a plain format in the label in the \xfile{aux} file, so that it can be extracted
%             to perform calculations with it.
%   \end{Version}
%   \begin{Version}{2023-10-14 v2.0e}
%     \item Replaced a |\PackageError| by |\PackageWarningNoLine| as suggested by \textsc{Ulrike Fischer}.
%   \end{Version}
%   \begin{Version}{2024-04-27 v2.1a}
%     \item Utilizes the new |\@currentHpage| provided by the \LaTeX -kernel.
%     \item The issue with |\thepage| (former |\lastpage@nonnumeric| command) has has been fixed
%             at 2023-11-07, \url{https://github.com/latex3/hyperref/issues/303}.
%     \item |\lastpage@IfNumericTF| is no longer needed, but has not been removed
%             so as not to break older documents that use it in a different context.
%   \end{Version}
%   \begin{Version}{2024-07-03 v2.1b}
%     \item The \xclass{beamer} class loads \xpackage{hyperref} partially,
%             so that \xpackage{hyperref} is incorrectly detected as having been loaded.
%             The defining operation for |\lastpage@Hy| has been changed to catch this
%             and other different cases.
%   \end{Version}
%   \begin{Version}{2024-07-07 v2.1c}
%     \item With help from \textsc{David Carlisle} |\lastpage@rmpage| no longer assumes
%             |\@currentHpage| to begin with \mbox{\textquotedblleft |page.|\textquotedblright .}
%   \end{Version}
%   \begin{Version}{2024-11-24 v2.1d}
%     \item The \xpackage{pageslts} package has been repaired, thus here the warnings have been removed.
%     \item Several small changes in documentation and |lastpagemodern.sty| because of the updates
%             of \LaTeX-format (to~2024-06-01), \xpackage{hyperref} package (to~2024-10-30, v7.01k),
%             and \xpackage{pageslts} package (to~2024-11-20, v2.0a).
%     \item Added a warning message about missing\newline
%             |\ProvideTextCommand{...}{PD1}{...}|\newline
%             (cf.~\autoref{subsec:fnsymbol}: \nameref{subsec:fnsymbol}, page~\pageref{subsec:fnsymbol}).
%     \item Documentation section about alternatives rewritten.
%   \end{Version}
%   \begin{Version}{2025-01-27 v2.1e}
%     \item Reorganized beginning of |\lastpage@makeHy|.
%     \item Documentation update.
%   \end{Version}
% \end{History}
% \bigskip
%
% When you find a mistake or have a suggestion for an improvement of this package,
% please send an e-mail to the maintainer, thanks! (Please see BUG REPORTS in the README.)
% \newpage
% \PrintIndex
% \Finale
\endinput|
% \end{quote}
% Do not forget to quote the argument according to the demands
% of your shell.
%
% \paragraph{Generating the documentation.\label{GenDoc}}
% You can use both the \xfile{.dtx} or the \xfile{.drv} to generate
% the documentation. The process can be configured by a
% configuration file \xfile{ltxdoc.cfg}. For instance, put the following
% line into this file, if you want to have A4 as paper format:
% \begin{quote}
%   \verb|\PassOptionsToClass{a4paper}{article}|
% \end{quote}
%
% \noindent An example follows how to generate the
% documentation with \pdfLaTeX :
%
% \begin{quote}
%\begin{verbatim}
%pdflatex lastpage.dtx
%makeindex -s gind.ist lastpage.idx
%pdflatex lastpage.dtx
%makeindex -s gind.ist lastpage.idx
%pdflatex lastpage.dtx
%\end{verbatim}
% \end{quote}
%
% \subsection{Compiling the example}
% The example file, \textsf{lastpage-example.tex}, can be compiled via\newline
% \indent |latex lastpage-example.tex|\newline
% or (recommended)\newline
% \indent |pdflatex lastpage-example.tex|\newline
% and will need at least two compiler runs to get all references right.
%
% \section{Acknowledgements}
%
% I (\textsc{H.-Martin Münch}) would like to thank
% \textsc{Jeffrey P. Goldberg} \mbox{(jeffrey+}\allowbreak news at goldmark dot org) for
% inventing the \xpackage{lastpage} package as well as for allowing me
% to update it. Further I would like to thank \textsc{Heiko Oberdiek}
% for providing a~lot~(!) of useful packages (from which I also learned everything
% I know about creating a file in \xfile{dtx} format, OK, say it: copying).
% Thanks to \textsc{David Carlisle} for the new code for |\lastpage@rmpage|.
% Thanks for bug reports go to \textsc{Ulrike Fischer} (several times),
% \textsc{Sebastian Bank}, \textsc{James Hedges}, \textsc{Mikhail Titov},
% \textsc{Micha\l{} Herman}, \textsc{Matteo Gamboz}, and \textsc{James Small}.
% Thanks to \textsc{Sven Siegmund} for pointing out a necessary further
% explanation in the documentation.
%
% \phantomsection
% \begin{History}\label{History}
%   \begin{Version}{1994/06/17 v0.99a}
%     \item First shot by \textsc{Jeffrey P. Goldberg}.
%   \end{Version}
%   \begin{Version}{1994/06/25 v0.1b}
%     \item Last version number created by \textsc{Jeffrey P. Goldberg}.
%   \end{Version}
%   \begin{Version}{1994/07/20 v0.1b (again)}
%     \item Documentation updated by \textsc{Jeffrey P. Goldberg}.\newline
%             The main source code of the \xpackage{lastpage} package 1994/07/20,
%             v0.1b, was:
%             \begin{verbatim}
%              \NeedsTeXFormat{LaTeX2e}[1994/06/01]
%              \ProvidesPackage{lastpage}[1994/07/20 v0.1b
%                LaTeX2e package for refs to last page number (JPG)]
%              \def\lastpage@putlabel{\addtocounter{page}{-1}%
%                \immediate\write\@auxout{\string
%                \newlabel{LastPage}{{}{\thepage}}}%
%                \addtocounter{page}{1}}
%              \AtEndDocument{%
%                \message{AED: lastpage setting LastPage}%
%                \clearpage\lastpage@putlabel}%
%              \endinput
%             \end{verbatim}
%             and then the \xpackage{hyperref} package (2012/11/06, v6.83m)
%             even redefined |\lastpage@putlabel|; the \xpackage{REV\TeX{}4}
%             class 2022-06-05, v4.2f, still does this.
%   \end{Version}
%   \begin{Version}{2010/02/18 v1.1}
%     \item Proposed |LastPages| label by \textsc{H.-Martin Münch}
%             on \Newsgroup{comp.text.tex}, see e.\,g.\ %
%             \url{https://groups.google.com/g/comp.text.tex/c/Ad8pO2Rw_HY/m/8EfHqT1JB0QJ};
%             now available in the \xpackage{pageslts} package.
%   \end{Version}
%   \begin{Version}{2010/07/29 v1.2a}
%     \item Complete rewriting of the package; upgrade from \xpackage{fancyheadings}
%             to \xpackage{fancyhdr} package, then removed the need for the
%             \xpackage{fancyhdr} package at all.
%     \item Included \textsf{lastpage209.sty} for \LaTeX2.09{}.
%     \item Replacement of |\filedate|, |-version|, |-name|,\ldots{} because
%             of \LaTeX{}~bug 2705:\newline
%             Synopsis: Possible problem with |\fileversion| and |\filedate|\newline
%             \url{https://www.latex-project.org/cgi-bin/ltxbugs2html?category=LaTeX&responsible=anyone&state=anything&keyword=lastpage&pr=latex/2705}
%     \item Example |lastpage-example.tex|.
%     \item Alternatives listing (section \ref{sec:Alternatives}).
%     \item Listing of \TeX{} sources (subsection \ref{ss:Downloads}).
%     \item Really a lot of details.
%     \item Complete rewriting of the documentation.
%     \item Everything in \texttt{DTX} framework.
%     \item Included a |\CheckSum|. [Removed in v2.0a.]
%     \item Complete rewriting of the README file.
%   \end{Version}
%   \begin{Version}{2010/08/12 v1.2b}
%     \item Bug fix: |\@PackageInfoNoLine| is only available,
%             if the \xpackage{hyperref} package is loaded.
%             (Bug reported by \textsc{Ulrike Fischer}, thanks!)
%     \item Bug fix: |\ifHy@pageanchor| etc. do not work without
%             \xpackage{hyperref}, and |\else| related to |\ifHy@pageanchor| was
%             wrongly associated with a preceding |\if|, and everything went wrong.
%             Now everything should work again also without \xpackage{hyperref}.
%     \item Renamed |\lastpage@putlabel| to |\lastpage@putl@bel| to get rid of
%             the conflicts with other classes and packages and resulting
%             multiple definitions of the \texttt{lastpage} label.
%   \end{Version}
%   \begin{Version}{2010/08/23 v1.2c}
%     \item Bug fix: Additionally to checking for the \xpackage{hyperref} package
%             |\AtBeginDocument|, when placing the \texttt{lastpage} label it is also
%             checked for the |\hyperref| command, in case \xpackage{hyperref} was not
%             loaded at |\begin{document}| yet. (Bug reported by \textsc{Sebastian Bank},
%             thanks!)\newline
%             [\pkg{lastpagemodern.sty} just uses |\IfPackageLoadedT{hyperref}| and
%              |\IfPackageLoadedTF{hyperref}|.]
%     \item Changed the |\unit| definition (got rid of an old |\rm|). [Removed in v2.0a.]
%     \item Changed |\lastpage@puthyperlabel| to |\lastpage@putlabelhyper| analogous to
%             |\pagesLTS@putlabelhyper| of the \xpackage{pageslts} package.
%     \item Updated version number and date of \xpackage{pagesLTS} package (especially
%             for the check for outdated versions). [Removed in v2.0a.]
%     \item Removed wrong \%\ from the driver file.
%   \end{Version}
%   \begin{Version}{2010/08/25 v1.2d}
%     \item Bug fix: also \xpackage{tcilatex} defines the |\hyperref| command,
%             therefore for \xpackage{hyperref} package detection this had to be
%             changed to |\Hy@Warning|.\newline
%             [\pkg{lastpagemodern.sty} just uses |\IfPackageLoadedTF{hyperref}|.]
%   \end{Version}
%   \begin{Version}{2010/09/12 v1.2e}
%     \item \textsc{James Hedges} pointed out, that there was no instruction in the
%             documentation about suppressing hyperlinks: added (also to the example).
%     \item Diverse small changes.
%   \end{Version}
%   \begin{Version}{2010/09/24 v1.2f}
%     \item Updated to version 2010/09/13 v6.81n of the \xpackage{hyperref} package.
%     \item New version of REV\TeX{}4\ 2010/07/25, v4.1r, old problem.
%     \item New version of \xpackage{pagesLTS} package, 2010/09/22, v1.1k.
%     \item Moved the package from \texttt{.../latex/muench/lastpage/...} to\newline
%             \texttt{.../latex/lastpage/...}.
%   \end{Version}
%   \begin{Version}{2011/02/01 v1.2g}
%     \item Updated to version 2010/04/24 v0.19 of the \xpackage{holtxdoc} package.
%     \item New version of \xpackage{pagesLTS} package, 2011/02/01, v1.1m.
%     \item Updated to version 2010/12/16 v6.81z of the \xpackage{hyperref} package.
%     \item Minor details.
%   \end{Version}
%   \begin{Version}{2011/07/03 v1.2h}
%     \item The \xpackage{holtxdoc} package was fixed, therefore the warning
%             in \xfile{drv} could be removed.~-- Adapted the style of this documentation
%             to new \textsc{Oberdiek} \xfile{dtx} style.
%     \item New versions of \xpackage{pagesLTS}, \xpackage{ulem}, \xpackage{hyperref},
%             \xpackage{papermas} packages.
%     \item Corrected references in the README and manual.
%   \end{Version}
%   \begin{Version}{2011/08/08 v1.2i}
%     \item The \xpackage{pagesLTS} package has been renamed to \xpackage{pageslts}:
%             2011/08/08, v1.2a.
%     \item Some details.
%   \end{Version}
%   \begin{Version}{2011/08/31 v1.2j}
%     \item Updated to \TeX{} Live 2011 (for compiling the documentation and example).
%     \item New version of \xpackage{papermas} package, 2011/08/22, v1.0h.
%     \item Adapted for the use together with packages, which sometimes prevent writing
%             to the \xfile{aux} file. (Bug reported by \textsc{Mikhail Titov}.)
%   \end{Version}
%   \begin{Version}{2011/09/01 v1.2k}
%     \item Fixed |\thepage{}| to \mbox{|\thepage{}|\textvisiblespace,}
%             where there should be a space.
%     \item New version of the \xpackage{hyperref} package, 2011/08/19, v6.82h,
%             but still problem with links to pages with
%             page-\textquotedblleft number\textquotedblright{} in \texttt{fnsymbol}
%             pagenumbering scheme. [Fixed since v6.83m as of 2012/11/06.]
%     \item Documentation update about \textquotedblleft No write access to the
%             \xfile{aux} file\textquotedblright .
%   \end{Version}
%   \begin{Version}{2013/01/28 v1.2l}
%     \item Updated to \TeX{} Live 2012 (for compiling the documentation and example).
%     \item New versions of the packages
%             \xpackage{endfloat}, \xpackage{holtxdoc}, \xpackage{hypdoc},
%             \xpackage{hyperref}, \xpackage{pageslts}, \xpackage{regstats},
%             \xpackage{ulem}, and \xpackage{zref} have become available.
%     \item The \xpackage{nameref} package redefines |\label| to have five arguments
%           instead of two, therefore |\newlabel{LastPage}{{}{\thepage}{}{}{}}| instead of
%           |\newlabel{LastPage}{{}{\thepage}}| must be used.
%           (Bug reported at \url{https://tex.stackexchange.com/q/95541},
%           thanks to \textsc{Micha\l{} Herman}!) Fixed.\newline
%           [Since \LaTeX{} release 2023-06-01 five arguments are standard.]
%   \end{Version}
%   \begin{Version}{2015/03/29 v1.2m}
%     \item Updated to \TeX{} Live 2014 (for compiling the documentation and example).
%     \item Updates to really a lot of details in the documentation (manual \& README).
%   \end{Version}
%   \begin{Version}{2021/09/03 v1.2n}
%     \item Updates to the documentation (manual \& README), to the example, and
%             several small changes in code.
%   \end{Version}
%   \begin{Version}{2023-03-07 v2.0a}
%     \item Removed use of \pkg{ulem}.
%     \item Removed |\unit|.
%     \item \pkg{lastpage} should now determine automatically, whether to load
%             its \TeX~2.09 version, classic \LaTeX2e-version, or modern version
%             with \eTeX, hook{-}man\-age\-ment etc.
%     \item Converted to UTF-8.
%     \item Updated to [then] current \LaTeX{} format 2022-11-01.
%     \item Extensive updates to the documentation (manual \& README) and to the example.
%   \end{Version}
%   \begin{Version}{2023-04-12 v2.0b}
%     \item Bug fix: What should have been |}}{}| was |}{}}|.
%             Thanks to \textsc{Matteo Gamboz} for the bug report!
%   \end{Version}
%   \begin{Version}{2023-07-24 v2.0c}
%     \item Since \LaTeX{} release 2023-06-01 labels have always five arguments.
%     \item Removed the incompatibility warning regarding \pkg{endfloat}
%             before 1994-06-01~(!) from \pkg{lastpagemodern.sty}.
%     \item Replaced |\immediate\write| by a form of (pseudo-code!) |\immediate\protected@write|.
%             (Thanks to \textsc{Ulrike Fischer} for the error report and solution!)
%     \item Update of the |\lastpage@putl@bel| code to \pkg{hyperref} 2023-07-08, v7.01b,
%             which now allows fnsymbol as page numbering scheme.
%   \end{Version}
%   \begin{Version}{2023-10-06 v2.0d}
%     \item For plain arabic page numbers only, \pkg{lastpage} again writes the page number
%             in a plain format in the label in the \xfile{aux} file, so that it can be extracted
%             to perform calculations with it.
%   \end{Version}
%   \begin{Version}{2023-10-14 v2.0e}
%     \item Replaced a |\PackageError| by |\PackageWarningNoLine| as suggested by \textsc{Ulrike Fischer}.
%   \end{Version}
%   \begin{Version}{2024-04-27 v2.1a}
%     \item Utilizes the new |\@currentHpage| provided by the \LaTeX -kernel.
%     \item The issue with |\thepage| (former |\lastpage@nonnumeric| command) has has been fixed
%             at 2023-11-07, \url{https://github.com/latex3/hyperref/issues/303}.
%     \item |\lastpage@IfNumericTF| is no longer needed, but has not been removed
%             so as not to break older documents that use it in a different context.
%   \end{Version}
%   \begin{Version}{2024-07-03 v2.1b}
%     \item The \xclass{beamer} class loads \xpackage{hyperref} partially,
%             so that \xpackage{hyperref} is incorrectly detected as having been loaded.
%             The defining operation for |\lastpage@Hy| has been changed to catch this
%             and other different cases.
%   \end{Version}
%   \begin{Version}{2024-07-07 v2.1c}
%     \item With help from \textsc{David Carlisle} |\lastpage@rmpage| no longer assumes
%             |\@currentHpage| to begin with \mbox{\textquotedblleft |page.|\textquotedblright .}
%   \end{Version}
%   \begin{Version}{2024-11-24 v2.1d}
%     \item The \xpackage{pageslts} package has been repaired, thus here the warnings have been removed.
%     \item Several small changes in documentation and |lastpagemodern.sty| because of the updates
%             of \LaTeX-format (to~2024-06-01), \xpackage{hyperref} package (to~2024-10-30, v7.01k),
%             and \xpackage{pageslts} package (to~2024-11-20, v2.0a).
%     \item Added a warning message about missing\newline
%             |\ProvideTextCommand{...}{PD1}{...}|\newline
%             (cf.~\autoref{subsec:fnsymbol}: \nameref{subsec:fnsymbol}, page~\pageref{subsec:fnsymbol}).
%     \item Documentation section about alternatives rewritten.
%   \end{Version}
%   \begin{Version}{2025-01-27 v2.1e}
%     \item Reorganized beginning of |\lastpage@makeHy|.
%     \item Documentation update.
%   \end{Version}
% \end{History}
% \bigskip
%
% When you find a mistake or have a suggestion for an improvement of this package,
% please send an e-mail to the maintainer, thanks! (Please see BUG REPORTS in the README.)
% \newpage
% \PrintIndex
% \Finale
\endinput|
% \end{quote}
% Do not forget to quote the argument according to the demands
% of your shell.
%
% \paragraph{Generating the documentation.\label{GenDoc}}
% You can use both the \xfile{.dtx} or the \xfile{.drv} to generate
% the documentation. The process can be configured by a
% configuration file \xfile{ltxdoc.cfg}. For instance, put the following
% line into this file, if you want to have A4 as paper format:
% \begin{quote}
%   \verb|\PassOptionsToClass{a4paper}{article}|
% \end{quote}
%
% \noindent An example follows how to generate the
% documentation with \pdfLaTeX :
%
% \begin{quote}
%\begin{verbatim}
%pdflatex lastpage.dtx
%makeindex -s gind.ist lastpage.idx
%pdflatex lastpage.dtx
%makeindex -s gind.ist lastpage.idx
%pdflatex lastpage.dtx
%\end{verbatim}
% \end{quote}
%
% \subsection{Compiling the example}
% The example file, \textsf{lastpage-example.tex}, can be compiled via\newline
% \indent |latex lastpage-example.tex|\newline
% or (recommended)\newline
% \indent |pdflatex lastpage-example.tex|\newline
% and will need at least two compiler runs to get all references right.
%
% \section{Acknowledgements}
%
% I (\textsc{H.-Martin Münch}) would like to thank
% \textsc{Jeffrey P. Goldberg} \mbox{(jeffrey+}\allowbreak news at goldmark dot org) for
% inventing the \xpackage{lastpage} package as well as for allowing me
% to update it. Further I would like to thank \textsc{Heiko Oberdiek}
% for providing a~lot~(!) of useful packages (from which I also learned everything
% I know about creating a file in \xfile{dtx} format, OK, say it: copying).
% Thanks to \textsc{David Carlisle} for the new code for |\lastpage@rmpage|.
% Thanks for bug reports go to \textsc{Ulrike Fischer} (several times),
% \textsc{Sebastian Bank}, \textsc{James Hedges}, \textsc{Mikhail Titov},
% \textsc{Micha\l{} Herman}, \textsc{Matteo Gamboz}, and \textsc{James Small}.
% Thanks to \textsc{Sven Siegmund} for pointing out a necessary further
% explanation in the documentation.
%
% \phantomsection
% \begin{History}\label{History}
%   \begin{Version}{1994/06/17 v0.99a}
%     \item First shot by \textsc{Jeffrey P. Goldberg}.
%   \end{Version}
%   \begin{Version}{1994/06/25 v0.1b}
%     \item Last version number created by \textsc{Jeffrey P. Goldberg}.
%   \end{Version}
%   \begin{Version}{1994/07/20 v0.1b (again)}
%     \item Documentation updated by \textsc{Jeffrey P. Goldberg}.\newline
%             The main source code of the \xpackage{lastpage} package 1994/07/20,
%             v0.1b, was:
%             \begin{verbatim}
%              \NeedsTeXFormat{LaTeX2e}[1994/06/01]
%              \ProvidesPackage{lastpage}[1994/07/20 v0.1b
%                LaTeX2e package for refs to last page number (JPG)]
%              \def\lastpage@putlabel{\addtocounter{page}{-1}%
%                \immediate\write\@auxout{\string
%                \newlabel{LastPage}{{}{\thepage}}}%
%                \addtocounter{page}{1}}
%              \AtEndDocument{%
%                \message{AED: lastpage setting LastPage}%
%                \clearpage\lastpage@putlabel}%
%              \endinput
%             \end{verbatim}
%             and then the \xpackage{hyperref} package (2012/11/06, v6.83m)
%             even redefined |\lastpage@putlabel|; the \xpackage{REV\TeX{}4}
%             class 2022-06-05, v4.2f, still does this.
%   \end{Version}
%   \begin{Version}{2010/02/18 v1.1}
%     \item Proposed |LastPages| label by \textsc{H.-Martin Münch}
%             on \Newsgroup{comp.text.tex}, see e.\,g.\ %
%             \url{https://groups.google.com/g/comp.text.tex/c/Ad8pO2Rw_HY/m/8EfHqT1JB0QJ};
%             now available in the \xpackage{pageslts} package.
%   \end{Version}
%   \begin{Version}{2010/07/29 v1.2a}
%     \item Complete rewriting of the package; upgrade from \xpackage{fancyheadings}
%             to \xpackage{fancyhdr} package, then removed the need for the
%             \xpackage{fancyhdr} package at all.
%     \item Included \textsf{lastpage209.sty} for \LaTeX2.09{}.
%     \item Replacement of |\filedate|, |-version|, |-name|,\ldots{} because
%             of \LaTeX{}~bug 2705:\newline
%             Synopsis: Possible problem with |\fileversion| and |\filedate|\newline
%             \url{https://www.latex-project.org/cgi-bin/ltxbugs2html?category=LaTeX&responsible=anyone&state=anything&keyword=lastpage&pr=latex/2705}
%     \item Example |lastpage-example.tex|.
%     \item Alternatives listing (section \ref{sec:Alternatives}).
%     \item Listing of \TeX{} sources (subsection \ref{ss:Downloads}).
%     \item Really a lot of details.
%     \item Complete rewriting of the documentation.
%     \item Everything in \texttt{DTX} framework.
%     \item Included a |\CheckSum|. [Removed in v2.0a.]
%     \item Complete rewriting of the README file.
%   \end{Version}
%   \begin{Version}{2010/08/12 v1.2b}
%     \item Bug fix: |\@PackageInfoNoLine| is only available,
%             if the \xpackage{hyperref} package is loaded.
%             (Bug reported by \textsc{Ulrike Fischer}, thanks!)
%     \item Bug fix: |\ifHy@pageanchor| etc. do not work without
%             \xpackage{hyperref}, and |\else| related to |\ifHy@pageanchor| was
%             wrongly associated with a preceding |\if|, and everything went wrong.
%             Now everything should work again also without \xpackage{hyperref}.
%     \item Renamed |\lastpage@putlabel| to |\lastpage@putl@bel| to get rid of
%             the conflicts with other classes and packages and resulting
%             multiple definitions of the \texttt{lastpage} label.
%   \end{Version}
%   \begin{Version}{2010/08/23 v1.2c}
%     \item Bug fix: Additionally to checking for the \xpackage{hyperref} package
%             |\AtBeginDocument|, when placing the \texttt{lastpage} label it is also
%             checked for the |\hyperref| command, in case \xpackage{hyperref} was not
%             loaded at |\begin{document}| yet. (Bug reported by \textsc{Sebastian Bank},
%             thanks!)\newline
%             [\pkg{lastpagemodern.sty} just uses |\IfPackageLoadedT{hyperref}| and
%              |\IfPackageLoadedTF{hyperref}|.]
%     \item Changed the |\unit| definition (got rid of an old |\rm|). [Removed in v2.0a.]
%     \item Changed |\lastpage@puthyperlabel| to |\lastpage@putlabelhyper| analogous to
%             |\pagesLTS@putlabelhyper| of the \xpackage{pageslts} package.
%     \item Updated version number and date of \xpackage{pagesLTS} package (especially
%             for the check for outdated versions). [Removed in v2.0a.]
%     \item Removed wrong \%\ from the driver file.
%   \end{Version}
%   \begin{Version}{2010/08/25 v1.2d}
%     \item Bug fix: also \xpackage{tcilatex} defines the |\hyperref| command,
%             therefore for \xpackage{hyperref} package detection this had to be
%             changed to |\Hy@Warning|.\newline
%             [\pkg{lastpagemodern.sty} just uses |\IfPackageLoadedTF{hyperref}|.]
%   \end{Version}
%   \begin{Version}{2010/09/12 v1.2e}
%     \item \textsc{James Hedges} pointed out, that there was no instruction in the
%             documentation about suppressing hyperlinks: added (also to the example).
%     \item Diverse small changes.
%   \end{Version}
%   \begin{Version}{2010/09/24 v1.2f}
%     \item Updated to version 2010/09/13 v6.81n of the \xpackage{hyperref} package.
%     \item New version of REV\TeX{}4\ 2010/07/25, v4.1r, old problem.
%     \item New version of \xpackage{pagesLTS} package, 2010/09/22, v1.1k.
%     \item Moved the package from \texttt{.../latex/muench/lastpage/...} to\newline
%             \texttt{.../latex/lastpage/...}.
%   \end{Version}
%   \begin{Version}{2011/02/01 v1.2g}
%     \item Updated to version 2010/04/24 v0.19 of the \xpackage{holtxdoc} package.
%     \item New version of \xpackage{pagesLTS} package, 2011/02/01, v1.1m.
%     \item Updated to version 2010/12/16 v6.81z of the \xpackage{hyperref} package.
%     \item Minor details.
%   \end{Version}
%   \begin{Version}{2011/07/03 v1.2h}
%     \item The \xpackage{holtxdoc} package was fixed, therefore the warning
%             in \xfile{drv} could be removed.~-- Adapted the style of this documentation
%             to new \textsc{Oberdiek} \xfile{dtx} style.
%     \item New versions of \xpackage{pagesLTS}, \xpackage{ulem}, \xpackage{hyperref},
%             \xpackage{papermas} packages.
%     \item Corrected references in the README and manual.
%   \end{Version}
%   \begin{Version}{2011/08/08 v1.2i}
%     \item The \xpackage{pagesLTS} package has been renamed to \xpackage{pageslts}:
%             2011/08/08, v1.2a.
%     \item Some details.
%   \end{Version}
%   \begin{Version}{2011/08/31 v1.2j}
%     \item Updated to \TeX{} Live 2011 (for compiling the documentation and example).
%     \item New version of \xpackage{papermas} package, 2011/08/22, v1.0h.
%     \item Adapted for the use together with packages, which sometimes prevent writing
%             to the \xfile{aux} file. (Bug reported by \textsc{Mikhail Titov}.)
%   \end{Version}
%   \begin{Version}{2011/09/01 v1.2k}
%     \item Fixed |\thepage{}| to \mbox{|\thepage{}|\textvisiblespace,}
%             where there should be a space.
%     \item New version of the \xpackage{hyperref} package, 2011/08/19, v6.82h,
%             but still problem with links to pages with
%             page-\textquotedblleft number\textquotedblright{} in \texttt{fnsymbol}
%             pagenumbering scheme. [Fixed since v6.83m as of 2012/11/06.]
%     \item Documentation update about \textquotedblleft No write access to the
%             \xfile{aux} file\textquotedblright .
%   \end{Version}
%   \begin{Version}{2013/01/28 v1.2l}
%     \item Updated to \TeX{} Live 2012 (for compiling the documentation and example).
%     \item New versions of the packages
%             \xpackage{endfloat}, \xpackage{holtxdoc}, \xpackage{hypdoc},
%             \xpackage{hyperref}, \xpackage{pageslts}, \xpackage{regstats},
%             \xpackage{ulem}, and \xpackage{zref} have become available.
%     \item The \xpackage{nameref} package redefines |\label| to have five arguments
%           instead of two, therefore |\newlabel{LastPage}{{}{\thepage}{}{}{}}| instead of
%           |\newlabel{LastPage}{{}{\thepage}}| must be used.
%           (Bug reported at \url{https://tex.stackexchange.com/q/95541},
%           thanks to \textsc{Micha\l{} Herman}!) Fixed.\newline
%           [Since \LaTeX{} release 2023-06-01 five arguments are standard.]
%   \end{Version}
%   \begin{Version}{2015/03/29 v1.2m}
%     \item Updated to \TeX{} Live 2014 (for compiling the documentation and example).
%     \item Updates to really a lot of details in the documentation (manual \& README).
%   \end{Version}
%   \begin{Version}{2021/09/03 v1.2n}
%     \item Updates to the documentation (manual \& README), to the example, and
%             several small changes in code.
%   \end{Version}
%   \begin{Version}{2023-03-07 v2.0a}
%     \item Removed use of \pkg{ulem}.
%     \item Removed |\unit|.
%     \item \pkg{lastpage} should now determine automatically, whether to load
%             its \TeX~2.09 version, classic \LaTeX2e-version, or modern version
%             with \eTeX, hook{-}man\-age\-ment etc.
%     \item Converted to UTF-8.
%     \item Updated to [then] current \LaTeX{} format 2022-11-01.
%     \item Extensive updates to the documentation (manual \& README) and to the example.
%   \end{Version}
%   \begin{Version}{2023-04-12 v2.0b}
%     \item Bug fix: What should have been |}}{}| was |}{}}|.
%             Thanks to \textsc{Matteo Gamboz} for the bug report!
%   \end{Version}
%   \begin{Version}{2023-07-24 v2.0c}
%     \item Since \LaTeX{} release 2023-06-01 labels have always five arguments.
%     \item Removed the incompatibility warning regarding \pkg{endfloat}
%             before 1994-06-01~(!) from \pkg{lastpagemodern.sty}.
%     \item Replaced |\immediate\write| by a form of (pseudo-code!) |\immediate\protected@write|.
%             (Thanks to \textsc{Ulrike Fischer} for the error report and solution!)
%     \item Update of the |\lastpage@putl@bel| code to \pkg{hyperref} 2023-07-08, v7.01b,
%             which now allows fnsymbol as page numbering scheme.
%   \end{Version}
%   \begin{Version}{2023-10-06 v2.0d}
%     \item For plain arabic page numbers only, \pkg{lastpage} again writes the page number
%             in a plain format in the label in the \xfile{aux} file, so that it can be extracted
%             to perform calculations with it.
%   \end{Version}
%   \begin{Version}{2023-10-14 v2.0e}
%     \item Replaced a |\PackageError| by |\PackageWarningNoLine| as suggested by \textsc{Ulrike Fischer}.
%   \end{Version}
%   \begin{Version}{2024-04-27 v2.1a}
%     \item Utilizes the new |\@currentHpage| provided by the \LaTeX -kernel.
%     \item The issue with |\thepage| (former |\lastpage@nonnumeric| command) has has been fixed
%             at 2023-11-07, \url{https://github.com/latex3/hyperref/issues/303}.
%     \item |\lastpage@IfNumericTF| is no longer needed, but has not been removed
%             so as not to break older documents that use it in a different context.
%   \end{Version}
%   \begin{Version}{2024-07-03 v2.1b}
%     \item The \xclass{beamer} class loads \xpackage{hyperref} partially,
%             so that \xpackage{hyperref} is incorrectly detected as having been loaded.
%             The defining operation for |\lastpage@Hy| has been changed to catch this
%             and other different cases.
%   \end{Version}
%   \begin{Version}{2024-07-07 v2.1c}
%     \item With help from \textsc{David Carlisle} |\lastpage@rmpage| no longer assumes
%             |\@currentHpage| to begin with \mbox{\textquotedblleft |page.|\textquotedblright .}
%   \end{Version}
%   \begin{Version}{2024-11-24 v2.1d}
%     \item The \xpackage{pageslts} package has been repaired, thus here the warnings have been removed.
%     \item Several small changes in documentation and |lastpagemodern.sty| because of the updates
%             of \LaTeX-format (to~2024-06-01), \xpackage{hyperref} package (to~2024-10-30, v7.01k),
%             and \xpackage{pageslts} package (to~2024-11-20, v2.0a).
%     \item Added a warning message about missing\newline
%             |\ProvideTextCommand{...}{PD1}{...}|\newline
%             (cf.~\autoref{subsec:fnsymbol}: \nameref{subsec:fnsymbol}, page~\pageref{subsec:fnsymbol}).
%     \item Documentation section about alternatives rewritten.
%   \end{Version}
%   \begin{Version}{2025-01-27 v2.1e}
%     \item Reorganized beginning of |\lastpage@makeHy|.
%     \item Documentation update.
%   \end{Version}
% \end{History}
% \bigskip
%
% When you find a mistake or have a suggestion for an improvement of this package,
% please send an e-mail to the maintainer, thanks! (Please see BUG REPORTS in the README.)
% \newpage
% \PrintIndex
% \Finale
\endinput|
% \end{quote}
% Do not forget to quote the argument according to the demands
% of your shell.
%
% \paragraph{Generating the documentation.\label{GenDoc}}
% You can use both the \xfile{.dtx} or the \xfile{.drv} to generate
% the documentation. The process can be configured by a
% configuration file \xfile{ltxdoc.cfg}. For instance, put the following
% line into this file, if you want to have A4 as paper format:
% \begin{quote}
%   \verb|\PassOptionsToClass{a4paper}{article}|
% \end{quote}
%
% \noindent An example follows how to generate the
% documentation with \pdfLaTeX :
%
% \begin{quote}
%\begin{verbatim}
%pdflatex lastpage.dtx
%makeindex -s gind.ist lastpage.idx
%pdflatex lastpage.dtx
%makeindex -s gind.ist lastpage.idx
%pdflatex lastpage.dtx
%\end{verbatim}
% \end{quote}
%
% \subsection{Compiling the example}
% The example file, \textsf{lastpage-example.tex}, can be compiled via\newline
% \indent |latex lastpage-example.tex|\newline
% or (recommended)\newline
% \indent |pdflatex lastpage-example.tex|\newline
% and will need at least two compiler runs to get all references right.
%
% \section{Acknowledgements}
%
% I (\textsc{H.-Martin Münch}) would like to thank
% \textsc{Jeffrey P. Goldberg} \mbox{(jeffrey+}\allowbreak news at goldmark dot org) for
% inventing the \xpackage{lastpage} package as well as for allowing me
% to update it. Further I would like to thank \textsc{Heiko Oberdiek}
% for providing a~lot~(!) of useful packages (from which I also learned everything
% I know about creating a file in \xfile{dtx} format, OK, say it: copying).
% Thanks to \textsc{David Carlisle} for the new code for |\lastpage@rmpage|.
% Thanks for bug reports go to \textsc{Ulrike Fischer} (several times),
% \textsc{Sebastian Bank}, \textsc{James Hedges}, \textsc{Mikhail Titov},
% \textsc{Micha\l{} Herman}, \textsc{Matteo Gamboz}, and \textsc{James Small}.
% Thanks to \textsc{Sven Siegmund} for pointing out a necessary further
% explanation in the documentation.
%
% \phantomsection
% \begin{History}\label{History}
%   \begin{Version}{1994/06/17 v0.99a}
%     \item First shot by \textsc{Jeffrey P. Goldberg}.
%   \end{Version}
%   \begin{Version}{1994/06/25 v0.1b}
%     \item Last version number created by \textsc{Jeffrey P. Goldberg}.
%   \end{Version}
%   \begin{Version}{1994/07/20 v0.1b (again)}
%     \item Documentation updated by \textsc{Jeffrey P. Goldberg}.\newline
%             The main source code of the \xpackage{lastpage} package 1994/07/20,
%             v0.1b, was:
%             \begin{verbatim}
%              \NeedsTeXFormat{LaTeX2e}[1994/06/01]
%              \ProvidesPackage{lastpage}[1994/07/20 v0.1b
%                LaTeX2e package for refs to last page number (JPG)]
%              \def\lastpage@putlabel{\addtocounter{page}{-1}%
%                \immediate\write\@auxout{\string
%                \newlabel{LastPage}{{}{\thepage}}}%
%                \addtocounter{page}{1}}
%              \AtEndDocument{%
%                \message{AED: lastpage setting LastPage}%
%                \clearpage\lastpage@putlabel}%
%              \endinput
%             \end{verbatim}
%             and then the \xpackage{hyperref} package (2012/11/06, v6.83m)
%             even redefined |\lastpage@putlabel|; the \xpackage{REV\TeX{}4}
%             class 2022-06-05, v4.2f, still does this.
%   \end{Version}
%   \begin{Version}{2010/02/18 v1.1}
%     \item Proposed |LastPages| label by \textsc{H.-Martin Münch}
%             on \Newsgroup{comp.text.tex}, see e.\,g.\ %
%             \url{https://groups.google.com/g/comp.text.tex/c/Ad8pO2Rw_HY/m/8EfHqT1JB0QJ};
%             now available in the \xpackage{pageslts} package.
%   \end{Version}
%   \begin{Version}{2010/07/29 v1.2a}
%     \item Complete rewriting of the package; upgrade from \xpackage{fancyheadings}
%             to \xpackage{fancyhdr} package, then removed the need for the
%             \xpackage{fancyhdr} package at all.
%     \item Included \textsf{lastpage209.sty} for \LaTeX2.09{}.
%     \item Replacement of |\filedate|, |-version|, |-name|,\ldots{} because
%             of \LaTeX{}~bug 2705:\newline
%             Synopsis: Possible problem with |\fileversion| and |\filedate|\newline
%             \url{https://www.latex-project.org/cgi-bin/ltxbugs2html?category=LaTeX&responsible=anyone&state=anything&keyword=lastpage&pr=latex/2705}
%     \item Example |lastpage-example.tex|.
%     \item Alternatives listing (section \ref{sec:Alternatives}).
%     \item Listing of \TeX{} sources (subsection \ref{ss:Downloads}).
%     \item Really a lot of details.
%     \item Complete rewriting of the documentation.
%     \item Everything in \texttt{DTX} framework.
%     \item Included a |\CheckSum|. [Removed in v2.0a.]
%     \item Complete rewriting of the README file.
%   \end{Version}
%   \begin{Version}{2010/08/12 v1.2b}
%     \item Bug fix: |\@PackageInfoNoLine| is only available,
%             if the \xpackage{hyperref} package is loaded.
%             (Bug reported by \textsc{Ulrike Fischer}, thanks!)
%     \item Bug fix: |\ifHy@pageanchor| etc. do not work without
%             \xpackage{hyperref}, and |\else| related to |\ifHy@pageanchor| was
%             wrongly associated with a preceding |\if|, and everything went wrong.
%             Now everything should work again also without \xpackage{hyperref}.
%     \item Renamed |\lastpage@putlabel| to |\lastpage@putl@bel| to get rid of
%             the conflicts with other classes and packages and resulting
%             multiple definitions of the \texttt{lastpage} label.
%   \end{Version}
%   \begin{Version}{2010/08/23 v1.2c}
%     \item Bug fix: Additionally to checking for the \xpackage{hyperref} package
%             |\AtBeginDocument|, when placing the \texttt{lastpage} label it is also
%             checked for the |\hyperref| command, in case \xpackage{hyperref} was not
%             loaded at |\begin{document}| yet. (Bug reported by \textsc{Sebastian Bank},
%             thanks!)\newline
%             [\pkg{lastpagemodern.sty} just uses |\IfPackageLoadedT{hyperref}| and
%              |\IfPackageLoadedTF{hyperref}|.]
%     \item Changed the |\unit| definition (got rid of an old |\rm|). [Removed in v2.0a.]
%     \item Changed |\lastpage@puthyperlabel| to |\lastpage@putlabelhyper| analogous to
%             |\pagesLTS@putlabelhyper| of the \xpackage{pageslts} package.
%     \item Updated version number and date of \xpackage{pagesLTS} package (especially
%             for the check for outdated versions). [Removed in v2.0a.]
%     \item Removed wrong \%\ from the driver file.
%   \end{Version}
%   \begin{Version}{2010/08/25 v1.2d}
%     \item Bug fix: also \xpackage{tcilatex} defines the |\hyperref| command,
%             therefore for \xpackage{hyperref} package detection this had to be
%             changed to |\Hy@Warning|.\newline
%             [\pkg{lastpagemodern.sty} just uses |\IfPackageLoadedTF{hyperref}|.]
%   \end{Version}
%   \begin{Version}{2010/09/12 v1.2e}
%     \item \textsc{James Hedges} pointed out, that there was no instruction in the
%             documentation about suppressing hyperlinks: added (also to the example).
%     \item Diverse small changes.
%   \end{Version}
%   \begin{Version}{2010/09/24 v1.2f}
%     \item Updated to version 2010/09/13 v6.81n of the \xpackage{hyperref} package.
%     \item New version of REV\TeX{}4\ 2010/07/25, v4.1r, old problem.
%     \item New version of \xpackage{pagesLTS} package, 2010/09/22, v1.1k.
%     \item Moved the package from \texttt{.../latex/muench/lastpage/...} to\newline
%             \texttt{.../latex/lastpage/...}.
%   \end{Version}
%   \begin{Version}{2011/02/01 v1.2g}
%     \item Updated to version 2010/04/24 v0.19 of the \xpackage{holtxdoc} package.
%     \item New version of \xpackage{pagesLTS} package, 2011/02/01, v1.1m.
%     \item Updated to version 2010/12/16 v6.81z of the \xpackage{hyperref} package.
%     \item Minor details.
%   \end{Version}
%   \begin{Version}{2011/07/03 v1.2h}
%     \item The \xpackage{holtxdoc} package was fixed, therefore the warning
%             in \xfile{drv} could be removed.~-- Adapted the style of this documentation
%             to new \textsc{Oberdiek} \xfile{dtx} style.
%     \item New versions of \xpackage{pagesLTS}, \xpackage{ulem}, \xpackage{hyperref},
%             \xpackage{papermas} packages.
%     \item Corrected references in the README and manual.
%   \end{Version}
%   \begin{Version}{2011/08/08 v1.2i}
%     \item The \xpackage{pagesLTS} package has been renamed to \xpackage{pageslts}:
%             2011/08/08, v1.2a.
%     \item Some details.
%   \end{Version}
%   \begin{Version}{2011/08/31 v1.2j}
%     \item Updated to \TeX{} Live 2011 (for compiling the documentation and example).
%     \item New version of \xpackage{papermas} package, 2011/08/22, v1.0h.
%     \item Adapted for the use together with packages, which sometimes prevent writing
%             to the \xfile{aux} file. (Bug reported by \textsc{Mikhail Titov}.)
%   \end{Version}
%   \begin{Version}{2011/09/01 v1.2k}
%     \item Fixed |\thepage{}| to \mbox{|\thepage{}|\textvisiblespace,}
%             where there should be a space.
%     \item New version of the \xpackage{hyperref} package, 2011/08/19, v6.82h,
%             but still problem with links to pages with
%             page-\textquotedblleft number\textquotedblright{} in \texttt{fnsymbol}
%             pagenumbering scheme. [Fixed since v6.83m as of 2012/11/06.]
%     \item Documentation update about \textquotedblleft No write access to the
%             \xfile{aux} file\textquotedblright .
%   \end{Version}
%   \begin{Version}{2013/01/28 v1.2l}
%     \item Updated to \TeX{} Live 2012 (for compiling the documentation and example).
%     \item New versions of the packages
%             \xpackage{endfloat}, \xpackage{holtxdoc}, \xpackage{hypdoc},
%             \xpackage{hyperref}, \xpackage{pageslts}, \xpackage{regstats},
%             \xpackage{ulem}, and \xpackage{zref} have become available.
%     \item The \xpackage{nameref} package redefines |\label| to have five arguments
%           instead of two, therefore |\newlabel{LastPage}{{}{\thepage}{}{}{}}| instead of
%           |\newlabel{LastPage}{{}{\thepage}}| must be used.
%           (Bug reported at \url{https://tex.stackexchange.com/q/95541},
%           thanks to \textsc{Micha\l{} Herman}!) Fixed.\newline
%           [Since \LaTeX{} release 2023-06-01 five arguments are standard.]
%   \end{Version}
%   \begin{Version}{2015/03/29 v1.2m}
%     \item Updated to \TeX{} Live 2014 (for compiling the documentation and example).
%     \item Updates to really a lot of details in the documentation (manual \& README).
%   \end{Version}
%   \begin{Version}{2021/09/03 v1.2n}
%     \item Updates to the documentation (manual \& README), to the example, and
%             several small changes in code.
%   \end{Version}
%   \begin{Version}{2023-03-07 v2.0a}
%     \item Removed use of \pkg{ulem}.
%     \item Removed |\unit|.
%     \item \pkg{lastpage} should now determine automatically, whether to load
%             its \TeX~2.09 version, classic \LaTeX2e-version, or modern version
%             with \eTeX, hook{-}man\-age\-ment etc.
%     \item Converted to UTF-8.
%     \item Updated to [then] current \LaTeX{} format 2022-11-01.
%     \item Extensive updates to the documentation (manual \& README) and to the example.
%   \end{Version}
%   \begin{Version}{2023-04-12 v2.0b}
%     \item Bug fix: What should have been |}}{}| was |}{}}|.
%             Thanks to \textsc{Matteo Gamboz} for the bug report!
%   \end{Version}
%   \begin{Version}{2023-07-24 v2.0c}
%     \item Since \LaTeX{} release 2023-06-01 labels have always five arguments.
%     \item Removed the incompatibility warning regarding \pkg{endfloat}
%             before 1994-06-01~(!) from \pkg{lastpagemodern.sty}.
%     \item Replaced |\immediate\write| by a form of (pseudo-code!) |\immediate\protected@write|.
%             (Thanks to \textsc{Ulrike Fischer} for the error report and solution!)
%     \item Update of the |\lastpage@putl@bel| code to \pkg{hyperref} 2023-07-08, v7.01b,
%             which now allows fnsymbol as page numbering scheme.
%   \end{Version}
%   \begin{Version}{2023-10-06 v2.0d}
%     \item For plain arabic page numbers only, \pkg{lastpage} again writes the page number
%             in a plain format in the label in the \xfile{aux} file, so that it can be extracted
%             to perform calculations with it.
%   \end{Version}
%   \begin{Version}{2023-10-14 v2.0e}
%     \item Replaced a |\PackageError| by |\PackageWarningNoLine| as suggested by \textsc{Ulrike Fischer}.
%   \end{Version}
%   \begin{Version}{2024-04-27 v2.1a}
%     \item Utilizes the new |\@currentHpage| provided by the \LaTeX -kernel.
%     \item The issue with |\thepage| (former |\lastpage@nonnumeric| command) has has been fixed
%             at 2023-11-07, \url{https://github.com/latex3/hyperref/issues/303}.
%     \item |\lastpage@IfNumericTF| is no longer needed, but has not been removed
%             so as not to break older documents that use it in a different context.
%   \end{Version}
%   \begin{Version}{2024-07-03 v2.1b}
%     \item The \xclass{beamer} class loads \xpackage{hyperref} partially,
%             so that \xpackage{hyperref} is incorrectly detected as having been loaded.
%             The defining operation for |\lastpage@Hy| has been changed to catch this
%             and other different cases.
%   \end{Version}
%   \begin{Version}{2024-07-07 v2.1c}
%     \item With help from \textsc{David Carlisle} |\lastpage@rmpage| no longer assumes
%             |\@currentHpage| to begin with \mbox{\textquotedblleft |page.|\textquotedblright .}
%   \end{Version}
%   \begin{Version}{2024-11-24 v2.1d}
%     \item The \xpackage{pageslts} package has been repaired, thus here the warnings have been removed.
%     \item Several small changes in documentation and |lastpagemodern.sty| because of the updates
%             of \LaTeX-format (to~2024-06-01), \xpackage{hyperref} package (to~2024-10-30, v7.01k),
%             and \xpackage{pageslts} package (to~2024-11-20, v2.0a).
%     \item Added a warning message about missing\newline
%             |\ProvideTextCommand{...}{PD1}{...}|\newline
%             (cf.~\autoref{subsec:fnsymbol}: \nameref{subsec:fnsymbol}, page~\pageref{subsec:fnsymbol}).
%     \item Documentation section about alternatives rewritten.
%   \end{Version}
%   \begin{Version}{2025-01-27 v2.1e}
%     \item Reorganized beginning of |\lastpage@makeHy|.
%     \item Documentation update.
%   \end{Version}
% \end{History}
% \bigskip
%
% When you find a mistake or have a suggestion for an improvement of this package,
% please send an e-mail to the maintainer, thanks! (Please see BUG REPORTS in the README.)
% \newpage
% \PrintIndex
% \Finale
\endinput